<DOC>
<DOCNO>FSP950101-001</DOCNO>
<DOCID>FSP950101-001</DOCID>
<DATE>950101</DATE>
<CATEGORY>PRIMEIRA_PÁGINA</CATEGORY>
<TEXT>
Datafolha revela que 70% esperam governo ótimo ou bom; Collor teve 71% antes de assumir a Presidência 
O sociólogo Fernando Henrique Cardoso, 63, toma possa hoje às 16h30 no Congresso como o 38º presidente do Brasil. Eleito em 3 de outubro com 34,3 milhões de votos, 54,28% dos válidos, FHC sucede a Itamar Augusto Cautiero Franco, 63, de quem foi ministro das Relações Exteriores e da Fazenda.
O Datafolha revela que 79% apóiam o real, moeda lançada pela equipe de FHC. Para 70%, seu governo será ótimo ou bom. Em 90, 71% tinham a mesma expectativa em relação a Fernando Collor.
O programa de FHC definiu como prioridades a manutenção do real e as reformas na Constituição. 
</TEXT>
</DOC>
<DOC>
<DOCNO>FSP950101-002</DOCNO>
<DOCID>FSP950101-002</DOCID>
<DATE>950101</DATE>
<CATEGORY>OPINIÃO</CATEGORY>
<TEXT>
Novo comandante do barco Brasil, Fernando Henrique Cardoso será, ao menos nos primeiros tempos de sua gestão, um presidente em busca de uma âncora mais sólida na qual apoiar a estabilização da economia, por ele mesmo iniciada como ministro da Fazenda.
O novo titular da Pasta, Pedro Malan, já avisou, em entrevista concedida a esta Folha: "Uma política de estabilização não pode estar baseada só em câmbio e em política monetária austera".
Ora, é exatamente na sobrevalorização do real em relação ao dólar e em uma austeridade monetária, ainda que apenas relativa, que se ancora hoje o Plano Real.
A âncora que falta e atrás da qual irá FHC, a partir do momento em que se instalar no Planalto, é exatamente a mais difícil de se obter. Chama-se âncora fiscal e significa um equilíbrio mais estrutural das contas públicas.
Para que se obtenha um ajuste fiscal efetivo, é preciso reformular algumas passagens da Constituição brasileira, ponto em que estão de acordo quase todos os economistas, de direita ou de esquerda, do governo ou de fora dele.
O próprio presidente, enquanto candidato, deixou claro que suas primeiras prioridades são as reformas constitucionais voltadas para o saneamento das contas públicas.
Mais exatamente, trata-se das reformas tributária, da Previdência Social e do chamado pacto federativo, ou seja, a distribuição de funções e receitas entre a União, os Estados e os municípios.
Conjunto tão amplo de reformas, como é óbvio, mexerá com inúmeros interesses estabelecidos e hábitos arraigados. Logo, não será nada fácil aprová-las, ainda mais porque a Constituição exige um quórum elevado (3/5 de cada Casa do Congresso, em duas votações) para a aprovação de emendas.
Até certo ponto, o ministério de FHC reflete essas prioridades. Em busca de uma maioria no Congresso, negociou politicamente não apenas com os partidos que compuseram a coligação que o elegeu (PFL e PTB), mas também com um teórico adversário, o PMDB, a legenda de maior bancada em ambas as Casas do Congresso.
O resultado dessa composição só se verá mais adiante e tende a ser decisivo para que o novo presidente satisfaça ou não a expectativa positiva que cerca a sua posse. De acordo com levantamento do Datafolha, são 71% os brasileiros que esperam de Fernando Henrique um governo "ótimo/bom".
É uma porcentagem superior à que deu a vitória a FHC, um sinal positivo. Mas, em contrapartida, é praticamente a mesma porcentagem dos otimistas em relação a Fernando Collor, às vésperas da posse deste, em 1990.
Vê-se, portanto, que a confiança do público nada assegura de antemão ao governante.
Vai ser necessária ação, muita ação –e muito rápida. E de pouco adiantará resolver o desafio da âncora e, por extensão, de uma estabilização mais sólida e duradoura, se ela não servir de plataforma para se enfrentar o desafio social.
A miséria e a desigualdade social no Brasil, é sempre bom lembrar, não foram criadas pela superinflação, ainda que esta tenha agravado profundamente o quadro. Logo, a estabilização é condição necessária mas não suficiente para se iniciar o processo de reversão do escândalo social que é o Brasil.
No caso de Fernando Henrique Cardoso, o compromisso com a questão social não é apenas do candidato, mas uma história de vida, como sociólogo primeiro e como parlamentar depois.
O que se vai ver agora é se as novas alianças que o intelectual costurou para a campanha funcionarão ou não, no governo, como anestésicos para a sua preocupação com o social.
Se funcionarem, o Brasil corre o risco de repetir, 30 anos depois, uma emblemática frase do então presidente, o general Garrastazu Médici: "O país vai bem, mas o povo vai mal".
Seria um triste epitáfio para uma carreira até aqui vitoriosa.
</TEXT>
</DOC>
<DOC>
<DOCNO>FSP950101-003</DOCNO>
<DOCID>FSP950101-003</DOCID>
<DATE>950101</DATE>
<CATEGORY>OPINIÃO</CATEGORY>
<TEXT>
A idéia do futuro ministro da Justiça, Nelson Jobim, de alterar a legislação antidrogas brasileira para diferenciar melhor consumidores de traficantes é de fato interessante. Pode-se até mesmo dizer que chega relativamente tarde ao Brasil.
Países como Espanha e Holanda e mesmo alguns Estados dos EUA já toleram o consumo pessoal de determinadas drogas. No Brasil, contudo, uma pessoa que porte quantidades ínfimas de substância estupefaciente pode ser condenada a até 15 anos de reclusão.
É evidente que um único dia na cadeia –o que não dizer de 15 anos– só vai agravar ainda mais a situação psicológica de alguém já suficientemente abalado para ter cedido ao vício.
Diferenciar claramente o usuário do traficante é, portanto, uma modernização mais do que necessária no Direito Penal brasileiro. Manter a legislação como está é confundir o criminoso com sua vítima. Equivale a punir um assaltado por ter-se deixado roubar.
E já que o futuro ministro parece disposto a promover uma discussão transparente sobre as drogas, por que não fazer avançar, como esta Folha defende, um debate sereno e isento sobre a proposta de, de fato, descriminar o uso de drogas?
Diante do fracasso que se revelou a política de repressão ao tráfico, que apenas consome muito dinheiro dos contribuintes sem ter conseguido reduzir de forma significativa o comércio de tóxicos e toda a violência que o acompanha, vem ganhando adeptos a tese de simplesmente legalizar o uso de tóxicos, a exemplo do que ocorre atualmente com o álcool na maior parte do mundo. Essa idéia tem defensores insuspeitos como o papa da ortodoxia econômica, Milton Friedman, e o liberal semanário britânico "The Economist".
Para os defensores da legalização, a descriminação das drogas privaria os traficantes de sua principal fonte de lucro, o chamado imposto da ilegalidade, que é justamente o que lhes dá o poder de gerar tamanha violência e corrupção. Também tenderia a ocorrer, dizem, uma diminuição nos crimes cometidos por viciados em busca de recursos para comprar drogas.
Os advogados da descriminação dizem ainda que os tóxicos poderiam ser taxados e os recursos daí advindos poderiam ser usados em programas de prevenção e recuperação de drogados, o que, em médio prazo, poderia trazer resultados talvez mais eficientes do que os obtidos com a simples repressão.
Enfim, trata-se de uma discussão complexa em que muitos prós e contras têm de ser analisados e cuidadosamente pesados. E o próximo governo não se pode furtar a essa discussão. Ela envolve, tanto no plano econômico como no da violência e principalmente no da saúde pública, o futuro do país.
</TEXT>
</DOC>
<DOC>
<DOCNO>FSP950101-004</DOCNO>
<DOCID>FSP950101-004</DOCID>
<DATE>950101</DATE>
<CATEGORY>OPINIÃO</CATEGORY>
<TEXT>
Clóvis Rossi 
SÃO PAULO – Para servir de mensagem de Ano Novo, retiro do baú dos guardados poema de Jorge Luís Borges, enviado pelo eterno moleque travesso Carlito Maia, junto com tantos outros escritos. O mesmo poema foi o tema do cartão de Natal do sindicalista Luiz Antônio Medeiros.
Serve para o leitor, mas gostaria que, pelo menos desta vez, também o presidente da República lesse o recado.
Nesse mundo de bom-mocismo, de tentar agradar a todos, os amigos de sempre junto com os adversários da véspera e também os de amanhã, parece se estar perdendo a capacidade de ousar, de inventar ou de chutar o pau da barraca, para servir de contraponto ao tratamento castiço que Borges dá ao idioma.
Uso Borges nem tanto por uma particular paixão pelo notável escritor argentino. Mas porque todo o mundo sabe que não se trata de um revolucionário ou um esquerdista, essas categorias caídas em desuso no mundo do bom-mocismo. Era um reacionário assumido e tudo leva a crer que os reacionários são, hoje, mais ouvidos do que nunca.
Escreve Borges: "Se pudesse viver novamente minha vida, na próxima trataria de cometer mais erros. Não tentaria ser tão perfeito, relaxaria mais. (...) Correria mais riscos, faria mais viagens, contemplaria mais entardeceres, subiria mais montanhas, nadaria mais rios".
Continua o poema: "Eu era um desses que nunca ia a parte alguma sem ter um termômetro, uma bolsa de água quente, um guarda-chuva e um pára-quedas. Se voltasse a viver, viajaria mais leve. Começaria a andar descalço no começo da primavera e continuaria assim até o fim do outono".
Se, como é mais provável, o presidente não ler ou der de ombros, fique o leitor com Bakunin, o anarquista: "Sou um partidário convicto da igualdade econômica e social, porque sei que, fora dessa igualdade, a liberdade, a justiça, a dignidade humana, a moralidade e o bem-estar dos indivíduos, assim como a prosperidade das nações, serão nada mais do que mentiras".
</TEXT>
</DOC>
<DOC>
<DOCNO>FSP950101-005</DOCNO>
<DOCID>FSP950101-005</DOCID>
<DATE>950101</DATE>
<CATEGORY>OPINIÃO</CATEGORY>
<TEXT>
Gilberto Dimenstein 
BRASÍLIA – Fernando Henrique Cardoso assume hoje a Presidência com uma ótima notícia: 70% dos brasileiros, segundo pesquisa Datafolha, acreditam que ele vai fazer um governo ótimo ou bom. Conta com esmagadora confiança do país, o que lhe dá força para executar as reformas. A pesquisa traz, porém, uma advertência: o Fernando que iniciou seu mandato presidencial, em 1990, tinha os mesmos níveis de otimismo. Todos sabem como acabou.
Opinião pública é volúvel, sempre atenta a resultados concretos. Isso, na prática, significa melhores níveis de emprego, salário, educação, saúde, lazer. Fernando Henrique recebe uma população eufórica.
Não é para menos: a classe média vive uma febre de consumo, em meio, pasme-se, à inflação em queda e sem controle de preços. Nunca se viu nada parecido na história do Brasil. Daí as altíssimas taxas de popularidade de Itamar Franco e de Ciro Gomes.
Fernando Henrique está na cômoda situação de pegar o país com inflação baixa e alta taxa de confiança. Mas, ao mesmo tempo, essa comodidade carrega um árduo desafio. Ele se desgasta se a inflação subir ou o consumo cair. A meta da estabilização ainda é um longo percurso, exige sacrifícios e não é rápida.
O gosto da inflação baixa já é conhecido e, certo ou errado, atribuído em grande parte a Itamar Franco. O diferencial de Fernando Henrique será não apenas manter os indicadores econômicos que já se conseguiu, mas melhorá-los –ou seja, baixar ainda mais a inflação. Não é só.
Ao final de seu mandato, os indicadores sociais devem ser menos vexaminosos, o que depende, óbvio, de entrosamento com Estados e municípios. Estou convencido, porém, de que se ele tiver mesmo a "paixão pelo possível", terá sucesso.
Cito, aqui, apenas um detalhe entre milhares de exemplos sobre o efeito da paixão pelo possível. Todos sabem que a Índia é um país miserável. Um dos seus Estados chama-se Keyralla, onde a renda per capita é de US$ 200 por ano. Lá se investiu em educação: a taxa de mortalidade e matrícula escolar são várias vezes melhores do que no Brasil, onde a renda per capita é 14 vezes maior.
</TEXT>
</DOC>
<DOC>
<DOCNO>FSP950101-006</DOCNO>
<DOCID>FSP950101-006</DOCID>
<DATE>950101</DATE>
<CATEGORY>OPINIÃO</CATEGORY>
<TEXT>
Carlos Heitor Cony 
RIO DE JANEIRO – Já acreditei em Deus mas sempre me recusei a acreditar em Papai Noel. Nada contra o bom velhinho. Apenas suspeitava de sua bondade, achava cretina a sua disponibilidade. Em compensação, acreditei em coisa mais fantástica do que Deus e Papai Noel.
Não lembro quem me contou, talvez um tio, um vizinho, talvez a lembrança confusa de primeiras leituras. O fato é que, no dia 31 de dezembro, eu me esforçava para ficar acordado e ver, no céu em cima de minha cabeça, a mudança do ano. Garantiram-me que, à meia-noite em ponto, surgiria um velho muito magro e descarnado, caindo aos pedaços, velho de apenas 365 dias mas alquebrado como se carregasse nas costas 365 séculos. Esse velho traria uma faixa, como as misses e os presidentes da República que se empossam. Na faixa, o número do ano que chegava ao fim.
No mesmo instante, vindo de outro canto, eu veria uma criança, alguma coisa parecida com o menino Jesus que colocam nos presépios. Além da fralda, o guri também traria uma faixa com o número do ano que se inicia.
Durante muitos anos, durante todos os anos antigos do passado, eu esperei por essa cena no céu sobre minha cabeça. Às vezes havia nuvens, outras eu não aguentava, quando saía do cochilo, a meia-noite já havia passado, ficaria para o próximo ano. Depois de certo tempo desanimei –como desanimei de ser maquinista da Central do Brasil e de ser padre.
Só agora, com a chegada dos invernos mais recentes, talvez numa epifania do fim, ou certamente no prenúncio da noite e do coma na memória, começo a embaralhar as recordações. E volta e meia me surpreendo, não acreditando, mas vendo realmente a mudança de ano, o velhinho fatigado e terminal, o guri roliço e noviço assumindo o comando do destino.
Neste final de ano, abri o champanhe de sempre, fumei um Monte Cristo e nem tive coragem de olhar para cima. Não sei chutar cão atropelado. Não iria dar pontapé num velho carcomido, caindo aos pedaços, exausto de meus dias.
</TEXT>
</DOC>
<DOC>
<DOCNO>FSP950101-007</DOCNO>
<DOCID>FSP950101-007</DOCID>
<DATE>950101</DATE>
<CATEGORY>OPINIÃO</CATEGORY>
<TEXT>
Antonio Ermírio de Moraes 
O ano de 1994 nos trouxe algumas boas lições. Relato aqui três delas, aparentemente desconexas mas, na realidade, muito relacionadas entre si.
1) Acabo de ler um relatório que revela um desperdício de alimentos no Brasil de quase US$ 2,5 bilhões decorrente de mau armazenamento e transporte inadequado de arroz, feijão, milho, soja, trigo, hortaliças e frutas. É lamentável ver isso acontecer num país em que tanta gente passa fome por falta de acesso ao que comer.
Além disso, há o desperdício do consumidor, especialmente das classes mais altas, que insiste em pôr no lixo muita coisa que os povos mais ricos guardam na geladeira para comer no dia seguinte. É difícil precisar o volume desse desperdício. Mas, considerando-se que aquelas classes possuem uma renda de aproximadamente US$ 120 bilhões –dos quais uns US$ 30 bilhões vão para alimentação– e considerando-se ainda um desperdício médio de 10%, isso significa uma perda adicional de mais US$ 3 bilhões. Ou seja, o país joga fora mais de US$ 5 bilhões de comida a cada ano. Isso é um absurdo. E, ao mesmo tempo, é um alerta para que tomemos providências para reduzir essa perda.
2) Outra lição importante do final de 1994 veio do México. Ela mostrou que apoiar um plano de estabilização em uma âncora cambial apenas é extremamente perigoso. A estabilização econômica só dá certo quando o ajuste fiscal é feito de modo completo e definitivo. O Brasil precisa levar isso a sério. Do contrário, em dois ou três anos, nossas reservas também irão para o espaço e a crise cambial nos atingirá em cheio –como atingiu o México.
3) Uma outra lição importante, finalmente, veio do Mercosul. A Argentina mostra-se cada vez mais preocupada com o novo mercado por julgar que, tendo a sua indústria sido sucateada por anos de recessão e abertura descontrolada das importações, o Brasil entrará em nítida vantagem comparativa passando a mais vender do que comprar no terreno dos produtos industriais.
Esse raciocínio pressupõe que nossa indústria continue crescendo e se modernizando –o que não está garantido com o atual câmbio deprimido –que afeta os exportadores– e com a entrada desregrada de produtos importados –que afeta os produtores do mercado interno.
O Brasil exporta cerca de US$ 42 bilhões por ano. Isso é menos de 10% do nosso PIB enquanto que, na média, os demais países do mundo exportam cerca de 20%. Temos um grande espaço para crescer nesse campo. Já deveríamos estar exportando US$ 100 bilhões –o que representaria muito mais divisas, empregos, salários e bem-estar.
Neste momento de euforia pelo Ano Novo e pelo novo governo, vale a pena olharmos para esse passado imediato e buscar soluções com base nas três lições. Temos de reduzir drasticamente o vergonhoso desperdício do que produzimos, realizar o ajuste fiscal para garantir a estabilidade da moeda e proteger com muito carinho o parque industrial brasileiro. Mais do que nunca, é imperioso pensar e acreditar no Brasil. Dessa forma, o Brasil terá não apenas um bom ano, mas um final de século altamente promissor.
Antonio Ermírio de Moraes escreve aos domingos nesta coluna.
</TEXT>
</DOC>
<DOC>
<DOCNO>FSP950101-008</DOCNO>
<DOCID>FSP950101-008</DOCID>
<DATE>950101</DATE>
<CATEGORY>OPINIÃO</CATEGORY>
<TEXT>
Não me ocorreu a idéia, nos anos 60, de que FHC estava prometido para um "destino nacional" 
Contrariando uma visão preconceituosa, eis um intelectual que chegou ao mais alto cargo político 
GÉRARD LEBRUN 
Quer o dia 1º de janeiro seja ou não uma virada na história do Brasil (e eu desejo, pessoalmente, que seja), é certo que essa data já é marcante na vida de cada um daqueles, colegas ou alunos, cujo caminho cruzou o do sociólogo que se torna hoje presidente. E eu não devo ser a única destas testemunhas de antanho a colocar-me a mim próprio duas questões sem dúvida fúteis, mas dificilmente evitáveis. A primeira: já me teria ocorrido a idéia nos anos 60 de que Fernando Henrique Cardoso estava prometido para um "destino nacional"? A segunda: do pouco que sei dele, das conversas ou dos seminários dispersos em minha memória, posso tirar alguma indicação do homem de Estado que ele será?
À primeira questão, respondo não, e para minha vergonha, pois o carisma nascente do professor Cardoso não impedia em absoluto tal prognóstico. O que me desviava dele era um preconceito eminentemente francês: a certeza, não formulada, de que o corte entre a vida acadêmica e a vida política é tal, por princípio, que um intelectual notório tem sem dúvida o direito de aspirar ao "Collège de France", mas que ele certamente não está destinado a exercer um alto cargo político, notadamente a primeira magistratura. Desse modo, a partilha é clara: para os intelectuais, o trabalho sobre os conceitos; para os políticos, as "combinaziones"... Esse dogma, o destino de Fernando Henrique o desmente com estrondo: eis aí um homem que obteve as honras universitárias (inclusive, justamente, uma oferta para ensinar no "Collège de France") e que, além disso, acedeu ao mais alto cargo político. Com o distanciamento, parece-me também que Fernando Henrique sempre recusou, de fato, a partilha que acabo de evocar e que ele jamais separou seu trabalho acadêmico ou científico de sua finalidade e ressonância práticas. Esse "pragmatismo" espontâneo era tanto mais notável porque não era comandado por nenhuma sede de honras, por nenhuma ambição visível (vale dizer: medíocre). Esse traço se manteve em seguida. Como suspeitar da menor preocupação com a "carreira" em alguém que, por pelo menos duas vezes, declinou duas ofertas de prestígio que recebera no estrangeiro? Não porque ele se reservasse para Brasília (eram os tempos mais sombrios da ditadura), mas simplesmente, creio, porque os cargos oferecidos o teriam desviado demais do centro de interesse científico que sempre foi o seu, a saber, o Brasil. Tanto é verdade que o patriotismo do novo presidente é indissociável de sua carreira e sua vocação.
Eis então um intelectual no poder. Que comportamento se deve esperar dele? Não se pode evitar refletir sobre essa outra questão. Mas a sua formulação não é muito feliz, tanto o sentido de intelectual foi corrompido pelo uso "mediático" do termo. Mais ou menos a meia distância entre "guru" e "bela alma", este induz a imagens que se ajustam particularmente mal à personalidade de Fernando Henrique, e que nos levariam a exigir dele performances que ele pouco se preocupa em realizar. Esse conceito que se tornou frouxo pertence antes às categorizações pré-fabricadas e pouco adequadas para esclarecer a ação de um governo. Assim seria certamente vão pretender determinar em suas primeiras decisões se a "ética da responsabilidade" prevalece sobre a "da convicção". Para ter a chance de apreciar com equidade, sobretudo nos primeiros momentos, a difícil navegação que vai começar, os "intelectuais" fariam melhor se colocassem entre parênteses as classificações acadêmicas. Seria melhor que eles evitassem aplicar sumariamente as codificações que lhes são familiares. Desejam uma referência erudita? Nestes tempos, é melhor reler o retrato do "homem prudente" na "Ética" de Aristóteles que o do "filósofo-rei" de Platão.
Eu não chegaria, de maneira nenhuma, a ponto de referir-me a Maquiavel. Pois isso seria insinuar que Fernando Henrique Cardoso é homem de sacrificar suas convicções às exigências da habilidade. Ora, se o novo chefe de Estado mostrou que sabe praticar a política como arte do possível, ele igualmente mostrou ao menos ao longo de sua vida que existem princípios sobre os quais ele não transige, e concessões que a flexibilidade tática jamais justificaria. E isso ele o fez à sua maneira: com firmeza, mas sem barulho. Razão a mais para lhe fazer a honra mínima de não comparar a prudência com que agiu até aqui com oportunismo.
O mais rápido exame de seu "cursus" não basta para nos prevenir contra essa confusão? O que deveria impressionar no longo percurso que, hoje, desemboca no exercício do poder supremo é que ele está isento de qualquer corte e, mais particularmente, de qualquer palinódia ou renegação. Um ambicioso medíocre e com pressa de chegar a seu objetivo não caminha dessa forma: ele toma atalhos, não teme guinadas. E o oportunismo nada mais é do que esta decisão de princípio de recorrer, em qualquer circunstância, a todo tipo de expediente. Nada na conduta passada e mesmo na conduta recente do novo chefe de Estado autoriza a lhe conferir tal forma de cinismo. Levar em conta o jogo de forças em uma tática eleitoral é, apesar de tudo, algo completamente diferente de aproveitar, por princípio e continuamente, toda "oportunidade" que se ofereça.
Dir-se-á então que um discurso assim edulcorado pelas necessidades do jogo político não conserva mais grande coisa das convicções outrora professadas? Essa crítica me pareceria ainda impertinente. Além do fato de que um discurso político não poderia ser julgado separadamente da ação determinada da qual faz parte, nada indica, nas circunstâncias, que esse discurso marque uma reviravolta. O tempo, simplesmente, transcorreu. E, se é verdade que aquele que entra no Planalto deixou atrás de si (e eu não digo muito atrás de si) o protagonista de certos seminários paulistanos sobre "O Capital", talvez seja porque a passagem dos anos, sem impor a rejeição do discurso ideológico dos anos 60, exigiu de alguma forma uma decantação deste. Uma separação das análises que, hoje, envelheceriam (no mesmo sentido em que bastam duas ou três décadas para que o estilo de um escritor ou a interpretação de um ator apareçam como datadas) e objetivos que merecem, estes sim, ser reformulados em função do novo teatro de operações, e perseguidos por intermédio de novos recursos. Isso, Fernando Henrique não julgou conveniente dizê-lo expressamente: é a sua prática que tornava manifestos os reajustes que ele considerou útil operar. Eu não digo "autocríticas", pois a palavra seria absolutamente imprópria –seria este um erro pelo qual seria preciso bater no peito dizendo "mea culpa" apenas por ter sido filho do seu tempo? E, sobretudo, de tê-lo sido com uma tão constante lucidez?
Dessa lucidez em circunstâncias críticas (nos primeiríssimos meses de 1964), eu sou um daqueles que poderiam testemunhar pessoalmente, a tal ponto minhas lembranças são precisas. Basta-me dizer que, durante a minha vida, muito poucas pessoas me ensinaram como ele, naquelas horas conturbadas, que a lucidez dos "fortes" não implica o ceticismo e não desvia do engajamento. Falar mais sobre isso seria faltar com a discrição e o respeito em relação ao chefe de Estado. Se essas linhas caírem sob os seus olhos, ele já as achará, talvez, personalizadas demais. Neste caso, que ele me desculpe. Eu não sou o único, no exterior, a me regozijar com o fato de que agora não se pode mais pensar no Brasil sem evocar a personalidade, desde há muito marcante, de Fernando Henrique Cardoso. E este era o único modo que eu tinha de desejar-lhe boa sorte mostrando toda a minha sinceridade.

GÉRARD LEBRUN, 64, professor da Faculdade de Filosofia, Letras e Ciências Humanas da Universidade de São Paulo e da Universidade de Aix-en-Provence (França). É autor, entre outros livros, de "Kant e o Fim da Metafísica" e "O que é Poder".

Tradução de Hélio Schwartsman
Ayrton Senna; Fascinação; Boas festas
</TEXT>
</DOC>
<DOC>
<DOCNO>FSP950101-009</DOCNO>
<DOCID>FSP950101-009</DOCID>
<DATE>950101</DATE>
<CATEGORY>BRASIL</CATEGORY>
<TEXT>
"Como cidadão, pagador de impostos e fiel cumpridor dos meus deveres, julgo-me no direito de, se um dia vier a necessitar, receber uma transfusão de sangue segura. Parabenizo este jornal por levar ao conhecimento da população a batalha que vem sendo travada por médicos que honram o código de ética e, acima de tudo, manifestam respeito pelo cidadão comum, como é o caso do professor Dalton Chamone."
Carlos Henrique Knapp (São Paulo, SP)
 
"Li com satisfação o artigo de 28/12 do professor Dalton Chamone sobre a retomada da vacinação contra a hepatite B. Congratulo-me com ele pelos ingentes esforços que vem desenvolvendo no sentido de melhorar a qualidade do sangue no Brasil, instituindo medidas de maior rigor na detecção de portadores de várias viroses, especialmente a hepatite B. Além disso, o alerta é procedente, já que a vacinação só é eficaz, em termos de saúde pública, quando a cobertura vacinal for realmente abrangente."
Edna Strauss, presidente eleita da Associação Latino-Americana para Estudo do Fígado (São Paulo, SP)
 
"A saúde do Brasil, que não anda muito bem, está em boas mãos com o ilustre médico e professor Adib Jatene no Ministério da Saúde. Certamente não será coisa fácil para o competente cirurgião, já que o sistema de saúde está na UTI do desgoverno administrativo, da fraude e da corrupção. Entretanto, o novo ministro, por seus méritos pessoais, poderá contar com o apoio incondicional da sociedade brasileira nesta luta em favor do direito à saúde e à vida."
Mário Negreiros dos Anjos, da Academia Fluminense de Medicina (Rio de Janeiro, RJ)

Torpedos e manchas
"O ponderado artigo de Marcelo Coelho sobre o 'affair' Weffort/Ministério da Cultura (Ilustrada, 28/12) convida a mais uma consideração de ordem política. É que ambos –Fernando Henrique ao fazer o convite, Weffort ao aceitá-lo sem antes desligar-se do PT (será que haveria convite se ele fosse cidadão privado, às voltas só com sua consciência?)– feriram regra básica do jogo democrático pelo qual sempre se empenharam. O próprio Fernando Henrique havia, antes, chamado a atenção para ela: deve haver oposição consistente ao governo. Portanto, nada de falsos consensos ou de 'torpedeamento' do partido oposicionista com manobras menores. Do jeito como a coisa foi feita, ambos saem manchados."
Gabriel Cohn (São Paulo, SP)
 
"O egoísmo partidário não pode se sobrepor aos interesses maiores da nação. Os técnicos mais competentes e os políticos mais habilitados de cada partido devem estar inteiramente à disposição do presidente eleito pelo povo, não devendo sofrer nenhuma restrição, sanção ou impedimento; ao contrário, os partidos devem se sentir orgulhosos de poder contar em seus quadros com pessoas competentes, que podem contribuir para que os problemas nacionais sejam resolvidos da melhor maneira possível."
Douglas Ribeiro Simões (Piracicaba, SP)

Salário
"A Folha de 25/12, em matéria assinada pela jornalista Silvana Quaglio, da Sucursal de Brasília, divulgou informações que merecem esclarecimentos. A média salarial do Banco do Nordeste situa-se em R$ 2.154,00 (posição de set/94), valor que, acrescido de todos os benefícios, alcança R$ 2.800,00, bem abaixo dos US$ 3.719,02 divulgados na matéria. Mesmo considerando que a fonte de dados utilizada pela jornalista foi estudo já defasado, não se explica que o BNB seja apontado como campeão da média mensal de remuneração, visto que os salários pagos pela instituição devem estar em patamar idêntico ao de outros organismos aos quais o BNB é equiparado. Por outro lado, a média relativamente alta para os padrões salariais da região –cerca de US$ 2.800 hoje, incluídos os benefícios– está associada ao envelhecimento dos quadros do banco. Além disso, a missão do BNB como agente de desenvolvimento regional torna imprescindível a existência de um quadro técnico da melhor qualificação."
Vladimir Spinelli Chagas, diretor de recursos humanos e patrimoniais do Banco do Nordeste do Brasil S/A (Fortaleza, CE)

Resposta da jornalista Silvana Quaglio – Os dados publicados na reportagem "Salário médio em empresa pública pode atingir US$ 3.700 por mês" constam de estudo oficial do governo federal.

Chefe avaliado
"A propósito da reportagem 'Agora é o funcionário quem avalia o chefe' do caderno Empregos (4/12), gostaria de, em primeiro lugar, elogiar a escolha do tema por sua relevância no desenvolvimento das pessoas e das empresas. Em segundo lugar, apontar restrições ao título escolhido para a mesma, que decorrem da seguinte análise: se 'agora' refere-se a 'neste momento', 'presentemente' ou 'atualmente', não corresponde à realidade, pelo menos no Rio Grande do Sul, no qual muito antes de 1993 –referência do texto– tem-se conhecimento de experiências de funcionário avaliando seus chefes. Concordo plenamente com a importância e a avaliação dos aspectos socio-afetivos do desempenho dos chefes. Frente às pressões competitivas da realidade externa à empresa, há que se buscar mais do que nunca o diálogo, o entendimento, a aproximação e coesão da equipe com o seu líder."
Maria Helena Schaan (Porto Alegre, RS)

Transparência
"Formidável a sugestão do leitor Francisco Cordeiro Afonso, no dia 26/12, sobre a criação de uma coluna semanal com a publicação dos nomes dos deputados federais e senadores que votam a favor e contra em assuntos polêmicos, criando uma alternativa ao festival de meias-verdades que se instala nos programas eleitorais."
Obadias de Deus (São Paulo, SP)

Ayrton Senna
"O fato mais marcante de 94 foi, infelizmente, a morte de Ayrton Senna. Que a vida digna e vitoriosa deste campeão sirva de inspiração para que 95 seja efetivamente um feliz ano novo."
Marcos Moreno (Varginha, MG)

Fascinação
"Pelo que se vê na TV e nos jornais, a vaidade de ocupar um cargo público ainda é um grande sonho para muitos. A acirrada disputa pelo secretariado do governador eleito Mário Covas e pelo ministério do presidente Fernando Henrique Cardoso prova que a função pública, apesar de todo o desgaste da classe política, ainda fascina muita gente."
Francisco das Chagas Fernandes (São Paulo, SP)

Boas festas
A Folha agradece e retribui as mensagens de boas festas que recebeu de: Pirelli; Volvo do Brasil; Parmalat; Andrade Gutierrez; Sadia; Heraldo Mu¤oz, embaixador do Chile; Consulado Geral Americano; Bolsa de Mercadorias & Futuros; Arno; Associação Paulista de Supermercados; Departamento de Comércio Exterior da Áustria; Pluna Lineas Aereas Uruguayas; Hyatt Hotels Corporation; Lazzati Engenharia; Ilace; Francisco F. M. Paes de Barros; Murillo Antunes Alves, vereador em São Paulo; hospital Cruzeiro do Sul; Câmara Municipal de Bastos; Antonio Paiva, vereador; Marson; Marfinite; Rendic International Corporation; TV1 Comunicação; Galeria de Arte André; Associação Brasileira das Indústrias de Queijo; Nelson Americo de Godoy; Refúgio Ecológico Caiman; Naoum Plaza Hotel; José Menino de Miranda; Henrique L. Alves; Renato Toledo de Queiroz; Logus Propaganda; Huélinton Cassiano Riva; Associação Nacional dos Funcionários do Banco do Brasil; Sindicato Rural de Cornélio Procópio; Carlos Alberto Nobre Safadi; Lar Escola São Francisco; Pannon Gráfica; Clínica Veterinária Rebouças.
</TEXT>
</DOC>
<DOC>
<DOCNO>FSP950101-010</DOCNO>
<DOCID>FSP950101-010</DOCID>
<DATE>950101</DATE>
<CATEGORY>BRASIL</CATEGORY>
<TEXT>
Itamar estava em Washington para um encontro com Bill Clinton. Como não fala inglês, precisou de um intérprete durante a reunião de dez minutos.
Ao sair da Casa Branca, foi abordado por uma jornalista da CNN, que, em espanhol, pedia sua opinião sobre a absolvição de Collor pelo STF.
Sem jeito, mas bem-humorado, Itamar disse que só falava "um poquito" de espanhol. E arriscou:
– Yo dizia que lo resultado da absolvição de Collor não nos cabe um julgamento.
Itamar ainda soltou um "los brasileiros" para completar a frase, com outras dez palavras em português.
A repórter tentou a mesma pergunta em inglês. Desta vez, Itamar confessou:
– Eu não saberei te responder em inglês.
Itamar estava acompanhado de alguns assessores, mas ninguém o ajudava a se desvencilhar da repórter americana. Como ela insistia em fazer perguntas em inglês, o presidente só teve uma escapatória. Virou-se para trás e disparou, antes de entrar no seu carro:
– No observations!
</TEXT>
</DOC>
<DOC>
<DOCNO>FSP950101-011</DOCNO>
<DOCID>FSP950101-011</DOCID>
<DATE>950101</DATE>
<CATEGORY>BRASIL</CATEGORY>
<TEXT>
A Folha, em editorial na quarta-feira sob o título "Chega de roubalheira", comenta a apresentação do relatório final da Comissão Especial de Investigação (CEI) sobre suspeitas de irregularidades no Executivo. O editorial afirma que é razoável imaginar que o relatório não revele mais que a "ponta do iceberg", mas que oferece um mapa inicial da corrupção na administração pública. "No caso dos transportes, por exemplo, um dos setores mais visados no relatório, o sobrepreço médio na construção de estradas, segundo a CEI, é de 40%", diz o editorial. "O próximo governo será então posto à prova desde o seu início, com o desafio de mostrar, com celeridade e ações concretas, se vai ou não compactuar com o binômio corrupção-impunidade que há tanto sangra o país".
</TEXT>
</DOC>
<DOC>
<DOCNO>FSP950101-012</DOCNO>
<DOCID>FSP950101-012</DOCID>
<DATE>950101</DATE>
<CATEGORY>BRASIL</CATEGORY>
<TEXT>
ANTONIO CARLOS SEIDL
 Da Reportagem Local 
O ano de 1995 será de "ou vai ou racha" para a indústria brasileira. Exposta à abertura comercial, ela reivindica, para sobreviver, a redução das condições de desigualdade com os importados.
O empresariado industrial considera de maneira unânime que o modelo de mercado fechado e de substituição das importações está esgotado e que a abertura "irreversível" é a nova estratégia de desenvolvimento do país.
A indústria quer, porém, que as importações sejam acompanhadas de incentivos a investimentos produtivos no país.
Na avaliação de empresários ligados à Fiesp (Federação das Indústrias do Estado de São Paulo) e ao Iedi (Instituto de Estudos para o Desenvolvimento Industrial) a abertura indiscriminada pode levar ao atraso tecnológico da indústria brasileira que passaria a se especializar em produtos "populares".
Carlos Eduardo Moreira Ferreira, presidente da Fiesp diz que o empresariado industrial não é contra a abertura às importações. A Fiesp, diz ele, vai insistir nas reformas estruturais.
Moreira Ferreira diz que é impossível falar a sério em competitividade da indústria brasileira sem a correção dessas distorções e enquanto o país não tiver juros iguais aos do mercado internacional.
Adauto Posa Ponte, presidente da Abifa (associação de fundição), diz que a abertura às importações estava atrasada mas muito mais atrasada está a modernização das instituições do país.
O empresariado acha que, diante do atraso das instituições, a abertura colocou o setor numa posição de desvantagem em relação aos manufaturados no exterior.
A Fiesp calcula que a indústria carrega uma carga de acréscimo de custos de 15%%, o chamado "fator Brasil", em comparação com os produtos importados.
O setor exportador da indústria brasileira se queixa ainda da defasagem cambial, a valorização de 15% do real frente ao dólar desde a introdução do real.
"É exigir demais que a indústria brasileira concorra com os importados, saindo com uma desvantagem de custos de 30% por questões fora do controle gerencial das empresas", diz Ponte.
Franz Reimer, executivo da Bosch e diretor da Fiesp, diz que o rápido crescimento das importações e a queda das exportações podem causar problemas para a indústria brasileira em 95. Ele prevê uma mudança no perfil da indústria que, para sobreviver, será menos sofisticada passando a se especializar em produtos populares.
Carlos Eduardo Uchôa Fagundes, presidente do sindicato do setor de iluminação, não concorda com a avaliação de Reimer. "Cerca de 500 empresas que operam no país obtiveram nos últimos dois anos a certificação de qualidade total da série ISO 9000, o que atesta a alta capacidade tecnológica de nosso parque industrial."
</TEXT>
</DOC>
<DOC>
<DOCNO>FSP950101-013</DOCNO>
<DOCID>FSP950101-013</DOCID>
<DATE>950101</DATE>
<CATEGORY>BRASIL</CATEGORY>
<TEXT>
Da Reportagem Local 
O empresariado industrial paulista descarta a hipótese do início de um período de recessão econômica em janeiro.
Eles apostam em uma taxa de crescimento anual entre um nível moderado de, no mínimo, 5% até um patamar de 8% em 95.
Carlos Eduardo Uchôa Fagundes, do setor de lâmpadas, diz que o Brasil começa em 95 um "círculo virtuoso" muito importante.
"Se conseguirmos fazer as reformas tributária e fiscal e desonerar a mão-de-obra e a produção vamos todos ficar ricos em dois anos", prevê.
Uchôa Fagundes acredita que o novo Congresso, apesar das pressões corporativistas e o fisiologismo político, apoiará o projeto de reformas estruturais do presidente Fernando Henrique Cardoso.
"A maioria do novo Congresso foi eleita por causa de seu apoio ao Plano Real e por entender a necessidade de reformas de sustentação da nova moeda", argumenta.
Franz Reimer, industrial do setor eletroeletrônico vê com otimismo as perspectivas econômicas em 95. Para ele, o crescimento das atividades industriais poderá chegar a 8% em 95. "A onda de consumo, impulsionada pelo fim do imposto inflacionário, vai levar a indústria para mais um ano muito bom".
Luiz Carlos Tripodo, diretor de Relações Externas da Bayer, diz que as atividades econômicas no primeiro trimestre de 95, por razões sazonais, não terão o mesmo ritmo de expansão dos últimos três meses de 94.
"Em termos gerais, 95 será de consolidação do crescimento observado principalmente no segundo semestre de 94 com a introdução do real", diz. A Bayer, afirma, traça um cenário de otimismo para 95, principalmente se as reformas estruturais forem aprovadas pelo novo Congresso.
Tripodo prevê um crescimento de 5% em 95, uma taxa, diz, "muito boa" para a consolidação da retomada do desenvolvimento. "A partir de 96, superadas as dificuldades estruturais, talvez consigamos taxas melhores", afirma. 
(ACS)
</TEXT>
</DOC>
<DOC>
<DOCNO>FSP950101-014</DOCNO>
<DOCID>FSP950101-014</DOCID>
<DATE>950101</DATE>
<CATEGORY>BRASIL</CATEGORY>
<TEXT>
RICARDO SEMLER 
Ufa. Todo final de ano tem cara de ufa. Este ano foi do Itamar. Dizem que o homem é sortudo mesmo. Há os que acreditam que a influência dele no rumo das coisas foi idêntica à sua interferência no tetra. Mas é sacanagem reduzi-lo a isto.
Gozaram bastante quando ele assumiu e eu fiz profissão de fé no efeito mineirinho. Sempre achei que o Brasil precisava de uma transição suave, de um estilo low-profile e de um endereçamento genérico na direção do sério-leve.
Tiram sarro do fusquinha, mas o Itamar foi o melhor Diretor de Planejamento Estratégico que as montadoras já tiveram. A Fiat cresceu horrores por conta, a Ford investe no pequeno Fiesta, o Corsa vai longe e até a Autolatina, tentativa desastrada de criar um truste, vai à lona.
Foi preciso um cliente preferencial da Ducal para demonstrar aos engomados de Giorgio Armani que no Brasil o que vende é o popular. Os saudosos da bandalheira, que são muitos, ainda vêem no Fernandinho de Aspen a verdadeira visão de estadista, e sonham com a sua volta.
Aliás, volta na hora que quiser e ainda esnoba os tais dos éticos se elegendo, junto com Maluf e Quércia, deputado federal pelo Estado de São Paulo.
Falando nisso, chegou a hora de São Paulo. Os empresários esfregam as mãos em antecipação à República Paulista. Nem café com leite teria alternado com Juiz de Fora tão bem. Lambem-se os beiços. Em vão.
Primeiro, porque São Paulo é basicamente um fracasso. Tem mais favelas do que o Rio, cresceu um terço o número de seus homicídios no último ano, é festival de motins de presídios, um Estado quebrado e tradicional eleitor de ladrões para seus líderes. Tem o trânsito do Cairo, poluição quase mexicana, esgoto aberto posando de rio Sena, e mais desvio de verba do que o Nordeste somado. A indústria foge e a Fiesp definha.
O paulistano, cheio de ares subdesenvolvidos em seus Jardins que arrotam sonhos cucarachas do figurino Miami–Cancún–Saks, escolhe mal até seus vereadores. Foi assim que se venderam nesta semana por cargos e favorezinhos, abrindo caminho para que o prefeito quebre a cidade da mesmíssima maneira que Quércia e Fleury faliram o Estado, via obras. Não é só o IPTU que aumentou seu valor venal.
Mas o Brasil está bem encaminhado e andará bem. Não vamos fazer de conta, porém, que isto tenha algo a ver com paulistas. Sejamos francos. Com exceção de alguns Estados ainda dominados por coronéis atrasados em fim de carreira, como a Bahia ou o Amazonas, o país escalou um time diversificado e de razoável competência.
Anos bons deverão vir, não na toada da sandice que a TV e a mídia andam prometendo em sua bajulação incontida do presidente eleito, mas virão. Receios de pefelização das verbas, má influência da ala da coligada turma da ditadura, tudo isto será secundário.
Aliás, Fernando Henrique é esperto o suficiente para arranjar uns pirulitos para esse pessoal se lambuzar. E lambuzar-se-ão. Prova é o ministério, balanceado e claramente respeitador da orientação do novo presidente, o mais preparado e inteligente que já tivemos.
Começa a raiar um ano novo, com ar de novo. Para os coitados que têm de comparecer à posse para mostrar fidelidade ou cavar influência, os pêsames da casa. Para os que se esticam em direção a uma caipirinha, votos de um ano mais descontraído. E para os que acham que está por raiar uma era paulista, a advertência de que a falta de protetor solar e moral deixa a pele vermelhinha, conhecida em Brasília por camarão à paulista. Dá-lhe 95!
</TEXT>
</DOC>
<DOC>
<DOCNO>FSP950101-015</DOCNO>
<DOCID>FSP950101-015</DOCID>
<DATE>950101</DATE>
<CATEGORY>BRASIL</CATEGORY>
<TEXT>
MARCELO LEITE 
Responda rápido: em que dia começa o século 21? Se pensou 1º de janeiro de 2000, errou. O próximo milênio começa daqui a exatos seis anos, e não cinco. Ou seja, no primeiro dia de 2001.
É simples, como explicou minha colega Gina Lubrano, ombudsman do jornal norte-americano "The San Diego Union-Tribune", em coluna recente.
O primeiro ano da Era Cristã foi o de número um, não zero. Portanto, o primeiro século terminou em 31 de dezembro do ano 100, e não 99. E assim por diante...
Será difícil convencer as pessoas de que teriam de esperar mais 366 dias pelo terceiro milênio. A maioria vai comemorar no réveillon de 2000. O de 2001 ficará reservado para os matematicamente corretos.
De minha parte, vou já preparando-me para festejar duas entradas milenares, o que nem Matusalém conseguiu. Vai ser "D+", como escreve a moçada.
 
Essa confusão comum com anos, séculos e milênios é só mais um exemplo da dificuldade que as pessoas tem com números. Ela é tão séria que se deveria adotar o neologismo "inumerado" (por analogia com "iletrado").
Lamento dizer que, nesta matéria, ninguém supera os jornalistas.
Em primeiro lugar porque todos eles, do mais beócio dos focas ao mais posudo dos editores, fetichizam os números. Em sua suposta assepsia, as cifras encarnariam a mítica da objetividade, confundindo-se com a própria noção de informação.
Julgando-se no papel de oráculos –ou "interface"– entre leitores mortais e esse Olimpo de infinitos deuses, repórteres e redatores acabam por confundir Ártemis com Palas-Atena e escrevem milhão em lugar de bilhão. Decerto acham que ninguém vai perceber.
Enganam-se cubicamente: há mais leitores atentos do que se poderia imaginar. Muitos deles terão notado que escrevi acima 366 e não 365 dias, para me referir ao último ano do segundo milênio. Boa parcela concluiria que foi erro de digitação, mas os verdadeiramente atentos se perguntarão: será que 2000 é bissexto?
É.
 
Em homenagem a esses leitores que reverenciam a verdadeira divindade dos números, a Exatidão, recolhi nestes três meses alguns delitos representativos dessa idolatria negligente que jornalistas dedicam aos números. Começo por um caso muito curioso do caderno Folhateen, um dos poucos que motivou resposta da Redação ao ombudsman.
A multiplicação dos padres - Em 31 de outubro, uma segunda-feira, o caderno dedicado a jovens ("teens", mais uma macaquice da classe média Miami-Reebok-celular) trazia embasbacante reportagem na capa. Contrariamente a tudo que se ouviu e viu em 20 anos, o título sustentava que "Vida nos seminários volta a atrair jovens".
O busílis aritmético estava na linha-fina, jargão de jornalistas para o subtítulo usado para explicar o título: "Procura pela atividade religiosa entre os adolescentes no Brasil aumentou quase dez vezes na última década" (destaque meu).
Ora, a leitura do texto revelava que o total de noviços em todo o Brasil passara de 64 em 1980 para 657 em 1990. Apontei na crítica interna da edição que redijo diariamente que a quantidade tinha aumentado quase 11 vezes, não 10, e supus confusão com o percentual citado no texto, de 926%.
Recebi então da editoria um arrazoado particularmente ilustrativo da mentalidade segundo a qual vale quase tudo para não publicar um Erramos. Leia:
"O aumento verificado foi de 926% (...). O aumento foi de 593 seminaristas, portanto, é de 9,26 vezes. Isso porque os 64 seminaristas iniciais não podem ser incluídos na conta; eles já estavam lá desde o princípio.
"A confusão que (...) o ombudsman cometeu se deve ao modo genérico de as pessoas se referirem a estes aumentos, que é, entretanto, inexato.
"A lógica da coisa é a seguinte: quando uma coisa aumenta uma vez, ela dobra, isto é, tem aumento de 100%. Se dissermos que aumentou duas vezes, o número terá triplicado. Por exemplo, se o número inicial for 10, um aumento de duas vezes levaria a cifra a trinta (10'(10x2)=30), triplicando o nosso dez. E assim por diante."
É ocioso dizer que a aula não convence. Os 64 noviços não "estavam lá desde o princípio" coisa nenhuma, pois em dez anos já tinham tido tempo suficiente para se ordenar ou desistir de vez do celibato. Os 657 de 1990 eram todos seminaristas frescos.
Isso para não argumentar que qualquer pessoa normal diria que o aumento foi superior a dez, se apresentada aos números 64 e 657. E é para pessoas normais que se escreve jornal.
As vendas negativas - Um mistério cerca as edições regionais da Folha (cadernos que circulam em parte do interior do Estado de São Paulo). Por duas vezes neste trimestre comprovaram em títulos de primeira página que as porcentagens fazem os jornalistas escorregarem também na queda, e não só na subida.
"Comércio tem queda de 550% nas vendas", alardeava a Folha Sudeste para a região de Campinas no último dia 13. Sua congênere Folha Norte (região de São José do Rio Preto) anunciara 16 dias antes: "Preço de importados é até 115% mais baixo".
Nada pode cair mais do que 100%, pois não é possível vender menos do que zero. A não ser, é claro, que comerciantes imbuídos do espírito natalino, ou inebriados com o perfume do real, tenham decidido pagar para que os clientes levem suas mercadorias.
A cidade dos deprês - No dia 16 de novembro, o jornal publicou uma curiosa reportagem: "50% dos moradores de Itapuí usam calmantes". Fiquei imaginando que maldição poderia ter levado essa cidadezinha 330 km a noroeste da capital paulista à beira de um ataque de nervos.
Impotente para resolver o enigma, apresentei à Redação o seguinte raciocínio, com base no pouco de bom senso que me resta após 15 anos de jornalismo:
"Parece exagero. Afinal, a cidade deve ter crianças e jovens, também. Se seguir a distribuição etária da população brasileira, arriscaria dizer que metade da população da cidade deve corresponder a toda a população maior de 18 anos. Todos, sem exceção, tomam calmantes?"
Em 22 de novembro, escreveu-me o itapuiense Antonio Carlos Bueno de Moraes. Seu fax demoveu-me de desacreditar definitivamente do dito do filósofo francês René Descartes (1596-1650), "o bom senso é a coisa mais bem-repartida do mundo":
"Considerando que a pequena Itapuí possui 9.051 habitantes (censo 1991) e que 41% têm até 19 anos (3.674 habitantes), praticamente o restante da população estaria consumindo calmantes."
Tanto Moraes quanto este ombudsman tinham deixado de ver, na edição do dia 19, uma nova reportagem retificando a informação alarmante (provavelmente por ter sido editada com destaque menor do que o delírio anterior). O contingente correto de consumidores de calmantes era 16% –da população adulta!
 
Resumo da ópera: o bom dos números, em sua clareza e distinção (para voltar a Descartes), é que não exigem talento especial para a verificação; o ruim de sua simplicidade é que jornalistas não lhes atribuem valor e não têm pudor de atropelá-los, como de hábito fazem com a gramática.
</TEXT>
</DOC>
<DOC>
<DOCNO>FSP950101-016</DOCNO>
<DOCID>FSP950101-016</DOCID>
<DATE>950101</DATE>
<CATEGORY>BRASIL</CATEGORY>
<TEXT>
Louve-se a boa intenção do redator. Provavelmente, acreditou que estava evitando um erro de concordância. Sem dó, cravou o acento circunflexo da terceira pessoa do plural do verbo "ter", fazendo-o concordar com "coisas" (é a única explicação que encontro para tamanho escorregão).
São muitos erros em um só. Para começar, "ter" não pode ser usado no sentido de "haver", neste caso. Esse uso já está consagrado na linguagem coloquial, invadiu mesmo a música popular ("Tem dias que a gente se sente / Como quem partiu ou morreu"), mas escrever isso em jornal já é demais.
Depois, mesmo que o emprego fosse legítimo, a concordância estaria errada, por dois motivos.
O primeiro é que mesmo "haver" não iria para o plural. "Coisas" é complemento e não sujeito.
Por fim, porque o complemento é "uma porção de coisas". Se fosse sujeito, o verbo ainda assim permaneceria no singular: "Uma porção de coisas contribui para que se erre tanto em jornal".
Tem gente que parece cega.

MARCELO LEITE é o ombudsman da Folha. O ombudsman tem mandato de um ano, renovável por mais um ano. Ele não pode ser demitido durante o exercício do cargo e tem estabilidade por um ano após o exercício da função. Suas atribuições são criticar o jornal sob a perspectiva do leitor –recebendo e checando as reclamações que ele encaminha à Redação– e comentar, aos domingos, o noticiário dos meios de comunicação. Cartas devem ser enviadas para a al. Barão de Limeira, 425, 8º andar, São Paulo (SP), CEP 01202-001, a.c. Marcelo Leite/Ombudsman. Para contatos telefônicos, ligue (011) 224-3896 entre 14h e 18h, de segunda a sexta-feira.
</TEXT>
</DOC>
<DOC>
<DOCNO>FSP950101-017</DOCNO>
<DOCID>FSP950101-017</DOCID>
<DATE>950101</DATE>
<CATEGORY>BRASIL</CATEGORY>
<TEXT>
CARLOS EDUARDO LINS DA SILVA
 De Washington 
A American Telephone and Telegraph Company, que em breve vai mudar seu nome formal para a sigla com que ficou conhecida nos 130 países em que opera –AT&T–, tem grandes planos para o Brasil, onde atualmente tem 400 empregados e escritórios em São Paulo, Rio e Brasília.
Tudo depende da rapidez com que o novo governo venha a liberalizar o mercado de telecomunicações o que, para ela, não significa apenas privatizar companhias estatais, mas —principalmente — incentivar a competição.
Quando isso ocorrer, a AT&T estará pronta para investir em pouco tempo mais do que no México, onde acaba de anunciar a criação de uma joint venture que vai aplicar US$ 1 bilhão em "alguns anos".
Essas informações foram dadas por Victor Pelson, vice-presidente executivo e líder da equipe de operações globais da empresa, em entrevista à Folha em Miami (EUA), no início de dezembro, durante a Cúpula das Américas.
A seguir, trechos da entrevista, da qual também participou o diretor de relações públicas da AT&T no Brasil, Fabio Steinberg:
 Folha – Quais são os planos da AT&T para o Brasil?
Victor Pelson - A AT&T está extremamente interessada em aumentar sua presença no Brasil. Nós estamos falando de uma das maiores economias do mundo, que atravessa dramáticas mudanças. Nós sabemos que a necessidade de infra-estrutura de comunicação estará entre as grandes prioridades para o Brasil nas próximas décadas e nós queremos estabelecer uma posição forte nesse mercado.
Por enquanto, a quantidade de negócios que temos feito no país é relativamente pequena. Mas à medida em que a economia se expandir e o mercado se abrir, nós esperamos que as coisas mudem.
Folha - Com que velocidade o sr. prevê que essas mudanças venham a ocorrer?
Pelson - Nós temos acompanhado ao redor do mundo, país por país, esse processo de liberalização da economia. Uma coisa é privatizar empresas estatais. Outra, muito diferente, é abrir o mercado para a competição. A privatização por si só não ajuda muito. O mais importante é encorajar a competição.
Com que velocidade isso vai ocorrer no Brasil, nós não estamos em posição de dizer. Só podemos esperar que, se a nova liderança do país estiver mesmo interessada em atrair investimentos externos e em ter os sistemas de comunicação de que precisa, isso aconteça com bastante rapidez.
Nós vimos isso ocorrer em várias partes da Ásia e agora começando no México e no Chile e esperamos que o Brasil, é claro, com o enorme tamanho de sua economia, também venha a fazer o mesmo.
Folha - O que vocês aprenderam com sua experiência em outros países que vocês não vão querer repetir no Brasil?
Pelson - É sempre difícil descrever nossos erros. Mas eu vou tentar responder.
Primeiro, é muito importante estabelecer boas relações e ter um bom entendimento com os agentes-chaves do governo e da indústria nacionais. A indústria da comunicação está sempre muito vinculada ao governo devido a questões de regulamentação.
Segundo, é importante se entender que se trata de estabelecer uma relação de longo prazo no país: não somos uma companhia em busca de lucratividade rápida; temos que ter um compromisso de longo prazo com o país em que investimos.
Terceiro, precisamos ter certeza de que a tecnologia que oferecermos ao país será a mais recente e que ela vai se adaptar às necessidades do país; temos que entender a cultura, a economia, a política do Brasil em vez de fazer negócios como costumamos fazer nos EUA ou na China ou no México.
Quarto, e isso é uma coisa que nós nem sempre fizemos em outros países, temos que fazer com que a liderança da AT&T no Brasil seja brasileira, seja formada por pessoas que entendam a cultura, a língua, os costumes brasileiros.
Folha - Vocês ainda não têm no Brasil um presidente da empresa, têm?
Pelson - Ainda não.
Folha - Estão procurando?
Pelson - Sim. E no momento certo, seu nome será divulgado.
Folha - Quando será o momento certo?
Pelson - Mais ou menos... depressa. Em pouco tempo. Quanto mais cedo, melhor. O cronograma vai ser ditado pelo momento em que acharmos a pessoa, não pelo nosso desejo de encontrá-la.
Folha - Se no México vocês investiram US$ 1 bilhão, é seguro supor que no Brasil a AT&T está preparada para aplicar mais?
Pelson - Se o Brasil abrir o seu mercado, nós estamos preparados para investir significativamente no país. Como a economia brasileira é bem maior que a mexicana, com certeza mais do que aplicamos lá.
</TEXT>
</DOC>
<DOC>
<DOCNO>FSP950101-018</DOCNO>
<DOCID>FSP950101-018</DOCID>
<DATE>950101</DATE>
<CATEGORY>DINHEIRO</CATEGORY>
<TEXT>
O espetáculo multimídia, com cinco grupos internacionais de teatro, não empolgou o público anteontem 
CRISTINA GRILLO 
Da Sucursal do Rio 
O espetáculo "Ópera Mundi - Um Sonho BOM" não empolgou as cerca de 30 mil pessoas (segundo avaliação do Corpo de Bombeiros) que foram anteontem à noite ao Maracanã (zona norte do Rio).
Os aplausos foram poucos. Do alto da arquibancada não se podia ver bem o que acontecia no gramado. Um dos pontos altos do espetáculo, que contou com a participação de cinco grupos internacionais de teatro, foi a "cachoeira humana". Em andaimes instalados em um setor das arquibancadas, um grupo de 20 alpinistas simulou uma queda d'água.
Outro momento que despertou o público foi a aparição de "El Corredor" –uma estátua articulada, de 7 metros de altura. Logo depois de a estátua dar uma volta olímpica no estádio, de um dos vestiários surgiu o jogador de futebol Zico.
Em breve aparição, Zico chutou uma bola de futebol na "onçapéia" –alegoria que representava a destruição da humanidade.
Preocupados com a quantidade de fogos de artifício que seriam usados durante o espetáculo, o Corpo de Bombeiros mandou ao Maracanã uma equipe reforçada. Cerca de 50 homens equipados com extintores de incêndio e muitas mangueiras foram ao estádio.
O espetáculo começou às 21h34, com 34 minutos de atraso. Depois de uma hora e meia de espetáculo, os quatro telões mostraram a imagem de Tom Jobim. O coral da Comlurb (Companhia de Limpeza Urbana do Rio) cantou o "Samba do Avião", de Jobim.
O fim do espetáculo foi comemorado pelos participantes. O gramado virou uma festa. Figurantes corriam cantando o refrão "U-tererê" (cantado pelos grupos que frequentam bailes funk).
Os acrobatas do grupo argentino De La Guarda fizeram sucesso ao atravessar os 230 metros de diâmetro do estádio pendurados em cabos de aço.
</TEXT>
</DOC>
<DOC>
<DOCNO>FSP950101-019</DOCNO>
<DOCID>FSP950101-019</DOCID>
<DATE>950101</DATE>
<CATEGORY>DINHEIRO</CATEGORY>
<TEXT>
A Tecnologia Bancária –do Bamerindus, Nacional e Unibanco– informa que o pagamento de contas através de cartão magnético –cheque eletrônico– está crescendo a uma taxa média de 10% ao mês. Segundo a empresa, a Telesp prepara-se para disponibilizar no serviço de Teletexto, através do Banco Interativo de Serviços, a possibilidade de utilização do cheque eletrônico a partir de um microcomputador. 

Banco Fenícia se une a Laffer Canto 
Criada a Fenícia/Laffer Investment Partner com sede em Miami (EUA). A empresa é uma associação do Banco Fenícia, da Laffer Canto e do Banque House. O objetivo é estruturar operações especiais para os clientes. 

Carbex investe US$ 1 mi em novos projetos 
A Carbex vai investir US$ 1 milhão em 1995 nas áreas industrial e comercial no desenvolvimento de novos produtos e programas de treinamento para os funcionários que trabalham nas duas fábricas. Roberto Sacchi, diretor da empresa, diz que vai destinar uma verba de US$ 1,2 milhão para a execução da estratégia de marketing da empresa no mesmo período.
</TEXT>
</DOC>
<DOC>
<DOCNO>FSP950101-020</DOCNO>
<DOCID>FSP950101-020</DOCID>
<DATE>950101</DATE>
<CATEGORY>DINHEIRO</CATEGORY>
<TEXT>
JOEL MENDES RENNÓ 
Para vencer a batalha do petróleo, vencemos a batalha da ciência e da tecnologia 
Ultimamente os temas petróleo e Petrobrás têm merecido maior atenção dos meios de comunicação e muitos são induzidos a crer que greves e demais fatos noticiados sobre a estatal constituem pontos de interesse prioritário. No entanto, vale considerar outros aspectos, estes sim muito importantes, os quais, embora não apresentem o mesmo potencial jornalístico de assuntos polêmicos, constituem na verdade as bases para a elaboração de um projeto nacional.
A Petrobrás pode e deve ser vista sob vários ângulos, cabendo à sociedade cobrar-lhe transparência, resultados e, sem dúvida, eficiência e modernidade no desempenho.
No entanto, o pensamento criativo deve buscar os fundamentos de cada questão e estabelecer parâmetros de debate para que possa questionar e eventualmente alterar paradigmas existentes.
Considere-se, por exemplo, alguns resultados da Petrobrás e a projeção de suas perspectivas, o que permitirá confrontar um quadro estrutural de otimismo realista, de expressiva participação no desenvolvimento nacional, com o quadro negro pintado sobre movimentos e problemas episódicos, triste exceção numa conduta altamente positiva.
Os campos da desinformação sempre enfrentam ervas daninhas e resultam em colheitas contaminadas. Assim, procura-se anular esforços e conquistas, na tentativa de impedir que a nação sequer ouse buscar seu destino de grande potência. Queremos distância desses terrenos insalubres e inférteis.
O país vem se destacando no setor do petróleo nesses 41 anos de existência da Petrobrás. E o que abona e assegura essa afirmação?
Em primeiro lugar, a sociedade teve sempre à sua disposição produtos petrolíferos na porta das refinarias nacionais, na quantidade demandada e em todos os pontos do território brasileiro, mesmo nos mais escondidos rincões, graças a um eficiente sistema de transporte e de bases de suprimento estrategicamente construídas. Trata-se de inegável conquista, considerando a dimensão continental do país. Esses produtos nunca faltaram no mercado interno, embora o povo brasileiro já tenha sofrido o desabastecimento de muitos outros produtos que consome.
Em segundo lugar, a Petrobrás, graças à escala alcançada no negócio petrolífero e sua eficiência em todas as fases do setor, vendeu e vende seus produtos na porta de suas refinarias por preço abaixo dos internacionais. Hoje, cerca de 20% inferiores. Ao comercializar tais produtos com essa redução de preço, transfere para a sociedade um ganho de cerca de US$ 2 bilhões ao ano. Melhor ainda, graças à performance tecnológica e operacional na produção de petróleo, a estatal continuará sustentando, sem limitação de prazo, o padrão de preços inferiores, e assim repassando ao país ganhos crescentes. Este é um fato da maior importância, que vale nos dias de hoje e para o futuro.
Em terceiro lugar, o povo brasileiro pode estar seguro e tranquilo quanto ao seu futuro energético relacionado com o petróleo. Enquanto a grande maioria das empresas privadas estrangeiras dispõe de 46 bilhões de barris de petróleo em reservas, tendo um horizonte de produção de 11 anos –por isso se vendo obrigadas a novas conquistas na faixa de 4 bilhões de barris por ano– o Brasil caminha para delimitar seus 10 bilhões de barris de petróleo e gás equivalentes, já descobertos, que nos garantem 32 anos de produção aos níveis atuais.
A melhor notícia, porém, consiste em saber que a Petrobrás já identificou o filão do ouro negro nacional. Anualmente, graças a sua tecnologia, incorpora grandes reservas. E caminha com segurança para buscar outros 20 bilhões de barris que se espera estejam reservados pela geologia nas profundezas do solo brasileiro, principalmente na plataforma submarina.
Tal é o significado desses números que o Brasil foi, nos últimos dez anos, o segundo país a realizar maiores descobertas de petróleo, somente superado pela Venezuela. Essas descobertas totalizam 7,6 bilhões de barris, no decênio.
Há mais ainda: a Petrobrás vem aumentando a produção de petróleo segundo uma taxa três vezes superior à da demanda, após a crise de 1979. E poderá continuar a fazê-lo nos anos vindouros, pois desenvolveu tecnologia de Primeiro Mundo na área e todos os campos disponíveis contam com projetos de magnitude em andamento.
Em quarto lugar, para vencer a batalha do petróleo, vencemos a batalha da ciência e da tecnologia nesse segmento. A natureza nos destinou petróleo nas profundezas do mar, originado há milhões de anos nos lagos existentes entre os continentes africano e brasileiro. E, desde o primeiro instante que o preço do petróleo permitiu, a Petrobrás o buscou na fronteira marítima, tornando-se líder nessa área a partir da década de 70.
É nesse ponto, então, que vemos altear-se o papel da Petrobrás na grande batalha deste fim de século e nas próximas décadas. Ajudar o Brasil a ser vitorioso nos complexos desafios e nas sedutoras perspectivas da ciência e da tecnologia.
Agora, todos nós percebemos com mais clareza que o sucesso do Brasil na inserção internacional depende, principalmente de dois fatores: o estímulo firme às empresas nacionais e o desenvolvimento científico-tecnológico.
Nesse contexto, a Petrobrás pode, sem dúvida, contribuir de forma positiva. Isso graças à sua escala, ao seu centro de pesquisas –que já investe de US$ 160 a US$ 180 milhões por ano em tecnologia– e fundamentalmente graças ao fato de ser um laboratório real onde interagem e convivem praticamente todas as tecnologias existentes no mundo.
Entendendo, assim, a importância para o país do desenvolvimento científico e tecnológico, a Petrobrás esboçou, no seu plano estratégico, o projeto Centros de Excelência, os quais vêm permitindo integrar, com as universidades e indústrias brasileiras, o seu esforço próprio de desenvolvimento tecnológico.
Nesse sentido, milhares de profissionais, dentro e fora da empresa, buscam a excelência na produção de petróleo em águas profundas, no refino, na recuperação de petróleo em suas rochas, na geoquímica, nos transportes, na gestão pela qualidade total, no planejamento estratégico e energético e em muitos outros campos.
O que se pretende é identificar todos os segmentos em que a Petrobrás possa estimular um desenvolvimento tecnológico maior ainda do que o alcançado. E, de forma conjunta –universidades, indústria privada e estatal– criar uma sólida capacitação científica e tecnológica, conferindo ao Brasil maior competitividade na exportação de produtos, bens e serviços e valorização do conteúdo desses mesmos itens no contexto interno.
Por esse caminho, e se o movimento qualitativo for seguido por todas as grandes empresas brasileiras, será superado um dos maiores desafios para todos nós, que temos a mística do progresso: vencer em ciência e tecnologia.
E para melhor aproveitar todo o potencial colocado diante de nós, a Petrobrás também decidiu aprimorar e incrementar seu programa de parcerias societárias nas áreas abrangidas pelo monopólio estatal do petróleo. Idem em relação às suas parcerias operacionais, tanto na área do monopólio ("leasing" de plataformas, ampliação de terceirização em trabalhos complementares como perfuração, sísmica e apoios industriais) quanto nas áreas não cobertas pelo monopólio, mas umbelicalmente a ele ligadas, como o suprimento de utilidades, as plantas industriais produtoras de vapor, de energia elétrica, de gases industriais, armazenamentos de derivados etc.
Algo como US$ 10 bilhões em projetos estão em análise para efetivação de parcerias nas áreas de refinarias, de transporte, de construção de navios, de oleodutos e gasodutos, de armazenamento. Dezenas de parceiros nacionais e internacionais discutem hoje com a Petrobrás a sua participação nessa façanha, que será a construção de um Brasil potência dentro do novo projeto nacional.
Desse modo, a Petrobrás busca espaços em que possa contribuir para o desenvolvimento nacional, deixando de lado o pessimismo e a valorização artificial de problemas conjunturais. Estes, por certo, devem ser sempre objeto de atenção, investigação, análise e soluções, mas nunca motivo de estéreis tentativas de destruição de uma conquista nacional irreversível.

JOEL MENDES RENNÓ, 56, engenheiro, é presidente da Petrobrás.
</TEXT>
</DOC>
<DOC>
<DOCNO>FSP950101-021</DOCNO>
<DOCID>FSP950101-021</DOCID>
<DATE>950101</DATE>
<CATEGORY>DINHEIRO</CATEGORY>
<TEXT>
ANTONIO KANDIR 
Cheguei ao México em plena crise cambial. Dizer que havia grande irritação contra o governo é "dourar a pílula". Assisti a uma mistura de raiva e decepção que demandaria mais espaço e talento literário para descrever com um mínimo de exatidão. Essa triste experiência reforçou-me algumas convicções antigas e me fez refletir sobre que lições se poderia extrair do episódio. A primeira das lições diz respeito ao erro de origem da experiência mexicana; as demais são relativas ao manejo da crise propriamente dito.
Lição 1: A âncora cambial não compensa. Ancorar o processo de estabilização no câmbio é uma estratégia que pode render frutos a curto e médio prazos, mas que encerra riscos incontroláveis a prazo mais longo. Pior, como procurei ressaltar em artigo nesta coluna ("Estabilização sem Atalhos Enganosos", 11/12), a ancoragem cambial, dadas as distorções que se vão acumulando, sobretudo em economias pouco abertas, impõe, mais à frente, um ajuste fiscal com custos sociais e econômicos muito elevados.
Lição 2: Ajuste cambial não se posterga. O governo do ex-presidente Salinas de Gortari já havia identificado a gravidade do problema cambial do país há algum tempo. Preferiu não fazer ajuste algum até o final de seu mandato. A fragilidade cambial do México agravou-se e o ajuste foi feito em condições quase dramáticas, sem que haja certeza se será capaz de salvar o país de nova e mais profunda crise de liquidez. Dissemina-se, no México e nos EUA, a opinião de que o governo de Salinas foi um governo de mentira.
Lição 3: Minimizar a profundidade da crise é o caminho mais seguro para agravá-la. Não bastasse o adiamento, o ajuste foi feito de modo desastrado. Na tentativa inútil de minimizá-lo, as autoridades econômicas do México resolveram anunciar uma "elevação de 15% da banda superior da variação cambial". A iniciativa detonou uma onda especulativa contra o peso que obrigou o governo a abandonar o sistema de bandas, menos de 24 horas depois. Movida por expectativas exacerbadas quanto à fragilidade cambial do país, a desvalorização da moeda mexicana frente ao dólar alcançou à casa dos 50%. Ficaram patentes o erro de diagnóstico e a debilidade do governo mexicano para fazer face à crise.
Lição 4: Dar explicações estapafúrdias tende a agravar a crise. Dos muitos equívocos cometidos no manejo da crise, não foi dos menores o de atribuir grande importância ao conflito de Chiapas no processo de perda de reservas internacionais. O governo pretendia minimizar a crítica ao modelo econômico atribuindo a crise a um fator externo. Com isso, colheu resultado justamente inverso ao pretendido, na medida que os agentes econômicos fizeram a seguinte leitura da explicação oficial: se a crise cambial decorre de um conflito armado que o governo tem tido grande dificuldade de encerrar, então a crise cambial vai se manter ou se aprofundar.
Lição 5: Induzir ao erro é letal à credibilidade. O sentimento de traição espalhou-se entre mexicanos e investidores norte-americanos. Pudera. Dias antes da crise, na presença de representantes de grupos financeiros dos Estados Unidos, o secretário da Fazenda e Crédito Público afirmou, de modo taxativo, que não haveria mudanças no câmbio. Induzidos a erro pela declaração da autoridade mexicana, os clientes de fundos de investimento com aplicações naquele país contabilizaram perdas superiores a US$ 15 bilhões. Jamais se havia visto em Nova York clima de hostilidade semelhante ao que se respirou nas reuniões entre o ministro mexicano Jaime Serra Puche e representantes do mundo financeiro, nos dias posteriores à crise cambial. Dificilmente o México poderá recuperar a credibilidade junto aos investidores internacionais depois desse episódio, o que certamente prejudicará, e muito, o fluxo regular de recursos externos àquele país.

ANTONIO KANDIR, 41, engenheiro, doutor em Economia, é deputado federal eleito pelo PSDB de São Paulo. Foi secretário de Política Econômica do Ministério da Economia (governo Collor). É autor, entre outros livros, de "Brasil Real: a Construção da Cidadania, da Moeda e do Desenvolvimento" (Klick Editora, 1994) e "Brasil Século 21: Tempo de Decidir" (Editora Atlas, 1994).
</TEXT>
</DOC>
<DOC>
<DOCNO>FSP950101-022</DOCNO>
<DOCID>FSP950101-022</DOCID>
<DATE>950101</DATE>
<CATEGORY>DINHEIRO</CATEGORY>
<TEXT>
ÁLVARO ANTÔNIO ZINI JR. 
"Na sua viagem para Ítaca deseje que sua jornada seja longa, cheia de aventuras, plena de descobertas. Lestrigãos, Ciclopes, o irado Poseidon – não os tema: você nunca os encontrará em seu caminho, conquanto seus pensamentos voem alto, conquanto uma fina excitação incendeie seu espírito e seu corpo. Lestrigãos, Ciclopes, o feroz Poseidon – você não os encontrará, a menos que os traga em suas entranhas, a menos que sua alma os coloque ante de si." C. Cavafy (Ithaca, 1910)

A posse do governo de Fernando Henrique Cardoso é o fato maior deste início do ano e merece ser saudada por seu potencial de renovação. O Brasil prepara-se para um novo ciclo de desenvolvimento e há um bom otimismo no ar. Mas devemos atentar para alguns desafios.
O Brasil atravessou 15 anos angustiantes de 1980 a 1994. Inflação elevada, constante incerteza sobre a economia e a política e a exposição das mazelas que dominam o setor público. É lógico que após tanta frustração todos estejamos querendo ver o país na rota do desenvolvimento, com preços estáveis.
Vamos caracterizar as perdas trazidas por essa crise tomando o produto per capita (isto é, na média, quanto cada brasileiro produz). Se tomamos a renda per capita como sendo 100 em 1980, no final de 1994 esta renda foi de 97,7 (supondo um crescimento do PIB de 5% no último ano): estamos 2,3% mais pobres do que em 1980.
A régua adequada para medir esse número, no entanto, é compará-lo com o que poderia ter sido o produto. Se a renda per capita tivesse crescido 3% ao ano entre 1980-94 (a média que crescemos de 1940 a 1980), no final de 1994 a renda per capita teria sido de 151,3.
Ou seja, cada um de nós –você leitor, seus herdeiros, eu– está, na média, com uma renda 54,9% menor do que poderia estar, se a economia tivesse crescido em sua trajetória normal, ou se os erros do endividamento do final dos anos 70 tivessem sido corrigidos a tempo.
O otimismo que existe na população com relação ao novo governo tem fundamentos. A inflação está baixa, as finanças públicas apresentam um melhor controle, ainda possuímos uma base produtiva de porte (o nono ou décimo parque industrial do mundo) e, principalmente, temos uma cultura que é favorável ao crescimento econômico.
Talvez até mais importante do que isso seja o fato de que o novo governo procurou formar alianças políticas explícitas para poder reformar o setor público. Deixando o sectarismo de lado, será um fato auspicioso se essa aliança política, que vai da centro-direita à centro-esquerda, se puser de acordo em como governar o país.
Mas vamos adiante. A atual calmaria nos índices de inflação ainda está longe de indicar que a guerra da estabilização já esteja ganha. Basicamente a atual calmaria deve-se a alguns fatores.
As contas públicas estão minimamente equilibradas, a introdução da URV, e sua passagem para o real, permitiu um razoável alinhamento dos preços relativos, a manutenção da lucratividade das empresas e uma desindexação moderada. Os diversos compulsórios adotados, tanto sobre a captação bancária como sobre as operações de empréstimo, terminaram se constituindo em fator de contenção monetária.
Por fim, a apreciação da taxa de câmbio e a ameaça das importações também ajudaram. Mas aqui chegamos a um campo problemático, fruto de uma inconsistência nos fundamentos monetários e cambiais.
A apreciação da taxa de câmbio real não se deve a um grande ganho de competitividade do país ou a fatores estruturais da economia, mas ao grande diferencial entre a taxa de juros doméstica e as taxas de juros de curto prazo no exterior.
A não-sustentação no tempo desta combinação está sendo vista no caso mexicano. Parece que o México despertou o deus dos mares revoltos. E, como diz Cavafy, o problema vem de dentro...
Há menos de duas semanas o presidente Ernesto Zedillo parecia no ápice. Tomara posse no início de dezembro e em duas semanas fez aprovar pelo Congresso uma reforma radical do setor judiciário e uma proposta orçamentária que projetava um crescimento de 4% para 1995, inflação também de 4%, além de déficit público zero.
Em 19 de dezembro, os rebeldes zapatistas anunciaram que iriam iniciar nova ofensiva (fato que o governo conhecia). Zedillo, sabendo que as reservas internacionais do país estavam baixas, resolveu usar o fato como bode expiatório e anunciou que deixaria o peso flutuar em uma banda maior do que a anterior, consoante com a liberdade do mercado.
Aí o pesadelo apareceu. Não só o peso desabou, como uma profunda crise de credibilidade se formou. Até 29 de dezembro o peso perdera 40% do seu valor, a taxa de juros subira para mais de 40% ao ano. Os desdobramentos disso, que implicam perdas de capitais significativas para empresas mexicanas, bancos e para investidores estrangeiros, ainda se farão sentir nos próximos meses.
Aliás, vamos notar algo paradoxal. Se algo não segurou o câmbio no México foi sua âncora fiscal. O país era a menina dos olhos dos economistas do FMI, sendo apontado como "o" modelo de ajuste fiscal dos últimos anos. Não só eliminou o déficit, como gerou superávits operacionais significativos.
Com um extenso programa de privatização, acumulou um significativo fundo de amortização, que seria usado para reduzir sua dívida interna de curto prazo (não foi utilizado até então porque pagar a dívida teria um impacto monetário expansionista forte).
Errou onde? Por não ter equacionado seu problema de endividamento de curto prazo, o México viu-se obrigado a pagar taxas de juros elevadas o tempo todo. Taxas de juros altas e cambio apreciado desestimularam a produção interna.
Com a produção estagnada, o governo permitiu formar um déficit em contas correntes de US$ 20 bilhões (8% do PIB), sinal de que o demanda estava superando a oferta. Agora, após a desvalorização, surgirão novas dificuldades fiscais pois o serviço das dívidas está mais caro.
Voltemos para o Brasil. Como o leitor deve ter percebido, apenas o ajuste fiscal e privatização, embora necessários (no nosso caso, uma prioridade), não são suficientes.
O que o exemplo mexicano vem ensinar é que há algumas noções básicas de equilíbrio fiscal, regras monetárias consistentes, taxa de câmbio alinhada com a competitividade do país, e situação de endividamento equacionada, sem as quais não se estabiliza um país.

ÁLVARO A. ZINI JR., 41, é professor de economia internacional da Faculdade de Economia e Administração (FEA) da USP e autor do livro "Taxa de Câmbio e Política Cambial no Brasil" (Editora da USP).
</TEXT>
</DOC>
<DOC>
<DOCNO>FSP950101-023</DOCNO>
<DOCID>FSP950101-023</DOCID>
<DATE>950101</DATE>
<CATEGORY>DINHEIRO</CATEGORY>
<TEXT>
No plano pessoal, Itamar Franco talvez tenha sido o mais desequilibrado presidente que este país já teve. Impulsivo, detalhista, desinformado, com parca capacidade de discernimento, tanto em relação a fatos quanto a pessoas, e uma necessidade quase doentia de auto-afirmação, tudo indicava que seu interinado produziria uma hecatombe nacional.
Suas virtudes –de homem simples, bem-intencionado, menos afeito a práticas fisiológicas e ao uso da máquina do Estado que a maioria dos políticos– pareciam insuficientes para contrabalançar seus vícios de temperamento.
Num autêntico desafio à teoria das probabilidades, errou praticamente todas as vezes que precisou tomar uma decisão. Com exceção de uma: a indicação de Fernando Henrique Cardoso para o Ministério da Fazenda.
Até lá, seu governo foi uma sucessão de desastres gratuitos. Humilhou seu primeiro ministro da Fazenda, Gustavo Krause, e seu segundo ministro, Paulo Haddad, até que pedissem demissão. Em ambos os casos, mostrou-se sinceramente surpreso com os pedidos, numa demonstração anormal de falta de discernimento nas relações pessoais.
Bem-intencionado, mas sem rumo, Itamar mudava permanentemente de opinião, de acordo com o último interlocutor. Qualquer pessoa que conseguisse desenvolver uma técnica de influenciá-lo, sentia-se dona do país.
O início de seu governo foi uma loucura, com pessoas do nível do consultor-geral José de Castro, do ministro da Justiça Maurício Correa e do presidente do Banco do Brasil Alcir Calliari dando palpites diários sobre economia, agitando o país com declarações sem nexo, numa barafunda infernal.
Virando o jogo 
A indicação de FHC mudou o jogo. Não que tecnicamente sua gestão tivesse sido boa. A de Paulo Haddad foi infinitamente mais responsável, no plano estrutural. Não fossem as loucuras de Itamar, o plano econômico teria saído mais cedo –e com muito mais consistência.
O grande mérito de FHC foi ter segurado Itamar –que, no fundo, era a variável irremovível e fora de controle, em todas as avaliações sobre a estabilização da economia. A admiração infinita do presidente por seu ministro, a paciência infinita do ministro com o presidente mereciam uma descrição literária. Itamar chegava a ligar 15 vezes por dia para FHC, para consultá-lo a respeito de tudo e de nada.
Graças a essa habilidade, FHC logrou anular a influência do grupo palaciano e trazer Itamar sob controle até o fim. Está certo o articulista Marcelo Coelho em afirmar que foi o sucessor que fez o presidente.

Opinião pública 
Se FHC acalmou Itamar no plano pessoal, o grande fator de equilíbrio institucional foi o amadurecimento da opinião pública. Em economias modernas, cabe a ela, através da mídia, exercer esse papel de equilíbrio –segurando rompantes de governantes, combatendo excesso de poder de políticos ou grupos empresariais etc.
Cada vez que Itamar cometia excessos, o mundo caía-lhe na cabeça. E, felizmente, ele recuava.
O governo Sarney foi um assalto indiscriminado ao Orçamento e privilégios indecentes a amigos. O governo Collor, um contraste absurdo entre idéias modernizantes e a apropriação do Estado por uma quadrilha.
Itamar deixa o governo com a imagem correta de um patriota, atrapalhado mas bem-intencionado. No plano da moral pública, foi semivirgem, restringindo sua caixinha política à Telemig, Telerj e ao sistema Telebrás.
Fora do governo, o país respira aliviado. Suas ranhetices não terão mais implicações sobre a estabilidade do país. Ele volta a ser a pessoa neurastênica, mas simpática, que atravessou o rio sem saber nadar e não se deslumbrou com as pompas do poder. Aliás, o poder é um conceito que transcende sua capacidade de compreensão.
Fora do governo, não haverá uma só pessoa que não lhe deseje felicidades.
</TEXT>
</DOC>
<DOC>
<DOCNO>FSP950101-024</DOCNO>
<DOCID>FSP950101-024</DOCID>
<DATE>950101</DATE>
<CATEGORY>DINHEIRO</CATEGORY>
<TEXT>
Preço reduzido determina estoques do varejo, lançamentos e importações 
Da Reportagem Local 
A indústria e o comércio já têm planos traçados para conquistar os consumidores de renda mais baixa e ampliar os negócios este ano.
A partir de amanhã a Marcyn, por exemplo, que confecciona lingerie feminina, despeja no mercado 120 mil peças de uma nova coleção 40% mais barata.
Com uma produção de 2 milhões de unidades nessa linha, a empresa quer aumentar 20% suas vendas este ano.
Para viabilizar o projeto, que custou US$ 350 mil, conta Mauro Zaborowsky, diretor comercial, a empresa passou a usar um tipo de tecido de lycra que estica para apenas um lado e permite maior rendimento na confecção da lingerie.
Além disso, adotou o saco plástico simples como embalagem para reduzir, ainda mais, os custos. O público da linha Dedicate é o consumidor das camadas C e D.
"Apostamos no crescimento do mercado de produtos baratos e com qualidade", diz Zaborowsky.
É que com a queda da inflação, diz ele, dobrou o poder de compra dos assalariados desde julho.
Em pesquisa realizada em abril, antes do Plano Real, a empresa detectou que o consumidor mais pobre deseja comprar um produto em conta, mas com qualidade.
É na trilha dessa pesquisa que a Azaléia quer conquistar quem ganha até três salários mínimos.
Para atingir esse mercado ávido por produtos baratos está trazendo da China 11 modelos de calçado popular, em borracha vulcanizada com tecido, que vão estar nas lojas em março. A Azaléia já montou um escritório em Hong Kong.
"O produto importado da China é 30% mais barato do que o fabricado aqui, apesar do imposto de importação de 20%", diz Geovani Sita, gerente.
A empresa chegou a fabricar duas linhas de calçados popular. Mudou de idéia porque constatou que era mais em conta importar.
Em 94, a empresa vendeu 700 mil pares "made in China". Pretende mais do que dobrar esse volume em 95.

Tíquete 
Atentas para esse mercado emergente, grandes lojas contabilizaram, no ano passado, os resultados de estratégias adotadas desde o início do Plano Real.
Na loja de departamento Magazan, de Belém (PA), os negócios cresceram 200% de outubro para novembro, depois que a empresa passou a vender mercadorias mais baratas importadas da China.
"Pensando no consumo popular, alteramos o 'mix' de produtos. Com isso, conseguimos reduzir o tíquete médio de venda, entre R$ 120 e R$ 150, para R$ 30 e R$ 40 e ganhamos no volume", diz João Rodrigues, diretor.
A loja, que tem metade da clientela com renda de um a três salários mínimos e os outros 50% entre três e seis mínimos, planeja aumentar os estoques de importados. Dos 15.000 itens vendidos, hoje 20% vêm do exterior por um preço mais barato do que o nacional. A meta, este ano, é aumentar para 45% a fatia dos importados.
"O que o consumidor quer é preço barato. O consumo popular é que vai dar o tom das vendas a partir de agora", diz Rodrigues.
Reestruturação semelhante no "mix" de produtos fez outra rede de lojas, a Le Postiche, alcançar bons resultados.
Alessandra Restaino, diretora comercial da rede franqueada, conta que as vendas dobraram de julho a novembro em relação a igual período de 93, depois que a empresa reforçou os estoques de produtos de maior giro, levando em conta a faixa de preço.
Resultado: o tíquete médio de vendas caiu de R$ 40 para R$ 20.
Segundo ela, a sua loja reduziu preço porque está trabalhando com volume maior de importados.
(Márcia de Chiara)
</TEXT>
</DOC>
<DOC>
<DOCNO>FSP950101-025</DOCNO>
<DOCID>FSP950101-025</DOCID>
<DATE>950101</DATE>
<CATEGORY>DINHEIRO</CATEGORY>
<TEXT>
Da Reportagem Local 
A perda de poder aquisitivo da classe média após o Plano Real foi parcialmente recuperada por quem teve data-base a partir de agosto. Se o reajuste salarial ocorreu em dezembro, o trabalhador ficou até mesmo com uma "sobra" para gastar ou poupar.
Quem teve data-base antes, entretanto, já deve estar com o orçamento mais apertado. É que, no correr deste ano, irão vencendo os prazos de congelamento dos contratos, de 12 meses a partir da conversão para URV ou real.
Ao mesmo tempo, após o primeiro dissídio na fase do real, os salários só serão corrigidos dentro de 12 meses, por livre negociação.
Já aluguel, escola e plano de saúde, três despesas básicas da classe média, devem ter reajustes elevados e pesar mais no bolso dos consumidores este ano.
Várias escolas particulares de São Paulo, por exemplo, já aumentaram as mensalidades deste mês em 20%, em média.
"A anualidade teve um reajuste médio de 35%, distribuído nos meses de março e dezembro", diz Mauro Bueno, presidente da Aipa (Associação Intermunicipal de Pais e Alunos), baseado em contratos das escolas.
Além disso, em março, data-base dos professores, as mensalidades devem subir mais de 100%.

Aluguel 
Quem paga aluguel também deve preparar o bolso para o aumento desta despesa este ano.
Vencendo os 12 meses de congelamento, após a conversão para URV/real ou o último acordo, os valores serão reajustados pelo índice previsto no contrato, acumulado a partir de julho passado.
Além disso, a MP do real permite que o proprietário entre com um pedido judicial de revisão do valor do aluguel a partir deste mês. "Mas cerca de 30% dos contratos foram convertidos pela média, sem acordo", diz José Roberto Graiche, presidente da Aabic (Associação das Administradoras de Bens Imóveis e Condomínio).
Alugar um imóvel novo também significa comprometer um percentual alto de seu orçamento. Segundo Graiche, pesquisa mensal da Aabic indica que os aluguéis iniciais tiveram uma alta de 50% entre julho e novembro passados.

Saúde
Os valores de planos e seguros-saúde estão congelados por 12 meses desde a conversão para URV ou real. Vencido esse prazo, durante este ano, as mensalidades serão reajustadas pelo índice previsto no contrato, acumulado a partir de julho passado.
"Vale lembrar que, além da correção por um índice de preços, como IGP ou IGP-M, os contratos prevêem aumento de acordo com a alta dos custos setoriais (taxas hospitalares, honorários, remédios etc.)", diz Selma do Amaral, chefe do setor de saúde do Procon-SP.
(Vera Bueno de Azevedo)
</TEXT>
</DOC>
<DOC>
<DOCNO>FSP950101-026</DOCNO>
<DOCID>FSP950101-026</DOCID>
<DATE>950101</DATE>
<CATEGORY>DINHEIRO</CATEGORY>
<TEXT>
Preocupação é pagar impostos e adiar outros gastos 
Da Reportagem Local 
A classe média deve frear o consumo durante o mês de janeiro para colocar em ordem suas finanças, depois dos excessos de fim de ano.
Além disso, o pagamento de impostos e com escola são vistos pelos consumidores como prioritários e não vão deixar muito espaço para gastos que possam ser adiados.
"Agora é a hora de cortar o que for possível, mesmo porque os salários de janeiro e fevereiro estão completamente comprometidos no pagamento das prestações e de outros gastos que aparecem no início do ano", diz Alzira Conceição Simões, diretora comercial de uma loja de vestidos de noiva.
O representante comercial Nelson Lima diz que sua preocupação são os gastos no pagamento de impostos.
"Vou pagar R$ 300 no IPVA de uma Paraty, que devo parcelar em duas ou três vezes e tem o IPTU, que ainda não sei de quanto vai ser", diz.
Outro fantasma que ronda a classe média são as notícias que começam a surgir de aumentos nas mensalidades e nos preços do material escolar.
"É muito preocupante algumas escolas estarem aumentando seus preços, porque as mensalidades já estavam bastante caras no ano passado", diz a professora Regina Souza.
(Luiz Antonio Cintra)
</TEXT>
</DOC>
<DOC>
<DOCNO>FSP950101-027</DOCNO>
<DOCID>FSP950101-027</DOCID>
<DATE>950101</DATE>
<CATEGORY>DINHEIRO</CATEGORY>
<TEXT>
DEMIAN FIOCCA 
A ruína da estabilização mexicana e a situação extremamente delicada da Argentina mostram que a valorização do câmbio e a abertura agressiva das importações são políticas de integração internacional inconsistentes. Uma estabilização duradoura depende de um rigoroso ajuste fiscal e de uma rota de equilíbrio no setor externo.
A intervenção no Banespa e no Banerj mostrou quão pequena é a consciência de que os déficits públicos não podem ser permanentes e quão longe chegou o endividamento irresponsável dos governos estaduais. Brizola pelo menos investiu em educação. Fleury, nem isso.
A indicação de José Serra para o ministério do Planejamento favorece o reequilíbrio das contas públicas. Mas a mistificação dos custos trabalhistas –que são de 40% e não de 100% do que recebe o trabalhador– e a possível eliminação dessses custos sem a provisão de fontes alternativas de receita pode originar déficits orçamentários.
A redução dos encargos sobre a folha de pagamentos favorece o crescimento do emprego e tem um efeito antiinflacionário. É correta portanto. Mas a redução no volume de arrecadação é deletéria. O obstáculo ao ajuste fiscal brasileiro não é o excesso de proteção social, mas a concentração de renda. Cabe taxar quem pode pagar.

DEMIAN FIOCCA, 26, é economista e pós-graduando do curso de Economia da Universidade de São Paulo (USP).
</TEXT>
</DOC>
<DOC>
<DOCNO>FSP950101-028</DOCNO>
<DOCID>FSP950101-028</DOCID>
<DATE>950101</DATE>
<CATEGORY>DINHEIRO</CATEGORY>
<TEXT>
MARCOS CINTRA 
Diferentemente do que se imagina, o Brasil corre risco de estar na mesma trajetória do México. É evidente que o Brasil não acumulou déficit comercial de US$ 18 bilhões; e nem perdeu três quartas partes de suas reservas, que naquele país caíram em alguns dias de US$ 24 bilhões para US$ 6,5 bilhões. O risco não é imediato.
México, Argentina e Brasil avançaram na estabilização ancorados no câmbio. Mas esqueceram dos fundamentos da estabilidade econômica, ao não complementarem a política cambial com mudanças no regime monetário e fiscal.
A fixação do câmbio é uma medida de curto prazo. Os preços internos caem. Os importados abundam. O PIB e o poder aquisitivo aumentam. Produtos sofisticados surgem nos mercados de consumo e de bens de capital.
Some-se a isto o fato do governo brasileiro ter controlado o componente inercial da inflação. Resultado: aumento de preços de 2% ao mês. Com aumento de demanda e emprego.
Alguém paga por isto? Os lucros e salários dos setores de bens comercializáveis –exportadores e setores que competem com os importados.
Os problemas, no entanto, começam a aparecer. Pelo segundo mês consecutivo haverá déficit comercial de cerca de US$ 250 milhões. O fluxo financeiro também será negativo e já em dezembro haverá perda de reservas.
Trata-se de desajuste estrutural, que resultará, se correções não forem feitas, em experiências dolorosas, como no México e na Argentina.
As correções implicam reformas, dentre elas a tributária, a administrativa, a previdenciária. Com isto se estaria transferindo o custo do ajuste para os que se locupletam com a corrupção e com os privilégios do setor público.
Âncoras cambiais são instrumentos limitados em economias com baixo coeficiente de abertura. A inflação nos setores não-comercializáveis e nos serviços já corre solta, entre 5% e 10% ao mês em dezembro. Se nada for feito, a frágil âncora cambial será logo arrastada pelas correntezas estruturais da economia brasileira.

MARCOS CINTRA CAVALCANTI DE ALBUQUERQUE, 48, doutor em Economia pela Universidade de Harvard (EUA), é vereador da cidade de São Paulo pelo PL e professor titular da Fundação Getúlio Vargas (SP). Foi secretário de Planejamento e de Privatização e Parceria do Município de São Paulo (administração Paulo Maluf).
</TEXT>
</DOC>
<DOC>
<DOCNO>FSP950101-029</DOCNO>
<DOCID>FSP950101-029</DOCID>
<DATE>950101</DATE>
<CATEGORY>DINHEIRO</CATEGORY>
<TEXT>
Demonstrar a força do desastre do México pode ser nossa única esperança de abortar a "morte anunciada" 
MARIA DA CONCEIÇÃO TAVARES 
JOSÉ CARLOS MIRANDA 
Especial para a Folha 
A Cúpula das Américas, celebrada pouco antes do desastre do México, promete para o continente sul-americano um cenário de (des)integração comercial e financeira com os Estados Unidos para o ano 2005. Sem ficção científica, os primeiros resultados podem ser avaliados pela experiência do Nafta.
A esta altura do milênio, quando os nossos projetos locais do neoliberalismo acabam de obter tantas vitórias, nossa única esperança de abortar a "morte anunciada" reside na força do efeito demonstração do desastre alheio.
Dizem que quando os Estados Unidos espirram a América Latina pega pneumonia. A pneumonia mexicana, apesar de ter um longo período de maturação, tornou-se aguda a partir de dois pequenos espirros norte-americanos: a vitória republicana no Congresso, que põe em xeque a política Clinton de "boa vizinhança" e várias subidas graduais da taxa de juros do FED.
O efeito da subida da taxa de juros americana e as crises cambiais dramáticas são nossas conhecidas desde a crise da dívida externa. Mas a situação atual, dirão os analistas, é completamente diferente! Claro, agora é mais difícil de controlar por causa das políticas de desregulação financeira global.
Estas políticas levaram a uma maciça entrada de capitais de curto prazo desde 1990, que desabaram sobre "os mercados emergentes", qualquer que fosse a situação das balanças de transações correntes dos países receptores ou, até mesmo, a natureza e grau dos seus desequilíbrios macroeconômicos (veja-se a entrada de capitais no Brasil, que obviamente não era candidato "estável" nem "absorvedor líquido" de recursos externos).
Desta vez o México não precisou de um choque de juros à la Volcker, nem de um período de três anos de crise cambial para reverter o fluxo de capitais e provocar uma crise financeira de proporções. Por que sempre começa com o México? Ora, dirão os supersticiosos: porque está "tan lejos de Dios e tan cerca de Estados Unidos".
Para esclarecer os brasileiros que ainda se acham muito diferentes do México e sobretudo chamar a atenção das autoridades competentes, convém fazer uma pequena análise dos "fundamentos" da crise mexicana.
Para isso pedi a ajuda de meu colega da Unicamp, José Carlos Miranda, recém-chegado do México, especialista em estudos de política macroeconômica comparada.
A crise mexicana 
Com a estabilização do Plano Brady, o México pôde implementar uma estratégia de estabilização centrada numa âncora cambial, em ajustes fiscais permanentes e abertura comercial e financeira.
A idéia prevalecente entre os economistas oficiais era que se fosse mantida estável a paridade dólar-peso, este se tornaria uma moeda forte, credenciando novamente o México como importador de capitais. Seria o maior mercado emergente no novo quadro de globalização econômico-financeira.
Na realidade, o que se observou como contrapartida do "ajuste pela conta de capitais" foi uma elevação significativa dos passivos de curto prazo do México com o exterior, tanto privados quanto públicos.
Em 1982, estes eram de US$ 89 bilhões e hoje equivalem a US$ 211 bilhões, dos quais US$ 87 bilhões são dívida pública, US$ 26 bilhões são dívidas do setor privado bancário, US$ 26 bilhões do setor privado não-bancário, US$ 27 bilhões são bônus do Tesouro denominados em dólar e US$ 50 bilhões estão no mercado de ações.
A contrapartida deste endividamento crescente com o exterior foi a emissão de dívida pública para neutralizar o impacto da entrada de capitais sobre a liquidez interna e pública para neutralizar o impacto da entrada de capitais sobre a liquidez interna e financiar o serviço da dívida (velha e nova), submetendo a economia mexicana a ajustes fiscais permanentes.
Em 1994, o governo terá despendido US$ 30 bilhões só com o serviço de suas dívidas interna e externa. É a partir desta situação fiscal e da integração comercial e financeira do México com os Estados Unidos (que levou a um déficit comercial da indústria manufatureira de US$ 20 bilhões) que se deve avaliar os efeitos da brutal desvalorização do peso.
Segunda-feira, 19, o governo anunciou uma desvalorização de quase 15%, que levaria o dólar de 3,46 para 4 pesos –como se fosse uma flutuação na banda–, o que permitiu inicialmente aos "insiders" nacionais e depois aos investidores estrangeiros fazer uma extraordinária fuga de capitais.
No dia seguinte, foi anunciada a flutuação livre, com o que o peso entrou em queda contínua. Em três dias, chegou-se a mais de seis pesos por dólar com a perda de US$ 10 bilhões de reservas. Estas, que se encontravam no início do ano em mais de US$ 25 bilhões e em 01/11/94 em US$ 17,25 bilhões, estão hoje em menos de US$ 6 bilhões.
A Bolsa, apesar da brutal valorização do dólar (cerca de 70%, em relação ao dia 16 de dezembro) caiu em pesos. É daí que surge a maior ameaça de crise cambial, dado que o México tinha US$ 50 bilhões na Bolsa, a maior parte dos quais de fundos de pensão de todas as partes do mundo, mas em particular dos EUA.
As negociações que o governo mexicano está empreendendo com o governo americano não dizem respeito apenas ao apoio de US$ 6 bilhões do Fundo de Contingência que o FED deveria colocar à disposição do México. Este fundo, que ainda não está disponível, seria manifestamente insuficiente para sustentar o câmbio, a ser mantido o regime de livre flutuação cambial.
O que os mexicanos querem é o apoio do governo americano para manter no país os principais fundos de pensão. E parecem dispostos a pagar qualquer preço.
Esta negociação afigura-se bem mais difícil do que a da crise da dívida externa de 1982, tanto pelo montante da dívida privada diretamente em dólares, quanto pelos montantes da dívida pública interna em bônus de curto prazo denominados em dólar.
Nem a situação da balança de pagamentos (com um déficit em transações correntes de cerca de US$ 30 bilhões), nem a situação fiscal permitem imaginar a curto prazo um ajuste de US$ 29 bilhões (montante da dívida pública dolarizada), razão pela qual teme-se nas praças financeiras mexicanas que esta dívida possa ser forçosamente convertida em pesos.
Para evitar uma corrida definitiva, as taxas de juros dos Cetes (títulos do Tesouro) elevaram-se de 16% para 32% em menos de três dias. O resultado desta tentativa desesperada só poderá ser avaliado nas próximas semanas, à medida que forem vencendo os títulos públicos que, convém lembrar, são de curto prazo.
A elevação da taxa básica de juros jogou as taxas de empréstimo ao setor privado para 50% ao ano, colocando em posição de virtual insolvência as carteiras de aplicações dos bancos que operam no México que, por sua vez, têm dívidas de US$ 26 bilhões com o exterior. Assim, os bancos estão duplamente ameaçados pelos lados passivo e ativo de seus balanços.
Dada a impossibilidade de estatizar a dívida privada através da estatização bancária, como ocorreu em 1982, é provável que as ações dos bancos caiam fortemente de cotação, abrindo a possibilidade de consolidação das posições credoras e devedoras do sistema bancário internacional que opera no território mexicano. Afinal parece mais fácil mudar a lei bancária mexicana do que a norte-americana. Outra sugestão recente é a "privatização" do petróleo.
Os ventos da crise mexicana chegaram rapidamente à Argentina. Os capitais de curto prazo já estão saindo da Argentina, temendo que o peso argentino também se desvalorize. E a insolvência do banco Extrader coloca a nu a situação dos demais bancos: passivos dolarizados e possibilidade de inadimplência de seus devedores pela subida das taxas de juros.
Tal situação serve de alerta às autoridades brasileiras, no sentido de atentar para a precariedade de um ajuste centrado em déficits em transações correntes, forçando uma sobrevalorização progressiva da paridade cambial e de emissão de títulos públicos denominados em dólar como padrão de financiamento público.
Esta análise não precisa ser usada agora, para não estragar as festas de fim-de-ano, mas quem sabe lá para fins de fevereiro, quando o carnaval passar e os nossos recentes títulos cambiais estiverem vencendo.

MARIA DA CONCEIÇÃO TAVARES, 63, é professora emérita da Universidade Federal do Rio de Janeiro (UFRJ) e professora associada da Universidade de Campinas (Unicamp).
JOSÉ CARLOS MIRANDA, 44, é diretor do Centro de Relações Econômicas Internacionais e professor do Instituto de Economia da Unicamp.
</TEXT>
</DOC>
<DOC>
<DOCNO>FSP950101-030</DOCNO>
<DOCID>FSP950101-030</DOCID>
<DATE>950101</DATE>
<CATEGORY>DINHEIRO</CATEGORY>
<TEXT>
Os fundões tradicionais (FAFs) fecharam dezembro com rentabilidade média estimada de 2,70%, já descontado o IR de 5% sobre o rendimento nominal bruto, mas sem considerar o IOF que incide sobre os resgates até 15 dias úteis.
Já a rentabilidade bruta estimada dos fundos de renda fixa de curto prazo (novos fundões, criados em junho de 1994) é de 3,24%. Ela cai para 3,12% líquidos para as aplicações de 31 dias que forem resgatadas no próximo dia 2 por causa da tributação que combina o IR e IOF.
A valorização diária das quotas dos 15 maiores fundões tradicionais estava em 0,1087%, na média, enquanto a dos fundos de curto prazo era de 0,1233%.
A rentabilidade líquida das aplicações de curto prazo feitas a partir do próximo dia 2 irá sofrer alterações substanciais com a vigência da nova tributação e também por conta da redução dos juros sinalizada pelo Banco Central na última terça-feira.
</TEXT>
</DOC>
<DOC>
<DOCNO>FSP950101-031</DOCNO>
<DOCID>FSP950101-031</DOCID>
<DATE>950101</DATE>
<CATEGORY>DINHEIRO</CATEGORY>
<TEXT>
LUIZ CARLOS MENDONÇA DE BARROS 
Nesta última coluna de 94 gostaria de chamar a atenção do leitor da Folha para uma lição que este fim de ano reservou a nós brasileiros. A crise mexicana destes últimos dias e a situação de nossa economia no apagar de 94 resgatam o nome do Brasil nos mercados financeiros internacionais. O sucesso do Plano Real, com uma inflação de apenas 0,8% em dezembro em um sistema de liberdade de preços, é hoje incontestável. Nos principais jornais do Primeiro Mundo têm sido constante as afirmações de analistas financeiros aconselhando os investidores a trocar seus investimentos mexicanos por alternativas nos mercados brasileiros. Apesar das incertezas que ainda temos pela frente são ressaltadas as condições políticas e econômicas favoráveis de nosso país.
Durante anos o México era considerado o paradigma do sucesso de recuperação econômica no mundo em desenvolvimento. E o Brasil o oposto, um exemplo de bagunça, de descalabro administrativo e confusão política. Escapava destes analistas do senso comum as diferenças fundamentais entre estes dois países, principalmente quanto à situação política.
O Brasil, recém saído de mais de 20 anos de ditadura militar, ensaiava os primeiros passos de uma riquíssima experiência de redemocratização. O México, por outro lado, exercitava de maneira vigorosa seu sistema político rígido e oligárquico de tantas décadas. A gestão compartilhada entre uma geração de brilhantes economistas e os tradicionais dinossauros do PRI permitiram a implantação de um processo de reformas modernizadoras na economia, mas não na sociedade mexicana. Estão colhendo hoje os frutos comuns a todos os regimes politicamente fechados: os erros cometidos por falta de crítica e autocrítica.
O caso brasileiro é diferente. A modernização de nossa economia está sendo feita através de um longo amadurecimento e discussão na sociedade. Este sistema é mais complicado mas certamente leva a mudanças mais sólidas e com elevado grau de consenso social. Cabe ao governo de Fernando Henrique terminar este processo. A crise mexicana deve servir como lição para os últimos traços da política econômica para os próximos quatro anos.

LUIZ CARLOS MENDONÇA DE BARROS, 51, engenheiro, é diretor do Banco Matrix S/A e professor do curso de doutorado do Instituto de Economia da Unicamp.
</TEXT>
</DOC>
<DOC>
<DOCNO>FSP950101-032</DOCNO>
<DOCID>FSP950101-032</DOCID>
<DATE>950101</DATE>
<CATEGORY>DINHEIRO</CATEGORY>
<TEXT>
O cenário de curto prazo indica a manutenção do clima de turbulência nas Bolsas de Valores, com as intervenções do Banco Central no Banespa e no Banerj. Elas se somam ao impacto negativo da crise mexicana sobre os investimentos estrangeiros nas bolsas latino-americanas.
Na última semana do ano, o Índice Bovespa registrou alta de apenas 0,06%, mas acumula desvalorização de 6,49% em dezembro e de 20,60% nos últimos três meses. O Índice Senn, da Bolsa de Valores do Rio, subiu 2,73% na semana mas fechou o mês com queda de 3,50%.
</TEXT>
</DOC>
<DOC>
<DOCNO>FSP950101-033</DOCNO>
<DOCID>FSP950101-033</DOCID>
<DATE>950101</DATE>
<CATEGORY>DINHEIRO</CATEGORY>
<TEXT>
O presente de Natal do povo brasileiro é a transformação para melhor do país; que 95 traga esta realização 
OSIRIS LOPES FILHO 
Especial para a Folha 
É preciso reconhecer o óbvio. Neste período que antecede o Natal até o Ano Novo são significativos os traços de mudança do país. A classe média foi voraz e faminta às compras. Objetos antes privativos de consumo dos ricos, nas lojas caras de importados ou nos abastecimentos conspícuos do contrabando de alto coturno, agora, graças às medidas liberalizantes adotadas pela burocracia tecnocrática, estão democraticamente à venda nos bazares, lojas e shoppings. A variedade é imensa. Ao brasileiro com grana, as oportunidades de consumo são incontáveis, sem necessidade de emprego de pistolão para embarcar nos vôos para o exterior, que estão abarrotados.
Um conhecido conta-me maravilhas do abastecimento de produtos estrangeiros que entraram legalmente no país. Ou pelo menos esse é o pressuposto, tão profusa e escancarada é a exposição desses bens.
"Comprei um aparador de fio de cabelos para o nariz e um barbeador para eliminar "naps", fiapos e bolotas das roupas, ternos e principalmente blusas de lã", diz-me feliz.
Fico extasiado com essas invenções da sofisticação da indústria de consumo estrangeira. Peço detalhes. "O aparador de cabelos anti-estéticos e rebeldes do nariz funciona uma maravilha. Corta os pêlos sem ferir, objetiva e funcionalmente", descreve-me o conhecido. "E o barbeador de roupa? Será este o nome. Não seria melhor aparador de fiapos?", pergunto. Esclarece-me que o nome em inglês é "shaver" (barbeador) e que sua função é exatamente barbear as roupas daqueles fios desalinhados e bolinhas a lhes dar aspecto de velhas, mal cuidadas e rotas. E diz-me em conclusão: "Tem a mesma função do barbeador, que dá ao rosto uma feição de limpeza, asseio e civilidade."
Calo-me diante da explicação entusiasmada de quem encontrou satisfação total no seu consumo, adquirindo tão úteis utensílios, que provocam o milagre de evitar melecas suspensas em fios desalinhados das narinas e transformam roupas usadas, desgastadas, em roupas "up to date".
E reflito. Não diria que tudo mudou, mas muita coisa mudou. Dirigindo o automóvel, chego a essa conclusão final. Nos sinais, semáforos ou faróis, o nome vai conforme o gosto ou região, já não há mais o recolhimento compulsório de contribuições em dinheiro, em jóia ou do próprio veículo, feito por enérgicos e decididos senhores, que, para mostrar a sua firmeza de vontade e a força de seu argumento, apontavam para a cabeça do condutor do veículo o cano de um revólver calibre 38.
Há a inovação de crianças, com caixas decoradas com papéis de colorido natalino, a solicitarem uma doação e os motoristas, para alívio da consciência, não dão apenas moedas, mas notas, que depois de muita humilhação de nossa moeda, agora, valem mais do que o dólar! É isso. Muita coisa mudou.
O futuro presidente anuncia o ministério e justifica suas escolhas, democraticamente dando explicações. São muitas as inovações. Gente comprovadamente experiente de volta. Jatene, na Saúde, Stephanes, na Previdência, Dorothéa, na Indústria e Comércio, Bresser na Administração. Para a Agricultura um banqueiro versátil, o senador José Eduardo, que já ocupou a pasta da Indústria e Comércio e que seguramente aplicará à nossa agricultura a receita da prosperidade e dos lucros bancários. Mas a surpresa é o PFL. Tão combatido na fase de campanha, apresenta a sua contribuição expressiva, que deixa desestabilizados os que o atacavam de fisiologismo: Stephanes , Krause e Raimundo de Brito. Na Justiça, Jobim é efetivamente um jurista e não se repete o equívoco cometido por Collor, quando da nomeação de um professor dito da Sorbonne, cujos cursos lá ministrados eram do tipo "Walita", para legitimação técnica de folhetos de turismo com pretensão de aprimoramento profissional. No Planejamento, o senador José Serra representa a versão modernizada do "é proibido gastar" do presidente Tancredo Neves. Ministro que chega se declarando "pão-duro" é do que se precisa na bacanal anárquica do gasto público federal para lhe imprimir ordem e eficácia.
Mas a grande novidade é a indicação de Pelé. Primeiro, a ascensão da raça negra ao poder, que tem sido tão ativa para o reconhecimento de sua expressividade. Segundo, nomeia-se um cidadão do mundo. E dá-se ao ministério do presidente Fernando Henrique um caráter universal, de integração mundial, muito ao modelo do deputado Roberto Campos. Nada de ministério tupiniquim e, muito menos, paroquialista. Pelé é divisor de águas no esporte nacional. Sua nomeação indica o desprestígio de Ricardo Teixeira e João Havelange, seus desafetos na imprensa e em litígio judicial.
É, as coisas estão mudando nestas paragens, que já foram consideradas "tristes trópicos" e estão cheias de otimismo para se adentrar o Ano Novo.
Ao presidente Itamar, em reconhecimento ao seu talento, acena-se com a embaixada em Lisboa. Suas aparições em público são apoteóticas. Sinceramente apoteóticas, se se consideram não apenas o reconhecimento e a aceitação populares e as expectativas reais de retorno à Presidência, mas o fato de que poderá descer a rampa do Palácio do Planalto sob aplausos populares, aposentando a saída clandestina e prudente pelos fundos, de algumas gestões anteriores.
Millor Fernandes identificou uma característica de nossa gente, quando definiu com propriedade: "Brasileiro, profissão esperança." Valeu a pena ser brasileiro neste final de milênio e ter mantido acesa a chama da esperança.
O Natal, época de presentes, de sentimentos nobres e de solidariedade, está prenunciando uma nova era de otimismo e realizações para o povo brasileiro, no novo ano que se anuncia –a superação da esperança por realizações concretas em favor dos humildes, pobres e desfavorecidos, que não mais serão marginalizados.
É o processo que se reinicia, de aguardar que o novo governo consiga transformar seu discurso e promessas em realidades melhores para o povo brasileiro.

OSIRIS DE AZEVEDO LOPES FILHO, 55, é professor de Direito Tributário e Financeiro na Universidade de Brasília, advogado e ex-secretário da Receita Federal.
</TEXT>
</DOC>
<DOC>
<DOCNO>FSP950101-034</DOCNO>
<DOCID>FSP950101-034</DOCID>
<DATE>950101</DATE>
<CATEGORY>DINHEIRO</CATEGORY>
<TEXT>
Inquilino e proprietário devem analisar cuidadosamente o contrato para saber se vale a pena enfrentar a ação judicial 
Da Reportagem Local 
A partir de amanhã, qualquer proprietário pode ingressar na Justiça com uma ação revisional do valor do aluguel. É o que determina a MP do Real.
Esse procedimento, entretanto, pode não ser aconselhável, dependendo da situação do contrato.
Se foi feito um acordo recentemente, principalmente após o plano de estabilização, as chances de o proprietário ganhar a ação são muito pequenas.
Isso porque a MP estabelece como precondição para a ação a prova de que há um desequilíbrio econômico-financeiro do contrato.
Para o advogado Márcio Bueno, diretor jurídico do Creci-SP (Conselho Regional dos Corretores de Imóveis), isso não é fácil.
"O acordo tem como principal finalidade restaurar o equilíbrio do contrato. Nele, as partes manifestam, espontaneamente, sua satisfação com o novo valor do aluguel", explica Bueno.
Assim, ele acredita que dificilmente os juízes acatarão a ação revisional pouco tempo após um acordo amigável entre as partes.
Bueno lembra que não há relação entre o valor do aluguel e o desequilíbrio econômico-financeiro do contrato. "Se um proprietário fixou valor de aluguel inferior ao praticado pelo mercado e aceitou a regra de reajuste semestral, o baixo valor não prova, necessariamente, desequilíbrio econômico-financeiro do contrato", diz.
O inquilino também pode ter várias razões para evitar a revisional. Se o valor do aluguel está muito abaixo do preço de mercado, por exemplo, é bastante provável que ele perca a ação.
O mesmo acontece se não houve acordo após o Plano Real e o valor a partir de julho é resultado da simples conversão pela média dos meses anteriores.
Quem está com o contrato vencido ou perto de vencer também leva vantagem se fizer um acordo amigável. Desde que o prazo do mesmo seja prorrogado por um ou dois anos ou se faça um novo contrato, por mais 30 meses.
Saiba que, mesmo fazendo um acordo e fixando um novo valor para o aluguel, o inquilino pode ser vítima de "denúncia vazia" após o vencimento do contrato.
Lembre-se ainda que, após o acordo, o aluguel ficará congelado por um ano. Assim, aqueles que estão perto do mês de reajuste podem ser beneficiados se fixarem um novo valor agora.
(Vera Bueno de Azevedo)
</TEXT>
</DOC>
<DOC>
<DOCNO>FSP950101-035</DOCNO>
<DOCID>FSP950101-035</DOCID>
<DATE>950101</DATE>
<CATEGORY>DINHEIRO</CATEGORY>
<TEXT>
Da Reportagem Local 
Os acordos entre inquilinos e proprietários devem crescer este mês, para evitar a ação revisional prevista na MP do Real.
Se for esse o seu caso, não se esqueça de fazer um aditamento (adendo) ao contrato, com o novo valor de aluguel, a mudança da data de reajuste e, se possível, a prorrogação do prazo da locação.
"Acordos verbais podem não ser cumpridos integralmente por uma das partes", explica o advogado Márcio Bueno, diretor jurídico do Creci-SP.
Suponha, por exemplo, um inquilino que faça um acordo verbal com o proprietário, aumentando o valor do aluguel a partir deste mês e mudando a data de reajuste, que inicialmente era março, para janeiro do próximo ano.
Como não há um aditamento com essas novas regras no contrato, o proprietário pode alegar que aceitou apenas o aumento no valor do aluguel (fato que o inquilino pode provar através de recibos dos pagamentos) e aplicar novo reajuste já em março, que é a data prevista no contrato.
(VBA)
</TEXT>
</DOC>
<DOC>
<DOCNO>FSP950101-036</DOCNO>
<DOCID>FSP950101-036</DOCID>
<DATE>950101</DATE>
<CATEGORY>MUNDO</CATEGORY>
<TEXT>
Da Reportagem Local 
Passado o período de festas, a dona-de-casa deve ficar atenta para mais uma despesa. É que o governo decidiu na semana passada pagar um abono de R$ 15,00 para quem ganha salário mínimo.
Com a medida, que é válida a partir deste mês, nenhum trabalhador no país pode receber salário inferior a R$ 85,00 (US$ 100).
O empregado doméstico, que tem direito por lei a ganhar pelo menos um salário mínimo, também terá que receber esse abono.
O valor de R$ 15,00 será pago apenas no mês de janeiro e não será incorporado ao salário mensal. O pagamento deve ser feito até o quinto dia útil de fevereiro (dia 6).
Mas atenção. Só tem direito ao abono quem ganha um salário mínimo em dezembro. Suponha que você já pague mais de R$ 85,00 para sua empregada. Nesse caso, não terá que dar qualquer abono. Quem registrou na carteira profissional do doméstico seu salário em quantidade de salários mínimos também deve pagar o abono.
</TEXT>
</DOC>
<DOC>
<DOCNO>FSP950101-037</DOCNO>
<DOCID>FSP950101-037</DOCID>
<DATE>950101</DATE>
<CATEGORY>MUNDO</CATEGORY>
<TEXT>
Uma cerimônia marca hoje em Genebra o início das atividades da OMC (Organização Mundial de Comércio), sucessora do Gatt (Acordo Geral de Tarifas e Comércio). O irlandês Peter Sutherland, presidente do Gatt, fica interinamente à frente da OMC. Não há consenso sobre o titular e existem três candidatos: Renato Ruggiero (Itália), Carlos Salinas de Gortari (México) e Kim Sul-chu (Coréia do Sul). 

Reeleição de Clinton divide americanos 
Pesquisa feita pela Princeton Research Associates para a revista "Newsweek" indica que os norte-americanos estão divididos sobre se o presidente Bill Clinton deve tentar a reeleição em 1996. Para 47%, ele deveria ficar fora da disputa; 44% acham que ele deve concorrer. 

Instituto acompanha guerras 'esquecidas' 
41 ...é o número de guerras "esquecidas" ocorridas em 1994 –aquelas que raramente foram mencionadas na mídia, que se concentrou em conflitos como os da Bósnia e da Tchetchênia. O levantamento, do Grupo de Pesquisa sobre Causas da Guerra, de Hamburgo, diz que 6,5 milhões de pessoas morreram desde o início dos conflitos ainda em curso. Pelo menos 42 milhões de pessoas foram deslocadas de suas moradias em 1994 por causa de guerras. 

Concessão dos EUA desagrada Seul 
O governo dos EUA aceitou estabelecer conversações militares diretas com a Coréia do Norte para obter a libertação do piloto de helicóptero militar que caiu em território norte-coreano em 17 de dezembro. O subsecretário de Estado Winston Lord disse que a Coréia do Sul, aliada de Washington, ficou "muito desconfortável" com a concessão. 

Governo financiava GAL, dizem policiais 
O esquadrão da morte GAL (Grupo Antiterrorista de Libertação), que atuou ilegalmente entre 1983 e 1985 contra a organização separatista ETA (Pátria Basca e Liberdade), era financiado com fundos secretos do Ministério do Interior da Espanha. A afirmação consta dos diários de dois policiais presos por envolvimento no escândalo, publicados pelo "El Mundo". 

Mulher de 106 anos é caloura de faculdade 
Tabitha Barker, 106, vai entrar na faculdade este ano. Ela se inscreveu no curso de reminiscência da Farnborough College of Technology, no sul da Inglaterra. O curso é sobre o papel dos indivíduos na história. "Eu não poderia ter um aluno mais apropriado", comentou a professora Jennie Espiner
</TEXT>
</DOC>
<DOC>
<DOCNO>FSP950101-038</DOCNO>
<DOCID>FSP950101-038</DOCID>
<DATE>950101</DATE>
<CATEGORY>MUNDO</CATEGORY>
<TEXT>
Braço político do Exército Republicano Irlandês rejeita plebiscito só na Irlanda do Norte 
ROGÉRIO SIMÕES 
Enviado especial à Irlanda do Norte 
Os armamentos do IRA (Exército Republicano Irlandês) não deverão ser entregues a curto prazo e o Sinn Fein, seu braço político, não vai realizar esforços nesta questão.
Esta é a posição do vice-presidente do partido e ex-líder do IRA, Martin McGuinness, 44, que esteve à frente da delegação do Sinn Fein nas duas reuniões com o governo britânico, durante o mês de dezembro.
McGuinness recebeu a Folha para uma entrevista em um dos escritórios de seu partido na cidade de Londonderry, um sobrado simples que não dispensa o circuito interno de TV como medida de segurança.
Depois de comandar durante a década de 80 a base do IRA em Londonderry, onde nasceu, McGuinness passou a atuar na política e se distanciou das operações militares.
Apesar de negar ter contato com o grupo paramilitar, ele é considerado o político com maior influência dentro do IRA. Para muitos, enquanto McGuinness apoiar o processo de paz, o cessar-fogo do grupo será mantido.
 Folha - Quando a questão dos armamentos, dos dois lados do conflito, será resolvida?
McGuinness - Eu não sei. Esta é uma pergunta que o governo britânico deve responder porque foram eles que levantaram esta questão. Nós deixamos claro ao governo que isto não tem nada a ver conosco, é uma questão sobre a qual o Sinn Fein não tem nenhum controle. Não há nada que possamos fazer.
A realidade é que o IRA não é o único com armamentos. A comunidade unionista tem sido armada pelo governo britânico desde a fundação do Estado irlandês. E os esquadrões da morte unionistas têm usado essas armas por muito tempo.
Folha - O primeiro-ministro John Major diz que apenas a população da Irlanda do Norte participará de um plebiscito sobre o futuro da região. O Sinn Fein aceitaria isso?
McGuinness - Não. O que deve ser reconhecido é que esta ilha foi dividida e um Estado foi criado, pelo governo britânico, para os unionistas. Então estamos lidando com uma situação antidemocrática e isso é inaceitável para nós.
Folha - Alguns jornais afirmam que o IRA continua definindo novos alvos na Grã-Bretanha. Quanto tempo você acha que o cessar-fogo pode durar?
McGuinness - Nós vemos em muitas dessas reportagens a mão do serviço de inteligência britânico, que tenta criar esta impressão, de que o IRA está escolhendo alvos. Eu não acredito nisso.
Os republicanos irlandeses estão bastante sérios sobre alcançar o sucesso neste processo de paz. Acho que o IRA está tão ansioso em ver este sucesso como qualquer um.
Folha - O que você acha que vai acontecer com o IRA se houver realmente uma solução pacífica para o conflito?
McGuinness - Bem, se nós alcançarmos o fim da jurisdição britânica na Irlanda, se os britânicos aceitarem o direito do povo irlandês como um todo de autodeterminação e decidirem ir embora, e uma vez que as pessoas estejam satisfeitas, eu acho que não haverá mais necessidade de haver nenhum grupo armado nesta ilha.
Folha - Qual é a sua relação com o IRA hoje? Você fala com os líderes frequentemente?
McGuinness - Antes de o IRA anunciar o cessar-fogo, Gerry Adams e eu nos encontramos com os líderes para discutir a situação política.
Foi a última vez que nós falamos com o IRA, mais de três meses atrás.
Folha - Qual a diferença entre o ambiente no início dos anos 70 e o de hoje?
McGuinness - A grande diferença é que todos nós passamos por 25 anos de uma luta bastante amarga e intensa. O governo britânico sempre imaginou uma solução para o conflito em termos de uma vitória sobre o IRA, o que nunca aconteceu.
Hoje há grande apoio da comunidade européia ao processo. Todo mundo passou a aceitar os argumentos que têm sido colocados nos últimos dois anos, de que tem que aumentar o diálogo.
Os únicos resistentes a isso no atual estágio são o governo britânico e os unionistas.
Folha - Você disse em uma recente entrevista que sentia por algumas vítimas do conflito, especialmente duas crianças que morreram em uma explosão do IRA em Warrington, no ano passado...
McGuinness - Eu não disse que eu sentia especialmente sobre os dois garotos. Eu sinto da mesma forma pela morte de todas as pessoas inocentes.
É muito errado identificar dois garotos em Warrington e isolá-los das nossas crianças que foram mortas pelo exército britânico e pela RUC (polícia).
Antes de essas duas crianças serem mortas em Warrington, dúzias de nossas crianças foram mortas, mas a mídia britânica não escreveu nenhuma reportagem especial sobre elas.
Folha - Você acha que a situação atual seria diferente se os trabalhistas estivessem no governo do Reino Unido?
McGuinness - Sim, eu acho que a situação seria muito pior, porque todas as nossas experiências com trabalhistas foram piores do que as com os conservadores. Um novo governo britânico trabalhista não inspiraria nenhuma grande confiança em mim.
Folha - Você se candidataria para o parlamento de uma Irlanda unificada?
McGuinness - Isso vai depender do Sinn Fein. Provavelmente, nesta época eu já estarei pronto para me aposentar (risos).
</TEXT>
</DOC>
<DOC>
<DOCNO>FSP950101-039</DOCNO>
<DOCID>FSP950101-039</DOCID>
<DATE>950101</DATE>
<CATEGORY>MUNDO</CATEGORY>
<TEXT>
1992 - O líder do SDLP, John Hume, se reúne em sigilo com o líder do Sinn Fein, Gerry Adams.
Outubro de 1993 - Hume informa o governo da República sobre as negociações e exige do governo britânico uma resposta à sua iniciativa. Dez pessoas morrem em atentado do IRA em Belfast. Adams condena publicamente o ataque. Sete pessoas morrem em ataque unionista em Londonderry.
Novembro de 1993 - Em seu discurso anual a rainha Elizabeth coloca o processo de paz na Irlanda do Norte como prioridade.
Dezembro de 93 - Os governos do Reino Unido e da República da Irlanda escrevem a Declaração de Downing Street, em que se comprometem com um processo de paz na Irlanda do Norte.
Janeiro de 94 - Gerry Adams obtém visto para visitar os EUA. Governo britânico protesta.
Junho de 94 - Seis católicos são mortos por protestantes vendo jogo da Irlanda na Copa do Mundo.
31 de Agosto de 94 - O IRA renuncia à violência e se diz comprometido com o processo de paz.
Setembro de 94 - Governo retira a proibição de membros do Sinn Fein falarem em rádios e TVs e anuncia negociações com o partido antes do final do ano.
13 de outubro de 94 - Grupos unionistas anunciam cessar-fogo.
21 de outubro de 94 - John Major aceita como permanente o cessar-fogo do IRA e permite que Gerry Adams e Martin McGuiness entrem na Grã-Bretanha.
9 de dezembro de 94 - Governo e Sinn Fein se encontram pela primeira vez, em Belfast.
15 de dezembro de 94 - Paramilitares unionistas se reúnem com o governo também em Belfast.
</TEXT>
</DOC>
<DOC>
<DOCNO>FSP950101-040</DOCNO>
<DOCID>FSP950101-040</DOCID>
<DATE>950101</DATE>
<CATEGORY>MUNDO</CATEGORY>
<TEXT>
SÔNIA MOSSRI
De Buenos Aires
As investigações do governo argentino sobre dois atentados, no período de dois anos, a entidades judaicas na Argentina permanecem na estaca zero. Por que será que tanto a Polícia Federal, o serviço de inteligência e a Justiça patinam sobre pistas falsas?
Essa é a indagação que os jornalistas Jorge Lanata e Joe Goldman tentam responder no livro "Cortinas de Humo" ou "Cortina de Fumaça" (Planeta, 217 págs., US$ 18,00) depois de quatro meses de investigação.
Em 17 de março de 1992, uma bomba explodiu na Embaixada de Israel em Buenos Aires, depois de o governo norte-americano ter declarado que o aeroporto internacional de Ezeiza não tinha condições mínimas de segurança.
Dois anos depois, em 18 de julho de 1994, um atentado à sede da Amia (Associação Mutual Israelita Argentina) provoca a morte de 86 pessoas.
Semanas antes, o governo argentino havia sido alertado pelo Mossad (serviço secreto israelense) e pelo FBI para a possibilidade de um ato terrorista no país.
A Polícia Federal argentina, por exemplo, não revistou todos os terraços ao redor da sede da Amia. Deteve um garoto de apenas 9 anos como suspeito.
Um simples "dedo gordo" encontrado num apartamento já foi suficiente para a Polícia anunciar a tese de um motorista suicida.
Um dos principais capítulos do livro é a "conexão síria" - vínculos do governo argentino com o traficante sírio de armas e drogas Mozar Al Kassar. Kassar conheceu o presidente Carlos Menem em 1988, em Damasco.
Na ocasião, Menem já discutia com o presidente Hafez al Assad a famosa contribuição síria para a sua campanha presidencial, em 1989. Graças a interferência do presidente, afirma os autores, em 1990 Al Kassar obteve um passaporte argentino às pressas.
Dois anos depois, Al Kassar apareceu envolvido num escândalo de lavagem de narcodólares com a cunhada de Menem, Amira Yoma. Coincidentemente, dez dias depois do atentado à Amia, Al Kassar, acusado de tráfico de drogas, deixou "furtivamente" o país, segundo denúncia do Departamento de Migrações.
Germán Móldes, funcionário da Justiça Federal, estaria envolvido na saída de Kassar do país.
O governo argentino insiste que os explosivos do atentado à Amia entraram no país clandestinamente, pela mala diplomática do Irã. Lanata e Goldman demonstram a facilidade de adquirir no mercado argentino explosivos de alta potência, linha que não consta das investigações oficiais.
Outra tese do governo questionada é a atribuição do atentado à Amia ao grupo radical xiita Hizbollah. Entre os argumentos estão: seria o primeiro atentado na América; e não houve reivindicação pelo grupo terrorista, que sempre o faz.
'Cortina de Humo' relata "o trabalho medíocre da Side, um grupo que se converteu num Estado dentro do Estado" e se baseia em "pistas falsas, investigações sinuosas e respostas esquivas".
</TEXT>
</DOC>
<DOC>
<DOCNO>FSP950101-041</DOCNO>
<DOCID>FSP950101-041</DOCID>
<DATE>950101</DATE>
<CATEGORY>MUNDO</CATEGORY>
<TEXT>
GILSON SCHWARTZ 
Da Equipe de Articulistas 
Nas últimas semanas colocamos em discussão a idéia de âncora cambial e de mercados emergentes, sugerindo, contra a corrente, que a euforia com os casos que se vendia como sucesso exige cautela redobrada. O terremoto mexicano confirmou os piores temores.
A atuação das autoridades econômicas é um aspecto menos comentado da crise mexicana. Mas talvez seja um dos mais dramáticos. Afinal, apenas alguns dias antes da maxidesvalorização, o ministro Serra Puche assegurava ao mundo que a estabilização estava assegurada.
Depois, já no início da crise, a forma de divulgar a suspensão da banda cambial vigente foi desastrada. E, diante do descontrole, veio o silêncio. Nos mercados, a falta de diretrizes do governo foi como lenha na fogueira. Finalmente, na última quinta-feira, Serra Puche caiu.
A justificativa oficial para a ausência de diretrizes (tais como o desenho de uma nova política econômica) foi dada pelo próprio Serra. Para o agora ex-ministro, havia exagero nas reações especulativas. Os mercados deveriam, saberiam e logo poderiam ajustar-se.
Ocorreu, portanto, uma contradição patente entre a lógica do ministro e a lógica dos mercados. Para o ministro, os mercados teriam uma racionalidade imanente. Bastaria dar mais um tempo e viria uma espécie de "autocorreção".
Para os mercados, entretanto, o jogo era o oposto: a propagação do pânico dependia de os agentes vislumbrarem alguma racionalidade na política econômica. A racionalidade tornou-se assim uma espécie de batata quente que ninguém se julgava habilitado a segurar. O peso evaporou nesse vácuo.
É uma lição interessante, que vai além do tema estrito da "âncora cambial". Fica evidente que as regras do jogo econômico só fazem sentido quando Estado e mercados encontram um solo comum onde cada um pode pressupor, com alguma tranquilidade, a racionalidade do outro. Mas é uma pressuposição necessária e precária, como qualquer convenção.
Define-se desse modo um campo na economia que os estudiosos dos últimos anos aos poucos desbravam como a "economia das convenções". É o campo das expectativas. A racionalidade não é um dado, mas um problema.
É interessante comparar a atitude de Serra Puche com a de outros colegas latino-americanos, como Cavallo na Argentina e Gustavo Franco no Brasil. Cavallo tentou, a princípio, fingir que não era com ele. A crise se esparramava, mas nada de o ministro interromper suas férias em Isla Marguerita.
Até que a simulação ruiu diante dos mercados em polvorosa. Em vez de insistir no silêncio, Cavallo tomou um avião para Nova York, onde teve encontros com autoridades do banco central dos EUA.
No Brasil, indagado há algumas semanas sobre quais seriam, afinal, as regras do regime cambial brasileiro, o diretor do BC, Gustavo Franco, saiu-se com esta: as regras serão percebidas pelos agentes do mercado na medida em que as pratiquem.
É uma afirmação inteligente e especialmente oportuna nos momentos em que ninguém, afinal, sente-se tentado a duvidar que as regras existam. As regras econômicas muitas vezes são como as bruxas: ninguém as vê, "pero que las hay, las hay".
</TEXT>
</DOC>
<DOC>
<DOCNO>FSP950101-042</DOCNO>
<DOCID>FSP950101-042</DOCID>
<DATE>950101</DATE>
<CATEGORY>MUNDO</CATEGORY>
<TEXT>
Do enviado especial a Israel 
O processo de paz Israel-palestinos estancou, diz Ahmed Tibi, assessor especial de Iasser Arafat para as conversações de paz e assuntos israelenses. Segundo ele, Israel não quer cumprir parte do acordo, especialmente a redistribuição de tropas da Cisjordânia.
Tibi diz que o nível de vida dos palestinos não melhorou desde o acordo de paz. E, se persistir o impasse no diálogo, a Intifada (revolta nos territórios ocupados) pode ser retomada.
Tibi, 37, é um árabe com cidadania israelense. Formou-se em medicina pela Universidade Hebraica e trabalha como ginecologista na parte leste de Jerusalém. É o médico particular de Suha, mulher de Arafat, e foi ele quem deu ao líder palestino a notícia de que teria um filho.
Tibi é ainda observador do diálogo entre a Fatah (principal grupo da OLP) e a organização fundamentalista Hamas, estabelecido depois do confronto de 18 de novembro, que deixou 13 mortos.
O assessor de Arafat pensa em concorrer ao Parlamento israelense em 1996. Sua eleição criaria a situação sui generis de um conselheiro do líder palestino com poder de voto no Parlamento israelense. "Gosto dessas situações de singularidade", disse Tibi, na entrevista concedida à Folha.
Folha - Como está indo o processo de paz?
Ahmed Tibi - Não muito bem. Começamos bem, o acordo foi aceito e assinado pelos dois lados, mas nas últimas semanas o governo israelense está se comportando como se não quisesse respeitar a segunda parte do acordo interino.
Especialmente, Israel não quer redistribuir suas tropas estacionadas na Cisjordânia, como manda o artigo 13 da Declaração de Princípios. Lá está escrito que as tropas têm que ser deslocadas antes das eleições (para a autoridade palestina nos territórios ocupados). E as eleições deveriam ter sido em julho passado.
Folha - Israel diz que é por motivos de segurança.
Tibi - Sabe, você pode sempre dar a desculpa da segurança. Eu acho que eles estão hesitando em respeitar a totalidade do acordo. Se os palestinos dissessem  este artigo nós não queremos, porque não nos sentimos bem com ele, seria um estardalhaço. Mas, quando os israelenses dizem isso, parece que todo mundo justifica.
Folha - Não há hipótese de eleições sem retirada israelense?
Tibi - É inaceitável. As tropas têm que sair das cidades, campos e aldeias, como diz o acordo.
Folha - É esse o principal risco ao acordo de paz agora?
Tibi - Sim, esse é o maior risco. O segundo maior risco são os grupos que estão agindo contra o processo de paz, dos dois lados. Mas o maior risco é a vontade israelense de não cumprir o acordo.
Folha - O sr. diz que os israelenses têm a intenção deliberada de não cumprir o acordo?
Tibi - Eles dizem isso publicamente. O premiê de Israel, Yitzhak Rabin, disse a Arafat em Oslo (após o recebimento do Prêmio Nobel da paz):  Faça eleições sem retirada de tropas da Cisjordânia.
Folha - É um beco sem saída ou o impasse vai ser resolvido?
Tibi - Eu sou um otimista realista. Espero que isso seja resolvido. Mas por enquanto temos um problema, um grande problema.
Folha - Quanto tempo o sr. acredita que vai levar para a implementação do acordo?
Tibi - Acredito que o Estado palestino será constituído em menos de cinco anos.
Folha - A vida melhorou para os palestinos desde o acordo?
Tibi - Não há uma melhora óbvia no nível de vida do ponto de vista econômico. Há até mesmo deterioração, porque a ajuda dos países doadores não chegou.
Folha - Os países europeus dizem que Arafat não respeita exigências do Banco Mundial...
Tibi - Todas as exigências deles foram respondidas positivamente. Nós estamos prontos para transparência, para contabilidade. Mas eles estão querendo ditar a política econômica da Autoridade Nacional. Isto é totalmente inaceitável.
Folha - É verdade que Arafat está perdendo popularidade?
Tibi - Sempre que há miséria e pobreza, e a presença contínua do ocupante, é natural que a oposição ganhe peso. As pessoas querem ver progressos. Não há progressos visíveis, além do fato de que se formou a Autoridade Nacional Palestina. Mas Arafat ainda é o mais popular líder dos palestinos.
Folha - Há perigo de algum radical do Hamas tentar assassinar Arafat?
Tibi - Não acredito. Eles o aceitam como o líder do povo palestino, mesmo se discordam dele politicamente.
Folha - Há algum tipo de diálogo com o Hamas?
Tibi - Foi criado um comitê de trabalho conjunto entre o Hamas e a Fatah, com três membros de cada lado e dois observadores. Eu sou observador pelos palestinos dentro de Israel.
Folha - O que se conseguiu com o comitê até agora?
Tibi - Tranquilidade em Gaza, depois do conflito da mesquita. E estamos discutindo o relacionamento entre o Hamas e a Autoridade Nacional.
Folha - Eles vão participar das eleições?
Tibi - A decisão é deles. Alguns membros do Hamas já foram incorporados ao sistema, apontados para pertencer à Autoridade.
Folha - Pode haver recuo no processo de paz se o Likud ganhar as eleições em Israel?
Tibi - Seria um desastre. Eu não acho que eles iriam cancelar o acordo, mas esvaziariam de significado cada um de seus artigos.
Folha - A geração da Intifada mudou? Qual é o sentimento hoje dos jovens e crianças que até um ano atrás jogavam pedras nos israelenses?
Tibi - Na Cisjordânia, eles têm os mesmos sentimentos. Porque a ocupação está lá. Em Gaza, a maior parte das tropas se foi, mas eles estão muito nervosos porque os assentamentos ficaram. Se a situação continuar ruim, a Intifada pode voltar.
Folha - Como ginecologista, o sr. vai acompanhar a gravidez de Suha, mulher de Arafat?
Tibi - Eu sou o médico particular dela. Fui eu quem disse a Arafat que ele ia ser pai.
Folha - Como ele recebeu a notícia?
Tibi - Eu telefonei para ele daqui. Ele disse:  Obrigado, obrigado. Obrigado.
(David Cohen)
</TEXT>
</DOC>
<DOC>
<DOCNO>FSP950101-043</DOCNO>
<DOCID>FSP950101-043</DOCID>
<DATE>950101</DATE>
<CATEGORY>MUNDO</CATEGORY>
<TEXT>
DAVID COHEN 
Enviado especial a Israel 
O processo de paz árabe-israelense enfrenta em 1995 sua mais dura prova: a realidade.
A implementação da autonomia palestina nos territórios ocupados por Israel tem um prazo muito bem definido para se firmar: caso contrário, as eleições israelenses de 1996 podem mudar o cenário que propiciou o começo da solução do mais intrincado conflito do planeta e congelar o processo.
O ano passado encerrou a fase Oslo do processo de paz. A capital da Noruega foi sede primeiro do diálogo secreto entre israelenses e palestinos, em 93, e em 10 de dezembro serviu de palco para a entrega do Prêmio Nobel da Paz ao premiê israelense, Yitzhak Rabin, ao chanceler Shimon Peres e ao presidente da Autoridade Palestina, Iasser Arafat.
Agora, israelenses e palestinos têm que resolver a série de impasses e dúvidas que saem dos papéis que assinaram.
A começar pela segurança, obsessão nacional israelense. Para a opinião pública, a cessão de autonomia só é aceitável se for acompanhada por aumento de tranquilidade. Não é o que os israelenses estão sentindo.
"Todo o acordo está baseado no alívio de tensões. E isso não houve até agora, ou pelo menos o público não sente isso", diz um analista político do governo.
Segundo ele, os radicais que lutam contra o processo de paz têm chance de sucesso se conseguirem dar a impressão de que, não importa quantas concessões Israel faça, o país não estará mais seguro.
Por isso, o governo israelense titubeia no cumprimento dos itens do acordo de paz provisório.
Israel duvida cada vez mais do controle de Arafat sobre os palestinos que andam armados em Gaza e na Cisjordânia. E duvida sobretudo da determinação da Autoridade Palestina em reprimir os radicais do Hamas e da Jihad Islâmica.
Para os palestinos, a ótica é inversa. A segurança é apenas um argumento para Israel, que já teria se decidido a não respeitar integralmente o acordo de paz, segundo Ahmed Tibi, assessor especial de Arafat (veja na pág. 3).
O principal impasse é a retirada de tropas israelenses de cidades e povoados na Cisjordânia, antes das eleições para a Autoridade Palestina. As eleições deviam ter sido em julho passado.
Outra exigência palestina é a libertação de prisioneiros. Israel se recusa a libertar os envolvidos em atentados violentos.
Para complicar mais, a ajuda internacional prometida à Autoridade Palestina está emperrada por questões político-burocráticas.
Economistas palestinos calculam em US$ 11 bilhões o total necessário para dotar a faixa de Gaza de infra-estrutura adequada. Até agora, chegaram cerca de US$ 200 milhões em ajuda.
Os riscos ao processo de paz vivem de uma espécie de círculo vicioso. Enquanto a situação da região autônoma não melhora, aumenta o apoio ao radicalismo, que aumenta a ameaça à segurança dos israelenses, o que emperra avanços, o que impede que a situação melhore.
Em 94, rompeu-se o principal impasse na crise árabe-israelense, com a aceitação da fórmula de troca de territórios por paz, conforme as resoluções da ONU.  A questão agora é quanta terra, quanta paz, diz Ron Nachman, deputado do Likud, maior partido de oposição.
Esta é a negociação mais complicada. Pouca gente em Israel aceita a volta às fronteiras de 1967, antes da Guerra dos Seis Dias, que deixam o país vulnerável a combater em áreas densamente povoadas em caso de guerra. Daí a demora em acertar a redistribuição de tropas israelenses na Cisjordânia.
David Mena, também deputado do Likud, diz que se seu partido vencer as eleições de 96 o processo de paz será refreado. "O que nós queremos, antes de mais nada, é congelar o processo", diz, para "ver se os palestinos são sérios".
O mesmo raciocínio vale para as negociações com a Síria, sobre a devolução do Golã, conquistado em 1967. "Não podemos ficar negociando para sempre, temos um deadline", diz Yitzhak Levanon, outro analista político israelense, referindo-se a 96, ano de campanha eleitoral.
Mas tanto Levanon como Ahmed Tibi se definem como "otimistas realistas" e acreditam no avanço do diálogo.
Como disse Shimon Peres, no discurso de aceitação do Nobel da Paz, no dia 10 de dezembro, "houve um tempo em que se guerreava por falta de escolha. Hoje, a opção da qual não se tem escolha é a paz."
"Do momento em que se começa um processo de autonomia, não se pode mais parar. Mas esse processo pode levar tanto 5 como 100 anos", diz Levanon.

O jornalista David Cohen viajou a convite do governo israelense.
</TEXT>
</DOC>
<DOC>
<DOCNO>FSP950101-044</DOCNO>
<DOCID>FSP950101-044</DOCID>
<DATE>950101</DATE>
<CATEGORY>MUNDO</CATEGORY>
<TEXT>
Não devolver 
O slogan mais visto em toda Israel é  Aam im a Golan", O Povo com o Golã. Segundo pesquisas de opinião, a grande maioria dos israelenses é contra a devolução da região conquistada à Síria em 1967.

Rei David 
Um dos debates mais nervosos do Parlamento israelense deu-se em dezembro, quando o chanceler Shimon Peres cometeu o deslize de dizer que não concordava com tudo que o rei David fez. Imediatamente os deputados religiosos começaram uma batelada de ataques a Peres e três partidos submeteram moções de desconfiança ao governo. No dia seguinte, Peres pediu desculpas. Os deputados religiosos recuaram da polêmica porque a mera discussão sobre a honra do rei David pode ser pecado. Cinco dias depois, as moções de desconfiança foram derrotadas por 56 a 41 votos.

Drible na Corte 
O Yihud, uma facção de um partido de oposição, resolveu entrar para a coalizão de governo. Dois de seus três deputados foram atraídos com cargos no ministério. Como a Suprema Corte impedia que deputados que abandonam seus partidos de origem recebam cargos por pelo menos seis meses, a coalizão votou uma emenda à lei básica de governo, revertendo a decisão do judiciário.

Só um viaja 
Dos 17 ministros do gabinete israelense, só um (Agricultura) não é parlamentar. Como a coalizão tem maioria frágil, ele é o único que pode aceitar convites para viajar sem se preocupar. Os outros sabem que sua ausência pode afetar o futuro do governo em caso de voto de desconfiança no Parlamento.

Turismo 
O turismo se beneficia diretamente do processo de paz. Em 1994, 2,2 milhões de pessoas visitaram Israel, 300 mil a mais do que em 1993. Em janeiro e fevereiro, houve 25% a mais de turistas do que no ano anterior. O número caiu em março, depois do massacre de 29 palestinos em Hebron.

Ajuda a árabes 
Israel mantém um programa de ajuda a países do Terceiro Mundo, basicamente em agricultura, saúde e educação. Com o processo de paz, o governo pretende aumentar de 3.000, hoje, para 4.500 o número de bolsistas de países árabes em seus cursos.

Depois de ter assinado o acordo de paz com Israel, em 78, o presidente egípcio Anuar Sadat fez a "prova do tomate" em uma aldeia. Mostrou um tomate murcho aos camponeses, dizendo: "Este é o tomate da guerra". Em seguida, pegou um tomate vermelho e grande, plantado com técnica israelense, e disse: "Este é o tomate da paz".

Brothers in Arms 
Em Israel, praticamente toda a população serve o Exército. Homens, três anos. Mulheres, pouco menos de dois. Até os 50 anos, os homens são convocados cerca de 40 dias por ano para serviço militar.

Mas as Forças de Defesa rejeitaram no mês passado 20 mil novos imigrantes, todos com mais de 25 anos. Alegaram que os custos de treiná-los não compensavam.

Imigrantes 
Israel tem cerca de 5,4 milhões de habitantes, 4,4 milhões judeus. Nos últimos cinco anos, absorveu 600 mil imigrantes russos, sendo 450 mil em dois anos. Deve manter taxa de desemprego perto dos 11%.

"Sem as nossas cadeiras, Rabin não teria maioria para fazer o acordo de paz."
(Noaf Massalha, deputado e vice-ministro da Saúde, sobre os votos árabes a favor do processo de paz no Parlamento de Israel. Os árabes israelenses elegem entre 8 e 10 dos 120 deputados do Knesset.)

"Os imigrantes russos deram a chave do governo ao Partido Trabalhista."
(David Markish, responsável pela absorção russa, sobre os 200 mil votos dos novos imigrantes em 92.)

"Não dá para botar o peixe de volta no mar."
(Ron Nachman, deputado pelo Likud e prefeito do assentamento de Ariel, na Cisjordânia ocupada, sobre a possibilidade de seu partido reverter o processo de paz caso vença as eleições de 96.)
</TEXT>
</DOC>
<DOC>
<DOCNO>FSP950101-045</DOCNO>
<DOCID>FSP950101-045</DOCID>
<DATE>950101</DATE>
<CATEGORY>MUNDO</CATEGORY>
<TEXT>
Trabalhistas - São o partido majoritário na coalizão de governo do premiê Yitzhak Rabin. O grande rival de Rabin na liderança do partido é o chanceler Shimon Peres, maior impulsionador do processo de paz.

Likud - Bloco que governou o país até 92, iniciou as conversações de Madri, mas bloqueou avanços com posições intransigentes. Questiona o acordo de paz.

Partidos religiosos – São vários, têm em comum a luta por concessões especiais do governo como isenção do serviço militar para estudiosos, proibição de comércio aos sábados e verbas para escolas religiosas. Com poucas cadeiras no Parlamento, costumam ser o fiel da balança na formação de governos de coalizão, daí seu poder de barganha.

Árabes israelenses - São os árabes que permaneceram no território de Israel depois da guerra da independência (1948) ou na parte oriental de Jerusalém, anexada depois da conquista de 1967. Têm cidadania israelense.

Territórios ocupados – São as terras conquistadas na Guerra dos Seis Dias, em 1967, por Israel: a faixa de Gaza, a Cisjordânia e o Golã. Jerusalém Oriental foi anexada e Israel não admite discutir a soberania sobre o que considera sua capital eterna.
Assentamentos - Desde 1967, Israel manteve política de povoar os territórios ocupados com colonos judeus. Estes são na maioria contra o processo de paz. Um deles foi responsável pelo massacre de 29 muçulmanos que oravam na mesquita de Hebron, em março passado. A expansão de assentamentos está congelada.

OLP - Organização para a Libertação da Palestina, criada em 1964 e desde 1969 liderada por Iasser Arafat, que chefia sua principal facção, a Fatah. Promoveu guerrilha e atos terroristas com objetivo de criar o Estado palestino, até o acordo de paz de 1993.

Hamas - A palavra significa "ardor" e é formada com as iniciais de Movimento de Resistência Islâmica. É um movimento de massas que luta para implantar um Estado islâmico palestino. Seu braço armado têm promovido atentados para minar o acordo de paz Israel-OLP.

Jihad Islâmica - Grupo que prega a "guerra santa" contra Israel, difere do Hamas por ser uma organização de quadros, ao estilo dos grupos terroristas comunistas dos anos 70. Acham que o trabalho social feito pelo Hamas (escolas, religião, atendimento) é perda de tempo e se dedicam a atacar tropas israelenses.

Hizbollah - Outro grupo radical islâmico, este do sul do Líbano, financiado pelo Irã com complacência da Síria. Atacam tropas israelenses e a milícia que as apóia, o Exército do Sul do Líbano. É devido aos seus ataques ao norte de Israel que o governo israelense mantém a "zona de segurança" no país vizinho.

ESL - Exército do Sul do Líbano, milícia montada por Israel durante a guerra civil libanesa (75-90). A princípio formada por cristãos maronitas, hoje tem maioria de xiitas (seita muçulmana, a mesma do Irã e do Hizbollah) atraídos pelos serviços sociais que Israel fornece a moradores da região sul do Líbano, através da "fronteira da amizade", no norte de Israel.
</TEXT>
</DOC>
<DOC>
<DOCNO>FSP950101-046</DOCNO>
<DOCID>FSP950101-046</DOCID>
<DATE>950101</DATE>
<CATEGORY>COTIDIANO</CATEGORY>
<TEXT>
Precondições
– A crise do bloco socialista, no final dos anos 80, se reflete na decadência das ideologias pan-arabistas e se fortalece o pragmatismo nacional (progressos regionais passam a ser mais importantes do que objetivos históricos de construir uma só nação árabe).
– Com o fim da Guerra Fria, os grupos mais radicais perdem apoio soviético e governos árabes perdem esperança de solução militar para o conflito com Israel, sendo levados ao jogo político.
– EUA e Europa ocidental transformam-se, de inimigos, em possíveis fontes de financiamento.
Guerra do Golfo – A guerra contra o Iraque (janeiro-fevereiro de 91) catalisa o processo, com os EUA lutando lado a lado com países árabes contra outro país árabe.
– Parte do pacto da coalizão era um acordo tácito de que a questão palestina seria encaminhada depois da liberação do Kuait, no contexto das resoluções 242 e 338 da ONU (que estabelecem troca de terras ocupadas por paz)
– Depois da guerra, a pressão internacional levou Israel e vizinhos árabes à Conferência de Madri (outubro de 91).

Palestinos
– Em 92, os trabalhistas ganham a eleição em Israel prometendo romper impasses na paz.
– A OLP de Iasser Arafat (isolada internacionalmente por ter apoiado o Iraque, em crise financeira por ter perdido ajuda da Arábia Saudita e países do Golfo e perdendo terreno para o grupo radical islâmico Hamas nos territórios ocupados) percebe que perdia oportunidade histórica e estabelece diálogo secreto paralelo com Israel, em Oslo (Noruega).
– Em setembro de 93, o premiê israelense, Yitzhak Rabin, e o líder da OLP, Iasser Arafat, assinam acordo de paz nos EUA.
– O acordo, chamado de Gaza-Jericó Primeiro, estabelece autonomia palestina na faixa de Gaza e numa área em volta da cidade de Jericó, na Cisjordânia. A área de autonomia na Cisjordânia deve ser estendida num prazo de cinco anos, sendo que a partir do terceiro começam as discussões sobre o status permanente dos territórios.

Jordânia – Depois da OLP, segue acordo de paz com a Jordânia. O país praticamente não tinha conflito com Israel. Em outubro de 94 o rei Hussein e o premiê Rabin selam a paz. Em dezembro, Israel e Jordânia trocam embaixadas.

Síria – O diálogo mais emperrado para Israel é com a Síria. O governo sírio exige, antes de qualquer negociação efetiva, a devolução integral do Golã, conquistado por Israel na guerra de 1967. Israel só aceita devolver parte do Golã, em etapas, conforme avançar a normalização de relações com a Síria.
- O Golã é um altiplano estratégico. De 1948 a 1967, tropas sírias costumavam bombardear povoados no norte de Israel. Hoje, tanques israelenses se posicionam a cerca de 40 minutos de Damasco, a capital síria. Além disso, a partir de lá se pode controlar cerca de 30% do abastecimento de água do Estado judaico.

Líbano – O Líbano exige que Israel desocupe a faixa de até 12 km de profundidade no sul de seu território. Israel diz não ter pretensão territorial, mas que só retira tropas quando o governo libanês for capaz de impedir ataques da guerrilha islâmica Hizbollah ao norte israelense. O conflito é visto como apêndice da questão síria, que mantém mais de 30 mil soldados no país e praticamente dita a política libanesa.
</TEXT>
</DOC>
<DOC>
<DOCNO>FSP950101-047</DOCNO>
<DOCID>FSP950101-047</DOCID>
<DATE>950101</DATE>
<CATEGORY>COTIDIANO</CATEGORY>
<TEXT>
ALESSANDRA BLANCO 
PATRICIA DECIA 
Da Reportagem Local 
Balanço de vida é algo que se faz em véspera de casamento, velório de amigo ou nos últimos segundos do ano, quando já começou a contagem regressiva (10, 9, 8, 7...). A pedido da Folha, 60 personalidades pararam para pensar no que fizeram até agora e responderam à pergunta: "qual foi o melhor ano da sua vida?".
A enquete mostra que as pessoas mais díspares –como a atriz Sonia Braga, o cientista Robert Gallo e Pelé– deram respostas idênticas. Ladrões, cardeais, roqueiros, milionários e escritores "suaram" na hora de responder. "Nunca tinha pensado nisso" foi a primeira reação da maioria.
Alguns demoraram ("É muito difícil. Tenho que pensar na minha vida inteira...", disse o compositor João Gilberto) e dois intelectuais desistiram de vez. O professor e crítico literário Antonio Candido de Mello e Souza, 77, ficou com medo de arrumar briga com a filha ou a mulher. "Se disser que foi o ano em que casei, minha filha fica zangada. E imagine o que acontece se eu disser que foi quando conheci minha primeira namorada!" O linguista norte-americano Noam Chomski, 66, também negou fogo: "Não respondo perguntas pessoais como esta. Mais ou menos por princípio..."
No final, foi possível agrupar os depoimentos. Muitos homens e mulheres lembraram-se do ano em que nasceram os primeiros filhos (para desespero dos caçulas).
Ganhadores de Nobel, jogadores de futebol e políticos identificaram seus feitos profissionais como o melhor tempo de suas vidas. Veteranas do cinema e da televisão lembraram com nostalgia do "vigor da juventude".
Apenas um citou a infância e poucos falaram de conquistas amorosas –exceção para alguns casados que fizeram média com as mulheres e Adriane Galisteu, namorada de carteirinha de Ayrton Senna ("Apesar de ter só 21 anos, tenho certeza que 1993 sempre será o melhor ano de toda a minha vida. Amei, fui amada e continuo amando").
Entre os patriotas, que gostaram de 1994 por causa dos "progressos brasileiros", estão o novo presidente Fernando Henrique Cardoso, dom João de Orleans e Bragança e a advogada Meire Franco. O ex-presidente Itamar Franco também enumerou vários motivos nacionalistas para eleger 1994, mas poderia ser enquadrado entre os românticos: todo mundo se lembra que foi nesse ano que ele namorou June e sambou ao lado da desnuda Lilian Ramos.
Os mais felizes –Roberto Carlos, Ivo Pitanguy e o carteiro Sansão Santos de Oliveira– agradeceram a Deus e tiveram dificuldade em apontar um ano especial porque suas vidas têm sido uma sucessão de "momentos especiais".
Para os não tão felizes, restou a saída "Scarlet O'Hara", a heroína de "E o Vento Levou..." que, ao final de quatro horas de filme, exclama: "Amanhã é um outro dia". A bela e milionária Xuxa e o várias vezes campeão Emerson Fittipaldi estão nesse grupo. Os dois responderam que "o melhor ainda está por vir".
Uma "personalidade" não pode ser incluída em nenhum grupo. Por fax, ele pediu que fosse aberta uma exceção para que pudesse escolher o "pior ano de todos": "1994, quando perdi minha mulher e minha liberdade". Pense nos poucos ex-poderosos que acabaram o ano na cadeia e adivinhe quem é.

Colaboraram DANIELA ROCHA, de Nova York; Folha Sudeste e Sucursal do Rio

LEIA MAIS   
Sobre os melhores anos nas páginas 4 e 5
</TEXT>
</DOC>
<DOC>
<DOCNO>FSP950101-048</DOCNO>
<DOCID>FSP950101-048</DOCID>
<DATE>950101</DATE>
<CATEGORY>COTIDIANO</CATEGORY>
<TEXT>
Integração aduaneira com Argentina, Uruguai e Paraguai faz taxa cair para 16% 
EUNICE NUNES 
Especial para a Folha 
Hoje entra em vigor um novo estágio do Mercosul (Mercado Comum do Sul), que engloba Brasil, Argentina, Uruguai e Paraguai. A partir de agora, forma-se uma zona de integração aduaneira, na qual os quatro membros adotam uma Tarifa Externa Comum (TEC) para quase todo produto importado de outro país.
Entre as exceções à aplicação da TEC –e a que levará mais tempo para chegar à tarifa comum– estão os bens e serviços de informática e telecomunicações. Os quatro países terão até 2006 para chegar à TEC de 16% para tais produtos. 
"No Brasil, o Imposto de Importação incidente sobre bens de informática e telecomunicações acabados varia entre 35 e 40%. Na Argentina, por exemplo, a alíquota cobrada na importação dos mesmos produtos é zero", informa o advogado Georges Fischer, presidente da Associação Brasileira de Direito de Informática e Telecomunicações (ABDI).
A variação das alíquotas adotadas nos quatro países torna impossível a adoção imediata da tarifa comum na importação de bens de informática e comunicação.
O assunto foi debatido durante dois anos. A conclusão foi que o Brasil, progressivamente, até o ano 2006, baixaria a alíquota do Imposto de Importação até chegar a 16%. Num movimento contrário, os outros membros do Mercosul, no mesmo prazo, aumentariam o imposto até chegar a 16%.
Segundo Fischer, o Brasil não terá problemas por ter um Imposto de Importação mais alto. "Outros países pagarão menos imposto, mas lá não há indústrias de informática e telecomunicações", diz.
Na reunião de Ouro Preto (MG), realizada nos dias 15 e 16 de dezembro último, tendo em vista os diferentes níveis de desenvolvimento em que se encontram as indústrias de informática e telecomunicações dos quatro países, foi adotado um tratamento especial para o setor. Ficou resguardada a exigência de manufatura local.
Um produto, fabricado em um dos países membros, para circular livremente no Mercosul tem de ter 60% de conteúdo local, caso contrário pagará Imposto de Importação como um produto estrangeiro.
No Brasil, as regras são ainda mais restritivas no campo da informática. Aqui há o "processo produtivo básico", segundo o qual todos os componentes básicos (como placas e partes elétricas) têm de ser montados no país.
"Na reunião de Ouro Preto ficou acertado que os outros membros do Mercosul terão de se adequar e adotar processo produtivosemelhante ao do Brasil", explica Georges Fischer.
Na área de proteção intelectual do software ainda não há nada definido no âmbito do Mercosul. Mas deverá ser adotada a norma do GATT (Acordo Geral de Tarifas e Comércio), que protege o software durante a vida do autor e mais 60 anos após sua morte.
No Brasil, segundo a Lei do Software, a proteção é de 25 anos a contar da data de lançamento do programa. "Mas isso deve mudar com a nova lei do software. O Executivo vai mandar para o Congresso um projeto de lei que adere à decisão do GATT", diz Fischer.
O projeto de lei tem 13 artigos. Além de dar ao programa de computador a mesma proteção autoral conferida às obras literárias, acaba com a burocracia que obriga quem quer comercializar software estrangeiro a cadastrar o produto junto ao governo.
</TEXT>
</DOC>
<DOC>
<DOCNO>FSP950101-049</DOCNO>
<DOCID>FSP950101-049</DOCID>
<DATE>950101</DATE>
<CATEGORY>COTIDIANO</CATEGORY>
<TEXT>
Especial para a Folha
Os países do Mercosul não têm normas comuns no campo dos direitos do consumidor. O Brasil tem o Código de Defesa do Consumidor, considerado um dos mais modernos e avançados do mundo. A Argentina tem uma lei aprovada em 1994, mas que é menos ampla e rigorosa do que a brasileira. Paraguai e Uruguai não têm leis de proteção do consumidor.
"Faltam normas conjuntas, com regras mínimas sobre comercialização e segurança do consumidor a serem seguidas pelos quatro membros do Mercosul", diz Arystóbulo Freitas, advogado especialista em relações de consumo.
Por exemplo, as informações que devem constar das embalagens (composição, peso, prazo de validade e outros) devem ser as mesmas em todos os países. "Caso contrário aumentarão os custos das empresas, que precisarão ter embalagens diferentes para cada país", afirma Freitas.
O código brasileiro garante ao consumidor informações claras e precisas sobre o produto ou serviço. Mercadorias produzidas em outro país do Mercosul que não atenda tais requisitos poderá ter sua venda impedida no Brasil.
A solução é a doação de normas supranacionais que deverão ser absorvidas pelo direito interno de cada país, como na União Européia.
</TEXT>
</DOC>
<DOC>
<DOCNO>FSP950101-050</DOCNO>
<DOCID>FSP950101-050</DOCID>
<DATE>950101</DATE>
<CATEGORY>COTIDIANO</CATEGORY>
<TEXT>
Especial para a Folha
Há três maneiras possíveis para resolver controvérsias surgidas entre Brasil, Argentina, Uruguai e Paraguai, os países do Mercosul.
A primeira dá-se através de negociações diretas entre os envolvidos na questão. No geral, este procedimento não poderá ultrapassar o prazo de 15 dias, contados da data em que um dos países houver suscitado o conflito.
Quando as negociações diretas fracassarem, o problema será submetido à apreciação do Grupo Mercado Comum (GMC), que é o órgão executivo do Mercosul (formado por quatro titulares e quatro suplentes de cada país).
O GMC, em período não superior a 30 dias, avaliará a situação e formulará recomendações para solucioná-la. O prazo de 30 dias é contado a partir da data em que o litígio tiver sido oferecido à consideração do GMC.
Se a solução do GMC não for acatada, o conflito será resolvido por um tribunal arbitral, constituído para o fim específico de resolver a controvérsia.
O tribunal será composto por três árbitros e sua decisão será dada por escrito num prazo de 60 dias, prorrogáveis, no máximo, por mais 30 dias. A decisão do tribunal arbitral é irrecorrível, obrigatória para os países litigantes e deve ser cumprida imediatamente.
</TEXT>
</DOC>
<DOC>
<DOCNO>FSP950101-051</DOCNO>
<DOCID>FSP950101-051</DOCID>
<DATE>950101</DATE>
<CATEGORY>COTIDIANO</CATEGORY>
<TEXT>
ALMIR PAZZIANOTTO PINTO 

"Não fomos capazes de desenvolver mecanismos minimamente eficientes de conciliação e arbitragem" 

Mostram as estatísticas da Justiça do Trabalho que, de 1983 a 1994 (até o mês de junho), deram entrada nas Juntas de Conciliação e Julgamento 12.635.496 reclamações trabalhistas ajuizadas por um ou mais de um reclamante, ou pelos sindicatos de trabalhadores investidos no papel de substitutos processuais.
Em grande parte são processos de interesse de ex-empregados julgando-se credores de diferenças salariais, ou de outros direitos não liquidados por ocasião da rescisão contratual.
Imaginemos cada uma dessas reclamações a exigir, em média, dez horas de dedicação exclusiva das partes, advogados, juízes de várias instâncias, testemunhas, peritos, funcionários, nada menos do que 126.354.960 horas seriam consumidas para se concluir se as dívidas reivindicadas existem, e serão ou não pagas.
Esse dispêndio de tempo, mão-de-obra, dinheiro e energia ocorre porque não fomos capazes de desenvolver técnicas e mecanismos minimamente eficientes de conciliação e arbitragem, que funcionariam junto às empresas e sindicatos visando permitir que se convertessem em processos unicamente aqueles casos que, pela sua complexidade, merecessem apreciação da Justiça especializada.
Creio ser axiomático que, quanto maior o número de ações em andamento, menor o tempo disponível dos juízes para cada uma delas e, por contraditório que possa parecer, maior a demora registrada para que decisões definitivas sejam tomadas e cumpridas.
Também é verdadeiro que, ao se multiplicarem os órgãos de decisão mais variadas e entrechocantes serão as interpretações acerca dos mesmos temas jurídicos.
Buscando acompanhar essa brutal elevação de volume de serviço, a Justiça do Trabalho tem solicitado ao Executivo e ao Legislativo a criação de mais juntas, a ampliação dos tribunais regionais do trabalho, e a nomeação dos juízes e funcionários destinados a colocá-las em funcionamento.
Nada disso, porém, resolverá a crise em que se debate o Judiciário Trabalhista se medidas deixarem de ser adotadas com o salutar propósito de reduzir a quantidade de processos. Nesse sentido, aos empresários e aos sindicatos de trabalhadores incumbe, em nome do interesse nacional, realizar incessante esforço para liquidação, mediante acordo, do maior número das demandas em andamento.
Por outro lado, se conseguíssemos fazer com que, de cada dez reclamações embrionárias cinco fossem solucionadas diretamente pelas partes, e das outras cinco, pelo menos três findassem antes da audiência de instrução, a atual capacidade produtiva do Judiciário trabalhista seria multiplicada, sem a obrigatoriedade de instalação de novas juntas, ou da ampliação da capacidade de trabalho dos tribunais.
Ostentamos o comprometedor título de "campeões mundiais de processos trabalhistas", indesejável quando urge retomar o processo de crescimento econômico, interrompido há anos.
As incertezas e a insegurança características das relações de trabalho no Brasil converteram-se em fortes obstáculos para os investidores externos, habituados a administrarem recursos humanos sob princípios racionais e modernos, diferentes daqueles que temos tradicionalmente obedecido, e que fazem do conflito permanente, entre patrões e empregados, uma espécie de estilo de vida.

Almir Pazzianotto Pinto, 58, é ministro do Tribunal Superior do Trabalho (TST). Foi ministro do Trabalho (governo Sarney) e secretário do Trabalho do Estado de São Paulo (governo Montoro).
</TEXT>
</DOC>
<DOC>
<DOCNO>FSP950101-052</DOCNO>
<DOCID>FSP950101-052</DOCID>
<DATE>950101</DATE>
<CATEGORY>COTIDIANO</CATEGORY>
<TEXT>
WALTER CENEVIVA 
Da Equipe de Articulistas 
Pode parecer inacreditável, mas o ano de 1994 mostrou que a medida provisória é pior que o decreto-lei. Menos prática, pois seria eficaz num país cujos deputados e senadores pudessem cumprir, com presteza e qualidade, a difícil missão de legislar. Não é o caso do Brasil, como se evidenciou nos seis anos de vigência da Carta de 88.
A medida provisória tem sido repetida pelo Executivo (veja-se a do real), com desrespeito à relevância e urgência que a justificam e com ofensa ao princípio de que a função legislativa cabe ao Congresso, muito embora o Congresso não o tenha defendido.
Há, porém, um elemento de psicologia política nada desprezível em relação ao decreto-lei. Quem tenha vivido os anos de ditadura, sabe que o decreto-lei foi utilizado para impor a vontade do poder (nem sequer podia ser emendado), dando amparo aos casuísmos compatíveis com os desígnios de dominação do povo, completando dezenas de emendas constitucionais destinadas ao mesmo fim.
O modelo italiano –no qual nosso decreto-lei foi baseado– ainda pode oferecer alguma inspiração. A Carta de 1969 impunha prazos rigorosos ao Congresso, o que há de ser corrigido, antes de ser mesclado com as boas intenções que nortearam a medida provisória. Aceita a alternativa, proponho a criação do decreto executivo, com vigência provisória de 90 dias, de livre apreciação congressual, inclusive quanto a emendas, o qual passará a vigorar definitivamente, na forma original, se o Congresso não o votar dentro daquele prazo.
Para a máquina judiciária o final do ano trouxe importantes modificações, com as leis 8.950 e 8.953, que alteraram o Código de Processo Civil. Elas coroaram trabalho realizado sob a direção do ministro Sálvio de Figueiredo Teixeira, em comissão da qual tive a honra de participar ao lado de ilustres figuras do direito brasileiro. Consistiu num esforço simplificador do processo, visando a desburocratizá-lo. Aguardemos, para 95, o resultado prático desse trabalho.
1995 retomará, no começo, o debate sobre a revisão constitucional, em que o ministro da Justiça, Nelson Jobim, representará papel decisivo. A contar da posição do Brasil no plano externo, a revisão tratará de facilitar nosso relacionamento internacional. Nesse campo o deputado André Franco Montoro terá participação relevantíssima. Foi ele o criador do parágrafo único do art. 4º da Constituição atual, que afirma a vocação latino-americana do Brasil. Seu Instituto Latinoamericano (ILAM) desenvolveu longa atividade nos últimos anos, objetivando criar suporte constitucional e legislativo para dinamizar o relacionamento com os países vizinhos e com os demais da América Latina. Montoro é um entusiasta da relação latinoamericana e estará em justa evidência, nas iniciativas desse campo.
No plano interno, a revisão estará voltada, entre outros fins, para o reequilíbrio das fontes de renda da União, compatibilizando-as com suas despesas, o que não deverá, porém, ser desculpa para que o país retome a centralização governamental que preponderou desde os anos 30. A descentralização estadualizadora e municipalizadora mostrou bons efeitos, ao mesmo tempo em que a máquina federal exibiu seus males, sem disfarce.
O ano começa, porém, com otimismo. Será bom que os impulsos de revisão se espraiem para além das leis. Que cada poder revalie seus defeitos, de modo a corrigí-los, o que é especialmente importante, numa coluna de letras jurídicas, para o Judiciário.
</TEXT>
</DOC>
<DOC>
<DOCNO>FSP950101-053</DOCNO>
<DOCID>FSP950101-053</DOCID>
<DATE>950101</DATE>
<CATEGORY>COTIDIANO</CATEGORY>
<TEXT>
Da Reportagem Local 
Pessoas que tiveram vidas mais do que excitantes, como Sophia Loren e Caetano Veloso, elegeram como o melhor de suas vidas momentos que acontecem com quase todo mundo: o nascimento dos filhos.
A reprodução da espécie ficou em primeiro lugar na enquete, tendo sido citada por 12 entrevistados. Para Desmond Morris, 66, biólogo inglês, autor do best-seller "O macaco nu", o medo da morte explica o apego à maternidade. "Instintivamente, através das crianças, nos sentimos estranhamente imortais. Isso dá um prazer muito grande", disse Morris à Folha, por telefone, de Oxford.
Para os que aclamaram que o próprio nascimento foi o melhor momento de suas vidas, restou uma análise desanimadora. A psicanalista junguiana Denise Gimenez Ramos, 45, professora da PUC-SP, acredita que essas pessoas são essencialmente narcisistas. "É como se pensassem: 'Eu nasci e o mundo todo mudou"', interpreta Denise.
Segundo a professora, esses ilustres se sentem tão importantes que seu próprio nascimento é capaz de superar qualquer outro de seus grandes feitos.
Já quem não consegue apontar um ano especial –ou porque todos são ruins ou maravilhosos demais– foi tachado de "estranho" por Morris."É surpreendente esse tipo de resposta e muito triste também. Isso significa uma vida linear, de conquistas ou fracassos sucessivos", acredita o biólogo.
Falta de criatividade foi a explicação dada por Denise para quem não teve nada a dizer. "São pessoas que passam a vida em brancas nuvens e não percebem o que acontece com elas. Se está tudo igual, é porque não estão vendo absolutamente nada", diz.
A psicanalista Mirian Chnaiderman não concorda. Para ela, não existe um ano especial porque as pessoas "constroem as coisas no dia-a-dia". Mirian condena a idéia de um período marcante no passado. "Acho complicado viver nostálgico o resto da vida. É como se a alegria de viver tivesse ficado para trás", afirma.
A infância, endeusada na literatura e no cinema como a fase melhor e mais pura, foi esquecida por quase todos. Para Morris, isso acontece porque ninguém se lembra dos tempos de criança.
"Meu filho, aos 4 anos, teve o melhor ano de sua vida. Morávamos perto do mar, tínhamos um barco, ele ia à praia todos os dias. A vida não podia ser mais perfeita para um garotinho. Eu digo isso a ele hoje (ele tem 26 anos), mas ele me responde que simplesmente não se lembra", conta Morris.
O biólogo, que escreveu um livro sobre as idades, acredita que os 18 anos são o ápice do desempenho sexual masculino. Para as mulheres, só aos 36 anos. A criatividade pelo menos coincide: ambos chegam ao apogeu aos 38 anos.
O conselho, portanto, para quem for participar da enquete é tentar fazer um balanço geral. E não se apegar "apenas" aos sucessos mais recentes, como fizeram o jogador de futebol Zinho e o presidente Fernando Henrique Cardoso.
(ABl e PD)
</TEXT>
</DOC>
<DOC>
<DOCNO>FSP950101-054</DOCNO>
<DOCID>FSP950101-054</DOCID>
<DATE>950101</DATE>
<CATEGORY>COTIDIANO</CATEGORY>
<TEXT>

ZÉLIA GATTAI, 78, escritora - "É muito difícil escolher. Em 45, conheci Jorge, meu primeiro filho nasceu em 41, os outros dois filhos nasceram em 47 e 51. Todos estes anos foram maravilhosos. O melhor réveillon foi em 52, quando passamos no Kremlin. Havia escritores, artistas e músicos de toda parte do mundo no palácio. O baile nos grandes salões me impressionou muito."
</TEXT>
</DOC>
<DOC>
<DOCNO>FSP950101-055</DOCNO>
<DOCID>FSP950101-055</DOCID>
<DATE>950101</DATE>
<CATEGORY>COTIDIANO</CATEGORY>
<TEXT>

GUILLERMO CABRERA INFANTE, 65, escritor cubano - "1958. Conheci a atriz Miriam Gomes, compilei meu primeiro livro (Asi en la paz como en la guerra), escrevi o conto En el gran ecbó, nasceu minha filha Carola e Batista Fulgêncio (presidente na época) fugiu de Cuba. Estava cheio de ilusões democráticas."
</TEXT>
</DOC>
<DOC>
<DOCNO>FSP950101-056</DOCNO>
<DOCID>FSP950101-056</DOCID>
<DATE>950101</DATE>
<CATEGORY>COTIDIANO</CATEGORY>
<TEXT>

ALBERT EINSTEIN (1879-1955): Tarefa árdua indicar o ano mais feliz na vida de Einstein, a personalidade-síntese deste século. Não há consenso mesmo entre biógrafos. Depoimentos pessoais apenas revelam pistas. Por exemplo: Einstein admitiu que foi feliz em 1896, aos 17 anos, quando frequentou a escola da província de Aarau, Suíça. Lá, encontrou a liberdade que nunca teve no rígido sistema alemão. Era de se esperar que em 1905, ano da publicação de sua Teoria da Relatividade e outros dois trabalhos geniais, tenha sido o mais feliz. O mesmo vale para 1919, comprovação da Teoria da Relatividade Geral. Ou 1921, quando ganhou o Nobel. Mas um trecho sutil de "Einstein lived here" de Abraham Pais (Oxford University Press, 1994) pode contar a mais forte evidência (ainda que subjetiva) do ano –e até do momento– mais feliz na vida dele. Próximo à morte, em 55, ele recebeu a visita-reconciliação de seu filho mais velho, Hans Albert, então professor da Universidade da Califórnia. Talvez aí Einstein tenha se redimido de suas "inabilidades" como marido e pai. "Viver em harmonia duradoura com uma mulher foi uma tarefa na qual falhei desgraçadamente por duas vezes", disse ele certa vez. Como pai, não foi melhor. Não há indícios de que tenha sequer conhecido sua primeira filha, Lieserl, nascida antes do casamento formal com Mileva Maric. Eduard, seu filho mais novo, não mereceu maior atenção: esquizofrênico, perambulou entre clínicas até sua morte em 1965 em um centro psiquiátrico na Suíça. Hans Albert era, de certa forma, a redenção para o Einstein pai e marido. Era também pelo menos parte do testemunho de muitos momentos felizes do Einstein cientista, cidadão e pacifista. Depois do encontro, o maior cientista deste século com certeza morreu mais feliz –e enfim totalmente em paz. (por Cássio Leite Vieira, Especial para a Folha)

NAPOLEÃO BONAPARTE (1769-1821) - "O melhor ano da vida de Napoleão foi 1804, quando foi proclamado imperador da França. Foi um caminho longo para alguém que começou sua carreira como um simples tenente de artilharia, e ainda por cima nascido em uma província meio brega, a Córsega. Nesse ano Napoleão foi responsável por uma obra tão importante quanto suas batalhas: um código de leis, o código napoleônico, que influenciou toda a legislação ocidental desde então. 1805 poderia ser ainda mais agradável ao imperador corso, se não fosse um detalhe. Foi em 1805 que Napoleão venceu russos e austríacos em uma de suas mais magistrais batalhas, Austerlitz. Não é por nada que uma das estações de trem de Paris se chama Gare d'Áusterlitz."(por Ricardo Bonalume, Especial para a Folha)

ALBERT CAMUS (1913-1960): "Vivia na penúria, mas também numa espécie de regozijo". Assim o escritor Albert Camus descrevia sua juventude, na colônia francesa da Argélia. O lirismo dessa vida despojada, sob o sol mediterrânico, atingiria seu ápice em 1935, quando Camus escreve sua primeira obra literária: "O Avesso e o Direito". (por Manuel da Costa Pinto, da Reportagem Local)

SIGMUND FREUD (1856-1939) - "1899 porque ele terminou e mais tarde (em novembro) publicou a Interpretação dos Sonhos. Acho que ele ficou muito satisfeito porque foi seu primeiro grande livro e talvez sua obra decisiva. Na vida pessoal, ele já tinha seis filhos e parecia bem com a família. Apesar de ser ocupado, sempre teve tempo para a família, especialmente no verão e não havia conflitos. (por Peter Gay, biógrafo)
</TEXT>
</DOC>
<DOC>
<DOCNO>FSP950101-057</DOCNO>
<DOCID>FSP950101-057</DOCID>
<DATE>950101</DATE>
<CATEGORY>COTIDIANO</CATEGORY>
<TEXT>

RUBENS BARRICHELLO, 22, piloto de Fórmula 1 - "1990 porque foi o ano em que corri na Fórmula Opel e me sagrei campeão. Foi um ano muito difícil, meu primeiro na Europa, com apenas 17 anos e sem ainda falar bem o idioma. Senti muita solidão e saudade da família, que eu apenas conseguia esquecer quando estava pilotando."
</TEXT>
</DOC>
<DOC>
<DOCNO>FSP950101-058</DOCNO>
<DOCID>FSP950101-058</DOCID>
<DATE>950101</DATE>
<CATEGORY>COTIDIANO</CATEGORY>
<TEXT>
ELVIRA LOBATO 
Enviada especial ao Japão 
Os brasileiros descendentes de japoneses, que deixaram o país com suas famílias para trabalhar no Japão, estão enfrentando um problema novo: ensinar o português para as crianças que desaprenderam a língua e não conseguem se comunicar com os pais.
Cerca de 170 mil brasileiros vivem no Japão. Eles formam o principal contingente da mão-de-obra de descendentes nascidos no exterior e que lá são conhecidos como dekasseguis.
A grande maioria é de adultos que foram sozinhos fazer o trabalho pesado que a rica sociedade japonesa já se recusa a realizar, em troca de um salário muito alto para os padrões brasileiros: em média, US$ 2 mil por mês.
Em Hamamatsu, uma cidade industrial a 170 km ao sul de Tóquio, vivem 20 mil brasileiros. Eles representam 4% da população da cidade e criaram uma estrutura própria de organização.
Todos os domingos, 40 crianças, com idades de 6 a 11 anos, passam as tardes reaprendendo o português em duas salas de aula improvisadas em um prédio comercial no centro da cidade.
O curso é gratuito e foi criado pelo paulista João Toshiei Masuko. Há seis anos, ele foi para o Japão trabalhar como operário em uma fábrica de autopeças e hoje é o dekassegui mais bem-sucedido de Hamamatsu.
As aulas são dadas pela professora Célia Nakamura, que se aposentou há dois anos no Rio de Janeiro e foi se juntar aos parentes.
Ela conta que as crianças esqueceram o português porque os pais passam o dia todo no trabalho, enquanto elas ficam nas escolas japonesas.
"Há casos dramáticos de pais que não conseguem se comunicar com os filhos porque eles não aprenderam o japonês e elas não conseguem se expressar em português", diz Célia.
Com patrocínio de seis empresas, foi editada uma cartilha –"Primeiros Passos ma Língua Portuguesa"– para que as crianças possam ser alfabetizadas em sua língua de origem.
João Masuko, que tem quatro filhos em São Paulo, diz que as crianças que desaprenderam o português são de famílias que vivem isoladas dos demais brasileiros.
As famílias vão do Brasil direto para os alojamentos designados pelas firmas de contratação de mão-de-obra dekassegui e, na obsessão de juntar poupança, os adultos fazem horas extras diárias no trabalho, enquanto as crianças ficam sob influência da TV e das escolas.
Como no Japão as ruas não têm nome, nem os prédios são numerados, a dificuldade de locomoção aumenta o isolamento das pessoas.
"Hoje sabemos de casos de famílias que chegaram a comer ração para cachorro, porque não sabiam identificar os produtos nos supermercados, nem tinham como se comunicar com os vendedores", diz Masuko.

A repórter Elvira Lobato viajou ao Japão a convite da NEC do Brasil.
</TEXT>
</DOC>
<DOC>
<DOCNO>FSP950101-059</DOCNO>
<DOCID>FSP950101-059</DOCID>
<DATE>950101</DATE>
<CATEGORY>COTIDIANO</CATEGORY>
<TEXT>
Da enviada especial 
Rapidamente, os brasileiros que vivem em Hamamatsu estão criando uma estrutura de serviços para atender a própria comunidade.
Nos últimos quatro anos, surgiram cinco restaurantes, quatro mercados e até uma danceteria, chamada Cristal, onde só se fala do Brasil.
Aos sábados, um padre dekassegui celebra missa em português em um galpão alugado no centro da cidade.
A maior empresa brasileira, que tem o surpreendente nome de "Servitu", ocupa quatro andares de um prédio no centro da cidade.
O proprietário, João Masuko, era empregado de supermercado em São Paulo. Hoje tem seu próprio supermercado no Japão.
A "Servitu" virou uma empresa de negócios variados: providencia registro de casamento, alistamento militar, certidão de nascimento e atestado de óbito, seguro de carro, aluguel de telefone, além de funcionar como uma central de distribuição de produtos brasileiros para vários pontos do Japão.
Masuko não revela o quanto ganha por mês, mas já está abrindo um novo negócio: a padaria "Servipão", com um padeiro levado do Brasil.
Nos mercados de brasileiros pode-se alugar fitas de vídeo com os capítulos das novelas que passaram no Brasil na semana anterior. As revistas semanais chegam com apenas três dias de atraso.
Vendem ainda farinha, carne seca, feijão e uma infinidade de itens importados do país, como mandioca congelada, massa para pastel e até cocadas.
O Banco do Brasil, que há um ano instalou um posto de serviço em Hamamatsu, descobriu um mercado para poupança e investimentos que não imaginava.
Os 20 mil brasileiros, somados, acumulam US$ 20 milhões em poupança por mês e todo o dinheiro ficava guardado em casa, porque eles não sabiam se movimentar nos bancos japoneses.
O posto de serviço passou a funcionar como uma espécie de consulado e ponto de encontro, onde se pode fazer amigos e ler os jornais brasileiros com uma semana de atraso.
(EL)
</TEXT>
</DOC>
<DOC>
<DOCNO>FSP950101-060</DOCNO>
<DOCID>FSP950101-060</DOCID>
<DATE>950101</DATE>
<CATEGORY>COTIDIANO</CATEGORY>
<TEXT>
MOACYR SCLIAR 
Em briga de marido e mulher ninguém deve se meter, era o que ele ouvia desde pequeno. Mas era um conselho que não podia seguir.
Em primeiro lugar, porque era policial. E como policial não podia permitir que casais brigassem. Outros guardiães da lei não pensavam assim; que se arrebentem, se matem, diziam, se morrerem não tem importância, são baderneiros a menos para incomodar.
Ele, ao contrário, achava que deveria intervir, chamando as pessoas à razão: não briguem, as coisas se arranjam com bom senso.
Mas não era só isto. Era também o fato de que ele acreditava no casamento. Uma confiança um tanto absurda, reconhecia; afinal, conhecia tantos casais que viviam mal, maridos batendo em mulheres, mulheres vingando-se de maridos. Mas isto, dizia, é porque as pessoas que não sabem mais conviver, esqueceram como é boa a vida em família, o que é o amor.
Quando falava estas coisas, no boteco, todo o mundo ria. Você não passa de um ingênuo, diziam, você não vive neste mundo, você pensa que a vida é novela de tevê, que é filme. E por que não poderia a vida ser como uma novela de tevê, perguntava. Por que não poderiam as histórias ter um final feliz, como os filmes de antigamente?
Era nisto que pensava no dia em que viu o casal brigando. Era uma briga feia, muita gente estava olhando, mas ninguém pensava em se meter –porque evidentemente sobraria para a turma do deixa-disso. Mas ele estava de serviço, e mesmo que não estivesse de serviço, trataria de separar os brigões pela simples razão de que não podia ver um homem e uma mulher se agredindo daquele jeito.
aproximou-se, agarrou-os, com firmeza, mas sem violência. Que é isto, gente, era o que ele ia dizer, não façam uma coisa destas, parem de brigar, este não é um tempo de briga, é um tempo de amor, todos estão confraternizando; vamos lá, abracem-se, beijem-se, é uma coisa tão bonita marido e mulher se beijando, é o que todo o mundo aqui espera de vocês, beijem-se e vocês verão como todos batem palmas, será um final feliz, como o final de novela ou filme.
Isto era o que ele ia dizer, antes que a faca do homem o atingisse na garganta. E as mansas palavras que ele ia dizer não foram pronunciadas. Ele caiu em silêncio e enquanto morria pensava nesta coisa tão bonita, o amor, o amor...
</TEXT>
</DOC>
<DOC>
<DOCNO>FSP950101-061</DOCNO>
<DOCID>FSP950101-061</DOCID>
<DATE>950101</DATE>
<CATEGORY>COTIDIANO</CATEGORY>
<TEXT>
Escolha de presidente sociólogo aumentou procura pelas ciências sociais nos vestibulares depois das eleições 
AURELIANO BIANCARELLI 
Da Reportagem Local 
A chegada de um sociólogo à Presidência da República pode estar trazendo novo alento aos cursos de sociologia no país. O número de inscritos no vestibular de Ciências Sociais da PUC de São Paulo para 95 foi três vezes maior que no ano passado.
Coincidência ou não, as inscrições foram feitas entre 5 e 27 de novembro, um mês depois do primeiro turno que elegeu o sociólogo Fernando Henrique Cardoso. Para as 100 vagas do curso noturno, o número de candidatos da PUC saltou de 120 para 343. Para as 50 vagas do matutino, os candidatos passaram de 75 para 200.
O possível efeito FHC já tinha sido discretamente sentido no curso de Ciências Sociais da Universidade de São Paulo. Ná época das inscrições na USP –11 a 18 de setembro–, Fernando Henrique já aparecia nas pesquisas como candidato com maior chance. O número de candidatos por vaga nas Ciências Sociais foi de 6,74 contra 5,6 no ano anterior.
Na Escola de Sociologia e Política de São Paulo, que já teve o novo presidente entre seus professores e conselheiros, o clima é de expectativa. Lá, as inscrições vão até o dia 13.
No ano passado, a escola cancelou seu vestibular uma semana antes das provas por falta de candidatos. Neste ano, abriu 70 vagas à noite e espera candidatos para um segundo curso pela manhã.
"Na última semana, recebemos mais de 50 telefonemas por dia de interessados no curso", diz Heloisa Pagliaro, professora e vice-diretora da escola.
Bolivar Lamounier, 51, fez o curso de Sociologia e Política na Universidade Federal de Minas Gerais entre 1960 e 1964. Em seguida doutorou-se em Ciências Políticas nos EUA. Nessa época, as ciências sociais estavam em ascensão. Depois os cursos deixaram de atrair os estudantes.
"Pode estar ocorrendo uma volta do interesse pela sociologia, independentemente do presidente", diz Lamounier.
Heloisa Pagliaro também pensa assim. Ela acredita que está havendo um "saturamento das profissões tecnológicas, carecendo das profissões humanas". "Isso nada tem a ver com o Fernando Henrique".
Aos alunos que desanimam, Heloisa costuma dizer: "Vocês vão ver, as ciências sociais ainda vão ter um futuro neste país."
Já tiveram um passado. O prédio da General Jardim, na região central da cidade, onde ainda hoje funciona a Escola de Sociologia e Política, já foi frequentado por professores que vinham das universidades de Harvard, Columbia, Chicago, Berlim.
A escola foi a criada em 1933. Entre seus ex-alunos e professores estão Roberto Simonsen, Florestan Fernandes, Darci Ribeiro, Luiza Erundina.
O perfil dos estudantes mudou, observa Heloisa. Sindicalistas e profissionais já diplomados substituíram as elites que faziam o cursonos nos anos 70.
A escola mantém hoje os cursos de Sociologia, Biblioteconomia, Documentação e Museologia.
</TEXT>
</DOC>
<DOC>
<DOCNO>FSP950101-062</DOCNO>
<DOCID>FSP950101-062</DOCID>
<DATE>950101</DATE>
<CATEGORY>COTIDIANO</CATEGORY>
<TEXT>
Da Reportagem Local 
Os novos candidatos a sociólogo dizem que a intenção de fazer o curso é anterior à ascensão de FHC. Anterior mesmo ao fenômeno Betinho-Herbert de Souza, sociólogo que ganhou notoriedade à frente da campanha contra a fome.
O corretor de seguros Jonas Gonçalves, 32, que se inscreveu na Escola de Sociologia e Política, diz que "foi apenas coincidência". "Eu já pensava nisso há dois anos." Ele acha, no entanto, que um sociólogo na presidência pode "animar o curso e até o mercado".
A gerente de livraria Marli Ribeiro diz que FHC nada teve com sua escolha.
O escriturário José Divino da Silva, 24, diz que o presidente "pode até prejudicar". "É uma ilusão pensar que a sociologia tem um grande futuro. O Fernando Henrique vem de outro berço, estudou na Europa, é outra coisa."
Silva diz que seu interesse pela sociologia surgiu na periferia, onde mora. "Vejo os meninos da rua e as pessoas que bebem. A sociologia vai me ajudar a entender esse mundo." Silva não tem idéia de onde encontrará emprego depois de formado.
(AB)
</TEXT>
</DOC>
<DOC>
<DOCNO>FSP950101-063</DOCNO>
<DOCID>FSP950101-063</DOCID>
<DATE>950101</DATE>
<CATEGORY>COTIDIANO</CATEGORY>
<TEXT>
FERNANDA DA ESCÓSSIA 
Da Sucursal do Rio 
A casa oficial do carioca Fernando Henrique Cardoso no Rio de Janeiro é um palacete que foi palco de alguns dos mais agitados acontecimentos da política brasileira nos últimos 40 anos.
Em busca de prestígio político em um ministério com nove paulistas, o governador eleito do Rio, Marcello Alencar (PSDB), ofereceu a FHC o Palácio Laranjeiras (zona sul), para servir de residência oficial.
O Laranjeiras –onde morou o presidente Juscelino Kubitschek– viveu os últimos dias do Rio como capital federal e viu as mudanças políticas que levaram à instauração do regime militar no país.
Lá o presidente João Goulart recebeu, em 64, a carta assinada pelo general Mourão Filho, informando que os militares queriam sua renúncia.
"Jango já saiu do Laranjeiras praticamente deposto", conta sua mulher, Maria Tereza Goulart.
Quatro anos mais tarde, no salão imperial do Laranjeiras, o general Costa e Silva, então presidente da República, assinou o AI-5, que restringiu liberdades individuais e políticas.
Para o governador eleito do Rio, Marcello Alencar, oferecer o palácio a FHC é uma forma de garantir sua presença no Rio e ajudar a recuperar a imagem e o prestígio político da cidade.
"Queremos que o presidente venha aqui e tenha uma casa carioca, onde ele possa descansar e trabalhar. O Rio é a cidade mais brasileira do país e precisa recuperar seu papel", afirma Marcello.
Juscelino foi o único presidente que passou todo o mandato vivendo no Laranjeiras. Ele não quis morar no Palácio do Catete, onde Getúlio Vargas se suicidara.
Até então, o palacete de Laranjeiras servia para receber hóspedes ilustres do governo federal. Para isso, havia sido comprado pela União à família Guinle, em 1947.
O palácio foi construído em 1913 por Eduardo Guinle. Os móveis e todo o material foram importados da França. O prédio é tombado pelo Estado e pela União.
Em 1976, com a fusão dos Estados do Rio de Janeiro e da Guanabara, a União cedeu o Palácio Laranjeiras ao governo estadual, e o prédio se tornou a residência oficial dos governadores fluminenses.
Faria Lima (75-79) e Chagas Freitas (79-83) moraram no Laranjeiras. Leonel Brizola (83-87 e 91-94) e Nilo Batista (94) só o utilizaram para recepções oficiais.
O último governador a morar no Laranjeiras, Moreira Franco (87-91), lembra-se do palácio como "um lugar muito quente, bonito, e de pouca privacidade".
"Fizemos um jantar para um chanceler de Cuba na varanda, porque o calor dentro de casa era insuportável. Não tinha vento. Caiu uma folha de uma mangueira, e veio descendo, descendo. Não soprava um vento. A folha caiu no prato da mulher do chanceler", conta Moreira Franco.
Para receber o presidente eleito o Palácio Laranjeiras precisará de uma reforma no prédio e de reforço na segurança.
O Gabinete Civil do Estado, responsável pelo prédio, não quis informar o custo mensal da manutenção do palácio.
O governador eleito, Marcello Alencar, fala pouco sobre as reformas no Laranjeiras. Prefere deixar o assunto para depois da posse.
O administrador do palácio, Francisco Farias, anima-se com a idéia de ter um presidente por perto: "Com os salões vazios, é como se o tempo tivesse parado. Será bom ver o palácio mais vivo."
</TEXT>
</DOC>
<DOC>
<DOCNO>FSP950101-064</DOCNO>
<DOCID>FSP950101-064</DOCID>
<DATE>950101</DATE>
<CATEGORY>ESPORTE</CATEGORY>
<TEXT>
Esportes terá R$ 20 milhões para gastar em 1995, menos que o gabinete da Presidência 
MÁRCIA MARQUES 
Da Sucursal de Brasília 
Um orçamento de R$ 20 milhões é o que o futuro ministro extraordinário dos Esportes, Edson Arantes do Nascimento, o Pelé, terá para o próximo ano.
Os recursos, que vêm de parte dos sorteios da Sena, Quina e Loteria Esportiva, são menores até que os R$ 33,99 milhões destinados ao gabinete da Presidência da República em 1995.
Esse orçamento comprova o desprestígio da pasta que Pelé se dispôs a assumir.
Apesar da pompa e circunstância com que o nome de Pelé foi anunciado para o ministério, o futuro ministro terá que usar de seu prestígio junto à iniciativa privada se quiser mais dinheiro para melhorar a imagem do esporte.
O ministro Edson Arantes do Nascimento também terá que se apoiar na fama de Pelé para atingir um objetivo ambicioso: fazer do Brasil a sede da Olimpíada de 2004 e/ou da Copa de 2006.
A atual Secretaria Nacional do Desporto, que será transformada em Ministério Extraordinário dos Esportes, está mal localizada no mapa do poder de Brasília.
Subordinada ao Ministério da Educação, a secretaria divide o terceiro andar do descuidado prédio da FAE (Fundação de Assistência ao Estudante) com o escritório de representação na capital do governo do Tocantins.
Para o secretário Marcos André da Costa Berenguer, ex-subsecretário de Esportes de Pernambuco, a situação agora é boa.
"Ocupávamos cinco lugares diferentes e agora temos os 94 funcionários trabalhando no mesmo lugar", diz Berenguer.
O presidente Fernando Henrique Cardoso estuda, porém, a possibilidade de dar a Pelé um gabinete no Palácio do Planalto.
Segundo Berenguer, a secretaria já está cumprindo, por questão legal, a principal meta que Pelé quer atingir como ministro extraordinário: atender as crianças.
No caso do chamado "desporto educacional", previsto na Lei Zico, a secretaria destinou parte dos R$ 11 milhões do orçamento de 1994 para a organização de competições estudantis e a parceria com municípios na construção de quadras poliesportivas escolares.
Pela estrutura atual, a secretaria recebe, para formar o orçamento, 15% da arrecadação da Loteria Esportiva e 4,5% da Sena e da Quina. Destas últimas, é feito o repasse de 1,5% para o Estado onde o dinheiro foi arrecadado.
Deste total, entre 60% e 70% são destinados aos esportes educacionais. Neste ano, dos R$ 7 milhões destinados ao setor, R$ 1,5 milhão foram para competições, R$ 1,5 milhão para o esporte sócio-cultural e R$ 4 milhões para o educacional.
A secretaria cumpre burocraticamente suas funções. Berenguer diz que não entregará dívidas a seu sucessor, "Edson Arantes do Nascimento", como faz questão de chamar o ministro.
"A secretaria está funcionando bem, enxuta, estruturada e não tem débitos", disse o secretário à Folha na pequena sala do gabinete, onde ficam desarrumados alguns poucos troféus.
Outra parcela dos recursos em esportes tem sido destinada para os deficientes, segundo Berenguer. Nas Para-Olimpíadas, realizadas na Espanha no ano passado, a delegação brasileira tinha 20 atletas.
"Ganhamos três medalhas de ouro e duas de prata", disse Berenguer. A secretaria pagou as despesas da equipe brasileira.
</TEXT>
</DOC>
<DOC>
<DOCNO>FSP950101-065</DOCNO>
<DOCID>FSP950101-065</DOCID>
<DATE>950101</DATE>
<CATEGORY>ESPORTE</CATEGORY>
<TEXT>
Leitor fica indignado com meu artigo sobre a nomeação de Pelé para ministro extraordinário dos Esportes 
ALBERTO HELENA JR. 
Da Equipe de Articulistas 
Há muitos anos, quando pisei pela primeira vez esta redação, conheci um velho e adorável, em usa profunda irreverência, jornalista chamado Aristides Lobo. Era uma lenda viva para os jovens jornalistas daquela época, o limiar dos anos 60.
Filho ou neto, já nem sei mais, do famoso tribuno, já beirava à época os 70 anos. Alto, magro, calva inteiramente raspada, sempre de terno preto, camisa branca e gravata escura, não falava, trovejava. Cultivava as palavras com esmero e paixão.
Os antigos contavam que, jovem, havia sido o maior de todos os secretários de redação, como eram chamados então os editores-chefes de hoje. Socialista, anarquista, na juventude, flertava com o espiritismo no outono da vida. Mas seu maior atributo era poder indignar-se ainda com o cotidiano de seu ofício.
Certa tarde morna e sonolenta, ainda mais com o matraquear distante das máquinas de escrever, que atuavam como um toque hipnótico para mim, a redação toda despertou ao trovejar do velho Aristides. Dedo longo e descarnado em riste, levantou-se de sua mesa e disparou em direção a um atônito Tavares de Miranda, José, o repórter, doce colunista social, no auge de sua celebridade: "Vou-lhe bater na bunda, seu moleque!" Nunca soube a razão do destempero, nem devia ser nada importante. Era apenas a ira do velho Lobo, que caía em nós como uma graça, nos dois sentidos. Sobretudo, porque Tavares de Miranda era quase um seu contemporâneo. Homem feito e viajado.
Lembro-me do velho Lobo, em primeiro lugar, porque dele nunca me esqueci. Depois, porque noutra ocasião, a propósito das seculares cartas dos leitores, ele destilava seu veneno proverbial: "O nosso problema é que o leitor não lê. No máximo, tenta advinhar o que você escreveu. E fica indignado com o que você não disse e louva o que você condenou".
Logo depois, o poetinha Vinicius disse que a vida é a arte do desencontro. E, bem mais tarde, Caetano, inspirado pelos Pignataris e Campos da vida, acrescentou: é o avesso, do avesso, do avesso.
Pois foi o avesso do avesso da arte da vida que encontrei no Painel do Leitor de sexta-feira. Um leitor "indignado" com meu artigo sobre a escolha de Pelé para ministro dos Esportes. E conclui: "Pelé é exemplo de homem, de bons costumes para a juventude, correto nos seus negócios, sem estar comprometido com os partidos". Em pânico, li, reli o tal artigo. Nem uma vírgula sequer sugeria algo contrário.
Outra sentença: "Pelé é cidadão do mundo". Pergunto: quem duvida? Caymmi responde: nem eu, nem ninguém.
Muito a contragosto, sou obrigado a admitir que o velho Lobo tinha razão. Não tivesse ele a sabedoria de um velho e o instinto de um lobo.
 
Vítor no Corinthians? Um presente de Natal tricolor à nação alvinegra, que agradece, mas não dá o troco.

Alberto Helena Jr. escreve aos domingos, segundas e quartas
</TEXT>
</DOC>
<DOC>
<DOCNO>FSP950101-066</DOCNO>
<DOCID>FSP950101-066</DOCID>
<DATE>950101</DATE>
<CATEGORY>ESPORTE</CATEGORY>
<TEXT>
ANDRÉ FONTENELLE
De Paris  
JOSÉ HENRIQUE MARIANTE 
Da Reportagem Local 
Paris-Dacar, Granada-Dacar ou simplesmente Dacar. O maior e mais perigoso rali do mundo faz hoje, no sul da Espanha, sua 17ª largada.
E depois de tantos anos, o chamado "rali da morte" procura resgatar sua principal característica, a aventura.
O francês Hubert Ariol, vencedor do Paris-Dacar duas vezes de moto e organizador da prova, é quem faz o apelo: "Acredito sinceramente que o Dacar ainda faz sonhar. Mas é essencial garantir sua credibilidade esportiva. Acho que, nesse ponto, há um princípio de incompreensão do público".
Ariol se refere à perda de interesse da competição nos últimos anos. Além das mortes, 31 em 17 anos, o uso de recursos como navegação por satélite desequilibrou a disputa.
Para evitar isso, a organização da prova terá uma equipe de reconhecimento. Com antecedência de 48 horas, o grupo fará os trechos repassando as informações colhidas para quem não tiver o auxílio dos computadores e satélites.
Outra medida, foi tornar público o lado "humanitário" da prova.
Segundos os organizadores, 10% do dinheiro arrecadado no rali são repassados para as cidades, em sua maioria vilas pobres, para a construção de poços artesianos.
Outro dado aponta que os médicos das equipes de resgate acabam cuidando mais das populações locais do que dos participantes.
Nada disso, porém, impediu as manifestações de protesto por parte de grupos ecológicos e de defesa dos direitos humanos.
Para amenizar a propaganda negativa, um esforço de marketing foi feito por toda a Europa. Além de em Paris, ocorreram largadas simbólicas em Bruxelas, Milão e Barcelona.
Nesta edição se repetirão os duelos de 94. Quatro Citroen ZX e três Mitsubishi Pajero deverão lutar pelas primeiras posições entre os carros.
O fabricante francês contará com os finlandeses Ari Vatanen (vencedor em 87, 89, 90 e 91) e Timo Salonen, o espanhol Salvador Servia e o atual detentor do título, o francês Pierre Lartigue.
Já os franceses Bruno Saby (vencedor de 93) e Jean-Pierre Fontenay, além do japonês Kenjiro Shinozuka, contarão com um novo motor de 350 hp para dar o título à Mitsubishi.
Em duas rodas, os favoritos são o francês Stephane Peterhansel (Yamaha) e o italiano Edi Orioli (Cagiva), reeditando a disputa do ano passado vencida pelo segundo.
Pela oitava vez, o Brasil será representado pela dupla Kléver Kolberg e André Azevedo (leia texto abaixo), que disputará o título da categoria maratona, disputada por motos de série.
O percurso foi bastante alterado. Serão 10.109 km, cruzando o Marrocos, o Saara Ocidental, a Mauritânia e a Guiné, finalizando no dia 15 de janeiro em Dacar, no Senegal.
Do total, 6.138 km serão trechos especiais. O restante, 3.971 km, trechos de ligação.
O dia 8 de janeiro, conforme nova determinação da Federação Internacional de Automobilismo, será para descanso em Zouerat, na Mauritânia.
Além dos perigos naturais do deserto, problemas de ordem política podem surgir quando a caravana atravessar o Saara Ocidental, região em litígio disputada pelo Marrocos e a Frente Polisário.
Os organizadores já foram avisados que guerrilheiros podem interferir na prova em busca de espaço na mídia internacional.
</TEXT>
</DOC>
<DOC>
<DOCNO>FSP950101-067</DOCNO>
<DOCID>FSP950101-067</DOCID>
<DATE>950101</DATE>
<CATEGORY>MAIS!</CATEGORY>
<TEXT>
A editora Morrow acaba de publicar "Deadly Sins", volume com oito ensaios sobre os sete pecados capitais escritos por Thomas Pynchon, Mary Gordon, John Updike, William Trevor, Gore Vidal, Richard Howard, A. S. Byatt e Joyce Carol Oarwa. Sete desses oito artigos sobre as mais variadas debilidades morais humanas já haviam sido publicados pelo "The New York Times Book Review".

Ator Tom Hanks é eleito artista do ano 
O ator Tom Hanks ("Filadélfia" e "Forrest Gump - O Contador de Histórias") foi eleito esta semana o Artista do Ano pela revista semanal norte-americana "Entertainment Weekly". A revista divulgou ainda como melhores do ano o diretor Quentin Tarantino ("Cães de Aluguel" e "Pulp Fiction"), o comediante Jim Carrey ("O Máskara"), o escritor Michael Crichton, o grupo musical Boyz 2 Men e a atriz Heather Locklear.

Arqueólogo amador descobre tesouro viking 
Um arqueólogo amador dinamarquês encontrou em Ramloese (norte de Copenhague), com o auxílio de um detector de metais, uma arca viking de mil anos com cerca de 450 moedas, braceletes e barras de ouro e prata. O tesouro viking é o quarto maior já descoberto na Dinamarca.

Kate Moss estréia em abril no mercado editorial 
A top model Kate Moss vai seguir os passos de suas companheiras Naomi Campbell e Claudia Schiffer e também estreiar no mercado editorial. Segundo a editora Pavillion, "Kate" trará basicamente fotos da modelo de Calvin Klein e um texto de introdução escrito pela própria Moss. O volume sai em abril.
</TEXT>
</DOC>
<DOC>
<DOCNO>FSP950101-068</DOCNO>
<DOCID>FSP950101-068</DOCID>
<DATE>950101</DATE>
<CATEGORY>MAIS!</CATEGORY>
<TEXT>
Da Redação 
1995 é o ano Erico Verissimo. Duas datas –os 90 anos de seu nascimento e os 20 de sua morte– motivam uma série de eventos em sua homenagem, a partir do Rio Grande do Sul, estado onde nasceu o autor de "Incidente em Antares".
O principal deles ocorrerá em setembro, com a edição, pela primeira vez no Brasil, de "Brazilian Literature: An Outline" (Literatura Brasileira: Um Esboço), publicado nos Estados Unidos em 1945. O livro traz as aulas que Verissimo deu nos EUA.
A Folha antecipa-se às comemorações e publica com exclusividade no "Mais!" cartas recebidas por Verissimo dos escritores Monteiro Lobato, Lucio Cardoso, Guimarães Rosa, Jorge de Lima e John dos Passos.
As cartas fazem parte do Acervo Literário Erico Verissimo –coordenado pela professora Maria da Glória Bordini–, um conjunto de cerca de 10 mil itens, com grande número de documentos inéditos, inclusive os cadernos de notas de um romance nunca publicado, "A Hora do Sétimo Anjo". Doze pesquisadores atuam no trabalho de catalogação, que está ligado ao curso de pós-graduação em letras da Pontifícia Universidade Católica do Rio Grande do Sul (PUCRS), sob a responsabilidade da professora Regina Zilberman.
A edição das cartas ainda não está nos projetos do acervo, que fica na própria casa de Erico Verissimo em Porto Alegre, onde mora até hoje sua mulher, Mafalda.
As homenagens ao escritor em Porto Alegre começam em abril, com o lançamento da coletânea de ensaios "Criação Literária em Erico Verissimo". Em novembro, será realizado um seminário internacional na PUCRS sobre a obra do autor. Exposições fotográficas e de originais estão sendo organizadas. Estava previsto para este mês o lançamento na Coréia do Sul do primeiro volume de "O Tempo e o Vento".
Paralelamente às homenagens a Erico Verissimo, acontece a comemoração dos 60 anos da imigração para Porto Alegre do intelectual alemão Herbert Caro, um dos maiores amigos do escritor. Crítico de literatura, música e artes plásticas, foi também o tradutor para o português de Thomas Mann, Herman Broch e Herman Hesse. De seu acervo, a Folha publica duas cartas inéditas enviadas pelo escritor Elias Canetti, de quem Caro traduziu "Auto da Fé".
A comemoração permite reavaliar a contribuição de Caro, figura menos conhecida embora fundamental, do grupo de intelectuais europeus imigrados para o Brasil às vésperas da Segunda Guerra, composto por nomes como Otto Maria Carpeaux, Paulo Rónai e Anatol Rosenfeld.
As homenagens a Erico Verissimo propiciam a rediscussão de seu lugar na literatura brasileira. Escritor de enorme alcance popular –"Olhai os Lírios do Campo" vendeu 62.000 exemplares na época de seu lançamento, em 1938–, Verissimo soube conciliar o sucesso com a erudição, a pesquisa literária e a responsabilidade ética, realizando uma das mais completas obras da ficção brasileira deste século.
</TEXT>
</DOC>
<DOC>
<DOCNO>FSP950101-069</DOCNO>
<DOCID>FSP950101-069</DOCID>
<DATE>950101</DATE>
<CATEGORY>MAIS!</CATEGORY>
<TEXT>
S. Paulo, 15,1,942
Verissimo:
Venho agradecer o último livro. Que presente de fim de ano você fez ao país! Mas acho que só quem morou anos na América pode dar o devido valor a tão linda realização. Eu considero os Estados Unidos como uma dessas famosas composições musicais que são impostas a todos os grandes executantes afim de tirar a prova dos noves fora do seu valor real, a rapsódia húngara de Lizt (sic), certas fugas de Bach. Se o bicho não revela excepcional habilidade, naufraga. O grande tema do mundo hoje, para todos os artistas, é a Rapsódia Americana, dos grandes gênios da música política, George Washington e Abe Lincoln. E você, meu arquiamado Verissimo, a executou como ninguém.
Conheço muitos livros de impressão sobre a América, mas nenhum vale o "Gato Preto em Campo de Neve". Li-o a fundo, meditadamente, e assombrei-me da perícia com que você executou as passagens difíceis, as em que os falsos talentos inevitavelmente naufragam, arrastados pela humaníssima inferioridade do afirmar ou negar. Você não abandona jamais o terreno seguro do "procurar compreender".
E que estilo, Verissimo! A língua te obedece como massa de pão entre os dedos do bom padeiro. Absolutamente limpo de estáticas –maneirismos, exibicionismos, flores de cera, besteiras. Escrever bem é isso, Verissimo –é escrever como você escreve– organicamente, com o correntio duma função natural da nossa fisiologia. Escrever bem é mijar. É deixar que o pensamento flua com o à vontade da mijada feliz.
E pelo livro inteiro você vai revelando o maravilhoso romancista nato que é –o olho a que nada escapa, o captador do detalhe que revela tudo, o CRIADOR. Aquele capítulo do TOMMY é página de antologia, das que comovem. E revela o teu espírito criador. Foi uma cena arrancada à Vida pelo teu gênio. Você a provocou como aquele "Hey, boy!" donde tudo mais saiu. Foi pena não ter fixado a resposta final do menino da maneira exata com que ele a disse. "Gee, sir. Thanks a lot. It's swell. I'll tell father." Você revela-se ali como em parte nenhuma da tua obra –e revela a essência do americano "in the making".
Outra habilidade que me encanta é a finura com que você escapa à trágica situação de todas as criaturas agarradas pelo monstruoso percevejo do Lugar-Comum. O visitante tem que ouvir de todos os lados as mesmas perguntas obrigatórias da amabilidade, e cai no ridículo se dá respostas sérias, isto é, se responde a um lugar-comum da amabilidade com um lugar-comum do nosso ingênuo patriotismo. Você atende a essas situações com as piruetas do espírito. São respostas que correspondem a pequenas criações mais reveladoras de talento que qualquer outra coisa.
O livro inteiro é um prodígio de habilidade, leveza e penetração. Ainda não o li todo. Estou no meio. Só tenho pressa em concluir uma leitura quando ela não me satisfaz integralmente; quando antes pulo fora daquilo, melhor. Mas às coisas realmente boas –e como são raras!– leio-as como Narizinho come cocadas: pinicadamente, poupadamente, para prolongar o prazer. Hoje parei no FILHO NATURAL.
Aquela International House é um encanto e a mais alta realização social do mundo. É a amostra do mundo de amanhã se em vez de vencer Hitler, vence a idéia dos Estados Unidos do Mundo. Estive lá numa roda de rapazes e girls de 16 nacionalidades e notei que a tão mal compreendida estandartização americana os transformara de 16 "hostilidades sociais ou raciais" em 16 irmãos –16 cidadãos dos Estados Unidos do Mundo. Mas você foi mais feliz do que eu, porque viveu naquela célula do futuro vários dias e eu só horas.
Suponho que teu livro vai aparecer em inglês e espanhol. Nenhum merece mais essa projeção –e fará o regalo dos americanos que o lerem e ensinará muito hispânico a compreender os Estados Unidos –mero nome político do que no plano do tempo chamamos AMANHÃ.
Adeus, meu caro Verissimo. Eu queria escrever esta carta depois de finda a leitura. Mas o capítulo do TOMMY, lido ontem à noite, não deixou. Vai pois antecipada.
Um grande abraço e o meu absoluto reconhecimento de que você é o grande escritor que temos hoje, o abafador de todas as bancas.
Do teu velho admirador
Monteiro Lobato
</TEXT>
</DOC>
<DOC>
<DOCNO>FSP950101-070</DOCNO>
<DOCID>FSP950101-070</DOCID>
<DATE>950101</DATE>
<CATEGORY>MAIS!</CATEGORY>
<TEXT>

Rio, 27.9.37
Prezado Erico,
Aí vai finalmente o "Felicité" (1).
Já tinha recebido a sua carta, mas estive tão doente que não me animei a responder, caracterizando-se a minha moléstia por um absoluto horror à pena, lápis e papel.
Achei interessante o que você diz a respeito de Octavio de Faria. Quem de nós sabe como é realmente a vida? Para cada pessoa ela é diferente –e a estes, a quem Deus dotou de uma sensibilidade mais aguda, mais do que aos outros, é através dela, pelos caminhos sangrentos das experiências que podemos fixar essa coisa sem nome que é uma constante tragédia: a vida. Cada um vê de um modo, tanto tem razão Octavio de Faria quanto Graciliano Ramos, que entre parênteses, a encara de um modo não menos dramático.
E se pudéssemos invalidar certos autores porque a vida que nos apresentam não é a que julgamos certa, não existiria senão um número bem restrito a quem concederíamos o passe de salvação... No final de tudo, só restariam aqueles a quem a vida tivesse dotado de experiências idênticas às nossas...
Acho que não é lícito julgar assim, pois se a concepção de um trágico grego difere da de Pirandello, se um Shakespeare não pode ser anulado a favor de uma Rosamond Lehmann ou um Dostoiévski a favor de um Baring, não quer isto dizer que um ou outro nos mostre a vida falsa, mas assim como os pintores vêem de um modo diferente (a visão que um Degas tem das coisas está longe da de um Van Gogh...) assim também esses outros criadores não podem ver e sentir senão de acordo com a natureza que Deus lhes deu...
O que não é a vida, na minha opinião é a miséria que Jorge Amado acaba de publicar como romance. Pois se enquanto num Octavio de Faria ou num Graciliano Ramos sentimos bem as "pegadas" do autor, em "Capitães de Areia" não encontramos uma só parcela do que é o verdadeiro Jorge Amado, a não ser uma tentativa ingênua de realidade fabricada, essa péssima realidade que tem envenenado quase toda a nossa literatura nestes últimos anos.
Não posso explicar por uma simples carta todo o meu horror diante dessa monstruosidade que os nossos críticos têm considerado "grande romance" –mas posso afirmar sob a minha palavra que me sinto feliz em ser o último dos romancistas numa terra onde semelhante crápula é tido como escritor de grande brilho.
Pouco ou quase nada poderei lhe dizer a meu respeito, pois não tenho conseguido escrever uma só linha. Vinha tentando um romance há alguns meses, mas fui obrigado a parar, pois o mesmo não me agradava de forma alguma. Estava excessivamente "intelectual", antipático, pretensioso e besta.
Espero pacientemente passar esta crise, ouvindo música e tomando banhos de mar, o que tem me auxiliado a curar as chagas abertas pelas últimas brigas –que talvez você já saiba por aí, foram ardentes e definitivas.
Tenho o prazer de comunicar-lhe que, exceto dois ou três, já não tenho ligação com o meio literário brasileiro. Sinto que os meus nervos estão se aquietando e que uma nova vida começa para mim...
Escreva e dê notícias mais compridas sobre seus trabalhos; suas cartas são sempre pequenas e não dizem nada, para quem, como eu, tem sempre tão grande interesse por tudo quanto sai das suas mãos.
Agradeço-lhe sinceramente o que fez pelo meu livro e espero que consiga resolver as dúvidas, com o Katherine Mansfield que segue. Abraços do
Lucio Cardoso

(1) Livro da escritora neozelandesa Katherine Mansfield (1888-1923), "Bliss" no original. Verissimo, que o traduziu para o português, cotejou a tradução francesa emprestada por Lucio Cardoso
</TEXT>
</DOC>
<DOC>
<DOCNO>FSP950101-071</DOCNO>
<DOCID>FSP950101-071</DOCID>
<DATE>950101</DATE>
<CATEGORY>MAIS!</CATEGORY>
<TEXT>
Rio, 11 de agosto de 1964
Meu caro Erico Verissimo:
Por motivo de saúde (falta de), de ausência daqui, e porque estive semanas sem vir à Academia, só hoje foi que me alegrou a surpresa de receber sua carta de 28 de junho. De verdade. Pois, veja, há também equívocos que valem a pena. Este, por exemplo, que você explica e desfaz, do "Correio da Manhã", do Mello Alvarenga, do nosso "Grande Sertão: Veredas". Sem ele, eu não teria agora o prêmio de suas linhas, tão cordiais e simpáticas. Muito, muito obrigado.
Eu não lera, aliás, a nota no jornal, nem a notinha na "Manchete". Mas, você sabe, essas coisas de imprensa, do apressado, imperfeito e efêmero, não têm nenhuma importância. Importa, e, para mim, vivamente, a limpidez de seu fraterno gesto, ao escrever-me, a alta elegância de atitude, a generosidade de sua carta. Leio-a, a fundo.
Sempre é um primeiro contato direto, que também sinceramente tenho desejado, de há muito. Houve e há boas fadas, em volta de nós, aproximando-nos. Desde as longas vezes em que o nosso querido e saudoso amigo João Neves da Fontoura comigo e afetuosamente falava de você, com viva presença –do escritor e do homem. Depois, o caríssimo Moog. E, não menos, o bom Augusto Graeff, de Carazinho, jovem confrade e colega, cheio de gaúcha alegria e seguro talento, que o tem em elevada estima e a quem quero muito bem.
Assim, com prévia lembrança e espírito de irmão, é que espero, para prazer, o dia de nosso próximo encontro, em que de viva voz conversaremos de coisas antigas e assistiremos aos nossos anjos da guarda combinarem, isto é, confirmarem o combinado. Seus belos livros sempre estiveram comigo.
Até lá, com reta e inteira admiração e aumentada simpatia e estima, o forte, muito cordial abraço do seu
Guimarães Rosa
</TEXT>
</DOC>
<DOC>
<DOCNO>FSP950101-072</DOCNO>
<DOCID>FSP950101-072</DOCID>
<DATE>950101</DATE>
<CATEGORY>MAIS!</CATEGORY>
<TEXT>
Rio, 23 de julho de 1935
Meu caro Erico Verissimo,
Recebi ontem "Caminhos Cruzados" e nesses dias vou começar a ler o seu livro que a crítica vem elogiando. Muito agradeço. Não lhe posso agora mandar, como desejava, o meu "Calunga" pois a livraria editora enviou para o Rio uma pequena remessa já esgotada. Desta remessa, o meu amigo Estevão Cruz me deu dez exemplares apenas. É possível que ainda esta semana tenha alguns exemplares para distribuição: então lhe remeterei um. Estão dadas as desculpas. Agora lhe peço um obséquio: fazer com que o nosso comum amigo Bertaso (1) não deixe como atualmente a praça do Rio sem meus livros. Já se foi também a remessa de "Tempo e Eternidade" com a mesma rapidez da de "Calunga"!
Outra parte do obséquio: substituir em segunda edição de "Anjo" aquela biografia que está nas pestanas de "Calunga" e de "Tempo e Eternidade", mesmo porque há ali obras que eu não escrevi. Poderei mandar uma biografia, se quiserem. Segunda parte do obséquio: quero fazer a revisão de "Anjo". Recebi ontem carta de Mário de Andrade dizendo que em São Paulo não há meus livros.
Talvez lá já estejam, mas não expostos na vitrine. Que tal o agente de São Paulo? A livraria editora poderia ter mandado fazer cartazes. Asseguro-lhe que os nossos livros quase não figuram nas vitrines do Rio. Estão escondidos.
Os leitores procuram os nossos livros somente por via da crítica. Rio de Janeiro compra muito livro meu. E estou achando fraca a saída de "Calunga" apesar da boa imprensa que tem tido. Agora então com a falta nas livrarias vai ser muito pior, pois o camarada procurando o livro e não o encontrando pede emprestado e não gasta logicamente as peles.
Espero resposta sua a esse meu pedido complicado. Jorge Amado me falou bem de sua obra. Vou ler nesses dias com bastante amor. São Paulo, Belo Horizonte, Recife, Rio, Maceió, Bahia e Fortaleza são as praças que mais compram meus livros. É bom dizer ao Bertaso. Aceite um abraço amigo do:
Jorge de Lima

(1) Henrique Bertaso, editor-proprietário da Livraria do Globo, em Porto Alegre 
</TEXT>
</DOC>
<DOC>
<DOCNO>FSP950101-073</DOCNO>
<DOCID>FSP950101-073</DOCID>
<DATE>950101</DATE>
<CATEGORY>MAIS!</CATEGORY>
<TEXT>
Monteiro Lobato, José Bento (1882-1948) - Escritor paulista, autor de "Urupês" (1918), entre outros. Foi o criador da literatura infantil brasileira. Entre 1921 e 1931, viveu nos EUA, tornando-se um apaixonado "americanófilo". A estadia no país lhe rendeu o livro "América".
</TEXT>
</DOC>
<DOC>
<DOCNO>FSP950101-074</DOCNO>
<DOCID>FSP950101-074</DOCID>
<DATE>950101</DATE>
<CATEGORY>MAIS!</CATEGORY>
<TEXT>
 Uma das mais extensas correspondência dos arquivos de Erico Verissimo é a do escritor norte-americano John dos Passos (1896- 1970). Verissimo e Dos Passos se conheceram nos anos 40, quando o escritor gaúcho foi dar aulas nos EUA. Um dos mais destacados escritores dos EUA na primeira metade do século, Dos Passos tinha como tema a degeneração do  sonho americano. É autor de  1919 (1932). Descendente de portugueses, publicou em 69  The Portugal Story (A História de Portugal).
</TEXT>
</DOC>
<DOC>
<DOCNO>FSP950101-075</DOCNO>
<DOCID>FSP950101-075</DOCID>
<DATE>950101</DATE>
<CATEGORY>MAIS!</CATEGORY>
<TEXT>
Spence's Point
Westmoreland, Virginia
9 de julho de 1959
Amigo Verissimo,
Estamos todos bem e felizes em tê-los quase como vizinhos. Vivemos mais ou menos cem milhas a sudeste de vocês. Estaremos aqui até o fim de julho e, depois, a partir de 1º de setembro. Ficaríamos felizes se você e sua mulher viessem passar uma noite conosco. Como gaúcho, você gostaria, penso eu, da região extremamente rural em que vivemos, conhecida como o Northern Neck da Virgínia.
Você tem carro? Se tiver, pode vir quando quiser, passando por Fredericksburg e Montross. Se não, eu poderia apanhá-los em Arlington numa ocasião qualquer. Se preferir vir no outono, podemos combinar uma data assim que nos encontrarmos num almoço na casa do Dr. Mariz. No momento, estamos sofrendo uma das secas mais desastrosas da história, a safra de milho está quase perdida, estamos alimentando o gado com palha –coisa inaudita em julho.
Lucy está nadando e andando a cavalo num acampamento perto de Baltimore, mas iremos buscá-la no fim do mês. Como sempre, estou tentando ficar em dia com o trabalho, que já vai atrasado. Desta vez é algo entre um romance e um documentário, no estilo de certas partes de "USA".
Avise-nos quando quiser vir.
Telefone: Greenwood 2-2674 em Hague, Virginia (via Warsaw, Va)
Cordialmente
John Dos Passos
</TEXT>
</DOC>
<DOC>
<DOCNO>FSP950101-076</DOCNO>
<DOCID>FSP950101-076</DOCID>
<DATE>950101</DATE>
<CATEGORY>MAIS!</CATEGORY>
<TEXT>
Westmoreland, Va 22577
22 de dezembro de 1967
dia mais curto do ano
Querido Erico,
Ficamos todos felizes em receber sua carta, que deve ter vindo por navio a vela, já que levou mais de um mês para chegar aqui. Estão todos aqui para o Natal. Lucy voltou do Occidental College, na Califórnia, onde é caloura. Kif acabou de chegar de Washington, onde ainda trabalha para o  Washington Post. Todos parecem em boa forma. Suponho que você não tenha recebido os cartões-postais: estivemos em Portugal por cinco semanas em julho e agosto.
Ainda estou trabalhando num livrinho chamado –talvez–  A Empresa das Índias, que, num momento de loucura, prometi escrever à editora Doubleday. De qualquer modo, isto me fez ler montes de crônicas dos séculos 15 e 16, Fernão Lopes, Rui de Pina, Damião de Góis, etc, que acho deliciosas. Um dos fins da viagem a Portugal era o de comprar edições difíceis de achar aqui (1)
Depois, em novembro, Betty e eu fomos a Roma, parando em Madri para ver as maravilhosas tapeçarias portuguesas em Pastrana (feitas em Tournai, mas concebidas, possivelmente por Nuno Gonçalves, na época de Afonso 5º), tudo por conta do negócio com Feltrinelli.
O cenário era magnífico, recepção na Vila Farnesina, apresentação no Palácio Corsini, guardas presidenciais com penachos de crina de cavalo, tive o privilégio único de pôr um cardeal para dormir. Um camarada mais velho na primeira fila, com uma barba saída de Tintoretto, dormiu feito um bebê durante todo o meu pequeno discurso... foi um alívio voltar para o Northern Neck, onde as pessoas estão ocupadas com as ostras, a madeira e os preços lamentáveis da safra de milho.
Fico feliz em saber que estarão por aqui no próximo verão. Temos que encontrar novos lugares para fazer piqueniques. Com certeza estaremos aqui durante junho, ainda que talvez viajemos para o fim do verão.
Todos os tipos de "abraços" e votos de feliz Ano Novo de Betty, Lucy, Kiffy (2) e
Dos
Anticlímax: os 20 milhões de liras ainda não chegaram ao meu banco. Você acha que tudo não passou de uma peça?

(1) O livro acabou se tornando "The Portugal Story", publicado em 1969
(2) Mulher, filha e filho de John dos Passos
</TEXT>
</DOC>
<DOC>
<DOCNO>FSP950101-077</DOCNO>
<DOCID>FSP950101-077</DOCID>
<DATE>950101</DATE>
<CATEGORY>MAIS!</CATEGORY>
<TEXT>
Westmoreland, Virginia, 22577, USA
16 de janeiro de 1967
Querido Érico,
Fico feliz em receber notícias suas. Eu lhe escrevi por volta do fim do outono e em novembro mandei um exemplar de  The Best Times. Avise-me se devo mandar outro exemplar através de Doug Elleby, que, acho eu, está em Porto Alegre. Sua carta de 1/8 acabou de chegar, e por ela deduzo que o livro não apareceu por aí. É melhor você me escrever um bilhete com o endereço de Elleby. Eu mandarei um exemplar por via aérea através do número APO dele.
Sentimos saudades de vocês dois. Fico feliz em saber que seu coração está se comportando bem outra vez. Por agora, estou ocupado com uma espécie de continuação para  Midcentury, que por enquanto atende simplesmente por  13ª Crônica. Há também um livrinho sobre o lugar de Portugal no mundo, para uma série da Doubleday.
Lucy vai bem. Foi aceita numa pequena faculdade de humanidades, Occidental College, na costa perto de Los Angeles. A idéia foi dela. Kiffy está trabalhando no departamento de pesquisa do  Washington Post. Não é bem pago, mas é o primeiro emprego de que ele de fato gostou. Betty e eu continuamos bem de saúde.
Escreva-me dizendo se o livro chegou. Fiz-lhe a mesma pergunta na última carta, que com certeza você não recebeu.
Cumprimentos carinhosos de Ano Novo de todos nós para você e para Mafalda.
Sempre, seu
Dos
</TEXT>
</DOC>
<DOC>
<DOCNO>FSP950101-078</DOCNO>
<DOCID>FSP950101-078</DOCID>
<DATE>950101</DATE>
<CATEGORY>MAIS!</CATEGORY>
<TEXT>
Westmoreland, Virginia
22577 USA
19 de maio de 1970
Querido Érico,
Feliz com seu bilhete de 12 de maio, que chegou ontem. Os correios brasileiros parecem estar melhorando. Eu até consegui, depois de dois fracassos, mandar um livro para um primo distante em São Paulo.
A "nobre víscera" está se comportando bastante bem. Faço longas caminhadas no fim da tarde e trabalho um tantinho a mais no jardim. É claro que isto jamais interferiu com a minha labuta na escrivaninha. Gosto de imaginá-lo saltando despreocupadamente de um romance para uma novela etc etc. Comecei esta série de narrativas provavelmente há dez anos atrás e ainda estou ocupado com ela.
Nesse meio tempo, escrevi um pequeno relato de nossa viagem à Ilha de Páscoa, que a Doubleday vai publicar no próximo inverno como "livro de luxo" com 60 reproduções fotográficas.  Se Deus quiser, terei "A Crônica da Aflição" –não é este o título, mas é assim que ela me parece agora– pronta pelo outono, a tempo para uma curta viagem à Iugoslávia. Tenho por lá alguns dólares de umas traduções, que têm que ser gastos no país.
Lucy vai bem, bastante ocupada com um curso de botânica, como preparação para paisagismo. Ofereceram-lhe um bom emprego: está classificando plantas, ou melhor, ajudando a compilar um novo dicionário de plantas cultiváveis que estão preparando para publicação, como o Arnold Arboretum do Departamento de Botânica de Harvard; assim, teremos que ir a Boston para poder vê-la neste verão. Por isso, alugamos uma casinha perto de Wiscasset, em Maine, para onde esperamos atraí-la nos fins-de-semana em agosto. Teremos mais quartos do que precisamos.
Talvez pudéssemos atrair você e Mafalda para lá, mas antes disso vocês devem vir a Spence's Point e testar meu pequeno funicular. Kiffy está fazendo suas provas de segundo ano na faculdade de Direito da Universidade de George Washington em Washington DC, não sei se encarando os distúrbios à luz correta. Ele diz que a maior parte dos arruaceiros é profissional, e certamente não de estudantes.
Parte da raiz de muitos de nossos problemas está na organização comunista de subversão com centro em Praga; outra parte dos amplos recursos que estes vândalos têm à sua disposição vem de Havana e de alguns patetas de organizações como a Fundação Ford. O pobre Henry Ford deve estar se contorcendo na tumba. Betty recuperou-se das aflições do inverno passado e em minha opinião está mais bonita que nunca. Ela se junta à minha saudação para Mafalda. Mal podemos esperar para tirá-los do ônibus em Montross e cumprimentá-los em Spence's Point.
Abraços de nós dois para vocês dois.
Sempre, seu 
Dos
</TEXT>
</DOC>
<DOC>
<DOCNO>FSP950101-079</DOCNO>
<DOCID>FSP950101-079</DOCID>
<DATE>950101</DATE>
<CATEGORY>MAIS!</CATEGORY>
<TEXT>
Westmoreland, Va 22877
(na verdade ainda estou em Baltimore e não retornarei à Virginia antes da volta de Lucy em 10 de dezembro)
Querido Érico,
Acabei de emergir de um período de cama no Johns Hopkins Hospital. Meus excelentes médicos saturaram-me com diuréticos a tal ponto que perdi algumas libras em um dia. Tudo isto me deixou horrivelmente fraco, mas acho que estou recobrando as forças bem rapidamente. Agora que estou de volta a nosso pequeno apartamento, sinto-me outra vez humano... Um dos grandes prazeres aqui é o de termos uma sacada dando para as árvores e a grama verdes –além de termos um toca-discos.
É muito agradável ver a foto com você e Mafalda sentados em seu estúdio, ouvindo Mahler (quem dera pudéssemos estar aí). Minha dificuldade para terminar a crônica em que estou trabalhando é muito parecida com a sua, como reunir forças para lidar com a imundície que vem subindo das profundezas de nossa sociedade.
Muitas vezes me pergunto se vale a pena o esforço de viver nesta estranha época de violência.
Espero chegar gradualmente a um ponto que me permita viajar outra vez. Faremos uma espécie de viagem "convalescente" a Tucson, Arizona, pelo meio de janeiro.
Que datilografia horrorosa! Eu mal dava por mim mesmo quando saí do Johns Hopkins, mas pareço estar voltando aos sentidos dia a dia.
Concordo inteiramente com você quanto ao retrato que Baker fez de Hemingway. Ele o congelou. O verdadeiro Hemingway era caloroso, ativo e cheio de vida, gostássemos dele ou não. É impressionante o que os acadêmicos podem fazer de suas vítimas.
 Saudades, felicitações de Natal, e muitos abraços de toda a família. Seu amigo,
Dos
</TEXT>
</DOC>
<DOC>
<DOCNO>FSP950101-080</DOCNO>
<DOCID>FSP950101-080</DOCID>
<DATE>950101</DATE>
<CATEGORY>MAIS!</CATEGORY>
<TEXT>
Siga o caminho do escritor nas versões de um trecho de 'O Prisioneiro' 
É evidente a preocupação do escritor em alinhar forma e narrativa 
MÁRCIA IVANA DE LIMA E SILVA 
Especial para a Folha 
O Prisioneiro, de Erico Verissimo, teve sua primeira publicação em 1967 pela editora Globo, dando sequência à série de romances políticos do escritor gaúcho, iniciada com  O Senhor Embaixador, de 1965, e concluída com  Incidente em Antares, de 1971. A ação se passa em data e local indefinidos, mas tem por base a intervenção americana no Vietnã, conforme entrevistas e depoimentos do romancista. O ponto central da história é o dilema ético e moral de um tenente que questiona seu poder de decidir sobre a vida e a morte de seus semelhantes. Além disso, há a denúncia da crueldade da guerra, que brutaliza o ser humano, reduzindo-o a seu instinto mais primitivo.
Estudar esse manuscrito de Erico Verissimo significa penetrar em seu gabinete particular, com o intuito de desvendar seu processo de criação. E tentar explicar por que o autor desiste de determinada forma, optando por outra, sempre em busca do texto perfeito. Para chegar a ele, no entanto, são necessárias várias etapas, como apontam os esboços e os originais que se encontram no Acervo Literário de Erico Verissimo. O romance  O Prisioneiro apresenta, apenas de seu primeiro parágrafo, seis versões que passaremos a acompanhar.
A primeira versão de  O Prisioneiro está numa folha de ofício manuscrita a tinta azul e começa com:  As monções de inverno haviam passado, entrara a estação seca, e naquele entardecer de junho a abóbada celeste era como uma descomunal ventosa emborcada sobre a antiga cidade imperial que, estendida à beira do rio, parecia arquejar/sufocar, à míngua de ar/oxigênio. Na mesma folha, Verissimo passa um traço horizontal e escreve "variante", substituindo  naquele entardecer de junho por  no entardecer daquele dia.
Mais adiante, coloca como alternativa para  estendida,  esparramada, acrescentando  de ambas as margens do entre  beira e  do rio, e decide-se por  sufocar e  oxigênio. Essa pode ser considerada a segunda versão, na medida em que apresenta algumas modificações significativas que, inclusive, irão se manter no texto final.
A terceira versão é escrita a caneta azul numa folha de ofício e diz:  Haviam passado as monções de inverno, entrara a estação seca –e naquele entardecer mormacento a abóbada celeste era como uma enorme ventosa emborcada sobre a velha cidade imperial que dava a impressão de sufocar, ofegante e intumescida, à míngua de oxigênio. Nesse reinício, além de inverter a ordem sintática, a qual será mantida até a forma final, o romancista substitui  de junho por  mormacento,  descomunal por  enorme e  antiga por  velha.
Com tais modificações, principalmente pensando em  ofegante e  intumescida, Erico parece estar buscando uma descrição sensorial para a cidade, ligada às altas temperaturas da região em que ela se localiza. Além disso, há um enxugamento do texto, concentrando a idéia de calor sufocante numa só frase.
Em nova folha de ofício, Verissimo datilografa esse mesmo texto, mas altera a mão, com caneta esferográfica azul, toda a parte inicial até o hífen, optando por  Sopravam já as monções de sudoeste, e ainda substituindo  mormacento por  de junho.
Acima de  oxigênio, Erico escreve "How if the wind was blowing?", como sugestão para acrescentar a idéia de vento soprando. (Saliente-se que é muito comum encontrar nos manuscritos de Verissimo palavras, frases, trechos ou até mesmo cenas inteiras em inglês). É muito provável que Erico tenha feito a rasura no topo da folha após sugerir a inclusão do vento na descrição, pensando que esta ganharia em força, se apresentasse tal sugestão desde o início.
É atrás dessa mesma folha, porém, que o autor rabisca aquela que será a versão mais próxima da final:  Haviam já começado as monções de sudoeste, mas naquele entardecer/mormacento/ de junho os ventos cessaram/o ar estava imóvel de repente como se a abóbada celeste emborcada sobre a cidade lhe/aquele trecho de terra e mar/ tivesse sugado/chupado todo o ar e agora ali a velha capital de província/cidade imperial estivesse a sufocar –ofegante e intumescida, à míngua de oxigênio. Volta a dúvida sobre a caracterização do local, além da opção por manter mormacento e de junho juntos, mas parece, pela primeira vez, a idéia de calmaria, contrastando com a movimentação do vento, sugerida na versão anterior. Além disso, a utilização de vários adjetivos ligados à idéia de calor enfatiza o clima sufocante da cidade.
A versão definitiva do primeiro parágrafo do romance  O Prisioneiro situa a narrativa temporal e espacialmente, alertando que maio findava, haviam já começado a soprar as monções de sudoeste, mas naquele entardecer mormacento fizera-se uma súbita calmaria em toda a região. Era como se a abóbada celeste, emborcada como uma ventosa sobre a terra, tivesse sugado quase todo o ar dum largo trato de planície, montanha e mar. E a velha cidade imperial, de tão ilustres palácios, templos e tumbas, ali plantada sobre ambas as margens do rio, parecia um organismo vivo palpitante intumescido, a sufocar à míngua de oxigênio. A descrição cresceu a ponto de se tornar, ao mesmo tempo, precisa e plástica, cheia de cores e de luzes.
Mas não é apenas a plasticidade que nos envolve; a sensação de calor sufocante, que Erico buscava desde a primeira tentativa, marca a abertura do texto e permanece por toda a narrativa. É como se experimentássemos antes a opressão e o mal-estar que o tenente sentirá ao longo de todo o romance, atormentado não apenas por suas dúvidas, mas também pela falta de ar e pelo calor.
Através dessa descrição, o romancista faz com que os pensamentos da personagem se tornem táteis, relacionando sensações interiores com exteriores. Há, ainda, a mudança das coordenadas temporais, de junho para fim de maio, e o desdobramento da frase inicial em três, com a consequente ampliação das noções de tempo e de espaço.
O estudo do processo de criação de Erico Verissimo evidencia sua preocupação em alinhar a organização formal do texto com a história que está sendo contada. Sempre à procura da palavra mais adequada, da melhor maneira de descrever ou de narrar e da construção frasal mais apropriada, o autor gaúcho trabalho incessantemente, como comprovam os originais das seis versões do primeiro parágrafo de  O Prisioneiro. Isso confirma o apuro linguístico e narrativo que encontramos em seus textos editados e explica a imensa quantidade de leitores que Verissimo sempre teve e tem até hoje.

MÁRCIA IVANA DE LIMA E SILVA é pesquisadora do Acervo Literário de Erico Verissimo (ALEV), doutoranda em teoria da literatura na PUC-RS (Pontifícia Universidade Católica do Rio Grande do Sul), professora do curso de letras da Faculdade Ritter dos Reis
</TEXT>
</DOC>
<DOC>
<DOCNO>FSP950101-081</DOCNO>
<DOCID>FSP950101-081</DOCID>
<DATE>950101</DATE>
<CATEGORY>MAIS!</CATEGORY>
<TEXT>
ARTHUR NESTROVSKI 
Especial para a Folha 
Para um adolescente metido em livros, Porto Alegre, naquela época, era a cidade dele. Não só pelo prestígio literário, mas também pela postura corajosa e nobre de resistência à ditadura. Eu era muito pequeno para conversar com ele sobre essas coisas. Não sei até que ponto tinha ideais políticos bem definidos. Mas tenho gravado na memória um ideal humano, que ele defendia com as palavras e representava, melhor do que quase qualquer outro, com a própria vida.
Quando eu nasci, a minha família já era um pouco a família deles também, e vice-versa. Meus avós foram amigos dele e de dona Mafalda desde a infância, em Cruz Alta; a segunda geração manteve a tradição de amizade. Nos anos de que estou falando –décadas de 60 e início de 70– a casa dos meus avós era o seu ponto de encontro, nos sábados à noite. Quem quisesse encontrá-lo, ia para lá.
Encontraria, também com um pouco de sorte, outros nomes da intelectualidade gaúcha: Josué Guimarães, Moacyr Scliar, Herbert Caro, Celso e Lya Luft, sem falar no Luiz Fernando. De fora, também, vinham outros amigos e conhecidos: Aurélio Buarque de Hollanda, Paulo Rónai, José Olympio, Paulo Autran, Clarice Lispector, até Tarcísio Meira, quando veio ao Sul filmar o "Capitão Rodrigo".
Esta é a memória afetiva e parcial que eu tenho daquele tempo. Em retrospecto, hoje, penso com admiração ainda maior num outro período. Meu avô e ele compartilhavam da vida literária da cidade em seu auge, nas décadas de 30 e 40. Residiam, então, em Porto Alegre o poeta Mário Quintana, o editor Henrique Bertaso, o ensaísta Augusto Meyer e o compositor Armando Albuquerque, entre outros. O grande projeto editorial da Livraria do Globo foi um dos seus legados. Anos mais tarde, em outras circunstâncias, foi criada a Feira do Livro –outra bela herança desse modernismo gentil que eles construíram. Construíram, isto é, uma idéia de país, baseados num ideal de cultura que não é menos relevante hoje por ser inaplicável.
Eu tenho duas mágoas na vida: uma é a de não ter convivido o suficiente com meu avô e seu amigo, não ter conversado com eles como adulto. Outra é a de não poder mostrar a eles o que venho fazendo agora, seguindo seus passos como intelectual e editor. A mágoa é tanto maior porque tenho a consciência de que, no fundo, tudo o que eu tento é ser um bom ventríloquo: como tantos outros da minha cidade e da minha geração, mas talvez de uma forma mais pessoalmente comprometida, o que eu tento é ser digno de fazer ressoar e de distorcer, à minha maneira, as vozes de meu avô, Maurício Rosenblatt, e de seu grande amigo, meu quase avô Erico Verissimo.

ARTHUR NESTROVSKI é professor na pós-graduação em Comunicação da Pontifícia Universidade Católica-SP e autor de "Debussy e Poe" e "Riverrun - ensaios sobre James Joyce"
</TEXT>
</DOC>
<DOC>
<DOCNO>FSP950101-082</DOCNO>
<DOCID>FSP950101-082</DOCID>
<DATE>950101</DATE>
<CATEGORY>MAIS!</CATEGORY>
<TEXT>
Além de seu trabalho como tradutor, mais conhecido, o intelectual teve forte presença na cultura do Brasil 
ROSANA J. CANDELORO 
Especial para a Folha 
Se estivesse vivo, Herbert Moritz Caro seria amplamente festejado, em Porto Alegre, neste ano que inicia, pela passagem dos 60 anos de sua chegada ao Brasil. Foram seis décadas de significativas contribuições em múltiplas áreas do conhecimento: na música erudita, nas artes plásticas, na crônica jornalística e sobretudo no campo da tradução.
A referência ao intelectual alemão está muito além da efeméride, que coincide com a dos 50 anos do final da Segunda Guerra Mundial, período histórico que forçou a vinda ao Brasil de legítimos representantes da intelligentsia européia.
Todos conhecem um Otto Maria Carpeaux, um Paulo Rónai, um Anatol Rosenfeld. Os três deixaram uma marca estilística, associada a um rigor teórico, inconfundíveis na crítica literária brasileira.
Herbert Caro, por sua vez, é conhecido por muitos apenas como um nome expressivo da arte de traduzir, vertendo obras da literatura universal, de línguas inglesas e alemã, para o português, trabalho esse executado com habilidade e competência.
Neste país desmemoriado, que mantém na obscuridade relevantes nomes na formação da cultura nacional, quem conhece a obra interdisciplinar de Herbert Moritz Caro? Uma dica para os que querem se iniciar na vida e na produção intelectual do judeu alemão é procurar o verbete Caro no "International Biographical Dictionary of Central European Emigrés" (1). Curiosamente, abaixo de seu nome encontra-se o verbete Carpeaux, imigrante austríaco chegado a São Paulo em 1939.

Três mil palavras 
Herbert Caro chegou ao Brasil em 9 de maio de 1935, numa época em que o número de refugiados alemães aumentava drasticamente. Logo, a imigração foi restrita pelo governo brasileiro, o que tornou cada vez mais difícil a obtenção da permanência definitiva, imprescindível para que os imigrantes pudessem trabalhar legalmente, assegurando um mínimo de dignidade.
Preparando-se para sua vinda ao Brasil, por três meses Caro teve aulas de português ainda em Berlim. Com seus conhecimentos de latim e de grego, chegou a Porto Alegre com um vocabulário de três mil palavras.
Ele e sua mulher Nina passaram momentos de dificuldade (2). Para manter um padrão de vida razoável, desde os primeiros tempos Nina ministrou aulas de línguas e, mais tarde, chegou a publicar, no Brasil e na Alemanha, livros didáticos sobre o ensino de alemão. Caro, por sua vez, teve de trabalhar no ramo da indústria e do comércio, de 1935 a 1938, até ser contratado pela Editora Globo, como tradutor, dicionarista e pesquisador. Em depoimento à Folha, Bernard Wolff, amigo pessoal de Caro, conta que se lembra do dia em que o intelectual alemão entrou em seu escritório para comercializar fechaduras, todas perfiladas num mostruário.
A partir de 1939, é contratado para integrar, junto a Erico Verissimo, Leonel Valandro e Mário Quintana, a Sala dos Tradutores, iniciativa da Globo, que marcou a época de ouro da efervescência cultural porto-alegrense. Já em 1935, Erico Verissimo inaugura a fase gloriosa da Editora Globo com a tradução de "Contraponto", de Aldous Huxley.
A Sala de Tradutores era constituída de enciclopedistas, inseridos na mais requintada tradição humanista, que ficavam confinados, em uma espécie de "baias" envidraçadas. Também havia uma biblioteca e uma cozinha para os intelectuais conversarem nos intervalos do trabalho e exercitarem suas preferências gastronômicas.
Na época, o poeta Mário Quintana realiza um importante número de traduções do francês, em especial do "Em Busca do Tempo Perdido", de Proust. Em 1948, passando por momentos difíceis, a Editora Globo vê-se obrigada a fechar sua sala e, com isso, encerra seu período áureo.
Concomitante ao trabalho de tradutor, Herbert Caro firma-se como colaborador da "Revista do Globo", escrevendo ensaios e numerosos artigos para a mesma.

O balconista erudito 
A experiência sui generis de Caro como balconista de uma antiga livraria do centro de Porto Alegre –a Livraria Americana– pode ser desfrutada através da leitura de suas crônicas publicadas no "Correio do Povo".
Caro inicia a redação de suas memórias de livreiro, após cinco anos de experiência como balconista e gerente da seção de livros importados da Americana. Mantendo uma intensa correspondência com Erico Verissimo –quando este trabalhava nos EUA–, oportunidade não lhe faltava de submeter suas crônicas ao crivo apurado do amigo distante.
Timidamente, aparecem suas primeiras crônicas, nas páginas do "Correio", encabeçadas pelo título "Balcão de Livraria", que também batiza o volume de 17 crônicas selecionadas, integrando a série "Aspectos", editada pelo Ministério da Educação e Cultura (MEC), em 1960.
Muitas das questões abordadas e das discussões promovidas por Caro –ao longo de dezenas de textos– continuam atuais e revitalizam o debate de hoje. Exemplo disso é o trecho que segue:
"Em minha opinião, não há bastante revistas literárias no Brasil. Há pelo menos uma, que falta, e que poderia ser de grande utilidade para muita gente –livreiros, editores e leitores" (3).
Para divulgar o lançamento do volume publicado, "Balcão de Livraria", Geir Campos, à frente do programa "Movimento Literário", da Rádio do Ministério da Educação e Cultura, apresenta, em uma das noites, a leitura e os comentários da crônica jornalística de Caro. Esse programa foi ao ar em 14 de julho de 1960.
As crônicas mantêm uma veia estilística inconfundível, domínio invejável da língua portuguesa e um humor não raras vezes refinado.
</TEXT>
</DOC>
<DOC>
<DOCNO>FSP950101-083</DOCNO>
<DOCID>FSP950101-083</DOCID>
<DATE>950101</DATE>
<CATEGORY>MAIS!</CATEGORY>
<TEXT>
26 de novembro de 1976
Prezado senhor Caro,
Agradeço-lhe pela carta, que só recebi agora porque nas últimas semanas estava viajando.
Fico feliz em saber que a tradução de "O Ofuscamento" (Die Blendung) já está tão adiantada. Sou-lhe muito grato pela realização desse trabalho árduo.
Quanto ao título, prefiro "Auto da Fé". Atualmente, este é o nome do romance na maioria dos países; na França e nos Estados Unidos as edições mais recentes também foram rebatizadas com esse título. A opção por "O Ofuscamento" me parece correta apenas nos lugares em que essa expressão for usual; para todos os demais determinei "Auto da Fé", a fim de que o livro, que vem sendo traduzido para um número cada vez maior de idiomas, seja reconhecido por uma identidade comum.
Não há como dispensar uma nota sobre o significado da expressão "Der note Hahn" (O galo vermelho). Se o senhor a fizer já no capítulo "Umwege" (Desvios), como sugere, ela não causará tanto incômodo.
Tenho pouquíssimas fotos minhas e envio-lhe uma que nunca foi utilizada (por editora etc.) e assim, na realidade, destina-se exclusivamente ao senhor.
Meus melhores cumprimentos,
Seu,
Elias Canetti
</TEXT>
</DOC>
<DOC>
<DOCNO>FSP950101-084</DOCNO>
<DOCID>FSP950101-084</DOCID>
<DATE>950101</DATE>
<CATEGORY>MAIS!</CATEGORY>
<TEXT>
20 de dezembro de 1976
Prezado Senhor Caro,
Muito obrigado pela carta. O que o senhor escreve sobre problemas de tradução evidentemente me interessa muito. Infelizmente nada sei de português, um idioma que mal conheço. Apesar disso, desejo, por ocasião de sua tradução do "Auto da Fé", iniciar-me nessa língua.
Hoje, na verdade, escrevo para comunicar-lhe o número do meu telefone em Zurique: 47-0936. Certamente passarei aqui a maior parte de janeiro, em função de uma doença séria de minha mulher (ela mora e trabalha sempre em Zurique). Por favor, telefone, e caso minha mulher já esteja melhor, como espero que aconteça, terei enorme satisfação em conhecê-lo.
Meus melhores cumprimentos,
Seu,
Elias Canetti.

Traduções de ANDRÉ CARONE
</TEXT>
</DOC>
<DOC>
<DOCNO>FSP950101-085</DOCNO>
<DOCID>FSP950101-085</DOCID>
<DATE>950101</DATE>
<CATEGORY>MAIS!</CATEGORY>
<TEXT>
De figura quixotesca de um balcão de livraria, Caro passa a integrar a memória cultural do país, que em 95 fará jus a 60 anos de serviços à cultura 
Continuação da pág. 6-7
Há nelas, fatalmente, um profundo pesar que palpita, o lado silencioso de um filho que foi privado da melodia da língua materna, por quase 25 anos –até voltar a Berlim como turista. Do ponto de vista formal, entretanto, os textos de Caro apresentam um empolamento que causa estranheza, dada sua condição de dicionarista.
Chama a atenção do leitor, considerando-se o conjunto de crônicas e artigos publicados na imprensa, que a música ocupe um papel destacado em detrimento das outras manifestações artísticas, sobre as quais proferiu conferências no Brasil e no exterior.
Não sabemos se Caro sofreu alguma influência da obra filosófica de Schopenhauer. No entanto, é preciso levar em conta que o alemão naturalizado brasileiro foi um profundo conhecedor da criação literária de Thomas Mann que, por sua vez, bebeu da fonte schopenhaueriana, a ponto de dedicar ao filósofo um volume de estudos (4).
"Doutor Fausto", de Mann, que Caro traduziu, está carregado da atmosfera encontrada em "O Mundo Como Vontade e Representação", obra capital de Schopenhauer, no Livro 3, dedicado à estética. Segundo Schopenhauer, a música é cópia dum modelo que nunca pode, ele mesmo, ser representado diretamente. Em outras palavras, a música está fora da hierarquia das artes (tratada desde a arquitetura até o mais elevado dos gêneros poéticos, a tragédia) exatamente por ser capaz de existir fora do mundo fenomênico, do mundo das representações.
A paixão de Caro pela música erudita o levou a escrever resenhas de lançamentos discográficos no gênero, durante muitos anos nas páginas do diário "Correio do Povo" (5). Os mais de 2.500 discos em vinil, colecionados por ele, ao longo de seis décadas no Brasil, estão à disposição dos aficionados na Discoteca Pública Natho Henn, da Casa de Cultura Mário Quintana, em Porto Alegre.
Ele amava especialmente Bach e Mozart. A seguir, em ordem de preferência, Schubert, Beethoven e Brahms. Participou ativamente, em seus últimos anos da vida, da Câmara de Música e Teatro do Instituto Cultural Judaico Marc Chagall, como voluntário.
Além de invejável coleção de discos, Herbert Caro possuía uma maravilhosa biblioteca, que foi motivo de constante apreensão para ele, durante a Segunda Guerra. Viviam em sobressalto, temendo que os livros fossem confiscados, por serem cidadãos alemães.
Os livros de arte de sua biblioteca multidisciplinar estão hoje no acervo do Instituto de Artes da Universidade Federal do Rio Grande do Sul, doadas para a instituição em 1993.
Todo o material em papel existente em seu gabinete, sob a responsabilidade de seu amigo e procurador Ernst Leyser, foi doado ao departamento de memória do Instituto Marc Chagall.
No espólio, encontram-se os documentos pessoais de Caro e Nina, algumas fotos particulares, fotos autografadas dos escritores internacionais traduzidos por ele, os originais das traduções realizadas, muitos recortes de seus artigos nas imprensas alemã e brasileira e, por fim, cartas recebidas de escritores brasileiros e estrangeiros.
Lamentavelmente, não foram acusadas a presença de cartas redigidas por Thomas Mann. Sabe-se que Herbert Caro chegou a convidá-lo a prefaciar a edição brasileira de "Buddenbrooks". A resposta trazia o pedido de liberação de tal compromisso, em virtude de estar assoberbado de trabalho naquele momento. Essas cartas trocadas entre Caro e Mann devem constituir não apenas um epistolário técnico, mas podem conter elementos preciosos da criação literária do autor de "A Montanha Mágica".

A figura quixotesca 
Herbert Caro também viajou pelo Brasil, guiando turistas alemães. Das belezas paisagísticas e arquiteturais, bateu centenas de "slides", amadoristicamente. A exemplo do dramaturgo alemão exilado no Brasil, Wolfgang Hoffmann-Harmisch (6), Caro publicou um livro intitulado "Isto é o Rio Grande do Sul", no qual deixou suas impressões registradas da flora e da fauna gaúchas.
Em suas palestras no exterior, leva um farto material ilustrativo sobre Portinari, os precursores da pintura brasileira, a arte do Aleijadinho e até sobre a obra literária de Erico Verissimo. A partir de 1956, Caro inicia um ciclo de palestras sobre arte e literatura no Instituto Goethe de Porto Alegre. Para um público fiel, falava sobre a poesia expressionista, os pintores e escritores alemães, o Grupo da Ponte, os filmes do cinema expressionista alemão, etc.
Dadas as realizações deste intelectual quase brasileiro, nada mais justo do que o resgate de sua produção em várias áreas do conhecimento. De figura quixotesca de um balcão de livraria, Herbert Moritz Caro passa a fazer parte da memória cultural do país que, em 1995, fará jus aos 60 anos de serviços ininterruptos à cultura brasileira.

NOTAS
1. "International Biographical Dictionary of Central European Emigrés" - 1933/1945, v. II, part 1, 1983
2. Em 1940, os pais de Caro vieram para Porto Alegre
3. Caro, Herbert M. "Balcão de Livraria ", Rio de Janeiro: MEC, 1960
4. Mann, Thomas. "O pensamento vivo de Schopenhauer", São Paulo; Martins, 1960
5. Mais tarde, passou a colaborar no "Caderno de Cultura", de "Zero Hora"
6. Harnisch, W. H. "A Serra no R. G. do Sul" (em alemão, s.d.) e "O Brasil Que Eu Vi", (1937)

ROSANA J. CANDELORO é professora no ensino superior na área de filosofia e mestre em literatura brasileira pela Universidade Federal do Rio Grande do Sul. Atualmente, inventaria o espólio de Herbert Caro nos arquivos do Instituto Cultural Judaico Marc Chagall
</TEXT>
</DOC>
<DOC>
<DOCNO>FSP950101-086</DOCNO>
<DOCID>FSP950101-086</DOCID>
<DATE>950101</DATE>
<CATEGORY>MAIS!</CATEGORY>
<TEXT>
Da Redação 
A parte em papel do acervo Herbert Caro –reunindo cartas, artigos, fotos, documentos pessoais, etc– foi doada ao Instituto Cultural Judaico Marc Chagall, de Porto Alegre, e encontra-se em início de catalogação.
É do instituto também o mais longo depoimento dado pelo tradutor. Faz parte dos arquivos de história oral organizados por Sandra Lemchen Moscovich, que é a coordenadora do Departamento de Memória Judaica do Marc Chagall.
O instituto, que foi fundado em 1985, é constituído de várias câmaras –de música e teatro, de ciência e tecnologia, de artes, ciências sociais e de letras (esta coordenada pelo escritor Moacyr Scliar). Entre suas promoções mais significativas nos últimos anos, está a exposição "Chagall e a Bíblia", trazida da Alemanha para o Brasil em 1987.
</TEXT>
</DOC>
<DOC>
<DOCNO>FSP950101-087</DOCNO>
<DOCID>FSP950101-087</DOCID>
<DATE>950101</DATE>
<CATEGORY>MAIS!</CATEGORY>
<TEXT>
Washington, Agosto.3.1953
Amigo Caro:
São 10 da noite e estou escutando Bach no meu high fidelity, que é uma combinação assim:
(segue desenho do aparelho)
Creio que não é possível no momento conseguir coisa melhor. O som é perfeito. Não reconheci meus discos quando os toquei neste aparelho! Custou, montado, 700 dólares.
A semana passada encomendei (a 80 ou 90 cruzeiros o disco de 12") os últimos quartetos de Beethoven na nova gravação de Budapeste; os concertos de piano e orquestra de Mozart nºs 21 e 24; a "Elégie" de Fauré; os quartetos de Debussy, Ravel e Villa-Lobos (nº 6); três peças de Vivaldi e 3 de Bach. Um total de 20 discos.
Nossa casa de Upshen Street já está parecida com a da Felipe de Oliveira. Todas as semanas, nas noites de sexta ou sábado, os amigos aparecem para ouvir música, conversar e beber. E já substituí as horríveis gravuras de Mr. McDermott por quadros de Van Gogh e Renoir.
O calor aqui tem sido medonho. Chega a quase 40. Hoje tivemos um break. A temperatura caiu. A noite está deliciosa.
Continuo trabalhando tanto que quando volto para casa o que quero é paz e música. Acho que só poderei tentar escrever no outono, lá no (ilegível), que é um lugar delicioso.
Quarta-feira passada este teu amigo apareceu num programa de TV. Achei mais fácil e agradável que ficar na frente dum microfone de rádio.
Dia 30 deste vou ao México. Tenho duas conferências marcadas para a Filadélfia, uma para Wilmington, Delaware e outra para Washington. Tenho de escrever um artigo para uma revista de Paris.
Washington é pobre em matéria de livros estrangeiros. As duas principais livrarias no gênero –Brentano's e Whyte's– têm menos stock que a tua. Em Washington, meu velho!
Este outono teremos bons concertos. Tenho visto bom teatro aqui.
Que me contas da tua vida?
Escreve.
Lembranças para d. Nina e para a sra. tua sogra.
Quando vires o Rasgado mostra-lhe esta carta e dá-lhe um abraço meu.
Até breve!
Um saudoso abraço do
Erico
</TEXT>
</DOC>
<DOC>
<DOCNO>FSP950101-088</DOCNO>
<DOCID>FSP950101-088</DOCID>
<DATE>950101</DATE>
<CATEGORY>MAIS!</CATEGORY>
<TEXT>
1906 - Nasce Herbert Moritz Caro em Berlim, em 16 de outubro
1930 - Doutor em direito pela Universidade Ruperto-Carola, Heidelberg
1933 - É proibido de exercer a advocacia por ser judeu. É destituído do cargo de diretor da Federação Alemã de Tênis de Mesa, após ter jogado seis anos na seleção alemã
1934 - Exílio em Dijon, França. Frequenta curso de estudos greco-latinos (Universidade de Dijon)
1935 - Chega ao Brasil. Estabelece-se em Porto Alegre a 9 de maio. Em dezembro, casa-se com Nina Zapludowski, nascida em Bralistock, na Polônia. Em 1936, funda com outros judeus a Sibra (Sociedade Israelita Brasileira de Cultura e Beneficência). Até 1938, trabalha em indústria e comércio
1939 - Trabalha na famosa Sala dos Tradutores da Editora Globo, a convite de Henrique Bertaso e Erico Verissimo, até 1948. Escreve para a "Revista do Globo"
1940 - Os pais de Caro estabelecem-se em Porto Alegre
1942 - Traduz "Os Buddenbrook", de Thomas Mann. Obtém sua carteira de trabalho
1945 - Passa a escrever regularmente para publicações brasileiras
1948 - Naturaliza-se brasileiro
1949-1957 - Dirige a seção de importados da Livraria Americana, no centro de Porto Alegre
1958 - Após seu fechamento, passa a trabalhar em casa como tradutor autônomo e jornalista free-lancer do "Correio do Povo"
1960 - O MEC, na série "Aspectos", publica um volume de 17 crônicas de Caro, intitulado "Balcão de Livraria". Começa a dirigir a Biblioteca do Instituto Goethe de Porto Alegre
1974 - Recebe em Bonn, na Alemanha, a Cruz da Ordem do Mérito, Primeira Classe
1980 - Traduz "A Montanha Mágica", de Thomas Mann
1982 - Traduz "A Morte de Virgílio", de Hermann Broch
1983 - Recebe o prêmio da Associação Paulista de Críticos de Arte, pela tradução de Broch. Recebe o prêmio Machado de Assis.
1984 - Traduz "Doutor Fausto", de Thomas Mann.
1985 - Recebe o Prêmio Nacional de Tradução do INL (Instituto Nacional do Livro) por "Doutor Fausto". Publica artigo sobre a mãe brasileira de Thomas Mann.
1986 - Recebe o título de Cidadão Emérito de Porto Alegre. Viaja a Berlim nos seus 80 anos.
1987 - Traduz "As Cabeças Trocadas", de Thomas Mann
1991 - Morre de câncer, em Porto Alegre, em 24 de março.
</TEXT>
</DOC>
<DOC>
<DOCNO>FSP950101-089</DOCNO>
<DOCID>FSP950101-089</DOCID>
<DATE>950101</DATE>
<CATEGORY>MAIS!</CATEGORY>
<TEXT>
Os médicos brasileiros têm produzido mais artigos científicos que seus colegas de outros países latino-americanos –5.590, contra 4.535 do México, 3.147 da Argentina ou 4.371 do Chile. O índice de citações varia de 1,72 por artigo chileno para 2,92 para artigo brasileiro. Apesar disso, os artigos mais citados entre os maiores produtores de ciência da região são os dos venezuelanos: 4.56 citações por artigo.

GENÉTICA 
Apesar de o número de pesquisas em biologia molecular e genética ter aumentado bem mais no Brasil que nos outros países da região, o percentual em relação ao mundo mostra que o avanço brasileiro mal acompanhou a tendência mundial. Pior: o impacto desses artigos é o menor entre os principais países da América Latina.

TROPICAL 
Mais uma vez os brasileiros produzem mais com menor qualidade: foram 1.233 artigos em medicina tropical, bem mais do que Venezuela, Argentina, Colômbia e México juntos (total de 463 artigos). Em compensação, cada artigo brasileiro foi citado em média apenas 3,35 vezes, contra 3,37 da Venezuela, 3,44 da Argentina, 5,53 da Colômbia e 4,59 do México.
(RBN)
</TEXT>
</DOC>
<DOC>
<DOCNO>FSP950101-090</DOCNO>
<DOCID>FSP950101-090</DOCID>
<DATE>950101</DATE>
<CATEGORY>MAIS!</CATEGORY>
<TEXT>
Cientistas publicaram muito pouco nas principais revistas científicas em 1994 
RICARDO BONALUME NETO 
Especial para a Folha 
A ciência brasileira está precisando de um choque de qualidade. É o que indicam números de uma pesquisa feita com exclusividade para a Folha pelo  Institute for Scientific Information (ISI), a respeitada instituição que compila listas de artigos científicos e das citações que eles geraram.
A pesquisa mostra que são raríssimos os pesquisadores de uma instituição brasileira que publicam suas descobertas nas principais revistas científicas do planeta.
De janeiro a outubro de 1994, revela a pesquisa, nenhum brasileiro publicou qualquer artigo nas duas principais revistas multidisciplinares de ciência: a americana  Science, editada pela AAAS (Associação Americana para o Avanço da Ciência) e a britânica  Nature, publicação tradicional que comemorou 125 anos em novembro de 94.
Houve uma pitoresca exceção: uma carta de dois cientistas da Unicamp (Universidade Estadual de Campinas), publicada na seção de correspondência científica da  Nature de 17 de fevereiro.
A carta dava a opinião dos dois sobre o processo químico que estaria por trás da formação do resíduo marrom do chá.
Em compensação, brasileiros pelo menos emplacaram alguns artigos –oito, no total– em outra publicação prestigiada, a  Proceedings, da Academia de Ciências dos EUA, geralmente em biologia molecular e genética.
Em duas revistas médicas importantíssimas, a americana  The New England Journal of Medicine e a britânica  The Lancet, apenas 1 e 7 artigos foram respectivamente publicados.
A grande maioria dos artigos publicados pelos brasileiros foram em revistas de física. Entre as revistas de primeiro time selecionadas para a pesquisa, as de física têm mais espaço para textos.
Apesar do universo restrito da pesquisa, foi confirmada a posição da Universidade de São Paulo como a principal produtora de ciência no Brasil.
O primeiro problema da ciência do país é sua quantidade: para uma economia que costuma ser apontada como a décima do planeta, existe uma produção científica que não chega ao trigésimo posto. Apenas em termos latino-americanos a produção de  papers (artigos) tem números respeitáveis.
Mesmo assim há o problema da qualidade: os artigos de brasileiros são consistentemente menos citados pelos pesquisadores do resto do planeta do que os escritos por mexicanos, venezuelanos, colombianos ou peruanos.
</TEXT>
</DOC>
<DOC>
<DOCNO>FSP950101-091</DOCNO>
<DOCID>FSP950101-091</DOCID>
<DATE>950101</DATE>
<CATEGORY>MAIS!</CATEGORY>
<TEXT>
JOSÉ REIS 
Especial para a Folha 
Na reunião de novembro último da Sociedade de Neurociência em Miami (EUA), que entre outros assuntos tratou do possível papel da serotonina na agressão e no suicídio, veio à baila um massacre ocorrido em 1989 em Kentucky.
Um gráfico demitido entrou na oficina em que trabalhara e, atirando ao acaso, feriu e matou muitas pessoas, suicidando-se em seguida. Soube-se depois que ele sofria de depressão e se tratava com Prozac, uma das drogas mais recentes e ativas contra a doença.
Vários grupos que combatem a neuropsiquiatria (ou biopsiquiatria) por diversos motivos desencadearam campanha contra essa prática médica e o Prozac, cuja proibição reclamaram sem êxito.
Há muito se sabe que a serotonina, substância transmissora de sinais nervosos (neurotrasmissora) muito espalhado no organismo, quando em baixo teor contribui para a depressão. Por isso vários medicamentos que a aumentam são usados no tratamento clínico da moléstia nervosa.
A serotonina é fabricada, secretada e absorvida por células nervosas em surtos que se espalham por todo o cérebro, onde transmite sinais de uma célula a outra. Além desta função ela desempenha muitas outras no corpo.
Na reunião de Miami, John Mann, da Universidade Columbia (EUA), mostrou que mais de 95% dos suicidas revelam alteração no teor de serotonina cerebral.
Essa deficiência é muito acentuada em determinada região que fica logo acima dos olhos, o córtex orbital, formado de substância branca. Em 20 suicidas autopsiados praticamente nenhum revelava serotonina nessa região.
Mas nem todos os deprimidos que contêm baixo teor de serotonina manifestam propensão ao suicídio. Parece que outros fatores adjuvantes, como o estresse, são necessários para deflagrar o acesso violento.
Em observações humanas e em animais de laboratório já se confirmou que a baixa de serotonina e as manifestações agressivas (como o suicídio) correspondem a picos naquela baixa. Por isso há pesquisadores que investigam o desenvolvimento de testes que indiquem a iminência desses picos.
Esforço semelhante está sendo feito por neuropsiquiatras para distinguir por meio desse tipo de testes diversas formas de doenças nervosas, como a esquizofrenia,a depressão, a perturbação obsessivo-compulsiva, para melhor orientar o tratamento a ministrar e adequadamente monitorar o uso de medicamentos.
De todos os fatos apresentados na reunião de Miami se depreende que a biopsiquiatria, particularmente mediante a química, está fazendo acentuados progressos e talvez nos ajude a formular testes jamais imaginados para a previsão de comportamentos agressivos.
</TEXT>
</DOC>
<DOC>
<DOCNO>FSP950101-092</DOCNO>
<DOCID>FSP950101-092</DOCID>
<DATE>950101</DATE>
<CATEGORY>MAIS!</CATEGORY>
<TEXT>
Simone Sampaio de Souza, 15, São Paulo, SP
Maria Regina Alcântara, do Instituto de Química da USP, responde:
Mesmo sendo o hidrogênio um gás inflamável e o oxigênio um gás comburente, a água não pega fogo porque ela é uma substância completamente diferente das duas primeiras, com propriedades bastante distintas.
Não é porque ela é composta de dois átomos de hidrogênio e um de oxigênio que ela  herda as propriedades de seus produtos de origem. A estrutura molecular e o tipo de ligação química encontrado na água são diferentes dos encontrados nas moléculas de hidrogênio e oxigênio.
Tanto o gás oxigênio quanto o gás hidrogênio são formados por moléculas com dois átomos iguais, portanto formam ligações chamadas apolares. A água, por sua vez, é composta de moléculas com três átomos diferentes, formando ligações chamadas covalentes polares.
As moléculas de água apresentam fortes ligações intermoleculares conhecidas como pontes de hidrogênio, que originam a formação, entre as moléculas, de uma estrutura tridimensional, mesmo quando no estado líquido.
É esta rede tridimensional que faz com que o gelo (água no estado sólido) tenha uma densidade menor do que a água no estado líquido.
Essa diferença de densidade é um dos principais motivos para a formação da camada de gelo na superfície dos lagos.

Correspondência para a seção "Sem Mistério" deve ser encaminhada à Editoria de Ciência (al. Barão de Limeira, 425, 4º andar, CEP 01290-900, São Paulo - SP), com nome, idade, profissão e endereço. Não serão respondidas cartas com pedidos de aconselhamento médico ou psíquico.
</TEXT>
</DOC>
<DOC>
<DOCNO>FSP950101-093</DOCNO>
<DOCID>FSP950101-093</DOCID>
<DATE>950101</DATE>
<CATEGORY>MAIS!</CATEGORY>
<TEXT>
Bronzeado da pele é resultado de suicídio coletivo de células para proteger organismo de câncer 
Bronzeamento nunca é sadio, mesmo com a aplicação de filtros solares 
Pesquisa pode levar a pomada que bronzeia sem danos do ultravioleta 
CLÁUDIO CSILLAG 
Editor-assistente de Ciência 
O bronzeado e a pele descascada são muito mais do que o resultado da luz solar.
Duas descobertas científicas, publicadas com apenas três semanas de diferença entre si, mostram que eles são resultado da tentativa desesperada do organismo em eliminar o estrago feito pelo Sol.
Um dos artigos indica que a pele morta, que descasca após as queimaduras do Sol, não é composta unicamente por células destruídas pelos raios ultravioleta.
Boa parte delas, na verdade, realizou um gesto altruístico: sabendo que haviam sido perigosamente afetadas pela luz solar, elas literalmente se suicidaram, para evitar que o câncer de pele se instalasse nelas (e, em seguida, se espalhasse pelo corpo).
O outro artigo mostra que o bronzeado não é estimulado unicamente pelo Sol.
Embora os raios ultravioleta provoquem a formação do pigmento escuro responsável pelo bronzeamento –a melanina–, o grande impulso para que o pigmento seja produzido vem de uma mecanismo de defesa da célula –contra as mutações genéticas induzidas pela radiação solar.
Quanto mais mutações forem eliminadas por esse mecanismo de defesa, mais melanina será produzida, e mais escura ficará a célula.
Esse estudo mostra que o bronzeamento nunca é saudável, mesmo quando resultante de uma exposição controlada ao Sol, com uso de filtros protetores.
A cor escurecida é sempre resultado de uma reação –esta sim saudável– do organismo contra a violência dos raios solares.
As descobertas científicas trazem um lado alentador, entretanto, para quem gosta de se bronzear.
 Como sabemos como a defesa da célula contra as mutações acabam levando à produção de melanina, estamos trabalhando para conseguir um produto que possa causar o mesmo efeito sem a exposição aos raios ultravioleta, disse à Folha Barbara Gilchrest, que publicou sua pesquisa na edição de 1º de dezembro da  Nature.
De acordo com Gilchrest, que conduz suas pesquisas na Escola de Medicina da Universidade de Boston, nos Estados Unidos, um produto assim seria o primeiro do gênero.  Não existe nenhum preparado capaz de induzir um bronzeado verdadeiro.
Os resultados da pesquisadora Annemarie Ziegler, atualmente no Hospital Universitário de Zurique, na Suíça, são menos animadores.
Ziegler, que também publicou seus resultados na revista  Nature (na edição de 22/29 de dezembro), não prevê nenhum produto para os aficcionados do Sol. Pelo contrário, ela revela um novo mecanismo pelo qual a luz solar leva ao câncer.
Agente duplo 
Os raios ultravioleta causam câncer de duas maneiras. A primeira, já conhecida e muito eficaz, ocorre a partir das mutações que sofre o DNA, o ácido que contém a informação genética.
Em células normais, a atividade do DNA é controlada. Quando uma célula precisa de uma proteína para catalisar uma reação interna, por exemplo, um pedaço do DNA é lido pela maquinaria celular e uma enzima é produzida.
Quando uma célula se divide, outros trechos do DNA são ativados, o DNA se duplica e cada nova célula recebe metade dos genes. Quando termina a divisão celular, os trechos do DNA são desativados e tudo nele volta ao que era antes.
O câncer de pele, como qualquer outro câncer, ocorre quando um defeito no DNA faz a célula perder a capacidade de brecar a divisão.
Um único defeito em uma única célula pode ser suficiente (embora um defeito não seja certeza de que o câncer vá se desenvolver, pois há outros fatores atuantes).
Um célula sem a capacidade de controlar suas divisões se transforma em duas, depois quatro, oito e assim sucessivamente.
O grupo dessas células em divisão constitui um tumor maligno, ou câncer (a referência ao signo astrológico se deve à forma invasiva que a doença assume: ela se espalha pelo corpo como as pernas de um caranguejo).
A radiação ultravioleta costuma danificar através de uma mutação justamente um gene que age como freio nas divisões celulares.
Sem esse gene, chamado p53, as células se tornam cancerosas e se dividem descontroladamente. Por exercer essa função, o p53 é classificado como um gene supressor de tumores.
Essa origem do câncer de pele já foi bem estudada. Agora, a equipe de Annemarie Ziegler, que na época das investigações trabalhava na Universidade Yale, em Connecticut (Estados Unidos), revelou uma outra ação da luz solar.
 As queimaduras do Sol podem selecionar a expansão (...) de células mutantes, escreveu a cientista em seu artigo.
Os resultados de Ziegler mostram que existe uma outra função para o gene p53.
Além de preservar a função do DNA da célula, mantendo-o sob controle, o gene também preserva a integridade do tecido em que a célula se encontra.
Os raios ultravioleta danificam as células em vários pontos, não só no gene p53.
Nesses casos, as células também podem se tornar cancerosa (pois o p53 não é o único freio das divisões celulares).
Para impedir que o câncer se instale nessas células, o p53 intacto age com rigor: ordena a autodestruição celular (o que implica na destruição do próprio gene, dentro de cada célula).
Esse processo de suicídio celular, chamado de apoptose, é muito comum.
Faz parte dos mecanismos de proteção do organismo, mas em algumas doenças, como a Aids, ocorre indevidamente, levando células sadias à morte.
Como são muitas as células que recebem a radiação ultravioleta, muitas são afetadas e consequentemente condenadas à morte por seus próprios genes. Esse conjunto de células mortas descasca, e o risco de câncer é eliminado.
O problema é quando o próprio p53 é danificado. Ziegler observou em suas experiências com camundongos que o gene p53 mutante fica com a capacidade de ordenar o suicídio celular reduzida.
Isso significa que várias células danificadas –com mutações capazes de levar ao câncer no p53 e em outros genes– acabam escapando da  condenação ao suicídio.
Como explicou Ziegler, essas células com p53 mutante ganham uma vantagem competitiva: enquanto outras células vão sendo obrigadas a se matar, essas vão se acumulando e tomando o lugar das outras.
A chance de surgir um câncer, com o acúmulo de células mutantes, aumenta.
Molécula do ano 
O organismo tenta se defender como pode das mutações provocadas pelos raios solares.
Ao mesmo tempo que inúmeras células se sacrificam para preservar a pele do câncer, há outros mecanismos em ação.
Um deles é a remoção e destruição dos trechos de DNA danificados. O próprio gene p53, se danificado, poderia ser reparado.
Existe um tipo de proteína, chamada  enzima de reparo de DNA, que é fundamental nesse processo.
Sua atuação é tão importante que a Associação Americana para o Progresso da Ciência –órgão que publica a revista  Science, a rival da britânica  Nature– a elegeu  molécula do ano, há dez dias.
A equipe de Barbara Gilchrest investigou uma variedade dessa enzima, a T4N5.
Como todas as enzimas, ela acelera reações químicas –especificamente uma ligada à retirada de trechos de DNA danificados pela radiação ultravioleta.
Gilchrest descobriu, em animais de laboratório, que a T4N5 provoca um  efeito colateral: aumenta a produção de melanina.
A cientista não sabe explicar como isso ocorre, mas suspeita que o acúmulo de fragmentos retirados do DNA dentro da célula seja um sinal para o início da produção do pigmento.
Após exposição à luz solar, as células da pele ativam essa enzima para consertar o DNA.
Como são muitos os trechos de DNA danificados, há muita atividade enzimática e um aumento razoável na produção de melanina.
A tonalidade do bronzeado pode ser, portanto, um indicador do grau de eficiência do mecanismo anticâncer da célula.
Esse  efeito colateral da enzima acaba sendo útil, pois a melanina é um filtro eficaz contra os raios ultravioleta.
Com a camada externa da pele –a epiderme– repleta de melanina, as camadas mais internas ficam protegidas da ação nociva do Sol.
Pomada bronzeadora 
Os cientistas da equipe de Gilchrest foram adiante. Rasparam o pêlo de cobaias e aplicaram uma pomadinha composta por trechos de DNA semelhantes aos danificados pela luz ultravioleta.
Eles realizaram a aplicação duas vezes ao dia durante apenas cinco dias. Dez dias depois, as áreas que receberam o tratamento com a pomada estava mais escura.  A cor persistiu por mais de 60 dias, disse Gilchrest.
Exames ao microscópio mostraram que as células da epiderme dos animais estavam tomadas de melanina, de forma idêntica à observada na epiderme exposta a raios ultravioleta.
De acordo com a pesquisadora, uma pomada dessa espécie, em seres humanos, pode ser usada de maneira preventiva.
As pessoas poderiam usá-la durante uma semana, ficar bronzeadas antes do verão e poder ir à praia com um excelente protetor solar –a melanina– dentro do próprio corpo.
</TEXT>
</DOC>
<DOC>
<DOCNO>FSP950101-094</DOCNO>
<DOCID>FSP950101-094</DOCID>
<DATE>950101</DATE>
<CATEGORY>EMPREGOS</CATEGORY>
<TEXT>
'Arquivo móvel' pode revelar um pouco da personalidade e dos trejeitos do profissional 
DENISE CHRISPIM MARIN 
Da Reportagem Local 
Wilton Santos, 43, gerente da filial centro da Xerox, venceu uma compulsão que há anos o atormentava. Conseguiu se livrar do costume de carregar sua pasta de trabalho para onde quer que fosse –até mesmo em viagens de fim-de-semana e férias.
"Era uma paranóia", afirma Santos. "Eu ia a um restaurante no sábado à noite e levava a bendita pasta no carro."
A maioria dos executivos, entretanto, ainda se mantém presa a essa compulsão. Deixar o pequeno arquivo ao alcance das mãos a todo instante dá justamente o que eles precisam –segurança.
Jairo Gurman, 37, diretor financeiro da Semilog Componentes Eletrônicos, não foge à regra. Ele não desgruda da pasta de couro –comprada por reembolso postal por US$ 250– nem mesmo quando vai descansar em um sítio em Itu, no interior de São Paulo.
Há alguns meses, Gurman chegou em casa e notou que não trazia a pasta consigo. Ficou desesperado, procurou em todo o canto. Chamou a polícia. O problema foi uma falha de memória: havia esquecido a dita cuja no escritório.
"A pasta, para mim, é como um óculos para quem necessita deles", diz Gurman. "Preciso mantê-la comigo até mesmo quando não preciso trabalhar."
Em geral, as pastas oferecem um grau de versatilidade e funcionalidade que as tornam indispensáveis na rotina dos executivos.
Na maioria dos casos, fazem o papel de arquivos móveis, onde é possível carregar objetos e documentos pessoais e toda a sorte de material relativo ao trabalho.
Gavetas ambulantes, acabam se transformando em uma versão masculina das bolsas de mulheres –que também carregam as suas pastas, mas tendem a reservar seus espaços internos para materiais exclusivos do trabalho.
Assim que deixaram de ter a aparência de caixas pretas e ganharam outros estilos e materiais, as pastas acabaram revelando um pouco da personalidade e dos trejeitos do profissional.
"Elas se transformaram em símbolos de status", afirma Victória Bloch, 43, sócia da DBM do Brasil, consultoria especializada em recolocação de executivos.
Segundo ela, é possível fazer uma apreciação preliminar de um profissional apenas com a observação de sua pasta. "O executivo que traz uma surrada passa imagem de desleixo", diz Victória.
</TEXT>
</DOC>
<DOC>
<DOCNO>FSP950101-095</DOCNO>
<DOCID>FSP950101-095</DOCID>
<DATE>950101</DATE>
<CATEGORY>EMPREGOS</CATEGORY>
<TEXT>
Considera-se noturno o trabalho executado das 22h de um dia às 5h do dia seguinte. Para compensar o incômodo do trabalho realizado nesse período, a legislação prevê um adicional de 20% a cada 52 minutos e 30 segundos, calculados sobre o valor da hora normal (diurna). A hora de trabalho noturno é, portanto, reduzida se comparada à hora normal.
A carga de trabalho permitida para o período noturno não pode ultrapassar oito horas diárias e 44 horas semanais.
Estudos mostram que a alternância de horários de trabalho é ainda mais prejudicial ao organismo humano. Por isso, a legislação prevê que a jornada para quem trabalha em turnos ininterruptos de revezamento seja de seis horas.
A jornada de seis horas pode ser prorrogada por duas horas, que deverão ser pagas como horas extras.
O empregado com jornada reduzida para seis horas também faz jus ao adicional de 20% e à hora de trabalho de 52 minutos e 30 segundos, caso tal jornada seja cumprida no período noturno.
(Consultoria IOB)
</TEXT>
</DOC>
<DOC>
<DOCNO>FSP950101-096</DOCNO>
<DOCID>FSP950101-096</DOCID>
<DATE>950101</DATE>
<CATEGORY>EMPREGOS</CATEGORY>
<TEXT>
Da Reportagem Local 
Os títulos que o autor do currículo escolhe para cada item podem valorizar ou "derrubar" o esforço feito na redação dos objetivos e experiências profissionais.
O engenheiro Fábio Mauricio Corrêa elaborou um bom currículo, segundo a consultora em recursos humanos Eliane Pires, da Andersen Consulting. Alguns pontos, no entanto, podem ser melhorados.
O título do item "pretensão profissional" deve ser alterado para "área de interesse".
"Além dessa providência, o leitor deve colocar alguma definição do tipo de atividade pretendida –execução, controle, gerenciamento– para que o selecionador identifique seu nível profissional", diz Eliane.
Esses conselhos, avisa a consultora, valem principalmente quando o candidato não responde a uma vaga específica.
Outra troca de título que ajudaria a valorizar as informações do candidato é a substituição do item "outros" por algo menos genérico. "Esse nome menospreza o conteúdo", avalia a consultora.
Ela propõe que se altere o item para "qualificações adicionais" ou "conhecimentos adicionais".
A estética –maneira como as informações estão colocadas no papel– também pode ser melhorada. "O currículo poderia ser valorizado com um espaçamento maior", diz Eliane.
O item "dados pessoais" deve ser o primeiro a aparecer, no alto da página. As experiências profissionais devem ser listadas em ordem decrescente –da mais recente para a mais antiga (veja fac símile do currículo e outras sugestões no quadro ao lado).

Leitores podem enviar seus currículos, que serão selecionados exclusivamente para análise por consultores e publicação nesta seção, para o caderno Empregos - Seção "Seu Currículo" –al. Barão de Limeira, 425, 4º andar, CEP 01290-900, São Paulo, SP. A Redação considera que o autor do currículo consente com a publicação parcial ou total das informações nele contidas.
</TEXT>
</DOC>
<DOC>
<DOCNO>FSP950101-097</DOCNO>
<DOCID>FSP950101-097</DOCID>
<DATE>950101</DATE>
<CATEGORY>EMPREGOS</CATEGORY>
<TEXT>
O número de ofertas de empregos para executivos em 94 cresceu 36% em relação ao ano passado, segundo pesquisa de oportunidades de recolocação da consultoria Laerte Cordeiro e Associados. O levantamento é realizado com base em anúncios publicados nos principais jornais de São Paulo.
Durante 94, os executivos mais procurados foram os de marketing/vendas, que responderam por 45% dos anúncios. Em seguida, vieram os das áreas de produção (23%) e de administração/finanças/controle (19%).
A pesquisa também apontou que o cargo mais requisitado pelas empresas foi o de gerente de vendas. A indústria continua a ser a maior recrutadora, responsável por 72% das ofertas de emprego.
Em dezembro, particularmente, a oferta de emprego para executivos foi de 142 –o dobro da registrada no mesmo mês de 93. 
</TEXT>
</DOC>
<DOC>
<DOCNO>FSP950101-098</DOCNO>
<DOCID>FSP950101-098</DOCID>
<DATE>950101</DATE>
<CATEGORY>EMPREGOS</CATEGORY>
<TEXT>
Pesquisa de opinião realizada durante o 3º Encontro da Associação Brasileira de Logística (Aslog) revelou que "há um forte indicativo do mercado para a formação de alianças entre clientes e fornecedores". Segundo Kamal Nahas, presidente da Aslog, o objetivo foi "avaliar o grau de utilização de serviços logísticos e o impacto potencial do negócio". 
</TEXT>
</DOC>
<DOC>
<DOCNO>FSP950101-099</DOCNO>
<DOCID>FSP950101-099</DOCID>
<DATE>950101</DATE>
<CATEGORY>EMPREGOS</CATEGORY>
<TEXT>
Pesquisa da Vale Refeição em 768 restaurantes de 30 cidades, em novembro, apontou aumento de 7,4% no preço das refeições à "la carte" e dos pratos executivos. O crescimento foi de 4,5% no valor de sanduíches e de pratos comerciais. O preço das refeições na região da av. Paulista, centro de São Paulo, variou de R$ 4,29 a R$ 18,77. No centro do Rio, elas oscilaram entre R$ 2,81 e R$ 16,88.
</TEXT>
</DOC>
<DOC>
<DOCNO>FSP950101-100</DOCNO>
<DOCID>FSP950101-100</DOCID>
<DATE>950101</DATE>
<CATEGORY>EMPREGOS</CATEGORY>
<TEXT>
É cada dia mais comum incentivar o funcionário a produzir mais e a superar metas. Segundo Gisele Lima, da Mark Up Incentive Marketing, a premiação por resultados cresceu 20% em 94. Hoje, ela atinge quase todos segmentos da economia nacional: agrário, automotivo, beleza e financeiro, além de fabricantes de brinquedos, eletrônicos, construtoras, editoras e empresas jornalísticas.
</TEXT>
</DOC>
<DOC>
<DOCNO>FSP950101-101</DOCNO>
<DOCID>FSP950101-101</DOCID>
<DATE>950101</DATE>
<CATEGORY>EMPREGOS</CATEGORY>
<TEXT>
Quem espera aprender idiomas ou reforçar sua fluência durante as férias pode se valer dos cursos oferecidos por algumas escolas de São Paulo. A União Cultural Brasil-Estados Unidos –(011) 885-1022– abriu matrícula para cursos de inglês de verão. As escolas CCAA -(011) 914-9444– também vão manter cursos intensivos de inglês, para todos os níveis, e de iniciação ao espanhol. 
</TEXT>
</DOC>
<DOC>
<DOCNO>FSP950101-102</DOCNO>
<DOCID>FSP950101-102</DOCID>
<DATE>950101</DATE>
<CATEGORY>EMPREGOS</CATEGORY>
<TEXT>
A Leo Madeiras & Ferragens firmou parceria com o Senai para promover cursos de aperfeiçoamento para seus clientes preferenciais –os marceneiros. Os cursos oferecidos são "Preparação e Acabamento com Verniz", "Colagem e Acabamento para Fórmica e Lâmina de Madeira" e "Colocação de Portas". A empresa responde por 50% do custo das inscrições. Informações: (011) 227-8500.
</TEXT>
</DOC>
<DOC>
<DOCNO>FSP950101-103</DOCNO>
<DOCID>FSP950101-103</DOCID>
<DATE>950101</DATE>
<CATEGORY>EMPREGOS</CATEGORY>
<TEXT>
A partir do dia 9 de janeiro, nenhum funcionário da Calçados Sândalo poderá fumar nas dependências da empresa. A proibição ocorre depois de cinco anos de campanhas de conscientização e da criação de áreas para fumantes, os "fumódromos", que também serão extintas. Atualmente, 10% dos 700 funcionários da Sândalo fumam. Juntos, consomem cerca de 70 maços de cigarros por dia.
</TEXT>
</DOC>
<DOC>
<DOCNO>FSP950101-104</DOCNO>
<DOCID>FSP950101-104</DOCID>
<DATE>950101</DATE>
<CATEGORY>EMPREGOS</CATEGORY>
<TEXT>
(Fernando Moreira de Oliveira, Guarujá, SP)

Seguem os endereços de algumas empresas em São Paulo que mantêm programas de estágio:
Compaq Computer Brasil - r. Alexandre Dumas, 2.220, 12º andar, Chácara Santo Antônio, São Paulo, SP, CEP 04717-004;
HP Brasil - al. Rio Negro, 750, Alphaville, Barueri, SP, CEP 06454-000;
Microsoft - av. Nações Unidas, 17.891, 6º andar, Santo Amaro, São Paulo, SP, CEP 04795-100;
Microtec/Digital - r. Howard A. Acherson Jr., 393, Moinho Velho, Cotia, SP, CEP 06700-000.
 
"Tenho alguns clientes –inclusive uma empresa de aviação da Suíça– aos quais presto ocasionalmente serviços de análise de grafia. Em questão de minutos, faço a distinção, através da escrita, da personalidade, caráter, conflitos pessoais, confiabilidade etc. Estou montando um curso de noções básicas de grafologia, procurando mostrar o lado fisiológico do processo da escrita. Essa abordagem é rara em bibliografia e cursos. Espero que o assunto volte a ser tratado pela Folha." 
(Marco Aurélio Cardoso, psicólogo, Rio de Janeiro, RJ)

"Trabalho na área de recursos humanos e gostaria de obter indicação de alguns livros sobre iniciação à grafologia, técnica usada para avaliação de candidatos a empregos." 
(Mara Cristina Molina Capelão, São Paulo, SP)

Seguem indicações de algumas publicações sobre grafologia disponíveis em livrarias:
"Grafoanálise - A Nova Abordagem da Grafologia", Agostinho Minicucci, ed. Atlas;
"Escrita e Personalidade - As Bases Científicas da Grafologia", Augusto Vels, ed. Pensamento;
"Dicionário de Grafologia - A a Z da sua Personalidade", Peggy Wilson e Gloria Hargreaves, ed. José Olímpio;
"A Grafologia - Método de Exploração Psicológica", Suzane Bresard, ed. Europa-América.
 
"Sou arquiteto e estou me transferindo para São Paulo para fazer pós-graduação na Universidade Mackenzie. Gostaria de obter o endereço de algumas consultorias em recursos humanos que atuem na área de arquitetura para que eu possa enviar meu currículo." 
(Antonio Carlos Fernandez, Curitiba, PR)
Em geral, as empresas de consultoria e recolocação selecionam profissionais de formações variadas. O "Guia Brasileiro de Recursos Humanos", editado pela Associação Brasileira de Recursos Humanos, e o "Guia de Serviços para Recursos Humanos de São Paulo", da editora Garret, fornecem vários endereços de empresas que prestam serviços de recolocação e aconselhamento profissional.

Gostaria de obter o endereço do escritório da empresa Bob's nos Estados Unidos. (Maria dos Prazeres Campos da Cunha, São Paulo, SP)
A rede de lojas Bob's é empresa brasileira de capital holandês e pertence à holding Vendex. A sede é no Brasil e não há filiais nos Estados Unidos. O endereço é av. Brasil, 6.431, Bonsucesso, Rio de Janeiro, RJ, CEP 21040-360.
</TEXT>
</DOC>
<DOC>
<DOCNO>FSP950101-105</DOCNO>
<DOCID>FSP950101-105</DOCID>
<DATE>950101</DATE>
<CATEGORY>EMPREGOS</CATEGORY>
<TEXT>
De Nova York 
Empresas dos EUA aderem às 'células' 
Um estudo sobre tecnologia de produção industrial com 1.042 empresas norte-americanas mostra que elas estão adotando o sistema de produção em células –todas as pessoas de uma equipe são capazes de exercer todas as funções. Cerca de 75% das fábricas com cem ou mais funcionários estão organizadas dessa maneira. Gigantes japonesas, como Mazda, Toyota e Sony, aderiram ao sistema.
</TEXT>
</DOC>
<DOC>
<DOCNO>FSP950101-106</DOCNO>
<DOCID>FSP950101-106</DOCID>
<DATE>950101</DATE>
<CATEGORY>EMPREGOS</CATEGORY>
<TEXT>
De Nova York 
Inflação preocupa trabalhador chinês 
Manter os trabalhadores calmos tornou-se prioridade na China. O motivo da inquietação é o crescimento da inflação. Em outubro de 94, os preços estavam 27% mais altos que em outubro de 93, e não há um plano de ataque "compreensível" ao problema. Muitas empresas estatais –como a Anshan, com 400 mil empregados– estão perigosamente perdendo dinheiro.
</TEXT>
</DOC>
<DOC>
<DOCNO>FSP950101-107</DOCNO>
<DOCID>FSP950101-107</DOCID>
<DATE>950101</DATE>
<CATEGORY>EMPREGOS</CATEGORY>
<TEXT>
Via Verde já está recrutando engenheiros civis, agrônomos, técnicos, contatos publicitários, jardineiros e ajudantes 
Da Reportagem Local Um projeto que prevê a plantação de 1 milhão de árvores nas ruas de São Paulo deve criar cerca de 1.300 empregos diretos para engenheiros civis, agrônomos, técnicos, contatos publicitários, estagiários (em engenharia), jardineiros, ajudantes e serventes.
Lançado oficialmente em setembro de 1994, o projeto "1 Milhão de Árvores" foi elaborado pela empresa Via Verde Comunicação Visual em parceria com a Secretaria Municipal do Meio Ambiente.
"Temos orientação da prefeitura para empregar o máximo de mão-de-obra possível, evitando mecanização", diz Barcha. "Além de criar mais empregos, isso diminui o risco de danificarmos tubulações de água e gás", explica.
Algumas contratações já foram feitas. "Começamos o plantio em fase experimental na semana passada para calcular tempos e visualisar as dificuldades."
Os salários obedecerão aos pisos estabelecidos pelo Sindicato dos Trabalhadores da Construção Civil. Os contatos publicitários –que venderão espaços para publicidade nos protetores metálicos– receberão comissões.
Segundo o diretor da Via Verde, Flávio Barcha, 37, o objetivo do programa é cobrir 18 mil quilômetros de vias num prazo de dois anos –o que significa uma média de 1.600 árvores plantadas por dia.
Para chegar a esse ritmo, Barcha pretende estar com o quadro de profissionais completo até a metade do ano. "Nunca ninguém plantou tantas árvores em tão pouco tempo em uma região metropolitana", diz o diretor do projeto.
Para evitar transtornos –65% das mudas serão plantadas em grandes avenidas– e controlar o fluxo de trabalho, será montada uma central informatizada. É lá que trabalharão os estagiários.
Para formar as equipes de coordenação –o pessoal técnico–, serão selecionados profissionais "com espírito empreendedor que saibam trabalhar com autonomia e que sejam rápidos em tomar decisões" (veja endereço para envio de currículos no quadro ao lado).
O pessoal operacional –jardineiros, ajudantes, serventes– só será recrutado a partir de março.
Projetos semelhantes estão previstos para Fortaleza e Maceió.
</TEXT>
</DOC>
<DOC>
<DOCNO>FSP950101-108</DOCNO>
<DOCID>FSP950101-108</DOCID>
<DATE>950101</DATE>
<CATEGORY>EMPREGOS</CATEGORY>
<TEXT>
ALEX SPILLIUS 
Do "Independent"
Você tem esquecido coisas importantes? Se a resposta é sim, bem-vindo ao clube. A perda de memória está se tornando um importante fator de ansiedade entre os trabalhadores nos anos 90.
Uma pesquisa com 15 mil executivos e gerentes da Escola de Administração European revelou que, há cinco anos, 15% deles citavam a memória e a concentração como preocupações principais. Em 93, o número subiu para 25%.
Michael McGannon, diretor clínico do curso de saúde no trabalho, comentou: "A vida está acelerando na pista da esquerda. Os mercados estão ficando menores e as pessoas precisam aumentar sua contribuição. Em vez de digerir uma idéia durante uma tarde, sua memória e concentração têm que estar a postos o tempo todo".
Segundo ele, estresse e poluição fazem diferença. "Ter estresse significa ocupar a mente com 15 coisas diferentes ao mesmo tempo, enquanto a poluição aumenta o monóxido de carbono e diminui o oxigênio que você inspira e do qual o cérebro precisa."
Estudos após estudos chegam à conclusão de que o chumbo da gasolina e de outras fontes causam danos à capacidade intelectual e de atenção das crianças. A conclusão é que qualquer pessoa que nasceu depois dos anos 60 pode ter tido sua memória enfraquecida.
"Se as neurotoxinas afetam as crianças, devem provavelmente interferir com o desempenho normal dos adultos", afirma Robin Russel-Jones, médico consultor do Hospital de St. Thomas (Londres).
McGannon argumenta que as pressões do dia-a-dia criaram uma geração que se sobrecarrega com tarefas simultâneas. "Imagine pentear o cabelo, dar uma bronca em um filho, tomar um sorvete e tirar uma foto –tudo ao mesmo tempo. É isso o que eles estão tentando fazer. Acham que são capazes, mas não são."
Em um laboratório em Cambridgeshire estão sendo desenvolvidos dispositivos que eventualmente podem ajudar a evitar esses "desvios mentais".
O Centro de Pesquisas Rank Xerox está pesquisando métodos para ajudar as pessoas a lembrar das tarefas que precisam cumprir e das que já fizeram.
Enquanto isso, proliferam os livros sobre como aprimorar a memória. Empresas farmacêuticas também lutam para encontrar uma droga que combata definitivamente a perda de memória.
Esta, aliás, não é uma queixa restrita aos idosos. Aparentemente, qualquer um com cerca de 30 anos de idade tem o dúbio privilégio de apagar da memória nomes e rostos, trancar o carro com as chaves dentro e esquecer de cumprir uma tarefa um minuto depois de lembrar de fazê-la.
Há um sentimento crescente de que nossa memória simplesmente já não é tão eficiente como costumava ser e que, na verdade, está ficando pior. Às vezes, parece realmente que ela está "fazendo água" por todos os lados.
"Não há dúvida de que as pessoas, em geral, sentem que têm uma memória ruim", comenta Abigail Sellen, do Centro de Pesquisas Rank Xerox. "Se elas levam uma vida agitada, têm maior propensão à distração."
Se é assim, as formas de trabalho dos anos 90 apóiam a tese de que nossa memória está diminuindo. Quem tem emprego está trabalhando mais duro e por mais horas em ambientes frequentemente mais competitivos.

Tradução de GLADYS WIEZEL
</TEXT>
</DOC>
<DOC>
<DOCNO>FSP950101-109</DOCNO>
<DOCID>FSP950101-109</DOCID>
<DATE>950101</DATE>
<CATEGORY>EMPREGOS</CATEGORY>
<TEXT>
Do "Independent" 
O Centro de Pesquisas Rank Xerox está desenvolvendo o "parctab", computador de mão que exibe recados para lembrar seu usuário de qualquer tipo de compromisso. O equipamento deve chegar ao mercado nos próximos cinco anos.
O "parctab" poderá, por exemplo, ser programado para lembrar seu usuário de comprar uma garrafa de vinho no supermercado que fica no caminho para a sua casa. O computador reagirá a um sensor e apresentará a mensagem programada.
Outro exemplo: quando o funcionário da contabilidade –com quem você precisa discutir um assunto ainda naquele dia– entrar em seu escritório munido de um "parctab", o seu computador apresentará uma mensagem na tela para lembrá-lo de falar com ele.
(AS)
</TEXT>
</DOC>
<DOC>
<DOCNO>FSP950101-110</DOCNO>
<DOCID>FSP950101-110</DOCID>
<DATE>950101</DATE>
<CATEGORY>TUDO</CATEGORY>
<TEXT>
Cinco empresários de sucesso revelam em que atividade investiriam se tivessem que começar tudo de novo 
INÊS CASTELO 
Editora do Tudo 
Partindo do princípio de que ninguém melhor para dar boas idéias de negócio do que alguém que já "deu certo", a Folha convidou cinco grandes empresários para dar sugestões de investimentos.
Fez a todos a mesma pergunta: que negócio você montaria se tivesse que recomeçar com US$ 50 mil?.
As opiniões são bastante diferentes, já que cada empresário optaria por uma área distinta. O sistema de franquia, no entanto, é apontado como uma boa alternativa por três dos cinco entrevistados.
"É uma opção para começar um negócio com segurança", diz Abram Szajman, 54, presidente da empresa Vale Refeição e da Federação do Comércio de São Paulo, que apostaria no segmento de lazer.
Outro consenso recai sobre a necessidade de que a atividade seja escolhida em função das habilidades e gostos do investidor. "É preciso encontrar um ponto de equilíbrio entre o que o mercado quer comprar e o que o investidor sabe fazer", diz Hugo Marques da Rosa, sócio da construtora Método e secretário de obras do governo Mario Covas.
Leia em seguida que destino teriam os últimos US$ 50 mil desses empresários.
</TEXT>
</DOC>
<DOC>
<DOCNO>FSP950101-111</DOCNO>
<DOCID>FSP950101-111</DOCID>
<DATE>950101</DATE>
<CATEGORY>TUDO</CATEGORY>
<TEXT>
Manoel Magili Mark - Barroso, MG

Um representante comercial é mais que um vendedor. Ele é o elo de ligação entre o produtor e o cliente. Assume a responsabilidade pelo produto que comercializa e acompanha todo o processo de venda, desde o pedido até a entrega do produto ao cliente.
O sucesso desse tipo de negócio depende muito da capacidade do empreendedor para negociação dos produtos e de seus conhecimentos de mercado (incluindo fabricantes e clientes potenciais).
Os consumidores de uma representação comercial são os mais variados possíveis: comércio atacadista, comércio varejista e consumidor final. A representação também pode ser feita entre indústrias.
A escolha do setor a ser trabalhado dependerá exclusivamente do empreendedor, que poderá aproveitar uma oportunidade de mercado ou mesmo utilizar sua experiência com uma determinada mercadoria.
Uma empresa de representação comercial quase não tem riscos financeiros, isso porque atua em nome do fabricante.
Normalmente, é a empresa de representação que vai até o fabricante oferecer os seus serviços. As comissões de venda variam, em geral, entre 5% e 10% sobre o total faturado.
Os pedidos serão feitos no próprio talão de pedidos da empresa representada à qual serão repassados, para então serem providenciados dentro das condições de vendas pactuadas. A comissão só é paga sobre os pedidos entregues e pagos.
A montagem de uma representação comercial se limita a um escritório com mesas, cadeiras, arquivos, máquinas de escrever e de calcular e um telefone. Com aproximadamente US$ 5.000 dá para começar.

As cartas ao Balcão Folha-Sebrae devem ser enviadas para Caderno Tudo - al. Barão de Limeira, 425, 4º andar, CEP 01290-900, São Paulo, SP, com nome, profissão, endereço, telefone e ramo de atividade; elas serão respondidas pela equipe técnica do sistema Sebrae.
</TEXT>
</DOC>
<DOC>
<DOCNO>FSP950101-112</DOCNO>
<DOCID>FSP950101-112</DOCID>
<DATE>950101</DATE>
<CATEGORY>VEÍCULOS</CATEGORY>
<TEXT>
NELSON ROCCO 
Da Reportagem Local 
Um dos sistemas empresariais que mais cresce no mundo –o de franquias– ganhou legislação própria: a lei nº 8.955, sancionada pelo presidente Itamar Franco, em dezembro.
A nova lei, que regulamenta a compra e venda de franquias, determina que os franqueadores forneçam aos interessados em aderir ao sistema uma Circular de Oferta de Franquia, um documento com informações sobre a empresa.
A lei –que entra em vigor em fevereiro– deve dar mais segurança ao sistema, que faturou cerca de US$ 55 bilhões este ano (12% do PIB brasileiro), segundo dados do "5º Censo do Franchising", feito pela InterScience.
A circular deve deixar clara a situação financeira da empresa, sua composição acionária e apresentar os balanços dos últimos dois anos.
O documento terá que chegar às mãos do candidato no mínimo dez dias antes da assinatura do contrato (veja quadro ao lado).

Críticas
Bernard Jeger, presidente da ABF (Associação Brasileira de Franchising), considera a lei excelente, mas critica o prazo de entrega da circular. "Havia uma emenda, vetada pelo presidente, que aumentava o prazo para 30 dias. Seria mais factível", diz.
Marcelo Cherto, 40, diretor do Instituto Franchising, acredita que a lei "ajude a moralizar setor".
Segundo ele, os franqueadores terão que parar para pensar na qualidade do que estão oferecendo aos seus franqueados.

Garantias 
O autor do projeto, José Roberto Magalhães Teixeira, 57, ex-deputado federal (PSDB-SP) e atual prefeito de Campinas (99 km a noroeste de São Paulo), diz que a lei vai dar mais garantias aos interessados em entrar no setor.
Pela lei, o franqueado tem o direito de pedir anulação do contrato e receber de volta o que pagou, caso algum item da circular seja falso ou venha a ser descumprido.
Na circular, o franqueador deve indicar se a empresa tem alguma pendência jurídica e fornecer a relação dos franqueados ativos e dos que se desligaram nos últimos 12 meses, com endereço e telefone.
Além disso, devem constar todas as informações sobre investimento necessário, taxa de franquia e de royalties, valor estimado dos gastos com instalações, equipamentos e estoques.
Os benefícios oferecidos pelo franqueador, como supervisão, treinamento do franqueado e dos funcionários, manuais e auxílio na análise e escolha do ponto, também precisam ser especificados.
</TEXT>
</DOC>
<DOC>
<DOCNO>FSP950101-113</DOCNO>
<DOCID>FSP950101-113</DOCID>
<DATE>950101</DATE>
<CATEGORY>VEÍCULOS</CATEGORY>
<TEXT>
Da Redação 
Os testes do caderno Veículos são realizados em pista de asfalto totalmente plana e reta, ao nível do mar.
Os carros são submetidos inicialmente à aferição do velocímetro. A checagem garante a execução das medições de desempenho a partir de informações reais, sem as frequentes distorções causadas por erros do instrumento do carro (encontrados em todos os modelos, sem exceção até agora, e geralmente variam entre 2% e 15%).
A partir daí são feitos testes de aceleração (o carro é acelerado ao máximo e as marchas são trocadas no limite de rotações do motor), de retomada de velocidade (sempre em última marcha, sem reduções) e de frenagem (simulando uma situação de pânico, com travamento das rodas).
Os testes de consumo são feitos através de sucessivos abastecimentos completos em percursos-padrão que somam cerca de 1.000 quilômetros rodados, em média, com cada modelo.
A média de consumo publicada considera 60% de percurso urbano e 40% de trecho rodoviário, em condições normais (sem congestionamentos) de tráfego.
</TEXT>
</DOC>
<DOC>
<DOCNO>FSP950101-114</DOCNO>
<DOCID>FSP950101-114</DOCID>
<DATE>950101</DATE>
<CATEGORY>VEÍCULOS</CATEGORY>
<TEXT>
A Senna Import, representante oficial da alemã Audi no Brasil, ocupa o primeiro lugar em vendas entre as 200 importadoras da marca no mundo. A empresa foi eleita melhor importadora Audi no mundo na 2ª Convenção de Importadores da marca. A Senna Import comercializou em sete meses 1.500 unidades.

3.4 
... mil foi a quantidade de veículos para táxi que a Ford comercializou no atacado em novembro passado.
</TEXT>
</DOC>
<DOC>
<DOCNO>FSP950101-115</DOCNO>
<DOCID>FSP950101-115</DOCID>
<DATE>950101</DATE>
<CATEGORY>VEÍCULOS</CATEGORY>
<TEXT>
"Em dezembro de 1993, adquiri um Tempra na concessionária Firenze. No período de dois meses foi necessário levar o carro seis vezes para reparo, sem que os principais problemas tenham sido sanados. Seguindo orientação da Fiat, deixei o veículo três vezes na concessionária Fercoi. Como esta não conseguiu resolver todos os problemas, levei o carro por mais três vezes na revenda Autosole. Os problemas existentes desde o início ainda persistem." 
(Ricardo de Jesus Torres - São Paulo, SP) 

A Fiat informou que o veículo foi entregue ao cliente em perfeitas condições de uso e funcionamento no final de outubro.

 "Comprei um Fiat Tipo na concessionária Tekar de Goiânia. Ao receber o carro percebi três irregularidades: porta-malas batendo, pintura da grade do radiador descascando e maçaneta do bagageiro rachada. Como está prevista uma revisão aos 2.500 km, resolvi esperar. Retornando de uma viagem, o veículo apresentou problemas no motor e fui obrigado a rebocá-lo até a cidade. Deixei o carro na concessionária Zema para saber o que tinha ocorrido. Fui informado que o veículo veio com defeito de fábrica no motor, o que provocou dano ao quarto cilindro. Liguei para a Fiat em Belo Horizonte e informaram que não trocariam o motor e sim as peças danificadas e devolveriam o carro. Não aceitei a proposta e estou sem carro." (Reginaldo Kloss - Goiânia, GO) 

A Fiat informou que o veículo do cliente foi reparado pela concessionária Galileo, sendo entregue em 19 de setembro em perfeitas condições de uso.

 "Troquei os pneus de minha Parati 90/91 pelos Grand Prix S-70 da Goodyear. Após a troca, o veículo passou a apresentar problemas de aderência em ruas e pistas molhadas. Ao fazer pequenas curvas ou freadas normais, o carro desliza. Os pneus também 'cantam' em curvas de baixa velocidade, mesmo estando corretamente calibrados. Gostaria de saber da Goodyear qual o motivo dessa anomalia, pois com os pneus originais (Pirelli P-44) isso não ocorria." 
(Enio Ancheben - Jundiaí, SP) 

A Goodyear informou que foi feita uma análise dos pneus e da suspensão do veículo, com as seguintes conclusões: pneus sem desgaste irregular; pneus traseiros (antes dianteiros) com desgaste excessivo –o recomendado é até 1,6 mm de profundidade mínima dos sulcos da rodagem; geometria de direção com alguma distorção em cáster e câmber. O resultado da avaliação técnica é que as "cantadas" dos pneus devem-se ao desalinhamento da geometria da direção e à falta de aderência ideal pela condição de desgaste (liso) dos pneus traseiros.

 "Consultei vários mecânicos que disseram que os motores AP-1.800 da Volkswagen dos anos 1990 e 1991 saíram de fábrica com problemas. Gostaria de saber se essa informação é verdadeira e quais os procedimentos necessários para trocar esse motor junto ao fabricante. Tenho um Gol GL 1.8 91 que está com 34 mil km e esses mecânicos afirmaram que ele já está começando a 'rajar'. Espero que a fábrica troque esse motor." 
(Getulio Francisco da Silva - São Paulo, SP) 

A Volkswagen informou que ruídos internos do motor podem ser resultantes de várias causas. A empresa assegura que o ruído provocado pela movimentação lateral da biela não implica desgaste anormal dos componentes internos nem compromete o desempenho ou a durabilidade. Esclareceu que testes de fábrica concluíram que o ruído característico indicado, com o motor girando a aproximadamente 2.000 rpm, está dentro dos índices prescritos de tolerância, o que afasta toda possibilidade de identificá-lo como "defeito de fábrica".

 "Dia 10 de setembro encomendei dois veículos na concessionária Ibirapuera da marca Renault, um modelo 19 RN e um Twingo. Paguei um sinal de R$ 2 mil para cada um, conforme solicitado pelo vendedor. O prazo de entrega foi de 20 dias, sujeito a atraso em função da greve na Receita Federal. Em 8 de outubro, paguei o valor total do modelo 19 e informaram que o veículo estaria pronto para entrega dia 11 daquele mês. O carro só foi faturado em 17 de outubro e novamente prometeram o carro em dois dias. Até o dia 24, o veículo não tinha sido entregue. Cada vez que vou à concessionária encontro outras pessoas enfrentando os mesmos problemas. Obviamente cancelei a encomenda do outro carro." 
(Fábio Nogueira - São Paulo, SP) 

A Renault esclareceu que a greve na Receita Federal, que durou quase todo o mês de setembro, a atingiu num período de grande demanda, interrompendo a rotina de recebimento de veículos. Informou que em vez de lotes regulares passou a receber carros com lotes acumulados, congestionando o processo de recebimento, preparação e entrega dos veículos. Disse que o mesmo sofreu o setor de faturamento, o que explica o problema enfrentado pelo cliente. Conclui que a situação hoje está normalizada.

"A Fiat distribuiu a todos os seus revendedores autorizados uma carta datada de 26 de outubro solicitando que não aceitem mais pedidos de consorciados contemplados com o modelo ELX, quatro portas, por estar com toda sua produção comprometida com o sistema Mille On-Line. Por ser consorciado dessa administradora de consórcio e ter sido contemplado por sorteio em 19 de outubro, senti-me lesado quando, com toda documentação solicitada, fui preterido de solicitar o bem escolhido." 
(Abelardo da Silva Melo Júnior - João Pessoa, PB) 

A Fiat informou que em 7 de dezembro o cliente recebeu seu veículo através do Consórcio Nacional Fiat. Esclareceu que em 26 de dezembro o cliente foi contatado pela fábrica, quando lhe foram esclarecidas as dúvidas a respeito do consórcio e entrega do veículo, ficando o cliente satisfeito com o atendimento.

MANUAL DO PROPRIETÁRIO 
Os interessados em manuais da Ford devem escrever para Caixa Postal 5.064, CPI 2.043, CEP 09870-900, São Bernardo do Campo, SP, anexando à carta uma cópia do certificado de propriedade do veículo. Fiat, Volkswagen e GM recomendam a procura em revendedores. Elas não atendem ao consumidor diretamente.

PARA ESCREVER 
Envie sua carta para caderno Veículos, Redação, al. Barão de Limeira, 425, 4º andar, CEP 01290-900, Campos Elíseos, São Paulo, SP. Elas só serão respondidas através desta seção.
</TEXT>
</DOC>
<DOC>
<DOCNO>FSP950101-116</DOCNO>
<DOCID>FSP950101-116</DOCID>
<DATE>950101</DATE>
<CATEGORY>VEÍCULOS</CATEGORY>
<TEXT>
Para reduzir custos e melhorar a distribuição, as redes Asia Motors e Daihatsu vão trabalhar em conjunto. A partir deste mês, as revendas Daihatsu devem começar a vender modelos coreanos;
em fevereiro, a rede Asia comercializará carros da Daihatsu.

Rolls-Royce terá motores da BMW 
A alemã BMW venceu a Mercedes-Benz na concorrência para fornecer, por volta do ano 2000, motores para a Rolls-Royce. Sem dinheiro para desenvolver novos propulsores, a marca britânica possui motores com projeto de quase 30 anos.

Rodosat rastreia pilotos no Paris-Dacar 
Os brasileiros André Azevedo e Klever Kolberg, que participam do rali Paris-Dacar, estão sendo rastreados pelo sistema Rodosat de monitoramento por satélite. O sistema envia e recebe mensagens do Brasil e ajuda na localização geográfica dos pilotos.

Rádios europeus vão ter sistema 'inteligente' 
Um terço dos auto-rádios vendidos na Europa em 95 estará equipado com o sistema de rádio "inteligente" RDS (em vigor há sete anos), que ajuda a evitar congestionamentos –uma central de operações informa as rotas alternativas de trânsito.

Daewoo inaugura mais cinco concessionárias 
A coreana Daewoo completou em dezembro a inauguração de mais cinco concessionárias da marca, ampliando sua rede autorizada para 16 revendas. São mais duas em São Paulo, além de São José do Rio Preto (SP), Brasília (DF), Belém (PA).

Ford e Mazda se unem para produzir modelo 
A norte-americana Ford e a japonesa Mazda vão produzir um automóvel pequeno, baseado no Fiesta e que será vendido com a marca Mazda, para o mercado europeu a partir de 96. As marcas querem comercializar cerca de 25 mil unidades do carro.

Volvo deverá fabricar esportivos com a F-1 
A fabricante sueca de automóveis e caminhões Volvo poderá iniciar em 1995 a produção de 850 veículos esportivos em conjunto com Tom Walkinshaw, colaborador da Benetton, equipe que disputa o campeonato mundial de Fórmula 1.

VW recupera-se de prejuízo obtido em 93 
O grupo Volkswagen fechou o ano de 94 com lucro, recuperando-se do prejuízo de US$ 1,1 bilhão registrado em 1993. Foram vendidas em todo mundo 3,3 milhões de unidades da marca.
A Volkswagen ocupa o primeiro lugar em vendas na Europa.
</TEXT>
</DOC>
<DOC>
<DOCNO>FSP950101-117</DOCNO>
<DOCID>FSP950101-117</DOCID>
<DATE>950101</DATE>
<CATEGORY>VEÍCULOS</CATEGORY>
<TEXT>
LUCIANO SOMENZARI 
Da Reportagem Local
A falta constante de "populares" nacionais fez o consumidor ficar mais atento a alternativas de similares importados. O japonês Suzuki Swift Hatch 1.0 avaliado por Veículos é uma delas.
Com preço na faixa dos R$ 12.000, o preço do Swift se aproxima do dos "populares" cobrados com ágio. Mille ELX e Corsa Wind, por exemplo, são cotados nessa faixa no mercado paralelo.
Ágil, compacto e de agradável dirigibilidade, o Swift peca por oferecer acabamento despojado de qualquer luxo e suspensão áspera.
No trânsito urbano, o pequeno japonês está no seu "habitat natural". Compacto (dez centímetros mais longo que o Mille, mas cerca de 100 kg mais leve), com motor de 55 cv (um cv a menos que o similar Fiat), tem bons requisitos para desempenhar bem seu papel no dia-a-dia da cidade.
Para um carro de 993 cc, não existe a sensação, ao dirigir, de que está "amarrado" quando se pisa no acelerador, ao contrário de alguns similares nacionais.
Dados de fábrica indicam que em 17 segundos a partir da imobilidade, ele chega aos 100 km/h.
Outra vantagem são as respostas rápidas da direção, permitindo manobras ligeiras e passando impressão de maior agilidade.
O Swift também possui um curioso sistema de travamento elétrico das portas. Quando o carro atinge 4 km/h as travas se acionam automaticamente.
Um dos problemas do Suzuki é o acabamento interno, muito pobre. O tecido dos bancos é feito de material rústico que lembra uma espécie de feltro grosso.
A qualidade do vinil do painel também denuncia o pouco cuidado com o material empregado.
Externamente, o visual não é muito convidativo. Seu design defasado passa longe das modernas linhas presentes nos carros de sua categoria que rodam na Europa.
Outra falha é a ausência de qualquer tipo de controle para ajuste dos vidros retrovisores. O motorista é obrigado a tocar no espelho para poder ajustá-lo.
A suspensão do Swift também não ajuda muito quando o quesito é conforto. Apesar de firme o suficiente em curvas, transmite à cabine excessiva trepidação em pisos irregulares.

Disputa 
Com a queda da alíquota de importação de 35% para 20%, aumentou o interesse em trazer mais modelos importados para o país. Mesmo no segmento do Swift 1.0, a disputa está acirrada.
Muitos modelos dessa categoria e de outras mais sofisticadas começam a chegar a partir deste mês (veja reportagem na pág. 8-1).
Além dos "populares" nacionais cobrados com ágio, o Suzuki Swift 1.0 disputa mercado com os também importados Peugeot 106; Subaru Vivio, que chega em fevereiro; Renault Twingo; Hyundai Accent, à venda a partir de janeiro; e Corsa GL 1.4, entre outros.
</TEXT>
</DOC>
<DOC>
<DOCNO>FSP950101-118</DOCNO>
<DOCID>FSP950101-118</DOCID>
<DATE>950101</DATE>
<CATEGORY>IMÓVEIS</CATEGORY>
<TEXT>
Da Redação 
Apresentada no último Salão do Automóvel (em São Paulo, novembro passado) e no Salão de Paris um mês antes, a Ferrari 512M chega para substituir a lendária Testarossa (cabeça vermelha, uma alusão à cor dos cabeçotes do motor de doze cilindros do modelo).
Desenhada pelo estúdio italiano de Sergio Pininfarina, a nova Ferrari tem motor V12 central derivado do usado na Fórmula 1 e carroceria construída inteiramente em alumínio.
Em relação à Testarossa desenhada nos anos 80, os faróis escamoteáveis foram substituídos por lâmpadas elipsoidais fixas (mais leves), tecnologia que permite obter formas arredondadas que seguem o design do modelo.
O capô frontal foi reprojetado com entradas de ar para aeração da cabine. O pára-choque também ganhou novas linhas e lanternas.
O chassi do novo produto da "Casa de Maranello", como é conhecida a fábrica italiana fundada pelo comendador Enzo Ferrari nos anos 30 como uma escuderia para corridas, é tubular, em aço cromo-molibdênio, leve e rígido.
O motor tem 4.943 cc de cilindrada, com 440 cv a 6.750 rpm (rotações por minuto). Tem quatro comandos de válvulas (dois em cada cabeçote) e quatro válvulas por cilindro (total de 48 válvulas).
O câmbio é longitudinal de cinco marchas, com a tradicional guia de alavanca da marca. A tração é traseira, como em todos os Ferrari.
A direção é do tipo pinhão e cremalheira e os freios –a discos ventilados nas quatro rodas– possuem sistema ABS (impede o bloqueio das rodas em emergências).
O carro possui tanque de combustível duplo em alumínio, com capacidade para 110 litros.
</TEXT>
</DOC>
<DOC>
<DOCNO>FSP950101-119</DOCNO>
<DOCID>FSP950101-119</DOCID>
<DATE>950101</DATE>
<CATEGORY>IMÓVEIS</CATEGORY>
<TEXT>
Em regiões tradicionais, 70% dos proprietários de imóveis novos são antigos moradores, dizem construtores 
Da Agência Folha 
Ligações afetivas com família e vizinhança e a proximidade com o trabalho. Esses fatores criaram uma característica marcante nos compradores de imóveis novos em São Paulo: a fidelidade ao bairro.
Em regiões como Santana (zona norte), Lapa (oeste), Aclimação (sul) e Mooca (leste), 70% das unidades vendidas estão nas mãos de antigos moradores.
"Essas pessoas têm poder aquisitivo para comprar um imóvel em qualquer parte da cidade, mas querem continuar vivendo no lugar onde nasceram", diz Odair Senra, diretor da Gafisa.
"Geralmente frequentam a mesma igreja e a mesma feira que os vizinhos e conhecem os outros moradores há anos", afirma Simon Goldfarb, dono da construtora Goldfarb, que atua no Ipiranga.
"Tivemos empreendimentos nos quais todos os apartamentos foram vendidos para os moradores de uma região", diz Goldfarb.
Segundo Marcos Camarota, 35, diretor da Camarota Incorporadora e Construtora, além do bairrismo, outros dois fatores pesam na hora da escolha: a perspectiva de valorização e o preço dos imóveis –em média, 20% mais baratos que os da zona sul ou dos Jardins.
"O morador também compra um imóvel para os filhos no próprio bairro", afirma o vice-presidente do Secovi (Sindicato das Empresas de Compra, Venda, Locação e Administração de Imóveis Residenciais e Comerciais de SP), Sérgio Ferrador.
Para o presidente da Embraesp (Empresa Brasileira de Estudos do Patrimônio), Luiz Antônio Pompéia, a tendência é nítida nos bairros onde o processo de verticalização é mais recente.
"Em Higienópolis, os moradores não têm o mesmo comportamento. Podem comprar um imóvel nos Jardins sem que isso represente queda de padrão."
Segundo ele, os habitantes do Tatuapé, Ipiranga e Mooca, ao contrário, não admitem qualquer troca. "O avô nasceu ali, o pai também, e uma mudança implicaria em quebra da tradição", afirma Antônio Pompéia.
Para Ana Maria Heyden, diretora da imobiliária Lindenberg, da construtora Adolpho Lindenberg, o mercado imobiliário nos bairros mais tradicionais tem um comportamento atípico.
"Esses locais funcionam como municípios independentes e, mesmo em períodos de crise, os negócios permanecem aquecidos", afirma.
(Graziele do Val)
</TEXT>
</DOC>
<DOC>
<DOCNO>FSP950101-120</DOCNO>
<DOCID>FSP950101-120</DOCID>
<DATE>950101</DATE>
<CATEGORY>IMÓVEIS</CATEGORY>
<TEXT>
Da Agência Folha 
A qualidade do ar em Santana é um dos pontos fortes do bairro. De acordo com um levantamento feito pela Cetesb, o nível de poluição foi considerado inadequado em menos de um dia do ano passado (0,8%). Em Cerqueira César, essa média chegou a 25.6%.
"É um dos lugares menos poluídos da cidade", afirma a moradora Odete Ackel, 61. A família dela chegou a Santana no início do século, quando o avô montou sua primeira loja no bairro.
Segundo a moradora, há um certo clima de cidade do interior, outra característica marcante nos bairros mais tradicionais. 
(GV)
</TEXT>
</DOC>
<DOC>
<DOCNO>FSP950101-121</DOCNO>
<DOCID>FSP950101-121</DOCID>
<DATE>950101</DATE>
<CATEGORY>IMÓVEIS</CATEGORY>
<TEXT>
Índice de comercialização de apartamentos foi de 7,1% em novembro, após ter alcançado 11,4% no mês anterior 
Da Redação 
O mercado de venda de imóveis novos no litoral de São Paulo pisou no freio em novembro. Depois de um outubro, que acenava com um verão de bons negócios registrando velocidade de vendas média de 11,4%, fechou o mês com índice mais modesto: 7,1%.
Foram comercializadas 41 das 581 unidades colocadas à venda em todo o litoral. A região sul teve maior volume de negócios, com 23 imóveis comercializados. Santos (13), Praia Grande (7) e Mongaguá (3) dividiram o bolo.
Os 18 apartamentos vendidos no litoral norte foram negociados em uma única praia: Riviera de São Lourenço. As demais cidades da região –Caraguatatuba, Ubatuba, Guarujᖠnão tiveram vendas.
Os dados são de estudo mensal (veja quadro ao lado) sobre o mercado de apartamentos novos, realizado pelo Datafolha.
A Praia Grande foi a campeã de ofertas. Teve 163 unidades (28,1%), seguida pela Riveira de São Lourenço, com 131 (22,5%). Mongaguá (48 apartamentos) e Santos (34) foram o terceiro e o quarto colocados.
ABCD 
O movimento de vendas de novos na região do ABCD apresentou recuperação em novembro.
Mantendo uma característica do mercado na região este ano –houve altos e baixos que variaram de 34,5% (março) a 2,8% (junho), passando por 27,2% (agosto) e 7,7% (setembro)–, novembro apresentou resultado bem acima do mês anterior: 10,4% contra 7,6% em outubro.
Foram vendidos 67 dos 645 apartamentos colocados no mercado no penúltimo mês do ano. São Bernardo do Campo teve a maior velocidade de vendas (50,7%), metade dos negócios da região.
Santo André ficou em segundo (32,8%), seguido por São Caetano (7%) e Diadema (4%). O imóvel mais caro do mês na região foi uma cobertura de quatro quartos, negociada em São Caetano por R$ 325,4 mil.
(Pablo Pereira)

O balanço é um estudo do mercado de imóveis novos no litoral e região do ABCD, baseado no Roteiro de Imóveis e em informações sobre as vendas realizadas no mês de novembro, coletadas até o dia 19/12/94, junto a participantes do roteiro. Esse estudo é uma realização do Datafolha, sob a direção dos sociólogos Antonio Manuel Teixeira Mendes e Gustavo Venturi, tendo como assistente de planejamento e análise a estatística Karla Mendes. A coordenação dos trabalhos ficou a cargo de Marlene Peret, auxiliada por Sílvio J. Fernandes.
</TEXT>
</DOC>
<DOC>
<DOCNO>FSP950101-122</DOCNO>
<DOCID>FSP950101-122</DOCID>
<DATE>950101</DATE>
<CATEGORY>IMÓVEIS</CATEGORY>
<TEXT>
WALDIR DE ARRUDA MIRANDA CARNEIRO 
Com a implantação do Plano Real, surgiram muitas dificuldades concernentes à conversão dos aluguéis, antes estabelecida em cruzeiros reais.
O mecanismo de conversão era complexo, as variáveis numerosas e a incompreensão sobre o sistema não se limitava aos destinatários da norma, mas afligia, também, os seus próprios elaboradores.
Não bastasse isso, o resultado das conversões surpreendia a todos e, na maior parte do casos, não agradava os locadores, que acabavam por ter a oportunidade de ver, claramente, quão defasado estavam os seus aluguéis.
A norma que instituiu o Plano Real pode ter sido severa, mas é inegável que guardou extrema coerência ao transportar os contratos de locação de um contexto inflacionário para um outro, no qual a inflação não deveria estar presente.
Paralelamente ao sistema de conversão, das diversas obrigações pecuniárias, as regras que instituíram o real estabeleceram mecanismos de compensação e ajuste para diversas hipóteses.
Tal é o que ocorreu, por exemplo, com os aluguéis referentes a locações residenciais. Para eles, a atual MP 731, de 25 de novembro de 1994, repetindo o disposto nas medidas anteriores desde a instituição do plano, estabeleceu no parágrafo 4º de seu artigo 21, que "em caso de desequilíbrio econômico-financeiro, os contratos de locação residencial, inclusive os convertidos anteriormente poderão ser revistos, a partir de 1º de janeiro de 1995, através de livre negociação entre as partes, ou judicialmente, a fim de adequá-los aos preços de mercado".
Como se vê, os reclamos dos locadores residenciais, no que concerne ao valor dos seus aluguéis, estão com os seus dias contados. Caso não haja acordo com seus inquilinos para atualizar o locativo, aqueles poderão recorrer ao Judiciário para obter o acertamento.
De se observar que tal faculdade também assiste ao locatário que estiver pagando aluguel superior ao vigente no mercado, podendo, nesse caso, pleitear a redução do mesmo.

WALDIR DE ARRUDA MIRANDA CARNEIRO, 30, é advogado em São Paulo, e autor de "Teoria e Prática da Ação Revisional de Aluguel", entre outros livros
</TEXT>
</DOC>
<DOC>
<DOCNO>FSP950101-123</DOCNO>
<DOCID>FSP950101-123</DOCID>
<DATE>950101</DATE>
<CATEGORY>IMÓVEIS</CATEGORY>
<TEXT>
"Moro em um sobrado geminado, estilo colonial, onde pretendo fazer uma cobertura para garagem, sem que ela escureça o imóvel ou deixe a sala muito abafada. A frente da garagem deve ser de alvenaria, com um portão de correr. Gostaria também de diminuir a sala de estar, para aumentar a área do escritório, transformando-o em uma sala de TV. Ocasionalmente esse cômodo ainda poderá ser usado como quarto de hóspedes"
(Rosa Guimarães, 53, dona de casa, Osasco-SP)

Resposta do arquiteto Marcelo Katsuki, 27: "A garagem em alvenaria possui cobertura em laje impermeabilizada e piso em cerâmica rústica. Para evitar problemas de insolação e ventilação, a leitora pode adotar uma estrutura pergolada que acompanhe toda a extensão de parede direita da garagem. As aberturas na laje recebem uma cobertura translúcida, como telhas em fibra de vidro, deixando uma parte descoberta (junto à parede), por onde passa a ventilação. Sob essa abertura poderá ser feito uma pequeno jardim linear, que recolheria a área proveniente da laje. Essa parede poderá ser revestida com pedras, criando um belo efeito visual, integrado ao estilo da construção. Com relação ao fechamento frontal, é importante que o portão apresente aberturas que ajudem na ventilação do ambiente. O escritório foi aumentado, com um deslocamento de 80 cm da parede em direção à sala de estar. Dessa forma, o pequeno ambiente passa a ter dimensões suficientes para abrigar uma sala de TV e funcionar como quarto de hóspedes."
</TEXT>
</DOC>
<DOC>
<DOCNO>FSP950101-124</DOCNO>
<DOCID>FSP950101-124</DOCID>
<DATE>950101</DATE>
<CATEGORY>IMÓVEIS</CATEGORY>
<TEXT>
O m2 construído de apartamentos com vista para o parque Ibirapuera chega a custar R$ 2.200 na região 
CARMEN BARCELLOS 
Da Reportagem Local 
A proximidade do parque Ibirapuera transformou a Vila Nova Conceição num dos bairros mais caros de São Paulo.
Abrir a janela e vislumbrar a área verde do parque pode custar R$ 2.200 por m2 construído. Para apartamentos sem esse privilégio, o valor gira em torno de R$ 1.400 por m2.
"Um imóvel com vista para o parque pode ter preço entre 20% e 30% maior", avalia Renato Genioli, 34, diretor de incorporações da construtora Concyb.
O bairro tem o prédio apontado pelo mercado como um dos mais caros da cidade –o Clermont de Ferrand, construído pela Sucar. O último dos 24 apartamentos foi vendido por US$ 2,8 milhões.
Além da suntuosidade (700 m2 de área útil, quatro suítes e seis vagas na garagem), o prédio está no ponto mais valorizado do bairro: a praça Pereira Coutinho.
Na rua Escobar Ortiz, a construtora Otávio Pires prepara um prédio de 13 apartamentos, com quatro suítes, 500 m2 de área útil, mármores e metais importados. As vendas terão início após a conclusão da obra, por preços que deverão ultrapassar R$ 1 milhão.
Mas os preços não assustam os compradores. A rapidez com que estão sendo vendidos os dois prédios lançados pela Concyb na avenida Hélio Pellegrino mostra a alta demanda pelo bairro.
O Green Wood teve as 28 unidades esgotadas em uma semana. No Green Valley, lançado há duas semanas, restam 14 apartamentos.
Os dois empreendimentos são vizinhos, têm quatro quartos (três suítes), 160 m2 de área útil e custam entre R$ 129 mil e R$ 384 mil. Um terceiro edifício será lançado pela Concyb neste mês.
Os pólos comerciais da Vila Nova Conceição concentram-se na avenida Santo Amaro e na rua Afonso Brás. Nessas vias, são encontrados bancos, Correios, farmácias, entre outros serviços.
No miolo do bairro, o comércio se restringe a panificadoras e pequenas lojas. Na avenida República do Líbano, várias casas viraram escritórios, consultórios, videolocadoras e butiques.
</TEXT>
</DOC>
<DOC>
<DOCNO>FSP950101-125</DOCNO>
<DOCID>FSP950101-125</DOCID>
<DATE>950101</DATE>
<CATEGORY>REVISTA_DA_FOLHA</CATEGORY>
<TEXT>
A estudante Daniela Cristina Lombardi Villarino Campos, 14, pergunta: "Como faço para me corresponder com Letícia Spiller?"
Cartas para a atriz carioca Letícia Spiller, 21, devem ser enviadas para: Rede Globo, Departamento de Elenco, rua Pacheco Leão, 320, loja D, CEP. 22.460-030, Rio de Janeiro, RJ. Antes de interpretar a personagem Babalu, da novela global "Quatro Por Quatro", Letícia foi paquita no extinto "Xou da Xuxa" e fez a personagem Anitra na peça "Peer Gynt", dirigida por Moacyr Goés.

ANA PAULA
O estudante Ricardo Matsuro Kinjo, 14, quer saber o endereço para correspondência de Ana Paula, jogadora da seleção brasileira de vôlei.
Para entrar em contato com a meio de rede Ana Paula Rodrigues, 22, escreva para: Clube Atlético Sorocaba, rua da Penha, 1.393, Centro, CEP. 18.010-004, Sorocaba, SP. Mineira de Lavras, Ana Paula tem 1,83 m e joga atualmente no Leite Moça Sorocaba.

EDGARD
"Como faço para entrar em contato com o VJ Edgard, da MTV? Também gostaria de seber seu nome completo e a data de nascimento." O pedido é da estudante Lidiane Bandeira Faria, 18.
O endereço para se corresponder com o video-jóquei Edgard Wilson Piccoli Jr. é: MTV Produção, avenida Professor Alfonso Bovero, 52, Sumaré, CEP. 01254-000, São Paulo, SP. Edgard nasceu em 20 de maio de 1965, em Campinas. Ele é casado e tem uma filha chamada Mariana.
"Adoro o jogador Tande da seleção brasileira de vôlei. Gostaria de saber seu verdadeiro nome e também seu endereço para correspondência." O pedido é da estudante Fabíola M. Ribeiro, 15.
O nome verdadeiro do atacante Tande é Alexandre Ramos Samuel. Nascido em 1970, ele é campeão olímpico de vôlei –título obtido na Olimpíada de Barcelona, em 1992– e campeão mundial interclubes pelo Misura Mediolanum, da Itália. Para entrar em contato com o jogador, escreva para: Clube de Regatas do Flamengo, praça Nossa Senhora Auxiliadora, sem número, Gávea, CEP. 22441-050, Rio de Janeiro, RJ.

LEONARDO
"Resolvi escrever para a seção Correio da Revista da Folha porque gostaria de saber o endereço para correspondência do jogador Leonardo no Japão." O pedido é da estudante Letícia Olimpio de Lima, 16.
Para se corresponder com o zagueiro e meio-campista Leonardo Nascimento de Araújo, escreva para: Kashima Antlers, 2887, Higashiyama, Kashima - Machi, Kashima – Gun, Ibaraki 314, Kashima, Japão. Antes de ser contratado pelo Kashima Antlers, Leonardo jogou no Flamengo, no Valencia (Espanha) e no São Paulo. Lateral esquerdo da seleção brasileira tetracampeã, o jogador já foi campeão da Taça União pelo Flamengo, em 87, e campeão da Taça Intercontinental de Clubes pelo São Paulo, em 93. Natural de Niterói (RJ), ele nasceu em 5 de setembro de 69, tem 1,77 m e 71 kg.
</TEXT>
</DOC>
<DOC>
<DOCNO>FSP950101-126</DOCNO>
<DOCID>FSP950101-126</DOCID>
<DATE>950101</DATE>
<CATEGORY>REVISTA_DA_FOLHA</CATEGORY>
<TEXT>
"Estou inaugurando uma loja nos Jardins e gostaria de contratar alguns funcionários singulares, tipo: um homossexual discreto, um negro maravilhoso que não tenha 'CC', uma japonesa exótica ou um macho que usa brinco. Só não servem os que são fichados no Carandiru. Me passa esses contatos, Barbara, ou então me informa de onde vêm esses estranhos personagens que a gente vê em lojas e bares badalados."
"Debora", S.P.

–Debbie mental,
Só publico sua cartinha para mostrar aos leitores de bem a que ponto chega a insanidade humana. Faço votos de que sua loja vá à falência.

Público diferenciado
"Gostaria que você me respondesse por que, no programa do Silvio Santos, não há homens no auditório, só mulheres?"
Betania Maria dos Santos, S.P.

–Bebê Tantã,
Puxa, é mesmo! Nunca tinha me dado conta de que o auditório do Silvio Santos é composto exclusivamente por mulheres. Muito obrigada por chamar minha atenção e a de milhares de leitores que desconheciam esse fenômeno de suprema importância. Será que só dá mulher no auditório do Silvio Santos pela mesma razão que só dá homem em estádio de futebol? Não vou conseguir dormir enquanto você não me enviar sua opinião.

Juízo Final
"Estou desesperado. Sou estudante de direito e, para terminar o 3º período, dependo da professora de economia, que é uma mocréia. Desde o primeiro dia de aula ela pegou no meu pé. Só me dá prova personalizada e, a cada explicação, olha no meu olho e ri feito o Coringa do Batman. O que devo fazer para me livrar desse dragão?"
"J.O.", Belo Horizonte, M.G.

–Aprendiz de feiticeiro,
Como pode um estudante de direito vir com uma pergunta dessas? Ora! Faça como reza o adesivo: consulte um advogado.

Sogra é soda
"Estou casado há seis meses e ainda não conheço minha sogra, que mora no Ceará. O ministro Ciro Gomes é cearense e tem a língua mais solta do oeste. Tenho medo que minha sogra seja parecida. Devo ir a Fortaleza conhecê-la ou resisto às pressões da minha mulher?"
Amarildo Nascimento, Itapecirica da Serra, S.P.

–Mamá,
Resista até a morte. Sogra só é mansa em porta-retrato.
</TEXT>
</DOC>
<DOC>
<DOCNO>FSP950101-127</DOCNO>
<DOCID>FSP950101-127</DOCID>
<DATE>950101</DATE>
<CATEGORY>REVISTA_DA_FOLHA</CATEGORY>
<TEXT>
Na última semana da 22ª Bienal de São Paulo, a artista "de esquerda" Ida Natalia, 24, testou sua tática de guerrilha, driblou a segurança do Ibirapuera e expôs ali, clandestinamente, sua obra alijada do "sistema". Com etiqueta igual às dos trabalhos oficiais. E perto de Leda Catunda.
Como nos 60, a "intervenção" foi fotografada para a posteridade. Há provas, portanto, de que o serviço da moça atraiu olhares de visitantes. E o de fiscais. Segundo o curador, Nelson Aguilar, houve outras "instalações" do tipo, providencialmente desmanchadas por monitores. "É como aparecer num filme em cena de multidão ou dar tchauzinho na TV", disse.
"Cem Títulos" são cópias coloridas de fotos de um cão esmagado, arrematadas com a frase "O melhor amigo do homem" (partida ao meio, como o bicho). "A intrusão denuncia esse shopping center das artes, o critério de seleção e a fetichização das obras", diz Ida Natalia.
Sobre a obra: "É um cachorro atropelado. Se o melhor amigo do homem está nessa situação, que dirá o resto?", afirma, dissecando o óbvio. E emenda: "Sou contra essa feira medieval, uma esbugalhação onde todos mostram brinquedinhos." 
A guerrilheira do xerox é criação do diretor Antonio Rocco, 33, que estréia dia 13 a peça "Os Dez Mandamentos do Jogo dos Sete Erros", no teatro Crowne. Ele acha a obra de Ida de mau-gosto e idem da Bienal. Foi Rocco quem clicou o cadáver e invadiu a mostra. Ida Natália não existe, ainda bem. "Mas pode atacar de novo", ameaça.
–Heloisa Helvecia
</TEXT>
</DOC>
<DOC>
<DOCNO>FSP950101-128</DOCNO>
<DOCID>FSP950101-128</DOCID>
<DATE>950101</DATE>
<CATEGORY>REVISTA_DA_FOLHA</CATEGORY>
<TEXT>
Citado por Fernando Henrique Cardoso em sua primeira entrevista coletiva depois da eleição, Caetano Veloso fala sobre suas utopias para o país, diz que considera "auspicioso" o encontro de São Paulo com o Brasil, diz que as leis da cidadania não devem abolir o "jeitinho" e proclama-se um príncipe no reino vitorioso da música popular
</TEXT>
</DOC>
<DOC>
<DOCNO>FSP950101-129</DOCNO>
<DOCID>FSP950101-129</DOCID>
<DATE>950101</DATE>
<CATEGORY>REVISTA_DA_FOLHA</CATEGORY>
<TEXT>
Na véspera, ele e Gilberto Gil fizeram o último show da turnê "Tropicália Duo" –iniciada na Europa e encerrada no célebre Teatro Castro Alves. Nos camarins, já de bermuda e sandália, comentava discretamente com uma amiga o telefonema que recebera de Fernando Henrique Cardoso, convidando-o para a posse de hoje. Não sabia ainda se seria possível ir a Brasília.
O presidente tem destinado a Caetano Veloso um tratamento muito especial. Citou-o já em sua primeira entrevista coletiva, após as eleições, como uma espécie de intérprete da originalidade cultural brasileira e de suas perspectivas de afirmação no mundo.
Encarnação de um programa que pretende imprimir racionalidade à economia e à vida pública nacionais, Fernando Henrique aparentemente deixou-se seduzir pelo tom visionário de seu famoso eleitor, que meses antes formulara, diante de uma platéia reunida no Museu de Arte Moderna do Rio, uma sugestiva utopia sobre o Brasil –publicada, em versão reduzida, pelo caderno "Mais!", no dia 25 de setembro, na Folha. Guindado ao topo do poder político, o príncipe da sociologia estendeu a mão ao príncipe do monte Parnaso da música popular.
A deferência do candidato a "condottiere" do mais recente projeto de modernização do país parece denotar –um pouco como a convocação de Pelé para seu ministério– a intenção, ao menos simbólica, de acolher o "Brasil" num projeto de governo com contornos tão hegemonicamente "paulistas".
Para Caetano, este encontro da São Paulo da força da grana com o Brasil da pele escura poderá ser, em toda sua complexidade, um fato "auspicioso".
–São Paulo me dá sempre a impressão de estar ou numa posição de superioridade ou de inferioridade em relação à realidade cultural brasileira –nunca em sintonia orgânica e natural com ela. Na era de Getúlio, o mundo da Rádio Nacional, por exemplo, não foi vivido pelos paulistanos e paulistas como foi pelo resto dos brasileiros. A frase do Vinicius de Moraes, sobre o "túmulo do samba", que não se destinava a São Paulo, mas ao ambiente de uma casa noturna da cidade, acabou tornando-se tão célebre por existir, de fato, na própria cabeça dos paulistas, essa falta de sintonia entre São Paulo e o país, esse sentimento de estar aquém ou além da vivência do samba, nunca no lugar do Brasil propriamente.
Mas não posso deixar de reconhecer a chegada de outras áreas da realidade brasileira, notadamente o poder político, a um desejo mais ou menos expresso, mas sempre implícito, no gesto tropicalista de 67. Esse desejo era um desejo de "descarioquização" e de "paulistização" do Brasil. Tínhamos, na época, um interesse em valorizar São Paulo como um lugar com coisas a dizer ao Brasil.
–A música "Sampa" traduz bem este sentimento.
–Aquilo que, afinal, apareceu na letra de "Sampa" –"alguma coisa acontece em meu coração, que só quando cruza a Ipiranga e a avenida São João"– de alguma forma já estava presente nas nossas conversas e atitudes. Era uma necessidade de libertação do Brasil da perspectiva à qual estava habituado, para que pudesse ser um pouco diferente de si mesmo e poder se afirmar. Isso São Paulo oferece.
Portanto, o encontro entre São Paulo e o Brasil, nos termos em que estamos falando aqui, é auspicioso. Mas não deve ser um fim em si mesmo, nem levar à idéia de que é preciso substituir as imagens que o Brasil faz de si pelas imagens que São Paulo muitas vezes tende a fazer de si e do Brasil.
–São Paulo teria uma "missão civilizatória" em relação ao Brasil?
–Teria, mas é uma missão relativa.
–Então o Brasil teria também uma "missão civilizatória" em relação a São Paulo?
–Teria em relação ao mundo. São Paulo pode ser uma cidade do mundo, mas ela está no Brasil, e o fato de ela estar no Brasil a deixa exposta à força civilizatória do Brasil... Eu acho muito engraçado quando o professor Agostinho da Silva diz que Portugal já civilizou África, Ásia e América e que só falta agora civilizar a Europa...
Caetano parece ter em mente uma pendular e complexa equação para o país. Se, de um lado, vê a necessidade do fortalecimento das leis da cidadania e do desenvolvimento segundo certos cânones das sociedades economicamente mais avançadas, mostra-se igualmente zeloso quanto aos traços socioculturais que fazem do Brasil o que ele é. 
–Há dez anos recebi uma vaia sonoríssima, de dez minutos, no teatro Castro Alves, por me queixar dos motoristas de Salvador, que se portavam muito mal em relação às leis de trânsito. Disse que quando eu via isso na Bahia me sentia um sueco. Falo frequentemente nesse assunto dos sinais de trânsito como algo emblemático da questão de nossa aparente incapacidade para a cidadania. Sei que um dia mudarei este discurso. Ele é transitório, estratégico –e mantenho-me, por enquanto, como um sueco.
Mas eu sou, na verdade, Zorba, o grego. Estive este ano me apresentando na Europa. Passei pela Inglaterra, pela Alemanha e fui, pela primeira vez, a Nápoles, na Itália. Não há nenhum exemplo no Brasil que eu possa equiparar a Nápoles, em termos de desordem e desrespeito às leis de trânsito. Mas em Nápoles há alguma coisa de mais profundamente humano do que na mera observação das leis de cidadania.
–Até que ponto este lado informal e mesmo desorganizado da sociedade brasileira deve ser preservado numa utopia de país?
–Às vezes eu estou em lugares do mundo em que as leis da cidadania são muito respeitadas, mas sinto que isso não basta, que as coisas não estão humanamente bem.
</TEXT>
</DOC>
<DOC>
<DOCNO>FSP950101-130</DOCNO>
<DOCID>FSP950101-130</DOCID>
<DATE>950101</DATE>
<CATEGORY>REVISTA_DA_FOLHA</CATEGORY>
<TEXT>
–Você teme que o êxito do projeto de integração do Brasil à atual globalização capitalista tenha como contrapartida de seus eventuais benefícios uma institucionalização das relações sociais, inter-raciais e interpessoais à la americana?
–Eu acho que a gente deveria ser sagaz, astuto o suficiente para se utilizar do que essas coisas podem nos trazer para mitigar as escandalosas injustiças que se dão no nível social, sem nos expormos demasiadamente aos efeitos colaterais de que os norte-americanos dão mostras. Caso dessas supercompartimentalizações dos grupos. Parece que se botaram os direitos demasiadamente à frente ou acima das pessoas no modelo americano. Há algo de um gosto puritano brutal em tudo isso. Mas que dificilmente se repetiria aqui, já de início por não termos essa tradição puritana.
Resumindo, eu acho que o Brasil pode se utilizar de uma capacidade mínima que ele tem para a cidadania para impor seu estilo próprio. Mas no fundo, quando eu vejo Jorge Amado dizendo que o fim do "jeitinho" não seria bom, eu concordo com ele... Eu não gosto dessa fetichização da cidadania como uma panacéia. Essa coisa de que o Brasil não presta por que não tem isso, mas pode vir a ter e então vai se salvar. Eu acho que o Brasil presta como é. O Brasil é interessante por que é ele.
Falando do mirante da grande arte do país, a música popular, Caetano encontra em sua paisagem exemplos desse "estilo próprio" elevado a níveis de exigência sofisticados. A "linha evolutiva", como disse em outros tempos, continua passando pela bossa nova –elaboração estética avançada e emblemática das potencialidades brasileiras, mas que, segundo ele, não encontra ainda um país à sua altura.
–Afinal, o que o Brasil precisa para merecer a bossa nova?
–Precisa muito. Precisa trabalhar, precisa conseguir resolver problemas mínimos de cidadania, precisa ser eficaz, precisa saber como usar seu jeito próprio. Veja o João Gilberto, veja o estilo dele: tem tudo isso que estou falando.
Formal e informal, sério e bem-humorado, exigente, refinado, internacional e nacional, João Gilberto surge como o exemplo acabado de uma síntese que projeta no horizonte a imagem de um país forte e vitorioso. Se no território da economia e dos direitos o Brasil ainda está longe deste resultado –"é preciso uma verdadeira revolução econômica para que se solucione o problema do acesso aos direitos"– nos caminhos da música a pauta já estaria à frente.
Uma demonstração de que, mesmo em condições materiais precárias, o país é capaz de vôos audaciosos.
–Não aceito esse aparente consenso universal que parece seguro de que só a capacidade de gerar riqueza material é critério para se julgar um povo. Eu não acho isso. Tenho mais apreço, volto a dizer, pelo jeitinho de Jorge Amado e pelas fantasias de Ariano Suassuna do que por essas objetividades.
Mesmo porque, ainda que Caetano acredite que o Brasil esteja fechando um ciclo (sensação reforçada pela morte de Tom Jobim), ele não acha que, no plano econômico, o país deixará de pertencer, nos próximos anos, ao clube da periferia.
–Quando soube que o Tom morreu, a minha primeira impressão foi de que tinham se passado muitos anos. Chorei, vi inteira a grandeza do Tom e me senti como se um tempo enorme tivesse passado de repente. Havia, realmente, algo de fim de uma era, de encerramento de um ciclo, naquele sentimento. Pode ser uma impressão, mas tem algo a ver com a realidade.
O que me preocupa, porém, no Brasil, é que através dos anos ele tem sido apenas uma peça numa formação particular, chamada América Latina. E tem acompanhado mais como objeto do que como sujeito os interesses dos países ricos. Não vejo como vai deixar de ser assim. Não creio que a sensação de estar nesse submundo dos acontecimentos históricos mundiais possa se romper em função de alguma coisa que estou vendo agora.
Mas não cabe, afinal, aos artistas a tarefa dos governantes e economistas. Nem tampouco subordinar seus êxitos ou fracassos aos resultados da objetividade material do país. A corte da política e a corte da arte podem se relacionar, mas vivem em palácios diferentes. E nisso Caetano é de uma clareza cortante:
–Meu otimismo pode ser muito grande porque eu não preciso de nada disso. Mas minha ambição é maior. Como trabalho com música popular, estou acostumado a vencer, porque este é o destino da música popular no Brasil. Então, eu não tenho desejo de me queixar, nem de meramente me opor, nem de aderir.
Eu tenho desejo de utilizar. Eu acho que esses idiotas da objetividade trabalham para mim. Ou seja, para aquilo que eu, produzindo música popular brasileira, vejo: algo que é livre disso, algo que não depende disso, que não se submete a esse critério, que ri desse critério, que se expressa através da voz de João Gilberto, e para o qual esses escravos, idiotas da objetividade, mais ou menos brilhantes, trabalham, como escravos que são. Eu não sou. Eu sou príncipe. Esse é o raciocínio. Esse é o tropicalismo.
</TEXT>
</DOC>
<DOC>
<DOCNO>FSP950101-131</DOCNO>
<DOCID>FSP950101-131</DOCID>
<DATE>950101</DATE>
<CATEGORY>REVISTA_DA_FOLHA</CATEGORY>
<TEXT>
Por Claudia Gonçalves

Com 1,90 m de altura, 37 anos e o nome artístico de "Falcão", o cearense Marcondes Falcão Maia é hoje o cantor e compositor brega mais cult do Brasil. Seu estilo debochado e escatológico agrada roqueiros e troianos. O segredo do sucesso são versões em "inglês nordestino" de pérolas do repertório nacional, como "Eu Não Sou Cachorro Não" (virou "I'm not Dog No") e "Fuscão Preto" ("Black People Car")

–Fale sobre o início da sua carreira.
–Comecei a compor em 1979. Na época, ninguém tinha coragem de cantar música brega e eu não era cantor. Eu costumo dizer que Roberto Carlos não quis gravar, Frank Sinatra também não, Julio Iglesias muito menos. Quando entrei na faculdade de arquitetura comecei a cantar para os colegas e vi que o pessoal gostava, nem que fosse por sacanagem, por esculhambação. Então, participei de um festival no Ceará com a música "Canto Bregoriano nº 2". O júri todo me deu zero, mas a platéia exigiu a minha presença na final do festival. Comecei a me apresentar em bares e na faculdade. No final de 1988, virei moda em Fortaleza. Daí para gravar o disco foi um pulo.
–Seu primeiro álbum ("Bonito, Lindo e Joiado") vendeu 50 mil cópias. Como vai indo o segundo, "Dinheiro Não é Tudo, Mas é 100%"?
–Está uma loucura. Já esgotou várias vezes. Está bem na terceira ou quarta fornada.
–Onde você aprendeu o inglês nordestino das suas músicas?
–São traduções ao pé da letra. Eu pego a versão em português e um dicionário, de preferência das Edições de Ouro. Sem merchandising, por favor! Começo a traduzir palavra por palavra até encaixar na melodia. Eu boto uma palavra, tiro outra. Ninguém vai saber mesmo. Não falo inglês e nem quero aprender para não estragar minha pronúncia. O importante é o sotaque nordestino, o cearês. É o inglês do Nordeste, como quando um nordestino fala: "Djondi tchu vai?".
–Sua primeira música em inglês nordestino foi "I'm Not Dog No" (versão de "Eu Não Sou Cachorro Não")?
–Foi. E eu comecei a fazer a versão quase sem querer. Quando ouvi essa música do Waldick Soriano, que deve existir há uns 20 anos, me deu aquele estalo de fazer em inglês. Até o nome já pintou na cabeça. Como deu certo, fiz logo a versão do "Fuscão Preto" (de Almir Rogério), a "Black People Car".
–É verdade que as fãs rasgam sua roupa?
–É. Eu tenho vários tipos de público. Tem aquele que é mais de rock, que vai ao show pela irreverência, leva tudo na brincadeira. Sabe que eu estou ali fazendo uma sátira, uma caricatura do brega. Mas tem um público mais humilde, mais povão, que leva a sério. Pensa que eu sou realmente um ídolo, um Fábio Júnior, um Maurício Mattar pós-moderno.
–Isso lá no Nordeste?
–Não. É no Brasil todo. Tem cidades onde eu não consigo sair na rua, porque é aquela confusão de autógrafo, de beijar, de agarrar. Tem lugares onde tenho que levar um estoque dobrado de cuecas, pois elas rasgam tudo.
–Você se interessa por política?
–Eu e uns amigos de Fortaleza fundamos um partido político clandestino, que é o PCB do B, Partido dos Cornos e Bregas do Brasil. Em época de eleição a gente sempre lança um candidato. No plebiscito fui candidato a rei. Também me candidatei ao governo do Ceará. Nas últimas eleições tentei a Presidência, mas não deu de novo.
</TEXT>
</DOC>
<DOC>
<DOCNO>FSP950101-132</DOCNO>
<DOCID>FSP950101-132</DOCID>
<DATE>950101</DATE>
<CATEGORY>REVISTA_DA_FOLHA</CATEGORY>
<TEXT>
–Porque a maioria do povo brasileiro é brega, inclusive os que se dizem chiques. Fiz uma pesquisa na minha agência Data Falcon e concluí que o Brasil é onde tem mais brega e corno por metro quadrado. Tenho que atingir esse público, que é muito carente e discriminado.
–Você gosta de ser chamado de brega?
–Claro. O Brasil é genuinamente brega e eu sou um brasileiro genuíno. Mas eu sou um brega cult. Não um brega autêntico.
–Até o novo presidente você considera brega?
–Embora o Fernando Henrique e sua equipe sejam muito bregas, eu dou um crédito de confiança para eles. Se nos primeiros dez minutos não fizerem nada, já estarei malhando.
–Em quem você votou nas últimas eleições?
–Votei em Lula. Mas foi sem convicção. Votei nele porque não entrou na minha cabeça aquela história de o Fernando Henrique fazer coligação com o PFL e essas forças do além que estão por aí, como Antonio Carlos Magalhães e companhia.
–Como vai sua suposta amizade com o Mick Jagger?
–Esse pessoal do jet set internacional sempre procura as vanguardas e, como eu sou a vanguarda no Brasil, eles me assediam. Já combinei de ir à casa do Sting. Também fui amigo dos Beatles quando eles moravam em Liverpool. Uma vez liguei para lá, George Harrison atendeu e perguntei: "Ringo Starr? Ele respondeu: "Não. Ringo foi pô o MacCartney no correio".
–Seu inglês nordestino faz sucesso no exterior?
–Claro. Tenho shows marcados em Miami. Estou tentando chegar no público dos EUA. Sei que algumas rádios alternativas de Nova York tocam "I'm Not Dog No" e "Black People Car" e isso está deixando os americanos meio malucos. Alguns estão até pensando em criar uma escola com esse neo-inglês.
–Além da música, você tem outra atividade artística?
–Estou escrevendo um livro sobre as filosofias falconianas. Vai ter um nome anglo-português: "Dogs Au Au It's not Nhac Nhac", que traduzido é cão que ladra não morde. A frase é de um deputado do Acre, que falou isso no plenário.
–Quais são seus ídolos musicais?
–Sou eclético. Ouço desde Luís Gonzaga e Fagner até Bob Dylan, Rolling Stones e Beatles. E o pessoal dos anos 70, como Led Zepellin, Deep Purple, Pink Floyd. Fiz uma versão de "Another Brick in the Wall", mas Roger Waters (vocalista do Pink) ou Rogério Aquático, como eu o chamo, não gostou.
–Quando você era pequeno, o que pensava em ser?
–Já passou pela minha cabeça, além do chifre, ser jogador de futebol e até ator. Então veio a vontade de ser intelectual, um Chico, um Caetano. Não sou chegado na música deles, mas sei que eles têm seu valor. Não é igual ao meu, mas têm.
–Você se acha bonito?
–Quando eu era adolescente, me achava muito feio. Hoje eu me acho o protótipo do macho brasileiro. Sou o que o país tem de melhor a oferecer ao mundo.
</TEXT>
</DOC>
<DOC>
<DOCNO>FSP950101-133</DOCNO>
<DOCID>FSP950101-133</DOCID>
<DATE>950101</DATE>
<CATEGORY>REVISTA_DA_FOLHA</CATEGORY>
<TEXT>
Já que você não vai mesmo seguir aquela sábia recomendação de todos os verões — não tomar sol entre 10h e 16h —, o jeito é se proteger ao máximo antes de encarar praia ou piscina.
Quem tem pele clara e está começando a se bronzear deve escolher um filtro com fator de proteção entre os níveis 30 e 40, diminuindo gradualmente. As morenas podem arriscar um pouco mais, começando com o 15. Em qualquer um dos dois casos, porém, o certo é correr para o guarda-sol no horário fatídico.
Sol no rosto, nunca; recorra a chapéus ou viseiras. Produtos especiais para o rosto (e lábios) também são úteis, especialmente para quem tem a pele oleosa (alguns protetores agravam o problema). Óculos escuros (só os que filtram os raios ultravioletas) são companhia indispensável.
Nada de passar o protetor na beira do mar ou piscina. Para funcionar, os componentes químicos que bloqueiam o sol precisam entrar em contato com a pele pelo menos meia hora antes da exposição. Também não dá para confiar cegamente em avisos tipo "não sai na água" e "resistente ao suor". Reaplique no mínimo a cada duas horas (se possível, sempre que entrar na água).
Crianças e idosos são casos à parte. Donos de peles mais sensíveis e hábitos menos ortodoxos (os pequenos não saem da água por nada), eles devem se servir à vontade de bloqueadores (fator 30 ou mais), reaplicados em intervalos de uma hora.
– Marisa Adán Gil
</TEXT>
</DOC>
<DOC>
<DOCNO>FSP950101-134</DOCNO>
<DOCID>FSP950101-134</DOCID>
<DATE>950101</DATE>
<CATEGORY>REVISTA_DA_FOLHA</CATEGORY>
<TEXT>
Quem quer pegar aquela cor sem perder tempo pode estar se arriscando - os especialistas não são unânimes neste ponto. Segundo a dermatologista Ana Claudia Schor, produtos que prometem bronzeamento rápido são inofensivos, desde que contenham protetores solares. Seu único efeito seria estimular a produção de melanina (pigmento natural da pele). Outro dermatologista, Aurélio Ancona, aponta um possível perigo: segundo ele, esses produtos provocam uma sensibilização maior da pele. Ela ficaria, assim, mais exposta à ação nociva dos raios ultravioleta - que vai de queimaduras a câncer de pele (a longo prazo). Se desistir do bronzeador, você pode pelo menos tentar ficar com uma cor mais uniforme: basta trocar as longas horas de moleza na areia por uma boa caminhada pela praia.

ONDE ENCONTRAR: O Boticário: shopping Iguatemi, loja Y-17, Jardins, zona sul, tel. 211-4440; shopping Eldorado, loja 384, segundo piso, Pinheiros, zona oeste, Tel. 814-8364. Farmaervas: r. Pamplona, 1.739, Jardins, zona sul, Tels. 887-4667 ou 0800-12-2911. Natura: tel. 0800-11-5566. Céu Carmesim: r. Paes de Araujo, 163, Itaim Bibi, zona oeste. Tel. 881-2171. Suil: shopping Center Norte, loja 244, Santana, zona norte, Tel. 290-9398. Yves Rocher: tel. 0800-13-0013. Drogaria D.B.: R. Cardoso de Almeida, 695, Perdizes, zona oeste, Tel. 864-3011.

PROTETORES
1. Protetor Natural Chemical Free Sunscreen, da Banana Boat, fator 8. Céu Carmesim, R$ 10
2. Bloqueador Nivea Sun, da Nivea, fator 25. Drogaria D.B., R$ 18,33
3. Gel Golden Plus, de O Boticário, fator 6. R$ 13,90
4. Loção Baby Sunblock, da Banana Boat, fator 29. Céu Carmesim, R$ 13,90
5. Protetor Wrinkle Stick, da Clinique. Suil, R$ 40,91
6. Bloqueador Sundown, da Johnson & Johnson, fator 30. Drogaria D.B., R$ 22,60
7. UV Stopper, importado japonês, fator 12. Céu Carmesim, R$ 13,80
8. Protetor spray, da Natura, fator 15. R$ 18
9. Moderador Sundown Sport, da Johnson & Johnson, fator 4. Drogaria D.B., R$ 12,71
10. Protetor Only Face, de O Boticário, fator 15. R$ 12,50

BRONZEADORES
1. Óleo bronzeador Canela, de O Boticário, fator 3. R$ 12,95
2. Loção Classic Brown, da Johnson & Johnson, fator 2. Drogaria D.B., R$ 12,16
3. Emulsão bronzeadora Cap Soleil, da Yves Rocher, fator 4. R$ 22,10
4. Loção bronzeadora Summer Care, da Farmaervas, fator 4. R$ 5,04
</TEXT>
</DOC>
<DOC>
<DOCNO>FSP950101-135</DOCNO>
<DOCID>FSP950101-135</DOCID>
<DATE>950101</DATE>
<CATEGORY>REVISTA_DA_FOLHA</CATEGORY>
<TEXT>
Ela entra em qualquer trama. De vime, junco, palha, taquara, cipó ou macu, chega no Natal, carregada de presentes e delícias. Sempre útil, atravessa o Ano Novo armazenando desde flores até roupa suja. Trançada e transada, vai da praia ao campo sem deixar de ser urbana.
–Daniela de Camaret

1. Cesta grande, em vime claro. Cortinas Régio, R$ 15
2. Balde de alumínio revestido em vime. Jorge Elias Boutique, R$ 80
3. Cesta grande, em vime. Armando Cerello, R$ 85
4. Porta-garrafas em vime. Cortinas Régio, R$ 10
5. Em junco, importada da Indonésia. Se essa rua..., R$ 620
6. Feita em palha colorida, da Indonésia. Armando Cerello, R$ 35
7. Em taquara, com tampa e asa. Arte da Terra, R$ 15
8. Isopor forrado de vime, com tampa em madeira. Cecilia Dale, R$ 103
9. Cachepô grande, em vime claro. Cortinas Régio, R$ 15
10. Cesto indígena em macu, com 40 cm de diâmetro. Kariri, R$ 35
11. Artesanato em cipó, do Ceará. Kariri, R$ 40
12. Baú em palha de milho, com tampa. Palhão, R$ 10
13. Cesta em vime escuro. Cortinas Régio, R$ 15
 
ONDE ENCONTRAR: Armando Cerello Cestas e Arte: r. Dr. Mário Ferraz, 402, Itaim Bibi, zona oeste. Tel. 829-6852. Arte da Terra: r. Teodoro Sampaio, 640, Pinheiros, zona oeste. Tel. 883-7438. Cecilia Dale: r. Dr. Melo Alves, 513, Jardins. Tel. 853-7388. Cortinas Régio: r. Peixoto Gomide, 400, Jardins. Tel. 256-3704. Jorge Elias Boutique: al. Gabriel Monteiro da Silva, 453, Jardins. Tel. 282-4033. Kariri Comércio de Artesanato: r. Francisco Leitão, 272, Pinheiros, zona oeste. Tel. 64-6586. Se Essa Rua Fosse Minha: r. Oscar Freire, 849, Jardins. Tel. 881-2400. Palhão: r. Oiapoque, 89, Brás, região central. Tel. 229-1186.
</TEXT>
</DOC>
<DOC>
<DOCNO>FSP950101-136</DOCNO>
<DOCID>FSP950101-136</DOCID>
<DATE>950101</DATE>
<CATEGORY>REVISTA_DA_FOLHA</CATEGORY>
<TEXT>
Santas Rodinhas! Antes só rolavam no escritório, agora colocam ordem na casa. Você cansou da mesmice na disposição dos móveis? Rodinha neles! A família resolve ouvir som, ver TV e jogar videogame ao mesmo tempo e no mesmo ambiente? Dá-lhe rodinha! A cada empurrão, as redondas criam um novo espaço e, ainda, restauram a paz doméstica
– Daniela de Camaret

1. Carrinho para computador, design Eduardo Samsò. Esther Giobbi, R$ 1.350
2. Mesa em aço para TV ou computador. Mantovani Tabet, R$ 1.200
3. Mesa de centro em marfim e embuia, design Claudia Moreira Salles. Etel Interiores, R$ 3.930
4. Poltrona revestida com retalhos de veludo. Mantovani Tabet, R$ 1.020
5. Cadeira em estrutura e chapa metálicas com almofada de veludo. Useche Arquitetura e Design, R$ 858,56
6. Gaveteiro em couro e madeira, design Enrico Tonucci. Esther Giobbi, R$ 1.780
7. Estante em mogno e estrutura metálica. Useche Arquitetura e Design, R$ 1.973,44
8. Estante em rádica de embuia com rodas e dobradiças em madeira, design Etel Carmona. Etel Interiores, R$ 2.480
9. Bancada em mogno e estrutura de ferro, design Fulvio Nanni. Nanni Movelaria, R$ 2.520

ONDE ENCONTRAR: Esther Giobbi: r. Padre Manoel Chaves, 92, Jardins, zona sul, Tel. 853-9666. Etel Interiores: r. Doutor Virgílio de Carvalho Pinto, 599, Pinheiros, zona sul, Tel. 211-4770. Mantovani Tabet Arquitetura e Design: r. André Fernandes, 149, Itaim Bibi, zona sul, Tel. 852-1659. Nanni Movelaria: r. Augusta, 303, região central, Tel. 256-7223. Useche Arquitetura e Design: r. Profª Maria Edyvani do Amaral Dick, 165, Vila São Francisco, zona sul, Tel. 523-8580.
</TEXT>
</DOC>
<DOC>
<DOCNO>FSP950101-137</DOCNO>
<DOCID>FSP950101-137</DOCID>
<DATE>950101</DATE>
<CATEGORY>REVISTA_DA_FOLHA</CATEGORY>
<TEXT>
Sopra de Paris um novo conceito de androginia. Protagonista, o terno mistura os gêneros e (mal) disfarça formas opulentas sob tecidos masculinos. Nostálgico, o dito-cujo pinta como fundamental. Cheio de história, revive décadas. Todo sintético, banca o moderno e vira high-tech
–Patricia Carta
</TEXT>
</DOC>
<DOC>
<DOCNO>FSP950101-138</DOCNO>
<DOCID>FSP950101-138</DOCID>
<DATE>950101</DATE>
<CATEGORY>REVISTA_DA_FOLHA</CATEGORY>
<TEXT>
Quem ainda não aderiu à máquina três sabe o quanto custa manter a cabeleira limpa, cheirosa e esvoaçante o tempo inteiro. Cabelos compridos só precisam de dois dias sem água, xampu, creme rinse e escova para começar a adquirir aquele perigoso aspecto "molhado" –atestado de oleosidade. Mas nem sempre é possível encaixar uma sessão de lava-seca-penteia em um dia de trabalho infernal, correria desesperada e compromisso com hora marcada. Que tal, em um caso assim, queimar a etapa água –e ganhar preciosa meia-hora na operação?
É essa, precisamente, a promessa por trás do rótulo do Dry Shampoo, produto alemão que a Marinage/Algemarin está importando para o Brasil. Segundo as instruções, basta aplicar o xampu no cabelo seco e escová-lo. O químico se encarrega de absorver a "gordura supérflua" do cabelo. Na escovação, vão-se o xampu e a sujeira.
A modelo Fabiana Kanan, da agência L'Equipe, testou o xampu seco para a Revista da Folha e matou a charada: "Isto parece o truque do talco da vovó", disse, referindo-se ao velho expediente de desengordurar os cabelos com talco em pó, completando o serviço com a escova. Leia, a seguir, o resultado do teste.
–M.E.M.

Como o xampu é apresentado em spray, é fácil espalhá-lo no cabelo. O inconveniente é que ele se espalha também por toda e qualquer superfície exposta à volta: mesas, pia etc. O pó tem coloração esbranquiçada e é grudento. Clareia o cabelo. Depois da aplicação, Fabiana parecia ainda mais loira do que já é. Conforme prometido, o excesso de xampu sai na escovação. A surpresa começa na etapa final, quando a escova completa o truque. O cabelo da modelo ganhou volume, brilho e maciez, além de um leve perfume. A oleosidade foi embora e o efeito clareante do xampu persistiu. A sensação, segundo Fabiana, foi "de limpeza". O mesmo não pode ser dito do que estava na área em volta do local da aplicação: a mesa e a bolsa da modelo terminaram o teste cobertas do xampu em pó.

ONDE ENCONTRAR: Marinage Algemarin (tel. 283-1100 e 0800-130066). Preço: R$ 13,75.
</TEXT>
</DOC>
<DOC>
<DOCNO>FSP950101-139</DOCNO>
<DOCID>FSP950101-139</DOCID>
<DATE>950101</DATE>
<CATEGORY>REVISTA_DA_FOLHA</CATEGORY>
<TEXT>
Marcos Augusto Gonçalves 

O futebol brasileiro tem sido vitorioso não por causa de seus dirigentes, mas quase sempre apesar deles 

Ele tem sido, fora dos estádios, uma figura controvertida. Criticado por sua suposta omissão sobre os problemas raciais no Brasil, ridicularizado pelo sentimentalismo de suas declarações sobre as "criancinhas" ou a célebre sentença de que os brasileiros não saberiam votar, Pelé, o rei do futebol, chega ao cargo de ministro extraordinário dos Esportes ainda cercado por suspeitas. Até que ponto irá seu empenho na nova função? Saberá mover-se no terreno da política com a desenvoltura necessária para concretizar seus planos? Montará uma equipe de trabalho capaz? 
O tempo dirá. Mas uma coisa é certa: para aqueles que vêm acompanhando as inglórias tentativas de alguns poucos técnicos e dirigentes de dotar o futebol brasileiro de um mínimo de racionalidade, a escolha de Pelé surge com a expectativa de um drible no atraso e na ineficiência que, há anos, envolve a administração do esporte no Brasil.
Aquilo que o presidente hoje empossado chama, com propriedade, de "oligarquia esportiva" costuma reagir às críticas citando títulos conquistados –a vitória na Copa dos Estados Unidos é a última peça autopromocional do presidente da Confederação Brasileira de Futebol, Ricardo Teixeira. É preciso ser um perfeito idiota da objetividade para aceitar o argumento.
Com uma capacidade extraordinária de produzir bons jogadores, o futebol brasileiro tem sido vitorioso não por causa de seus dirigentes, mas quase sempre apesar deles. Artífices de campeonatos deficitários e mirabolantes, incapazes de organizar um simples calendário anual, permeáveis a toda sorte de concessões políticas, metidos em negócios nebulosos, os responsáveis pelo futebol, com as exceções honrosas de sempre, fazem parte de um Brasil de autarquias, de um Brasil de carimbos e mata-borrão, de um Brasil de politicagem e troca de favores.
Inépcia e anacronismo gerencial têm levado o futebol brasileiro a se transformar em exportador de matéria-prima –ou de "pé-de-obra", como disse o colunista Matinas Suzuki, editor-executivo da Folha– e provocado a virtual falência de clubes com enorme apoio de torcida, caso do Flamengo, o maior de todos, que deve cerca de... US$ 20 milhões! 
Pelé, cujas relações com a cúpula do futebol têm sido saudavelmente conflituosas, certamente encontrará muitas dificuldades para exercer sua nova função. Mas por sua experiência nos Estados Unidos, pelo que conhece do mundo esportivo europeu e pela sua própria história como jogador e empresário, tem tudo para exercer sobre a tal "oligarquia esportiva" uma pressão capaz de quebrar a inércia do atraso e da ineficiência. 

Marcos Augusto Gonçalves é editor da Revista da Folha
</TEXT>
</DOC>
<DOC>
<DOCNO>FSP950101-140</DOCNO>
<DOCID>FSP950101-140</DOCID>
<DATE>950101</DATE>
<CATEGORY>REVISTA_DA_FOLHA</CATEGORY>
<TEXT>
Quem esperar por um filme de ação tipo "Speed" não precisa nem passar na porta do cinema: esse não é o público de "Através das Oliveiras" (foto à esq.), do diretor Abbas Kiarostami. Mas quem tiver tempo (interior) para relaxar, ver e ouvir perceberá que está diante de um dos maiores cineastas vivos.
No enredo, um cineasta procura, entre habitantes do norte do Irã, atores para seu novo filme. A região foi devastada há alguns anos por um terremoto. O personagem encontra um rapaz e uma moça apaixonados um pelo outro.
Mas sobre seu amor pesa um interdito: ela é letrada e rica (isto é, tem uma casa); ele é analfabeto e pobre (perdeu a casa no terremoto). A família da moça proíbe o casamento. Ela não se dirige ao rapaz durante a filmagem. Ele, ao contrário, lhe diz coisas belíssimas.
A questão é tão elementar quanto eterna. O iraniano Abbas Kiarostami trabalha como um tapeceiro persa: trançando os fios com inteligência, precisão e delicadeza. "Através das Oliveiras" é um filme raro.
–Inácio Araujo 

Através das Oliveiras (Through the Olive Trees. Irã, 1994). Direção: Abbas Kiarostami. Estréia prevista para sexta, 6. Salas e horários a confirmar.
</TEXT>
</DOC>
<DOC>
<DOCNO>FSP950101-141</DOCNO>
<DOCID>FSP950101-141</DOCID>
<DATE>950101</DATE>
<CATEGORY>REVISTA_DA_FOLHA</CATEGORY>
<TEXT>
Não é bem um restaurante, o Dansk Corner. São apenas duas mesas, para quem prefere comer lá. Mas comer lá, embora possível, não é sua principal função. A casa nasceu, há um mês, como um delivery, um restaurante para entregas na região.
O cardápio é de fast food, com força nos sanduíches abertos, típicos da Escandinávia. São grandes canapés, montados sobre uma fina fatia de pão preto ou branco, com os ingredientes da smorgasbord (a mesa de frios dos países nórdicos).
O resultado é uma refeição leve, saborosa e bonita. Os pratos foram criados pelo chef islandês Jakob Jakobson, que conta com passagem de três anos pela centenária sanduicheria de Ida Davidsen, na Dinamarca.
À mesa, os escandinavos combatem o frio com o frio. Por isso, nenhum sanduíche do Dansk é quente. Suas coberturas podem ser salmão defumado (com mostarda doce ou curry), pastrami (com chucrute e pepino em conserva), carne cozida (com picles e raiz-forte) ou rosbife com molho agridoce, entre outras.
Rápido, o sanduíche que veio do frio é uma gostosa opção para este verão. –Josimar Melo

Dansk Corner. R. Baltazar Fernandes, 108, Morumbi, zona sul. Tels. 531-9733 e 543-2000 (pedidos). Segunda a sexta: 10h/19h. Sábado: 9h/15h.$
</TEXT>
</DOC>
<DOC>
<DOCNO>FSP950101-142</DOCNO>
<DOCID>FSP950101-142</DOCID>
<DATE>950101</DATE>
<CATEGORY>REVISTA_DA_FOLHA</CATEGORY>
<TEXT>
A exposição retrospectiva de Antonio Dias mostra 27 telas e 12 papéis, datados desde 1977. Ele expõe na Europa desde 1965. Talvez o mais internacional dos artistas brasileiros, é especial. Ele se diz estrangeiro, lá e cá. "Eu desconfio do quadro."
Um pintor que não gosta de pintar, cuja técnica "escarnece de si própria. E esta ironia atinge tanto o criador, quanto a obra e o seu consumo". Tem a astúcia do sujeito consciente de seu objeto.
Sua obra tem uma unidade de pensamento, uma liga alquímica-visual rara, na qual o ato de fazer se confunde com a fisiologia do olhar.
Seus símbolos –halteres disformes, ossos– e superfícies (em ouro, cobre, chumbo e grafite) são não-imagem, representação gráfica.
Suas influências são a "arte povera", o conceitual e a pintura, essa esfinge da modernidade. Antonio Dias é bom de enigmas, pode conferir.
–João Pedrosa

Antonio Dias. Paço das Artes. Av. da Universidade, 1, Cidade Universitária. Tel. 211-0682. Terça a sexta: 13h/20h. Sábado e domingo: 10h/18h. Até 8 de janeiro.
</TEXT>
</DOC>
<DOC>
<DOCNO>FSP950101-143</DOCNO>
<DOCID>FSP950101-143</DOCID>
<DATE>950101</DATE>
<CATEGORY>TV_FOLHA</CATEGORY>
<TEXT>
Era o que faltava. Estão abertas as inscrições para o concurso de beleza negra promovido pela agência de modelos New Company. A final será no descolado Balafon, no dia 10 de março de 1995. Para se inscrever, basta aparecer na agência ou no Balafon levando duas fotos (uma em close, outra de corpo inteiro). Lá, o candidato preenche uma ficha com dados pessoais e paga taxa de R$ 3.
O mercado brasileiro de modelos, escravo de um único padrão de beleza, não chega a consumir nem 10% do potencial de negros. Foi pensando nisso –e muito mais– que a ex-modelo Rita de Cassia Santos, 32, montou a New Company, responsável pela realização deste concurso. É uma grande oportunidade para quem quer trilhar os caminhos de moda, publicidade ou cinema –e um incentivo à beleza universal. Good trip. –Sonia Fabiano

New Company. Av. Rio Branco, 1.704, Campos Elíseos, região central. Tel. 222-1413. Balafon. R. Sergipe, 160, Higienópolis, zona oeste. Tel. 259-8341.
</TEXT>
</DOC>
<DOC>
<DOCNO>FSP950101-144</DOCNO>
<DOCID>FSP950101-144</DOCID>
<DATE>950101</DATE>
<CATEGORY>TV_FOLHA</CATEGORY>
<TEXT>
Marcos Palmeira (foto), Marcos Winter e Ilya São Paulo são os protagonistas de "Irmãos Coragem", novela das seis que estréia amanhã na Globo. O remake da obra de Janete Clair, exibida originalmente em 1970, é marcado por uma linguagem cinematográfica.
</TEXT>
</DOC>
<DOC>
<DOCNO>FSP950101-145</DOCNO>
<DOCID>FSP950101-145</DOCID>
<DATE>950101</DATE>
<CATEGORY>TV_FOLHA</CATEGORY>
<TEXT>
Arlete Montenegro não é Fernanda 
Ao contrário do que foi publicado na última edição do TV Folha, Arlete Montenegro não é o nome da atriz Fernanda Montenegro – que se chama Arlette Pinheiro Monteiro Torres. ERRAMOS
Travesso descansa nos EUA 
Antes de mergulhar de cabeça na produção da próxima novela do SBT, "Sangue do Meu Sangue", o diretor de teledramaturgia Nilton Travesso embarcou com toda a família para Orlando, EUA, onde passa o Réveillon. Levou junto um "concorrente", seu filho Marcelo, que vai assistir Jorge Fernando na direção de "A Próxima Vítima", novela de Sílvio de Abreu que substitui "Pátria Minha" na rival Rede Globo.

Xuxa exibe hits aos domingos 
Xuxa ganha espaço na programação dominical da Globo. Às 13h50, a apresentadora exibe seu "Xuxa Hits", que reprisa os principais sucessos musicais exibidos pelo "Xuxa Park" em 1994.

Bontempo é padre em 'Pupilas' 
O ator Roberto Bontempo gravou na última terça-feira suas primeiras aparições na novela "As Pupilas do Senhor Reitor", do SBT. Bontempo interpreta José, um padre conservador –e inimigo do Reitor, vivido por Juca de Oliveira– que é nomeado pelo bispo do Porto, insatisfeito com os serviços por demais liberais do Reitor. Em uma das cenas, o padre chega à aldeia e é recebido com festa pelas beatas locais. As gravações foram realizadas na cidade cenográfica do SBT, onde está construída a aldeia da novela.

Séries brasileiras têm reprise 
"Desejo", "Grande Sertão Veredas", "Agosto", "Anos Rebeldes", "As Noites de Copacabana" e "O Tempo e o Vento" são as séries brasileiras que a Globo condensou e vai reprisar este ano. A primeira a ir ao ar é "O Tempo e o Vento", que a emissora exibe a partir desta terça-feira, às 22h30.

Globo exibe seriados inéditos 
Enquanto não estréia sua nova programação, a Globo exibe séries inéditas. No dia 8, na faixa das 22h, estréia "Plantão Médico", série sobre o dia-a-dia em um pronto socorro urbano. Às 22h50, a emissora põe no ar o "Festival Charlie Chaplin", com sucessos do cineasta. Durante a semana, no horário das 16h45, a Globo mostra "Cobra" (segundas), "Model's" (terças), "M.A.N.T.I.S. - O Vingador" (quartas), "Operação Acapulco" (quintas) e "Thunder - Missão no Mar" (sextas).

Diretores reclamam de Vera 
A atriz Vera Fischer que se cuide. É crescente o mau humor de diretores da Globo em relação aos problemas que ela vem causando nas gravações da novela "Pátria Minha", em que vive a personagem Lídia Laport. A atriz vive se atrasando para as gravações e, há poucos dias, chegou a faltar para levar a filha Rafaella –que viajou para a Europa– ao aeroporto. Cansado de ter que esperar a atriz para gravar cenas em comum, o ator Tarcísio Meira –que interpreta Raul– agora só sai de casa quando a produção telefona avisando que Vera já está nos estúdios. O primeiro problema envolvendo Vera ocorreu quando a imprensa divulgou que ela quebrara o braço durante uma briga com o marido e ator Felipe Camargo. Insatisfeitos com a repercussão negativa do episódio, alguns diretores chegaram a defender seu afastamento definitivo da novela. O caso só foi resolvido quando o vice-presidente de operações da emissora, José Bonifácio de Oliveira Sobrinho, o Boni, decidiu só afastá-la do vídeo temporariamente, até que o caso esfriasse.
</TEXT>
</DOC>
<DOC>
<DOCNO>FSP950101-146</DOCNO>
<DOCID>FSP950101-146</DOCID>
<DATE>950101</DATE>
<CATEGORY>TV_FOLHA</CATEGORY>
<TEXT>
Estou chocada com a falta de respeito com a qual o jornal da CNT/Gazeta tratou uma manifestação de hare krishnas em São Paulo. Ao assistir a esse jornal no dia 21 de dezembro, às 20h45, foi com surpresa que vi o repórter se referir à passeata dos hare krishnas pela paz como "uma manifestação de carecas exigindo mais cabelo". Não é porque essa é uma religião de pouca representatividade que ela deve receber um tratamento menos respeitoso que qualquer outra. Falo isso como cristã. É lamentável que um noticiário que deveria informar imparcialmente consiga, ao mesmo tempo, não informar e não ser imparcial.

Denise Cristina S. Perrotti São Paulo, SP  

"Avallone critica demais o Telê" 
Acho que o apresentador do "Mesa Redonda", Roberto Avallone, anda criticando demais o técnico do São Paulo, Telê Santana. Algumas perguntas para o Avallone. Quem levou o tricolor a alcançar os títulos mais importantes do mundo? Quem fez vários jogadores indisciplinados entrarem na linha? Quem foi o ténico que você pediu para a seleção? Ele tem mais chances de conseguir títulos pelo São Paulo do que o seu programa de ter audiência.

Gilvete da Silva Lemos São Bernardo do Campo, SP
</TEXT>
</DOC>
<DOC>
<DOCNO>FSP950101-147</DOCNO>
<DOCID>FSP950101-147</DOCID>
<DATE>950101</DATE>
<CATEGORY>TV_FOLHA</CATEGORY>
<TEXT>
DALMO MAGNO DEFENSOR 
Especial para o TV Folha 
Planos econômicos sempre começam com o ministro da Fazenda em rede nacional de TV, sisudo, anunciando as medidas como se estivesse mobilizando tropas e convocando reservistas para expulsar o inimigo do solo pátrio.
No dia seguinte, a TV transmite a entrevista coletiva em que a "equipe" explica como o cidadão comum será afetado.
Depois, no noticiário, lá está o cidadão comum na fila do banco. O desconfiado quer ver se fizeram direito a conversão dos saldos, o xereta quer saber quanto sobrou depois do confisco, o exibido quer trocar rapidinho suas notas velhas pelo reluzente dinheiro forte.
Minha mulher assistia comigo aos anúncios dos planos, fazendo comentários de um sarcasmo atroz. Mas, quando a professora Maria da Conceição Tavares chorou ao falar do Cruzado, e ela percebeu um nó na garganta de seu marido economista, inventou algo para fazer na cozinha e saiu discretamente. Talvez para rir, mas foi muito gentil.
Apesar de não ter conhecimento técnico, ela chegava com sua lógica impecável às mesmas perguntas torturantes que eu me fazia, sem encontrar a resposta. Ao final, citava o filósofo africano Hardy Har-Har: "Não vai dar certo, Lippy."
No plano Collor, afinal, a unanimidade. Enquanto a ministra provava por alfa mais beta (a mais b é para os humildes) que seria melhor para nós ficarmos ano e meio sem o suado dinheiro da entrada do apartamento, fazíamos coro com a Lillian Witte Fibe. Rosnando.
</TEXT>
</DOC>
<DOC>
<DOCNO>FSP950101-148</DOCNO>
<DOCID>FSP950101-148</DOCID>
<DATE>950101</DATE>
<CATEGORY>TV_FOLHA</CATEGORY>
<TEXT>
por cento dos paulistanos com nível superior de escolaridade forneceram respostas incorretas quando inquiridos pelo Datafolha sobre qual série de TV gostariam de rever. Esse índice de erro cai para 36% entre os paulistanos com até segundo grau e 28% entre os que cursaram apenas o primeiro grau. 

9º 
foi o lugar ocupado pela novela "Tropicaliente" na lista das maiores audiências da Rede Globo na semana de 28 de novembro a 4 de dezembro, com um índice médio de audiência de 31%.

61 
por cento foi o índice médio de televisores ligados na Grande São Paulo na faixa noturna dessa mesma semana.
</TEXT>
</DOC>
<DOC>
<DOCNO>FSP950101-149</DOCNO>
<DOCID>FSP950101-149</DOCID>
<DATE>950101</DATE>
<CATEGORY>TV_FOLHA</CATEGORY>
<TEXT>
No dia da posse de Fernando Henrique Cardoso como presidente da República, a Cultura exibe o especial "FHC - A Trajetória de um Presidente". A emissora mostra a posse a partir das 16h. As redes Manchete e Bandeirantes começam antes, exibindo flashes a partir de 11h e 14h30, respectivamente. A Globo entra às 16h25.
</TEXT>
</DOC>
<DOC>
<DOCNO>FSP950101-150</DOCNO>
<DOCID>FSP950101-150</DOCID>
<DATE>950101</DATE>
<CATEGORY>TV_FOLHA</CATEGORY>
<TEXT>
Estréia amanhã às 18h o remake de um dos maiores sucessos de audiência de Janete Clair nos anos 70 
RONI LIMA 
Da Sucursal do Rio 
A Rede Globo começa a recontar amanhã, às 18h, as aventuras dos "Irmãos Coragem", uma das mais marcantes histórias da teledramaturgia brasileira, escrita por Janete Clair em 1970 (leia texto à pág. 7).
Com um elenco encabeçado por Marcos Palmeira, Letícia Sabatella e Marcos Winter, a nova versão está sendo atualizada por Dias Gomes (marido da autora, morta em 1983), Ferreira Gullar e Lilian Garcia.
A direção deste bangue-bangue cabloco está a cargo de Luiz Fernando Carvalho, responsável pelo último grande sucesso agreste da emissora, "Renascer" (1993).
Sua missão é a de mostrar, mais uma vez, a realidade do Brasil rural. Leia-se atores suados, corpos sujos de terra e rostos com barba por fazer.
O personagem central da trama é João Coragem, líder de uma família pobre que detém uma concessão de garimpo na cidade fictícia de Coroado.
Para interpretá-lo, a Globo não arriscou e escolheu Marcos Palmeira –consagrado por papéis semelhantes em "Pantanal" (Manchete) e "Renascer".
O trio dos irmãos Coragem é completado por Ilya São Paulo (Jerônimo) e Marcos Winter (o jogador de futebol Duda).
A disputa pelo poder é o fio condutor da trama, que envolve os irmãos Coragem e o coronel Pedro Barros (interpretado Cláudio Marzo, o Duda na primeira versão), dono de quase todos os garimpos da região.
A descoberta de um enorme diamante por João Coragem é o estopim do início da guerra entre eles.
Para apimentar o tom folhetinesco da história, a esquizofrênica Lara (Letícia Sabatella), filha do coronel vilão, se envolve com João.
Normalmente quieta e introvertida, Lara assume outras duas personalidades. No início da trama, incorpora Diana –uma cigana exagerada e sensual, que dorme com João Coragem. Depois, também vai personificar Márcia, espécie de síntese entre Lara e Diana.
"É um personagem maravilhoso, que me dá muitas coisas, me ensina muito", diz Letícia. "Os personagens, a concepção de cena, tudo é muito profundo nessa novela."
Palmeira elogia a estética cinematográfica de Luiz Fernando Carvalho (leia texto ao lado) e diz que a novela pode combater a idéia pré-concebida segundo a qual a TV aliena.
"Todos aqui estão buscando essa verdade, de pisar no chão, mexer na terra, pra mostrar como vive o povo. Televisão pode ser um veículo de divertimento com qualidade", diz o ator.
Suzana Faini –que na primeira versão viveu Cema e agora é Dalva– faz coro. Ela acha que o cenário de Curralinho (25 quilômetros de Diamantina, MG) escolhido para criar Coroado tem o mérito de mostrar o interior do país, "essa coisa rude".
</TEXT>
</DOC>
<DOC>
<DOCNO>FSP950101-151</DOCNO>
<DOCID>FSP950101-151</DOCID>
<DATE>950101</DATE>
<CATEGORY>TV_FOLHA</CATEGORY>
<TEXT>
Da Redação 
Não foi por acaso que a Globo escolheu "Irmãos Coragem" para comemorar seus 30 anos e, de quebra, tentar revitalizar sua faixa das seis.
Exibida entre 29 de julho de 1970 a 15 de julho de 1971, a novela é representativa do período no qual a emissora carioca tomou a liderança absoluta da audiência e da passagem do cetro de rainha das novelas de Glória Magadan para Janete Clair.
Os irmãos do título foram interpretados por Tarcísio Meira (João), Cláudio Marzo (Duda) e Cláudio Cavalcanti (Jerônimo). Glória Menezes e Regina Duarte reviviam com os dois primeiros, respectivamente, os mais famosos pares românticos da Globo à época.
Para o papel da índia Potira (o amor proibido de Jerônimo) foi escolhida uma estreante em novelas, Lúcia Alves. Outra que debutava no gênero era Sônia Braga.
A caracterização de Duda como jogador de futebol tinha sua razão de ser: o Brasil se sagrara tricampeão mundial em junho, no México –coincidentemente a novela volta depois de mais um título.
A cidade cenográfica foi destruída em meio à novela por um temporal. Na trama, Coroado foi queimada pelo coronel Barros (Gilberto Martinho) no último capítulo.
</TEXT>
</DOC>
<DOC>
<DOCNO>FSP950101-152</DOCNO>
<DOCID>FSP950101-152</DOCID>
<DATE>950101</DATE>
<CATEGORY>TV_FOLHA</CATEGORY>
<TEXT>
O cinema moderno começa, oficialmente, com "Cidadão Kane". Menos pela famosa utilização da profundidade de campo, ou pelos ângulos inesperados, ou pelos planos-sequência, ou pela narrativa em "flash back". Na verdade, o que faz de "Kane" um monumento é a dimensão da dúvida. À certeza plácida do classicismo, Orson Welles substituiu a pergunta: o que é um homem? E demonstrou que conhecer este homem era impossível. Quanto mais entramos na história de Charles Foster Kane, o magnata da imprensa, mais ele nos escapa, ao mesmo tempo em que a questão suscitada –o que é o cinema?– se torna mais presente. Num dia cheio de filmes significativos, "Kane" é, de longe, o que mais conta. Os outros são decorrência. (IA)

CIDADÃO KANE (Citizen Kane). EUA, 1941, 119 min. Direção: Orson Welles. Com Orson Welles, Joseph Cotten. P&B. Legendado. Na Cultura.
</TEXT>
</DOC>
<DOC>
<DOCNO>FSP950101-153</DOCNO>
<DOCID>FSP950101-153</DOCID>
<DATE>950101</DATE>
<CATEGORY>TV_FOLHA</CATEGORY>
<TEXT>
Jonathan Demme estava longe de ficar famoso, ganhar Oscar e tudo mais, quando fez este filme, talvez o mais delicado de sua carreira. Janet Margolin é uma jovem judia que busca vingar-se dos descendentes que causaram a desgraça de uma antepassada. Roy Scheider é uma de suas vítimas. Mas esses seres que tudo leva à oposição, apaixonam-se. Se tudo até ali era estranho, a paixão tornará as coisas ainda mais estranhas. O que anunciava um giro de 180o, desdobra-se: é toda a volta, os 360o que se deixam ver, ao longo de uma busca em que a dor se insinua a cada passo. É o tipo de filme bom para quebrar o pescoço de alguém menos talentoso. Mas perfeito para dar a real dimensão de um cineasta. E, no caso, de seus atores. (IA).

O ABRAÇO DA MORTE (The Last Embrace). EUA, 1979, 103 min. Direção: Jonathan Demme. Com Roy Scheider, Janet Margolin. Na Globo.
</TEXT>
</DOC>
<DOC>
<DOCNO>FSP950101-154</DOCNO>
<DOCID>FSP950101-154</DOCID>
<DATE>950101</DATE>
<CATEGORY>CADERNO_ESPECIAL_-_ANOS_FHC</CATEGORY>
<TEXT>
Mario Vargas Llosa está no centro do "Roda Viva" desta semana. Em programa previamente gravado, o escritor peruano fala sobre sua infância na cidade de Piura, sobre sua candidatura à Presidência de seu país e sobre "O Peixe na Água", seu mais recente livro, inspirado em sua incursão pela política.
RODA VIVA  Cultura, 22h30.

Cultura muda sua programação noturna 
A Cultura muda a sua programação infanto-juvenil noturna a partir desta semana. Às 18h volta a ser exibida a série "Contos de Fada". Kevin e seus "Anos Incríveis" voltam a aparecer de segunda a sexta no vídeo, na faixa das 20h. "Confissões de Adolescente" passa a ser reprisada aos sábados, às 20h.
ANOS INCRÍVEIS  Cultura, 20h.

Machline recebe João Bosco 
O compositor e violonista João Bosco e a cantora Simone são os convidados desta semana de José Maurício Machline, apresentador e produtor do programa "Por Acaso", da Rede Bandeirantes.
POR ACASO  Bandeirantes, 23h45.

Série explica o mundo da moda 
A Cultura exibe a série em três episódios "O Mundo da Moda", com depoimentos de, entre outros, Armani (foto) e Gaultier.
O MUNDO DA MODA  Cultura, 22h30.

Band transmite futebol americano 
A Rede Bandeirantes começa a exibir os jogos dos "play offs" (finais) da temporada de 1994 da liga de futebol americano.
FAIXA NOBRE DO ESPORTE  Bandeirantes, 19h15.

Show inédito dos Cranberries 
A MTV exibe show inédito no qual a banda irlandesa Cranberries toca "Linger", "Dreams", "Zombies", "How" e "So Cold in Ireland".
CRANBERRIES  MTV, 21h15.

Novo filme de Maitê na MTV 
O "Semana Cine" invade os bastidores de "16060", primeiro longa de Vinícius Mainardi, com Maitê Proença (foto) e Antonio Calloni no elenco.
SEMANA CINE  MTV, 19h.

Hancock homenageia Davis 
O "Free Jazz in Concert" traz uma homenagem a Miles Davis. O programa começa com a última apresentação do trompetista no Festival de Jazz de Montreux, em 91, ao lado de Quincy Jones. Nos blocos seguintes, ex-parceiros do mestre executam algumas de suas principais composições. Herbie Hancock (foto) e Wayne Shorter tocam "Footprints". John McLaughlin interpreta "Mother Tongues" e Chick Corea apresenta "Sicily".
FREE JAZZ IN CONCERT  Bandeirantes, 0h45.

Os vikings invadem a TV 
A Cultura passa a exibir a série norte-americana em seis episódios "Jornal da História", que dramatiza eventos dos séculos 11 a 15 em locações na Inglaterra, Espanha, Turquia e Índia. No primeiro programa, as conquistas vikings.
OS VIKINGS  Cultura, 21h.
</TEXT>
</DOC>
<DOC>
<DOCNO>FSP950101-155</DOCNO>
<DOCID>FSP950101-155</DOCID>
<DATE>950101</DATE>
<CATEGORY>CADERNO_ESPECIAL_-_ANOS_FHC</CATEGORY>
<TEXT>
JOSIAS DE SOUZA 
Diretor-executivo da Sucursal de Brasília 
Fernando Henrique Cardoso, 63, toma posse hoje, às 16h30, como o novo presidente do Brasil em meio a uma atmosfera de grande otimismo, que obscurece até mesmo a ascensão dos 27 novos governadores que também chegam ao poder ungidos pela maior eleição a que o país assistiu.
Esse otimismo contamina o próprio tucano. Ele está certo de que os  Anos FHC, que agora se iniciam, farão vingar uma tese lançada em 1941, no título do último livro do austríaco Stefan Zweig. "Brasil, País do Futuro" é como se chama a obra, pendente de comprovação há 53 anos. Leitor de Zweig, Fernando Henrique não é o único a crer que o êxito de sua gestão emprestará às palavras do escritor ares de profecia.
Pesquisa Datafolha mostra que é alto o teor de esperança no novo governo. 70% dos brasileiros acham que seu desempenho será ótimo ou bom. O percentual supera os 44,08% que Fernando Henrique obteve nas urnas -34,3 milhões de votos. Mas não se está diante de um índice inédito. Quando Fernando Collor de Mello assumiu em 90, 71% da população apostava no sucesso de seu governo. Para tentar manter essa taxa de expectativa, o tucano se impôs uma meta.
Sua prioridade será zelar pela saúde do Plano Real. Está convencido de que o patrimônio político que começou a amealhar no dia em que aceitou ser ministro da Fazenda de Itamar Franco virará pó se a estabilidade da moeda for, de algum modo, abalada.
A crise verificada no México abriu fissuras na convicção de alguns integrantes de sua equipe. Em público, continuam destilando confiança. Sob reserva, cultivam dúvidas aflitas. Não que considerem o Brasil à beira de um novo colapso. Bem ao contrário. Mesmo longe dos gravadores, vangloriam-se do fato de estar montados em indicadores bastante confortáveis.
Mencionam sobretudo a inflação de 2,19% registrada em dezembro, as reservas cambiais superiores a US$ 40 bilhões e a perspectiva de crescimento do PIB projetado em 4,5%. O que os deixa inquietos é a fragilidade da estabilização da economia brasileira. O processo de tonificação da moeda depende de reformas que o governo só submeterá ao Congresso a partir de fevereiro.
São temas espinhosos, tais como a mudança no sistema previdenciário, a flexibilização de monopólios e a reformulação do sistema tributário. Ao ser entronizado no cargo, em solenidade no plenário da Câmara, o novo inquilino do Planalto jurará respeito a uma Constituição que abomina.
O presidente FHC quer fatiar o texto constitucional que o senador constituinte Fernando Henrique Cardoso ajudou a redigir. Quer "desconstitucionalizar" o cotidiano do país. E deseja fazê-lo num prazo máximo de seis meses.
Por isso rateou uma parte da Esplanada dos Ministérios com legendas como o PFL, o PTB e o PMDB. Por isso também deve aceitar indicações políticas para postos de segundo e terceiro escalões. FHC precisa de apoio no Congresso. Também aqui deseja evitar um erro que o próprio Collor hoje admite ter cometido.
Ao seu tempo, o presidente do impeachment imaginava que conseguiria estabelecer ligações diretas com a sociedade. Deu de ombros para a oposição da academia e dos sindicatos. Indispôs-se com os políticos. Quanto a FHC, é difícil identificar-lhe os opositores. O PDT e o PT, que poderiam representar focos de resistência, mostram-se desarticulados. Buscam o próprio eixo. O mesmo se dá com o PMDB quercista, mais próximo da luta por cargos do que de qualquer trincheira oposicionista.
Também a atmosfera política parece conspirar a favor do tucano. Ele assume para dar continuidade à própria obra. Na prática, fez-se sucessor de si mesmo a partir de um governo que, até a sua entrada, era visto como desastrado. Entre os auxiliares mais íntimos de FHC, aliás, Itamar Franco não é visto como o presidente que fez o sucessor. Afirma-se que FHC é que fez o antecessor.
Anuncia-se, de resto, a instalação de uma nova era em Brasília, a era do processo. Nada de mudanças bruscas, de planos urdidos em segredo. Pé ante pé, o Brasil de Fernando Henrique se pautará pelo planejamento. A começar da própria posse. Anteciparam-se providências mais duras, como a intervenção no Banespa e no Banerj, para evitar comparações com o início da gestão Collor, marcada pela fase dos pacotes bombásticos.
Com o mesmo objetivo, adiaram-se outras providências, entre elas a reforma administrativa, hoje limitada à extinção de duas pastas e alguns poucos ajustes. O resto, Fernando Henrique (38º presidente do Brasil) planeja executar aos poucos. Terá até 31 de dezembro de 1999 para construir o Brasil imaginado por Stefan Zweig.
Além do desafio econômico, terá pela frente o conjunto de problemas representados pela hipoteca social brasileira. Algo que, durante a campanha, resumiu nos cinco dedos da mão: emprego, agricultura, segurança, saúde e educação.
</TEXT>
</DOC>
<DOC>
<DOCNO>FSP950101-156</DOCNO>
<DOCID>FSP950101-156</DOCID>
<DATE>950101</DATE>
<CATEGORY>CADERNO_ESPECIAL_-_ANOS_FHC</CATEGORY>
<TEXT>
1931
Nasce em casa,  no Rio,  a  18 de junho. Seu pai, Leônidas Cardoso, era coronel do Exército. Aos oito anos, sua família se transfere para São Paulo. Em 1946, é eleito diretor-geral do grêmio estudantil do Colégio São Paulo. No ano seguinte, é escolhido vice-presidente do grêmio e em 1948 volta a ser diretor-geral

1949
Ingressa nas Ciências Sociais da USP. Forma-se em 1952 e começa a lecionar na universidade. Casa-se com Ruth em 1953, e seu primeiro filho, Paulo, nasce em 1954. Publica "Cor e Mobilidade Social em Florianópolis", em 1960. Em 1961 conclui o doutorado ("Capitalismo e Escravidão no Brasil Meridional")

1964
Após o Movimento Militar, tem sua prisão pedida pelos militares. FHC se exila  no Chile, onde trabalha em um instituto da Cepal, organismo ligado à ONU. Publica "Dependência e Desenvolvimento na América Latina" em 1967. Retorna  ao Brasil em 1968.  Neste ano,  ganha o concurso para a cátedra de Ciência Política da USP.

1969
É aposentado compulsoriamente com base no AI-5. No mesmo ano, funda o Cebrap. Publica "O Modelo Político Brasileiro e Outros Ensaios" em 72. Em 73, critica a guerrilha de esquerda e defende a oposição legal ao regime militar. Publica "Autoritarismo e Democraticação" em 75

1978
Concorre a uma vaga  no  Senado pelo MDB paulista. Obtém 1.272.416 votos e torna-se suplente de Franco Montoro. Assume a vaga no Senado em 1983. Em 1985, concorre à Prefeitura de São Paulo. Senta-se na cadeira do prefeito antes da eleição, mas é derrotado por Jânio Quadros (obtém 1.431.300 votos)

1986
Com o sucesso do Plano Cruzado, é eleito senador pelo PMDB de São Paulo, com 6.223.995 de votos (fica em segundo lugar). Torna-se líder do partido  no Senado. Rompe com o governo José Sarney em 1988 e deixa o PMDB e funda o PSDB. Em 1989, apóia Lula, do PT, contra Collor no segundo turno da eleição presidencial

1990
Em março, elogia o Plano Collor. Em abril de 1992, um mês antes da abertura da CPI do Collorgate, defende a entrada do PSDB no governo collorido.  Era cotado para o Itamaraty. Em outubro, logo após o impeachment, é nomeado ministro das Relações Exteriores por Itamar Franco.  Durante sua gestão, visita diversos países

1993
Assume o Ministério da Fazenda em maio. Cria o Fundo Social de Emergência e coordena a elaboração do Real, lançado em 1º de junho de 1994. Deixa o governo em abril para se candidatar à Presidência pelo PSDB. Obtém o apoio do PFL e do PTB. É eleito no 1º turno com 34.377.198 (54,3% dos válidos), tendo o senador Marco Maciel (PFL) como vice.
</TEXT>
</DOC>
<DOC>
<DOCNO>FSP950101-157</DOCNO>
<DOCID>FSP950101-157</DOCID>
<DATE>950101</DATE>
<CATEGORY>CADERNO_ESPECIAL_-_ANOS_FHC</CATEGORY>
<TEXT>
Presidente eleito fez lista de convidados especiais, decidiu promover o show de música e escolheu artistas 
Da Sucursal de Brasília 
Fernando Henrique Cardoso cuidou pessoalmente de vários detalhes de sua posse hoje, como uma lista de 17 convidados especiais, todos estrangeiros.
Entre eles estão o ex-primeiro-ministro da França Michel Rocard e o diretor-geral da Unesco (Organização das Nações Unidas para a Educação, Ciência e Cultura), o francês Federico Mayor.
Sete confirmaram presença –inclusive Rocard. Vêm também dois representantes do Congresso Mundial Judaico, que tem sede nos Estados Unidos, e dois membros do Club Brésil, na França.
Da América Latina, FHC escolheu para integrar a lista, entre outros, o presidente eleito do Uruguai, Julio Maria Sanguinetti e o secretário-geral da Comissão Sul-Americana de Paz, o chileno Carlos Contreras.
Uma outra lista, de 300 nomes, também passou pelo crivo de FHC. São personalidades ligadas ao mundo intelectual, artístico e social do Brasil.
Também estão convidados todos os governadores atuais e eleitos, ex-presidentes da República e os candidatos à Presidência, que disputaram com ele a eleição de 3 de outubro. FHC só não convidou Fernando Collor.
A posse 
A cerimônia de posse acontecerá às 16h30, no Congresso Nacional. Às 17h30, no Planalto, haverá a passagem da faixa presidencial.
O público poderá ver o novo presidente no desfile que fará em carro aberto, a partir das 16h15, entre a catedral de Brasília e a rampa do Congresso.
A passagem da faixa, no parlatório do Planalto, poderá ser vista por quem estiver na praça dos Três Poderes –onde também haverá um show de música popular e queima de fogos.
Para a recepção noturna no Itamaraty foram distribuídos 2.800 convites, muitos para casais. Estima-se o comparecimento de 5.000 a 5.500 pessoas, que serão atendidas por 300 garçons.
Entre festa, hospedagem dos convidados estrangeiros, diárias de servidores e outras despesas, o Itamaraty está gastando R$ 3 milhões.
O governo brasileiro arca com as despesas de três pessoas em cada uma das 80 delegações que devem vir do exterior, o que significa hospedagem e transporte para quase 250 convidados.
O Itamaraty reservou 300 apartamentos nos hotéis de Brasília e alugou 120 carros para a posse. Uma suíte do Hotel Nacional foi reservada para o presidente de Cuba, Fidel Castro, mas até ontem havia mistério sobre sua presença.
Cuidados especiais 
É do presidente eleito a idéia de um show popular na praça dos Três Poderes.
Os músicos que vão tocar na festa oficial também foram escolhidos por FHC. São Daniela Mercury, Herbert Vianna, Dominguinhos, Oswaldinho e Hermeto Paschoal.
A animação da festa da também conta com um grupo de chorinho, comandado pelo flautista Carlos Poyares, e um conjunto da Escola de Música de Brasília.
Também foi Fernando Henrique quem determinou o traje para a festa do Itamaraty: rigor completo, o que significa smoking ou vestido longo.
FHC não quer ficar isolado durante essa festa. Rejeitou a sugestão de se sentar numa mesa principal, ao lado da mulher, Ruth, e de seu vice, Marco Maciel, e a mulher, Ana Maria.
A idéia é que FHC "abra o buffet", acompanhando o início do serviço dos 19 pratos frios, 12 pratos quentes e dez sobremesas diferentes. Depois, já avisou que vai circular entre os cerca de 5.000 convidados em dois andares de salões do Itamaraty.
A segunda festa é mais reservada. Será uma homenagem a Itamar Franco, dentro do Palácio do Planalto, antes de ele descer a rampa. Terá a participação de seus ministros e amigos.
A última cerimônia terá caráter ecumênico, desprezado nas posses anteriores. Está programado um "culto inter-religioso" para o meio-dia de amanhã, na Catedral de Brasília. Na celebração, falarão Marco Maciel, Ruth Cardoso, Ana Maria Maciel e Fernando Henrique Cardoso nesta ordem.
(William França, Silvana de Freitas e Gutemberg de Souza)
</TEXT>
</DOC>
<DOC>
<DOCNO>FSP950101-158</DOCNO>
<DOCID>FSP950101-158</DOCID>
<DATE>950101</DATE>
<CATEGORY>CADERNO_ESPECIAL_-_ANOS_FHC</CATEGORY>
<TEXT>
FERNANDO RODRIGUES 
Enviado especial a Brasília 
Dos 114 países representados na cerimônia de posse, 51 decidiram escalar seus embaixadores em Brasília para cumprimentar FHC. 
Foi oficializada e divulgada sexta-feira a presença de Fidel Castro, de Cuba. Com isso, subiu para 11 o número de chefes de Estado presentes.
FMI 
A última autoridade estrangeira que cumprimentará Fernando Henrique Cardoso hoje pela posse será José Fajgenbaun, representante do Fundo Monetário Internacional (FMI) na cerimônia.
Em julho de 91, Fajgenbaun foi criticado pelo ex-presidente Fernando Collor. Teve de deixar suas funções de representante da missão do FMI que estava no Brasil.
Fajgenbaun havia dito que reformas estruturais no Brasil exigiam "mudanças na Constituição". Collor respondeu: "Ele devia é reformar a casa dele".
A colocação de Fajgenbaun no final da lista de cumprimentos –há representantes de 114 países e de 11 organizações internacionais– não representa uma represália.
A lista dos cumprimentos obedece à ordem de confirmação das autoridades convidadas. Quem confirmou o nome de seu representante primeiro, cumprimentará FHC primeiro.
Em 3 de outubro, quando foi eleito, FHC recebeu o primeiro cumprimento pela vitória do FMI. Atendeu em sua casa em São Paulo um telefonema de Michel Camdessus, diretor-gerente do fundo.
Amanhã, antes dos organismos internacionais, todos os representates de países estrangeiros apertarão a mão do presidente brasileiro. Há duas exceções. A União Européia e a Organização para a Libertação da Palestina (OLP) foram colocadas na lista de países.
Juntos, os países dos 11 chefes de Estado presentes têm um Produto Interno Bruto (PIB, soma de tudo o que produzem em um ano) de US$ 294,9 bilhões. O valor é de 92, os últimos disponíveis. Isso equivale a 71,77% do PIB brasileiro.
Um representante real estará na cerimônia. O príncipe Abdul Aziz Althenaian Al-Saud representará a Arábia Saudita.
O Grupo dos Sete (G-7), que reúne os países mais ricos do planeta, enviou os seguintes representantes: Janet Reno (EUA, ministra da Justiça); Sheila Copps (Canadá, ministra do Meio Ambiente); Burkhard Hirsch (Alemanha, vice-presidente do Parlamento Federal); Vincenso Trantino (Itália, vice-ministro das Relações Exteriores); Michael Portillo (Grã-Bretanha, ministro do Trabalho); Simone Weil (França, ministra dos Assuntos Sociais, da Saúde e Urbanos) e Keizo Obuchi (Japão, deputado do Partido Liberal Democrático).
Dos representantes do G-7, FHC pretende conceder audiências separadas amanhã para Janet Reno, Keizo Obuchi e Simone Weil. Também deverá receber Narcís Serra Serra, vice-presidente do Governo da Espanha, e José Angel Gurría, secretário das Relações Exteriores do México.
</TEXT>
</DOC>
<DOC>
<DOCNO>FSP950101-159</DOCNO>
<DOCID>FSP950101-159</DOCID>
<DATE>950101</DATE>
<CATEGORY>CADERNO_ESPECIAL_-_ANOS_FHC</CATEGORY>
<TEXT>
LILIANA LAVORATTI;  VIVALDO DE SOUZA 
Da Sucursal de Brasília 
A principal preocupação da equipe econômica nos primeiros meses do próximo governo é aumentar a arrecadação e reduzir os gastos públicos para que o Plano Real deixe de se sustentar na política cambial, que deverá sofrer ajustes graduais.
O governo não pretende baixar nenhum pacote de medidas em janeiro, disse à Folha o novo ministro da Fazenda, Pedro Malan. FHC deve anunciar em janeiro regras para agilizar o programa de privatização.
As projeções feitas pela área econômica mostram que a perda de receita de R$ 4,5 bilhões com o fim do IPMF (Imposto Provisório sobre Movimentação Financeira) será compensada com recursos da privatização e do aumento do IR (Imposto de Renda) das pessoas jurídicas.
As medidas para equilibrar as contas também vão incluir cortes no Orçamento. A previsão da atual equipe econômica é de que a proposta orçamentária para 1995, enviada ao Congresso, contém um déficit de US$ 11 bilhões.
A área econômica continuará administrando o Plano Real para manter sob controle a inflação. A meta é não ultrapassar, durante o ano, a taxa de 24% a 26%, uma média mensal entre 1,8% e 1,9%.
A nova equipe já decidiu que a desindexação da economia será gradual. A Ufir, que era mensal, vai ser trimestral a partir de hoje. A intenção do governo é extingui-la somente em 1996.
A lei 8.880 prevê que os salários terão reposição pelo IPC-r até junho de acordo com sua data-base. A equipe chegou a estudar propostas de antecipar para janeiro o fim da reposição pelo IPC-r dos salários, mas o presidente Itamar Franco não concordou.
Se optar por mexer nas regras salariais, FHC pode editar ainda em janeiro uma MP (medida provisória) antecipando para fevereiro o fim da reposição automática pelo IPC-r. Caso isso ocorra, o IPC-r acumulado de julho a dezembro será repassado na data-base para quem ainda não o recebeu.
O governo não deve mudar a TR no primeiro semestre. O problema da TR, segundo Malan, é que o custo dos empréstimos para habitação e agricultura têm de estar casados com a remuneração paga aos aplicadores.
O governo deverá implantar, ainda no primeiro trimestre, o índice de inflação dessazonalizado -sem o efeito da alta ou queda dos preços causada por entressafra e problemas climáticos.
Outra preocupação no primeiro semestre é a renegociação das dívidas dos Estados. A negociação vai ser feita caso a caso, mas a proposta com maior probabilidade de ser adotada é a federalização das dívidas mobiliária (com títulos) dos Estados e municípios.
O problema de saneamento financeiro dos bancos estaduais já começou a ser resolvido na semana passada, com a intervenção no Banespa, de São Paulo, e no Banerj, do Rio de Janeiro.
</TEXT>
</DOC>
<DOC>
<DOCNO>FSP950101-160</DOCNO>
<DOCID>FSP950101-160</DOCID>
<DATE>950101</DATE>
<CATEGORY>CADERNO_ESPECIAL_-_ANOS_FHC</CATEGORY>
<TEXT>
JOSÉ ROBERTO DE TOLEDO 
Enviado especial a Brasília 
FERNANDO GODINHO 
Da Sucursal de Brasília 
Os poderes de José Serra no governo FHC vão aumentar ainda mais. Por decisão do presidente eleito, o contingenciamento do Orçamento de 95 ficará a cargo da SOF (Secretaria de Orçamento e Finanças), subordinada ao Ministério do Planejamento.
Ao gerenciar o contingenciamento, Serra passará a ter poder direto sobre os demais ministérios –pois influirá em suas prioridades– e sobre as negociações com os congressistas, que assinaram emendas ao Orçamento.
Até hoje, esta função era exercida pela Secretaria do Tesouro Nacional, que é vinculada ao Ministério da Fazenda, de Pedro Malan.
Segundo o novo titular da SOF, Waldemar Giomi, será necessário cortar entre R$ 10 bilhões e R$ 13 bilhões do Orçamento para se atingir a meta de eliminar o déficit operacional no próximo ano.
Giomi diz que há duas maneiras de se fazer isto: vetos presidenciais ao Orçamento já aprovado pelo Congresso e contingenciar as verbas dos ministérios.
O contingenciamento significa que a SOF determinará um limite de verbas para os programas prioritários de cada ministério. Isto será feito através de ajustes no Orçamento e não mais na liberação na boca do caixa, como acontece hoje no Tesouro Nacional.
A SOF tem melhores condições de administrar o contingenciamento porque tem acesso às previsões de despesa dos ministérios, bem como à lista de prioridades estabelecida pelo governo.
A necessidade de contingenciar o Orçamento de 95 é reforçada pelo abono de R$ 15 a ser dado em janeiro para quem ganha salário mínimo. O impacto previsto é de R$ 300 milhões por mês, ou R$ 3,6 milhões no ano –se o abono for incorporado ao salário.
Além disso, entre as receitas previstas no Orçamento estão R$ 4,3 bilhões que devem vir da privatização. É um dinheiro que não está garantido. Para alcançar esta meta, o governo precisaria, por exemplo, vender a Companhia Vale do Rio Doce.
As estatais que estão em processo mais avançado de privatização não somam a quantia necessária. A Ecelsa, por exemplo, está avaliada em R$ 400 milhões, e as empresas petroquímicas, em R$ 600 milhões.
As possibilidades de corte vislumbradas por Giomi são, por ora, duas: corte de parte dos gastos em investimentos e custeio previstos nos R$ 2 bilhões de emendas de parlamentares e um corte linear no aumento de despesas gerado pela correção monetária do Orçamento.
O Orçamento/95 foi reajustado em 20% (por conta da inflação). Mas Giomi afirma que os produtos que puxaram o IPC-r para cima (aluguel, carne, hortifrútis) não afetam as despesas dos ministério e, portanto, não haveria necessidade de receberem reajuste integral.
</TEXT>
</DOC>
<DOC>
<DOCNO>FSP950101-161</DOCNO>
<DOCID>FSP950101-161</DOCID>
<DATE>950101</DATE>
<CATEGORY>CADERNO_ESPECIAL_-_ANOS_FHC</CATEGORY>
<TEXT>
Problema do presidente eleito será conciliar interesses dos partidos aliados 
LUCIO VAZ;  EUMANO SILVA 
Da Sucursal de Brasília 
O presidente eleito, Fernando Henrique Cardoso, terá uma oposição inexpressiva numericamente, com pouco mais de cem parlamentares. Sua maior dificuldade no Congresso será conciliar os interesses dos partidos aliados.
A oposição mais definida ficará restrita aos partidos de esquerda, como PT, PC do B e PDT. Mesmo o PSB e o PPS, que apoiaram Luiz Inácio Lula da Silva (PT), admitem até participar do governo.
"A existência de oposição, com partidos não subordinados ao governo, é uma necessidade do processo democrático", afirmou o líder do PT na Câmara, José Fortunati (RS).
O líder do PDT na Câmara, Luiz Salomão (RJ), afirma que o partido vai definir a estratégia de ação numa reunião em Brasília, dias 17 e 18, com a presença do candidato derrotado do partido à Presidência, Leonel Brizola. Mas já está certo que a linha será de oposição.
Na opinião do presidente do PPS, senador eleito Roberto Freire (PE), a oposição somente ficará definida depois que FHC enviar as propostas de reforma ao Congresso.
Freire avalia que haverá uma oposição "pendular", que vai mudar em função do tipo de discussão.
"Se o presidente propuser uma reforma tributária progressiva, por exemplo, a oposição poderá ser da direita", disse.
O PPR promete manter uma posição de independência, analisando cada um dos projetos enviados por FHC ao Congresso. Mas a tendência do partido é aprovar as reformas propostas pelo novo governo.
O vice-presidente nacional do PPR, deputado Delfim Netto (SP), afirma que o PPR é um partido de oposição. E considera isso positivo para FHC: "O governo precisa de oposição para funcionar".
Delfim acrescenta, porém, que o PPR vai analisar os projetos do governo, apresentar sugestões e votar favoravelmente quando concordar com as propostas finais.
Aliados sem cargos 
As dificuldades com os aliados começam no relacionamento com os partidos que não receberam ministérios. É o caso do PP e do PL. A qualidade do apoio destes partidos dependerá dos cargos de segundo e terceiro escalões que receberem.
O líder do PL na Câmara, Valdemar Costa Neto (SP), afirma que o partido espera pelos cargos do governo: "Queremos crescer. E, para crescer, temos que participar do poder", diz.
Ele justifica por que o partido precisa dos cargos: "Temos que ter espaço para poder atender aos nossos prefeitos e deputados estaduais. Se não temos nada, vamos oferecer o quê?"
Ainda há risco de focos de oposição mesmo nos partidos que compõem o conselho político de FHC (PSDB, PFL, PMDB e PTB). Nesses partidos, FHC pode perder o apoio de parlamentares insatisfeitos com a distribuição dos cargos federais nos Estados.
O futuro presidente também terá de atender a interesses de parlamentares que se organizam fora dos partidos. O melhor exemplo é a bancada ruralista, que tem membros em todos os partidos e sempre condiciona seus votos ao atendimento de suas reivindicações específicas.
</TEXT>
</DOC>
<DOC>
<DOCNO>FSP950101-162</DOCNO>
<DOCID>FSP950101-162</DOCID>
<DATE>950101</DATE>
<CATEGORY>CADERNO_ESPECIAL_-_ANOS_FHC</CATEGORY>
<TEXT>
Novo chefe da Casa Civil diz que o fim de algumas pastas e a reforma do Estado começam a ser debatidos na sexta 
Há, no governo, o compromisso de que o Estado deve ser reduzido ao tamanho necessário 
Da Sucursal de Brasília 
Clóvis Carvalho, 54, novo chefe da Casa Civil, disse à Folha que o governo não desistiu de fazer uma ampla reforma administrativa.
Carvalho afirmou que os ministros do governo de Fernando Henrique Cardoso (PSDB) tomam posse hoje informados de que muitas pastas podem ser extintas.
A reforma do Estado será tema da primeira reunião de trabalho da equipe com o presidente Fernando Henrique. O encontro está marcado para acontecer sexta e sábado que vem.
A reforma começa hoje, com uma medida provisória que extingue, como previsto, as pastas da Integração Regional e do Bem-Estar Social. As demais mudanças virão num "processo", a ser iniciado ainda em 95.
Abaixo, os principais trechos da entrevista à Folha, concedida pelo novo chefe da Casa Civil na última sexta-feira:
Folha – Quando anunciou os nomes dos ministros, o presidente Fernando Henrique Cardoso apresentou Clóvis Carvalho como o seu "segundo". O que vai fazer o segundo homem da República?
Clóvis Carvalho – O presidente não deseja ter um agregado de ministros, mas um governo afinado, coordenado, com todos voltados para uma ação conjunta. E a Casa Civil fará essa coordenação das ações de governo.
Folha – O sr. é um político ou um técnico?
Carvalho – Eu sou um cidadão consciente das responsabilidades da vida em sociedade, que pressupõe o exercício da política. Mas me considero um técnico.
Folha – O sr. está mais para Golbery ou para Henrique Hargreaves?
Carvalho – Talvez esteja mais para Golbery. Ainda não sei distinguir. Mas o condutor político será o próprio presidente.
Muito provavelmente o conteúdo de ação política da Casa Civil será menor do que em outros governos. Minha característica será muito mais na direção da integração do governo.
Folha – O relacionamento do governo federal com o Congresso Nacional permanece entre as atribuições da Casa Civil?
Carvalho – Sim. Isso não será modificado.
Folha – O presidente disse que o lema do governo seria muito trabalho e pouca fofoca. Uma das fofocas em voga é a de que Clóvis Carvalho é ligado a José Serra. Isso é verdade?
Carvalho – Clóvis Carvalho é ligado a José Serra, a Paulo Renato, é ligado a Fernando Henrique, a Mário Covas, a Tasso Jereissati. Sou um cara que vive muito para o partido. Tenho uma amizade, talvez mais antiga, com José Serra. É verdade. Mas estamos todos a serviço do bem comum.
Folha – Imaginou-se durante a campanha que o novo governo teria apenas oito pastas. Depois, Fernando Henrique falou que nomearia "12 Jatenes". Foram nomeados, afinal, 20 ministros, fora os secretários. Por que se abandonou a idéia de enxugar a Esplanada?
Carvalho – Não se abandonou a idéia de fazer uma reforma do Estado, uma reforma administrativa. Só que optamos por fazê-la dentro do governo. Isso será feito. Tanto que se está constituindo uma estrutura dentro do Ministério da Administração. A pasta agora vai se chamar Ministério da Administração e da Reforma do Estado. Um dos objetivos é enxugar toda a estrutura do Estado. Apenas teremos um pouco mais de tempo para discussão interna, para não fazermos uma reforma de orelhada, como se fez no passado. Dois ministérios, o da Integração Regional e do Bem–Estar Social serão extintos já. O resto será mais amadurecido.
Folha – Não será mais difícil desmontar a máquina depois que ela estiver funcionando?
Carvalho – Claro. O processo autoritário é sempre mais fácil, mas não é necessariamente eficiente. Seria, sem dúvida, mais fácil chegar e dizer: vai ser assim daqui para a frente, doa a quem doer. Reconhecemos o risco das reações. Mas ele tem que ser enfrentado. Há uma integração de todas as pessoas que foram convidadas para o compor o governo. Há o compromisso de que o Estado deve ser reduzido ao seu tamanho necessário. O tema será objeto da nossa primeira reunião de trabalho, na semana que vem (sexta e sábado).
Folha – Quer dizer que os ministros estão entrando no governo avisados de que eventualmente o seu ministério pode ser eliminado?
Carvalho – Claro. Alguns serão responsáveis por eliminar. Todos os que estão aí estão sintonizados com o programa de governo.
Folha – Já se sabe qual é o tamanho ideal do Estado?'
Carvalho - Não. Há reflexões individuais a esse respeito. Mas ainda não se chegou a um ponto comum.
Folha – Existe um prazo para que a reforma seja feita?
Carvalho – Não. O primeiro ano é o ano da reflexão, do diagnóstico. Somos fiéis à nossa ideologia do processo. E o processo deve começar neste ano de 95. Não há nenhuma solução que aconteça por um estalo ou por um decreto.
Folha – Algumas acomodações administrativas serão feitas agora. Haverá medida provisória?
Carvalho – Sim. Está pronta. Tivemos algumas observações nos jornais, reações a algumas distribuições. Mas tudo está absolutamente apaziguado e definido. Essa ação específica da ação regional (Secretaria de Integração Regional, entregue a Cícero Lucena, do PMDB) continua sendo preocupação do presidente.
Junto com a questão social, é um dos braços com os quais nos preocupamos. Portanto, é algo prioritário. O que estamos fazendo é desmontando uma máquina de fisiologismo, que operava dentro da ação social e regional. Nós consideramos que a responsabilidade do governo central nessa matéria é traçar diretrizes.
A execução tem que ser descentralizada. Como se trata de definir diretrizes a Integração Regional está no Ministério do Planejamento, com status de secretaria especial.
Folha – A área social também fica com o Planejamento?
Carvalho – Não. Nesse caso, como não é propriamente uma área de definição de política, mas de coordenação e de ação, de mobilização de recursos da sociedade, o programa de comunidade solidarária ficará na Casa Civil que tem a função de coordenação de ação do governo.
Folha – Além destas mudandas e da transferência dos recursos hídricos para a pasta do Meio Ambiente há outros ajustes?
Carvalho – Há ajustes decorrentes da extinção dos ministérios do Bem-Estar Social e da Integração Regional. Há também a transformação da Secretaria de Assuntos Estratégicos, com o desmenbramento das áreas de informação e de estratégia.
Folha - Existe algum outro ato de governo a ser baixado no dia da posse?
Carvalho – Uma série deles, todos de adaptação. Nenhum susto, nada em relação à área econômica.
Folha – O sr. disse que a Casa Civil continuará se relacionando com o Congresso. A composição do segundo e do terceiro escalão passa por esse processo? Serão aceitas indicações políticas?
Carvalho – Quem está decidindo efetivamente as coisas básicas é o presidente. Os ministros têm a delegação para escolher os integrantes do segundo escalão, mas precisa sim passar pelo crivo da Presidência, para que a gente possa assegurar a harmonia. Os critérios de escolha são a competência e a integração. Não passa pela velha forma de jogar no computador e fazer as ponderações a partir da composição política.
(Josias de Souza)
</TEXT>
</DOC>
<DOC>
<DOCNO>FSP950101-163</DOCNO>
<DOCID>FSP950101-163</DOCID>
<DATE>950101</DATE>
<CATEGORY>CADERNO_ESPECIAL_-_ANOS_FHC</CATEGORY>
<TEXT>
Da Reportagem Local 
O futuro chefe da Casa Civil, Clóvis Carvalho, tem ligações com a iniciativa privada. Após trabalhar no governo de Franco Montoro (1983-1987), como secretário de Planejamento de São Paulo, foi vice-presidente de Recursos Humanos do Grupo Villares.
Amigo de José Serra, o novo ministro do Planejamento, Carvalho foi responsável por um projeto de reestruturação iniciado na Villares em 1990. O enxugamento que fez na empresa resultou na demissão de aproximadamente metade dos 23 mil funcionários da empresa.
Carvalho ocupou interinamente o Ministério da Fazenda, entre a saída de Fernando Henrique Cardoso e a posse de Rubens Ricupero. Nesse período, entrou em conflito com o então secretário da Receita Federal, Osiris Lopes Filho.
Carvalho também é amigo do presidente eleito. Fernando Henrique fez questão de levá-lo para a secretaria-executiva do Ministério da Fazenda. Agora, o mantém em seu ministério.
</TEXT>
</DOC>
<DOC>
<DOCNO>FSP950101-164</DOCNO>
<DOCID>FSP950101-164</DOCID>
<DATE>950101</DATE>
<CATEGORY>CADERNO_ESPECIAL_-_ANOS_FHC</CATEGORY>
<TEXT>
Da Sucursal de Brasília 
Fernando Henrique Cardoso herdará problemas que o próprio Itamar Franco admite estar deixando para seu sucessor: crise dos hospitais conveniados ao SUS (Sistema Único de Saúde) e uma dívida social que ele tentou amenizar com o abono de R$ 15 em janeiro para quem ganha salário mínimo e aposentados.
Além disso, ao assumir o cargo, FHC terá de prosseguir o processo iniciado no governo Itamar de concessão da isonomia salarial aos servidores do Executivo e punir os responsáveis pela corrupção em contratos do Ministério dos Transportes com empreiteiras.
"Ainda temos problemas na área de saúde e problemas sociais, mas fizemos o esforço que era possível em dois anos de mandato, que é muito curto", justificou-se Itamar numa entrevista aos jornalistas no último dia 29.
O presidente eleito encontrará o caos na rede hospitalar com ameaça da Federação Brasileira de Hospitais de suspender o atendimento já no início do mandato. Só metade da dívida de R$ 580 milhões em outubro pelos serviços prestados foi paga.
A equipe econômica de Itamar, sem entrosamento com a da Saúde, não deixou qualquer perspectiva para os pagamentos de novembro e dezembro, acumulando uma dívida superior a R$ 1 bilhão.
FHC sabe também que será pressionado para manter o abono de R$ 15 que elevou o salário mínimo a R$ 85, previsto apenas para janeiro. Já se prevê um rombo de R$ 240 milhões na Previdência no primeiro mês do ano.
Nas primeiras semanas no cargo, FHC terá de negociar com uma comissão formada por representantes do governo, centrais sindicais e Federação dos Aposentados as soluções imediatas para aumentar a receita da Previdência e manter o mínimo em R$ 85.
Itamar não quis punir empreiteiras e suspender contratos tidos como irregulares no Ministério dos Tranportes. Na mesma hora em que recebeu o relatório da CEI (Comissão Especial de Investigação do Executivo) com os indícios da corrupção, o presidente o repassou para Fernando Henrique Cardoso.
Na questão isonomia, Itamar deixou para seu sucessor na Presidência da República a decisão de pagar os 45% que ainda faltam para equiparar os servidores dos Três Poderes, com gasto de R$ 111 milhões por mês. Falta também pagar a diferença da gratificação aos servidores, que custa outros R$ 29,5 milhões mensais.
(Silvana de Freitas e William França)
</TEXT>
</DOC>
<DOC>
<DOCNO>FSP950101-165</DOCNO>
<DOCID>FSP950101-165</DOCID>
<DATE>950101</DATE>
<CATEGORY>CADERNO_ESPECIAL_-_ANOS_FHC</CATEGORY>
<TEXT>
Datafolha mostra que 70% dos brasileiros esperam do tucano uma gestão ótima/boa, como esperavam de Collor em 90  
CLÓVIS ROSSI 
Da Reportagem Local 
Fernando Henrique Cardoso e Fernando Collor de Mello têm algo mais em comum além do prenome idêntico e de alguns ministros: a expectativa de que FHC faça um governo "ótimo/bom" é praticamente a mesma que cercava Collor.
Em março de 1990 (a posse, naquele ano, foi no dia 15 de março), 71% dos brasileiros imaginavam que o governo Collor seria ótimo ou bom.
Hoje (ou, mais exatamente, nos dias 12, 13 e 14 de dezembro), vésperas da posse de Fernando Henrique, são 70% os que esperam do novo presidente um desempenho ótimo ou bom.
Mas a expectativa positiva é temperada por um gota de cautela. Apenas 21% dos pesquisados acreditam que o tucano cumprirá, totalmente, as promessas feitas durante a campanha eleitoral.
A maioria relativa (47%) acha que as promessas serão resgatadas apenas parcialmente. Quase um quarto dos brasileiros (23%) acha que o novo presidente simplesmente não vai cumprir o que prometeu nos palanques.
Tais números surgem de pesquisa feita pelo Datafolha com 14.151 eleitores distribuídos por 403 cidades de todas as unidades da Federação.
A pesquisa mostra que a expectativa de um governo "ótimo/bom" subiu oito pontos percentuais em relação ao levantamento anterior, feito a 19 de outubro, após a vitória eleitoral do então candidato do PSDB.
Os pessimistas (aqueles que esperam um governo "ruim/péssimo") caíram de 8% para 5%. Também nesse item, há um empate em relação às expectativas sobre o governo Collor (os pessimistas de então eram 4%).
A confiança em FHC é maior entre os mais ricos e os de escolaridade superior. Dos que ganham mais de 10 salários-mínimos, 74% esperam um governo "ótimo/bom", contra 68% entre os que recebem até 5 mínimos.
Números quase idênticos surgem entre os de nível universitário (75% de otimistas) e entre os que fizeram apenas até o 1º grau (69% de "ótimo/bom").
É uniforme o otimismo em todas as quatro regiões em que se dividiu a pesquisa, mas, pela natureza da cidade, há diferenças importantes: nas regiões metropolitanas, os plenamente otimistas são menos (64%) do que no interior (73%).
Por Estados, o campeão nacional da confiança em FHC é Goiás (78%). Os menos entusiasmados são os cariocas: 60% acreditam em um governo "ótimo/bom", dez pontos percentuais abaixo da média nacional.
Como era previsível, os peesedebistas são, entre os simpatizantes de partidos, os mais confiantes. Deles, 82% esperam um governo "ótimo/bom".
Igualmente previsível é o fato de o PT fornecer o maior contingente de pessimistas: 10% dos que se dizem simpatizantes do partido de Luiz Inácio Lula da Silva esperam um governo "ruim/péssimo", quando a média geral dessa qualificação é de apenas 4% ou 2,5 vezes menos.
Mas os eleitores de Enéas Carneiro (Prona) empatam com os petistas em pessimismo. Também são 10% os que acreditam que FHC fará um governo ruim ou péssimo.

A direção do Datafolha é exercida pelos sociólogos Antônio Manuel Teixeira Mendes e Gustavo Venturi, tendo como assistentes Mauro Francisco Paulino, Emilia de Franco e a estatística Renata Nunes César.
</TEXT>
</DOC>
<DOC>
<DOCNO>FSP950101-166</DOCNO>
<DOCID>FSP950101-166</DOCID>
<DATE>950101</DATE>
<CATEGORY>CADERNO_ESPECIAL_-_ANOS_FHC</CATEGORY>
<TEXT>
SILVANA DE FREITAS;  WILLIAM FRANÇA 
Da Sucursal de Brasília 
A metade dos ministros do governo Itamar Franco deve permanecer no governo Fernando Henrique Cardoso –seja no mesmo cargo, em outro ministério, como presidente de estatal ou em missão no exterior.
Dos 28 ministros, nove praticamente asseguraram permanência no novo governo e pelo menos quatro negociam cargo público.
Amigos íntimos de Itamar, Henrique Hargreaves (Casa Civil) e Djalma Morais (Comunicações) garantiram funções em estatais.
O primeiro presidirá a ECT (Empresa Brasileira de Correios e Telégrafos) e o segundo deve ficar com a Telesp.
O ministro Celso Amorim (Relações Exteriores) chefiará a missão brasileira na ONU (Organização das Nações Unidas).
Ficarão no cargo os ministros Zenildo Lucena (Exército) e Israel Vargas (Ciência e Tecnologia).
Romildo Canhim (Administração Federal), Sérgio Cutolo (Previdência), Fernando Cardoso (Gabinete Militar) e Arnaldo Leite Pereira (Estado-Maior das Forças Armadas) já têm destino acertado no novo governo.
Canhim será diretor da Telebrás em Campinas (SP); Cutolo presidirá a Caixa Econômica Federal; Cardoso comandará o Batalhão do Exército em Fortaleza (CE) e Pereira será observador militar do Brasil na ONU.
Devem ser confirmados ainda os nomes dos ministros Lélio Lôbo (Aeronáutica) para presidente da Telebrás; Ivan Serpa (Marinha) para diretor de Transportes da Petrobrás; Delcídio Gomez (Minas e Energia) para presidente da Eletrosul e Geraldo Quintão (advogado-geral da União) para permanecer no cargo.
Outros ministros continuarão exercendo funções públicas fora do Executivo. Mauro Durante (Secretaria Geral) presidirá o conselho deliberativo do Sebrae. Leonor Franco (Bem-Estar Social), futura primeira-dama em Sergipe, será a presidente do Conselho Nacional do Sesi.
O único que deve deixar o país com projeto pessoal é Ciro Gomes (Fazenda). Diz que vai estudar na Universidade de Harvard (EUA).
José de Castro 
O ex-consultor geral da União José de Castro Ferreira quer aproveitar a notoriedade que alcançou como principal interlocutor do presidente Itamar Franco e estender sua atividade como advogado a grandes centros do país.
Seu objetivo é se transformar em um profissional de renome, emitindo pareceres para grandes empresas privadas com honorários de até R$ 20 mil.
O advogado foi o assessor de Juiz de Fora mais influente nos 27 meses da gestão de Itamar.
No governo, Castro foi consultor-geral da República, advogado-geral da União e presidente da Telerj (Telecomunições do Rio de Janeiro). Nunca desejou assumir ministério. Preferia a discrição dos cargos que ocupou.
Nos próximos dois meses, Castro pretende escrever um livro –ainda sem título– sobre o que chama de "a história secreta do governo Itamar Franco".

Colaborou EVANDRO EBOLI, da Agência Folha em Juiz de Fora
</TEXT>
</DOC>
<DOC>
<DOCNO>FSP950101-167</DOCNO>
<DOCID>FSP950101-167</DOCID>
<DATE>950101</DATE>
<CATEGORY>CADERNO_ESPECIAL_-_ANOS_FHC</CATEGORY>
<TEXT>
Da Reportagem Local 
O tucano Mário Covas assume hoje o governo de São Paulo preocupado em recuperar as finanças do Estado, que tem hoje uma dívida de R$ 31 bilhões.
A partir de amanhã, Covas comanda uma "drástica" redução de gastos -como define o futuro secretário da Fazenda, Yoshiaki Nakano.
Essa redução prevê a paralisação de obras executadas pelo Estado, cortes de pessoal, parcelamento de pagamentos e um "choque de eficiência" na gestão das empresas estatais.
Nakano diz que a situação financeira do Estado é tão crítica que nem mesmo o pagamento do funcionalismo, em janeiro, está assegurado.
Como a arrecadação prevista até o dia 5 de janeiro não cobre o gasto de RS$ 600 milhões com funcionalismo, o pagamento pode ser parcelado em duas vezes.
Também a partir de amanhã, Covas tenta renegociar com o Banco Central a dívida do Estado com o Banespa, que chega hoje a US$ 8 bilhões.
O tucano quer alongar de 12 para 20 anos o prazo para pagamento da dívida e a redução dos juros, que hoje oscilam em torno de 60% ao ano.
Posse 
Três cerimônias vão marcar a posse e a transmissão de cargo ao governador eleito.
Elas foram concentradas na manhã e início da tarde para que Covas possa viajar a Brasília para assistir à posse do presidente eleito Fernando Henrique Cardoso (PSDB).
Covas mandou renovar seu guarda-roupa para as festas e os primeiros dias de seu governo. Ele encomendou ao seu alfaiate, Brasilino Tomasini, seis ternos e 30 camisas. Nos últimos quatro meses, Tomasini já lhe havia feito, sob medida, outros oito ternos. Cada terno feito pelo alfaiate custa cerca de R$ 700,00.
Na sua posse, segundo Tomasini, o governador eleito vai estar trajando um terno azul-marinho, de micro-fibra japonesa, uma camisa branca e uma gravata de seda mesclada de azul-claro e vermelho.
A primeira cerimônia vai acontecer a partir das 9h30 no plenário da Assembléia Legislativa do Estado, onde Covas e seu vice, Geraldo Alckmin Filho, participarão de uma sessão solene.
Eles farão as leituras de seus termos de compromissos constitucionais e serão declarados empossados em seus cargos.
Ao ler os termos de compromissos, os dois vão prometer respeitar as constituições federal e estadual e observar as leis no exercício de seus cargos. A sessão deve durar cerca de 40 minutos.
Logo depois, na frente da Assembléia, Covas vai participar de uma cerimônia militar. Ele vai receber a continência militar comandada por um oficial da Polícia Militar e passará em revista as tropas.
Da Assembléia, o tucano irá ao Palácio dos Bandeirantes, onde o governador Luiz Antônio Fleury Filho (PMDB) vai lhe transmitir o cargo por volta das 12h.
Durante o trajeto, seu carro será escoltado por batedores e pelos lanceiros do Regime de Cavalaria da PM.
Discurso 
No Palácio, Covas fará um discurso para anunciar algumas das medidas que pretende adotar em seu governo e assinará o termo de posse de seus secretários. Às 14h, viaja a Brasília.
Durante a posse, os secretários de Covas podem anunciar os futuros presidentes de estatais, entre elas as chamadas "energéticas" (Eletropaulo, Cesp, CPFL e Comgás).
</TEXT>
</DOC>
